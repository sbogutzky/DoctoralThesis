

% ******************************* PhD Thesis Template **************************
\documentclass[12pt, twoside=semi, DIV=calc, pagesize, parskip=half, listof=totoc, bibliography=totoc, open=right, listof=nochaptergap, pointlessnumbers, final]{scrreprt} 

% BCOR=12mm,
% ********************************** Preamble **********************************
% Preamble: Contains packages and user-defined commands and settings
% ****************************** Packages **********************************

\usepackage[utf8]{inputenc} % Um die Umlaute direkt tippen zu können, kann das Paket inputenc verwendet werden.
\usepackage[T1]{fontenc} % Zusammen mit dem inputenc Paket sollte das fontenc Paket verwendet werden. Dadurch lassen sich Probleme beim Trennen von Wörter, die Umlaute enthalten, vermeiden beziehungsweise sie können dann erst überhaupt getrennt werden.
\usepackage{lmodern} % Das lmodern (= Latin Modern) Paket verändert die verwendete Schriftart. Der Hauptunterschied ist die Darstellung der Schrift innerhalb von pdf Dateien, während diese bei der Verwendung der normalen Standard Schriftart von Latex - Computer Modern - doch recht pixelige auf den Betrachter wirkt, erscheint der Text der mit Latin Modern als Schriftart geschrieben worden ist doch um einiges flüßiger.
\usepackage[ngerman]{babel} % Aufgrund der Tatsache, dass LATEX im englischen Sprachraum entwickelt worden ist, sind einige Anpassungen nötig um es auch mit anderen Sprachen nutzen zu können.
\usepackage[babel,german=quotes]{csquotes}
%\usepackage{natbib} % Bei natbib handelt es sich um ein Paket, das erlaubt, den Bibliographiestil einfach und direkt zu verändern.
\usepackage[mla]{ellipsis} % ellipsis sind Auslassungspunkte. Die Option mla ist für die eckigen Klammern zuständig
\usepackage[onehalfspacing]{setspace} % Das Setspace Paket ermöglicht es auch recht einfach Weise den Zeilenabstand zu ändern.
\usepackage[dvipsnames, table, xcdraw]{xcolor} % Unterstützung von Farbmodellen
\usepackage[backend=bibtex, pagetracker=true, bibencoding=ascii, natbib=true, style=authoryear-ibid, backref=true, backrefstyle=all+, url=true, maxnames=3, minnames=1, uniquelist=true]{biblatex} % Paket bietet erweiterte bibliographischen Einrichtungen für die Verwendung mit LaTeX in Verbindung mit BibTeX
\usepackage{tikz} % Erlaubt Grafiken programmatisch zu erstellen
\usepackage{graphicx} % Ist Schlüssel-Wert-Schnittstelle für optionale Argumente für den "includegraphics" Befehl
\usepackage{rotating} % Das rotating Paket ermöglicht das rotieren von Grafiken
\usepackage{caption} % Das caption Paket bietet einem Mittel und Wege, das Erscheinungsbild der Bild- und Tabellenbeschriftungen den eigenen Wünschen bzw. Vorgaben anzupassen.
\usepackage{url} % Paket für Umbrüche in E-Mail-Adressen, URLs usw.
\usepackage[pdfauthor={Simon Bogutzky}, pdftitle={Objektivierung und Messung von Flow-Erleben beim Gehen und Laufen für mobile Applikationen}, colorlinks=true, citecolor=FlowCyan, urlcolor=FlowCyan, linkcolor=FlowCyan, linktoc=page, pdfpagelabels]{hyperref} % Das Hyperref Paket ist das Paket schlecht hin wenn es darum geht mit Hilfe von LATEX PDF Dokumente zu erstellen. Da es dem Benutzer nicht nur die Möglichkeit gibt Links und Verweise innerhalb von PDF Dokumenten zu erzeugen und zu setzen, sondern auch die Änderung von Einstellungen innerhalb des PDF Dokumentes zulässt.
\usepackage[acronym, toc, shortcuts, indexonlyfirst]{glossaries} % Das Paket glossaries unterstützt Akronyme und mehrere Glossare
\usepackage{tabu} % Mit longtable können Sie Tabellen zu schreiben, die auf der nächsten Seite weiter gehen
\usepackage{tabularx} % Erweiterung von tabular um eine Spalte des xten Größe zu erstellen
\usepackage{booktabs} % Anfertigen von hochwertigen Tabellen mit LATEX
\usepackage{dcolumn} % Ermöglicht das spezielle Anpassen einer Zelleninhalts einer Tabelle
\usepackage{ragged2e} % Setzt Flattersatz

% **************************** User-defined commands ***************************

\let\oldtabular\tabular
\renewcommand{\tabular}{\footnotesize\oldtabular} % Ändere Schriftgröße in den Tabellen

\let\oldlongtable\longtable
\renewcommand{\longtable}{\footnotesize\oldlongtable} % Ändere Schriftgröße in den Tabellen

\let\oldtabularx\tabularx
\renewcommand{\tabularx}{\footnotesize\oldtabularx} % Ändere Schriftgröße in den Tabellen

\renewcommand*{\nameyeardelim}{\addspace} % Entfernt Komma zwischen Author und Jahresjahr
\renewcommand{\arraystretch}{1.5} % 1.5-fache Tabellenzeilenhöhe

% ********************************** Settings **********************************

\graphicspath{{Abbildungen/}} % Abbildungspfad

\pdfinclusioncopyfonts = 1 % Nutze explizit die Schriftarten, die in der PDF eingebettet sind

\definecolor{FlowCyan}{RGB}{52, 128, 164} % RGB Farbe definiert
\definecolor{FlowRed}{RGB}{218, 44, 56} % RGB Farbe definiert
\definecolor{FlowDarkGrey}{RGB}{79, 80, 84} % RGB Farbe definiert

% Definition des DOI-Formats im Literaturverzeichnis
\DeclareFieldFormat{doi}{
\newline
\mkbibacro{DOI}\addcolon\space
\ifhyperref
{\href{http://dx.doi.org/#1}{\nolinkurl{#1}}}
{\nolinkurl{#1}}}

% Definition des URL-Formats im Literaturverzeichnis
\DeclareFieldFormat{url}{
\newline
\mkbibacro{URL}\addcolon\space
\ifhyperref
{\href{http://#1}{\nolinkurl{#1}}}
{\nolinkurl{#1}}}

% Definition des ISBN-Formats im Literaturverzeichnis
\DeclareFieldFormat{isbn}{
\newline
\mkbibacro{ISBN}\addcolon\space#1}

\newcolumntype{x}{D{,}{,}{3.3}} % Dcolumn-Format definiert
\newcolumntype{y}{D{;}{\pm}{5.4}} % Dcolumn-Format definiert

\AfterTOCHead{\singlespacing}
\KOMAoptions{DIV=last}

\sloppy
\usepackage{longtable}
\usepackage{ltablex}

% ************************ Thesis Information & Meta-data **********************
% Thesis title and author information, reference file for biblatex
%!TEX root = /Users/sbogutzky/Entwicklung/projects/bogutzky/repositories/2939413/final-draft.tex
\title{Objektivierung und Messung von Flow-Erleben beim Gehen und Laufen für mobile Applikationen}
\author{Simon Bogutzky}
\date{\today}

\bibliography{Bibliographie/bibliographie}

\begin{document}

\pagenumbering{arabic}

\begin{center}
\Huge Errata\\
\vspace{5mm}
\normalsize Simon Bogutzky (2016)\\
\normalsize Objektivierung und Messung von Flow-Erleben beim Gehen und Laufen für mobile Applikationen\\
\vspace{5mm}
\normalsize 21. Februar 2017\\
\end{center}

Das erste Erratum in der nachfolgenden Liste hat alle weiteren Errata zur Folge. Es bezieht sich auf die nicht korrekte Wiedergabe der Erkenntnisse der Korrelationsmatrix (Tabelle~\ref{tab:korrelationen_3}). Auf das Gesamtergebnis der von mir eingereichten Disseration hat die nicht korrekte Wiedergabe keinen Einfluss, da ich den nicht korrekt wiedergegeben Zusammenhang in der Effektanalyse auf \textbf{S. 135-133} belegt habe.


	\begin{tabularx}{\textwidth}{|c|c|X|X|} 
	\hline
	\textbf{S.} & \textbf{Abs.} & \textbf{vorherige Version} & \textbf{geltende Version}\\
	\hline
	\endhead
		130 & 1 & Ein weiterer linear positiver Zusammenhang wurde zwischen den beiden nicht normalverteilten impliziten Merkmalen mittlere HR und mittlerem normalisiertem Shannon Entropie Index festgestellt. & Den vermuteten linearen positiven Zusammenhang zwischen dem Generalfaktor der FKS und dem normalisierten Shannon Entropie Index der kardio-lokomotorischen Phasensynchronisation konnte ich nicht feststellen, dafür einen positiven linearen Zusammenhang zwischen \emph{Absorbiertheit} und dem normalisierten Shannon Entropie Index.\\
		\hline
		130 & 3 & Damit kann ich keinen direkten Zusammenhang zwischen Flow-Erleben und kardio-lokomotorischer Phasensynchronisation feststellen (Abschnitt~5.3.2, $H^1$). Aus diesem Grund testete ich im nächsten Schritt den Effekt einer \emph{physiologisch gemessenen AFP} auf das Flow-Erleben (Abschnitt~5.3.2, $H^2$). & Damit kann ich einen Zusammenhang zwischen einer der Dimensionen des Flow-Erlebens und der kardio-lokomotorischen Phasensynchronisation feststellen, aber die $H^1$ (Abschnitt~5.3.2) nicht bestätigen. Aus diesem Grund testete ich im nächsten Schritt den Effekt einer \emph{physiologisch gemessenen AFP} auf das Flow-Erleben (Abschnitt~5.3.2, $H^2$).\\
		\hline
		136 & 4 & Ich konnte keinen linearen Zusammenhang zwischen Flow-Erleben und kardio-lokomotorischer Phasensynchronisation feststellen. & Ich konnte keinen linearen Zusammenhang zwischen dem Gesamtkonstrukt des Flow-Erlebens, welches durch die FKS erfragt wird, und der kardio-lokomotorischen Phasensynchronisation feststellen.\\
		\hline
		137 & 5 & Da einige Flow-Forscher wie Peifer et al. (2014) sogar die \emph{Absorbiertheit} als eindeutigeres Flow-Merkmal ansehen, gehe ich bei den Ergebnissen von einer indirekten Verbindung von Flow-Erleben und kardio-lokomotorischer Phasensynchronisation aus. & Da einige Flow-Forscher wie Peifer et al. (2014) die \emph{Absorbiertheit} als eindeutigeres Flow-Merkmal ansehen, gehe ich bei den Ergebnissen von einem unmittelbaren Zusammenhang von Flow-Erleben und kardio-lokomotorischer Phasensynchronisation aus.\\
		
		\hline
		150 & 3 &  [...] konnte aber in den vorgestellten Studien keinen direkten Zusammenhang zum Flow-Erleben nachweisen. & [...] konnte aber in den vorgestellten Studien lediglich einen Zusammenhang mit einer der Dimensionen der FKS (\emph{Absorbiertheit}) nachweisen.\\
		\hline 
		157 & 1 & Ich konnte einen direkten intraindividuellen Zusammenhang zwischen Flow-Erleben und Doppelschrittfrequenz sowie einen indirekten Zusammenhang zwischen Flow-Erleben und der kardio-lokomotorischen Phasensynchronisation (Anschnitt~3.4.4) als eine implizit \emph{physiologisch messbare AFP} nachweisen. & Ich konnte einen intraindividuellen Zusammenhang zwischen Flow-Erleben und Doppelschrittfrequenz sowie einen interindividuellen Zusammenhang zwischen Flow-Erleben und der kardio-lokomotorischen Phasensynchronisation (Anschnitt~3.4.4) als eine implizit \emph{physiologisch messbare AFP} nachweisen.\\
		\hline
		158 & 2 & Diesen direkten positiven linearen Zusammenhang konnte ich für das Gehen und das Laufen nicht nachweisen. Hierfür sind weitere Studien notwendig. & Diesen positiven linearen Zusammenhang konnte ich für das Gehen nicht nachweisen. Hierfür sind weitere Studien notwendig.\\
		
		\hline
	\end{tabularx}

\setcounter{chapter}{5}
\setcounter{table}{5}

\begin{sidewaystable}
	\centering \caption[Korrelationsmatrix (Finale Studie: Laufen)]{Korrelationsmatrix der finalen Studie zum Flow-Erleben beim Laufen: Arithmetisches Mittel, Standardabweichung und Korrelationen\\
	\hspace{
	\textwidth}\emph{Anmerkung}: Bew. = Bewegungsaufwand \\
	\hspace{
	\textwidth}* Korrelation ist auf dem Niveau von 0,05 (zweiseitig) signifikant \\
	\hspace{
	\textwidth}** Korrelation ist auf dem Niveau von 0,01 (zweiseitig) signifikant} \label{tab:korrelationen_3} 
	\begin{tabular}
		{lxxxxxxxx} \toprule & M & SD & 1 & 2 & 3 & 4 & 5 & 6 \\
		\midrule 1. Generalfaktor & 4,85 & 0,79 & & & & & & \\
		2. Glatter Verlauf & 4,86 & 0,95 & 0,92** & & & & & \\
		3. Absorbiertheit & 4,85 & 0,87 & 0,78** & 0,47** & & & & \\
		4. HR ($1/min$) & 166,36 & 14,53 & 0,12 & 0,02 & 0,24 & & & \\
		5. Norm. Shan. Entr. Index & 0,09 & 0,13 & 0,29 & 0,18 & 0,36* & -0,04 & & \\
		6. Doppelschrittfrequenz ($1/min$) rechts & 82,68 & 4,86 & 0,29 & 0,22 & 0,29 & 0,21 & 0,02 & \\
		7. Bew. ($\times 10^3 \: m^2 \cdot s^{-5}$) rechts & 33,21 & 4,60 & -0,10 & 0,03 & -0,28 & 0,33 & -0,16 & 0,28 \\
		\bottomrule 
	\end{tabular}
\end{sidewaystable}

\end{document} 
\section*{Zusammenfassung}
Das Flow-Erleben wird weithin als eine Quelle der Kreativität und Innovation, der körperlichen und geistigen Höchstleistungen sowie des Wohlbefindens verstanden. Es definiert eine optimale Erfahrung einer Person, die eine Tätigkeit ausübt. Das Ziel der vorliegenden Arbeit ist Flow-Erleben als Maß in Apps für das Gehen und das Laufen zu verwenden. Lauf-Apps dienen als Anwendungsbeispiel, in denen Flow-Erleben als Maß des Wohlbefindens, die überwiegend leistungsbezogenen Kennzahlen wie z. B. zurückgelegte Strecke, Kalorienverbrauch, usw. ergänzt. Damit soll das Fundament geschaffen werden, Lauf-Apps zu entwickeln, die die Voraussetzungen verbessern, Flow beim Laufen zu erleben. Die vorliegende Arbeit beschreibt die Suche nach einem validen impliziten Messverfahren des Flow-Erlebens, das eine Echtzeitverarbeitung zulässt und auf einem Smartphone mit zusätzlichen technischen Geräten zu realisieren ist. Hierzu präsentiert sie Kandidaten für ein implizites Messverfahren des Flow-Erlebens und stellt drei Feldstudien mit Läufern und gehenden Personen vor, in denen die Zusammenhänge zwischen diesen potenziellen Kandidaten und expliziten Merkmalen des Flow-Erlebens gemessen mit der Flow-Kurzskala geprüft werden. Trotz widerlegbarer Ergebnisse, gibt die vorliegende Arbeit interessante Hinweise auf Zusammenhänge zwischen Flow-Erleben und mindestens zwei der Kandidaten beim Laufen. Damit bietet sie mehrere Anknüpfungspunkte für fortführende Forschungsarbeiten. 
\pagebreak

\section*{Abstract}
Flow-experience is widely understood as a source of creativity and innovation, physical and mental peak performances, and well-being. The flow-experience is defined as the optimal experience of a person performing an activity. The aim of this dissertation is to use flow-experience as a measure in apps for walking and running. Running apps serve as an application context, in which flow-experience as a measure of well-being, should complement the performance-related metrics such as distance traveled, calories burned, etc. Thereby it should create the foundation to develop run apps that improve the conditions of experiencing flow while running. The present work gives a description of the search for a valid implicit measurement method of flow-experience that allows real-time processing and can be implemented on a smartphone with additional technical equipment. The dissertation introduces candidates for an implicit measurement method of flow-experience and presents three field studies with runners and walking people, in which relationships between these potential candidates and explicit features of the flow-experience measured by the flow-short-scale, were tested. Despite rebuttable results, the present work provides interesting indications on relationships between flow-experience and at least two of the candidates while running. Thus, it offers several starting points for further research. 
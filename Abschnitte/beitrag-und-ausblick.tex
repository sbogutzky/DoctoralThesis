%!TEX root = /Users/sbogutzky/Entwicklung/projects/bogutzky/repositories/2939413/final-draft.tex
\chapter{Wissenschaftlicher Beitrag und Ausblick} 
\label{cha:wissenschaftlicher_beitrag_und_ausblick}
Im letzten Kapitel verdeutliche ich die wissenschaftlichen Beiträge, die ich mit der vorliegenden Arbeit geleistet habe. Abschnitt~\ref{sec:wissenschaftlicher_beitrag} gibt eine entsprechende Zusammenfassung, bevor ich abschließend in Abschnitt~\ref{sec:ausblick} einen Ausblick auf Anknüpfungspunkte für weitere Forschungsarbeiten gebe.

\section{Wissenschaftlicher Beitrag}
\label{sec:wissenschaftlicher_beitrag} 
\subsection{Theoretische Beiträge}
Ein Thema der vorliegenden Arbeit war Flow-Erleben mobil und im Verlauf des Gehens und Laufens zu messen. Ich identifizierte nach einer umfassenden Literaturrecherche potenzielle Kandidaten für ein implizites Messverfahren des Flow-Erlebens und führte eine zusammenfassende Beurteilung ihrer Eignung im Verlauf des Gehens und Laufens durch. 

Ich skizzierte nach der Beurteilung Messmethoden für ein Konzept einer mobilen Messanwendung (\ac{PPC}) für Smartphones. Zusätzlich motivierte ich den Einsatz der Flow-Zustand-Messung in mobilen Laufanwendungen als ergänzendes Maß zu den leistungsbezogenen Parametern wie z. B. Laufdistanz und Kalorienverbrauch.

\subsection{Empirische Beiträge} 
Ich untersuchte im Gegensatz zu den meisten anderen Arbeiten im Bereich der Flow-Forschung, die sich mit der Messung von Flow-Erleben anhand physiologischer Merkmale befassen, im Rahmen dieser Arbeit eine Tätigkeit, die eine mittlere bis hohe physische Belastung für den menschlichen Organismus darstellt. Ein zentraler wissenschaftlicher Beitrag ist die Beurteilung der Eignung der gewählten Messmethoden für die Tätigkeiten Gehen und Laufen. Zu diesem Zweck identifizierte ich unterschiedliche Forschungslücken und führte entsprechende Studien durch. Die Forschungslücken, die ich in der vorliegenden Arbeit im Feld der Flow-Forschung bearbeitet habe, sind in Abbildung~\ref{fig:8_1_forschungslueken} mit Cyan markiert. Die mit grün markierten Felder sind schon von anderen Wissenschaftlern aus der Flow-Forschung untersucht worden. 

\begin{figure}[t]
	\centering
		\includegraphics[width=1.00\textwidth]{8-1-forschungslueken.pdf}
	\caption[Bearbeitete Forschungslücken aus der Flow-Forschung]{Bearbeitete Forschungslücken aus der Flow-Forschung. Quelle: Eigene Darstellung}
	\label{fig:8_1_forschungslueken}
\end{figure}

Ich prüfte in drei Feldstudien mit Läufern und gehenden Personen den Zusammenhang zwischen potenzielle Kandidaten für ein implizites Messverfahren des Flow-Erlebens und expliziten Merkmalen des Flow-Erlebens gemessen mit der \ac{FKS}. Zu Flow-Erleben und impliziten Merkmalen, die auf der Biomechanik der Tätigkeit beruhen, gab es meiner Kenntnis nach bislang keine wissenschaftlichen Veröffentlichungen. 

\subsection{Technische Beiträge}
Mit der technischen Realisierung einer Messanwendung zur Messung von kardiologischen Daten und Körperbewegungsdaten durch externe Sensorik sowie der Abfrage von Fragenbögen (PPC), die eine Referenz zum Flow-Erleben geben, erprobte ich deren Echtzeiterhebung, die ich später mit der internen Sensorik eines Smartphones im Demonstrator der \emph{Flow-Maschine} realisierte. 

Ich entwickelte zur Auswertung der Daten eine Verarbeitungspipeline (\ac{PPP}), die die gesammelten Daten segmentiert, die biomechanischen Gangmerkmale identifiziert und die impliziten Merkmale zum Vergleich mit den expliziten Flow-Merkmalen berechnet. Die biomechanische Gangmerkmalerkennung passte ich an die Anforderung einer Echtzeiterkennung durch ein Smartphone an und integrierte sie in den Demonstrator der Flow-Maschinen. 

\section{Ausblick} 
\label{sec:ausblick} 
Die in der vorliegenden Arbeit präsentierten Apparate, Studien und Ergebnisse bieten an mehreren Stellen Anknüpfungspunkte für fortführende Forschungsarbeiten.

Im Bereich der software-technischen Analyse fehlt ein benutzerfreundliches Werkzeug zur Visualisierung zeitbasierter multimodaler Daten. Die von mir realisierte Verarbeitungspipeline übernimmt zwar einige Aufgaben eines solchen Analysewerkzeuges, ist aber alles andere als benutzerfreundlich. 

In Abschnitt~\ref{sec:zusammenfassung_5} stellte ich die These auf, dass die kardio-lokomotorische Phasensynchronisation eine Voraussetzung für Flow-Erleben beim Laufen darstellt. Diesen positiven linearen Zusammenhang gilt es zu manifestieren; das ist eine Aufgabe der Sportpsychologie. Sporthochschulen mit ihren menschlichen und technischen Ressourcen sind in der Lage, diese These anhand einer Laufbandstudie zu prüfen und auch den positiven linearen Zusammenhang zwischen Flow-Erleben und Bewegungsaufwand allgemeingültig nachzuweisen. 

Des Weiteren ist ein Vergleich der kardio-lokomotorischen Phasensynchronisation gemessen am Bewegungsablauf des Oberarms mit der kardio-lokomotorischen Phasensynchronisation gemessen am unteren Bein von hohem Interesse für die Entwicklung einer mobilen Laufanwendung, die das notwendige Equipment gering halten soll

Für genauere Werte des Bewegungsaufwands ist das in Abschnitt~\ref{par:r_programme_zur_erkennung} angesprochene Orientierungsproblem befriedigender zu lösen, damit wir die Gravitation präzise von der Benutzerbeschleunigung trennen können. 

Ist erst einmal die kardio-lokomotorische Phasensynchronisation als implizites Merkmal des Flow-Erlebens manifestiert, dann ist von höherem Interesse, Übergänge des Flow-Erlebens anhand der Musterveränderung im Synchrogramm festmachen zu können. Die Identifikation von Übergängen liefert Interaktionsgestaltern einen größeren Gestaltungsraum für Echtzeit-Rückmeldungen. Die Gestaltung der Rückmeldung, die Flow-Erleben beeinflusst, ist dann die nächste große Forschungsarbeit.
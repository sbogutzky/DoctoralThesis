

%!TEX root = /Users/sbogutzky/Entwicklung/projects/bogutzky/repositories/2939413/final-draft.tex
\section{Arbeitsschritte der technischen Arbeit} 

% (fold)
\label{sec:arbeitsschritte_der_technischen_arbeit}

Der erste technische Schritt der vorliegenden Arbeit (Abschitt~\ref{sec:herangehensweise} ist die Entwicklung einer mobile Messanwendung für Smartphones. Diese Anwendung mit dem Namen \ac{PPC} \ref{Bogutzky2016} entwickelte ich mit Jan Christoph Schrader, einer studentischen Hilfskraft, die im Kontext des \ac{BMBF}-Projekts eingestellt wurde. Der \ac{PPC} läuft auf dem Android OS. Die Anwendung ermöglicht es, kontinuierlich \ac{EKG}-Daten mittels des \ac{EKG}-Moduls auf einem tragbaren Shimmer \ac{IMU}, kontinuierlich Bewegungsdaten mittels der internen Smartphone Sensoren und mehreren externen tragbaren Shimmer \ac{IMU}s mit Gyro-Modul und kontinuierlich Positionsdaten mittels \ac{GPS} zu erheben. Befragungen können intervall-kontingent oder im Nachhinein mit psychometrische Skalen oder mit offenen Fragen durchgeführt werden.

Zur Identifikation eines impliziten Messverfahrens des Flow-Erlebens ist der zweite Schritt die offline Verarbeitung und Analyse. Hierzu entwickelte ich mit Phillip Marsch, einer weiteren studentischen Hilfskraft, die im Kontext des \ac{BMBF}-Projekts eingestellt wurde, eine Verarbeitungs- und Analysepipeline mit dem Namen \ac{PPP} \ref{Bogutzky2016a}. Die \ac{PPP} besteht aus Programmen, die wir in der Entwicklungsumgebung R realisierten, und Open Source Software. Eine detaillierte Darstellung des \ac{PPC}s und der \ac{PPP} folgt in den Abschnitten des Kapitels~\ref{cha:studien_zur_mobilen_und_prozessorientierten_messung}.

% section arbeitsschritte_der_technischen_arbeit (end)

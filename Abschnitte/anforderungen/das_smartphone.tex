

%!TEX root = /Users/sbogutzky/Entwicklung/projects/bogutzky/repositories/2939413/final-draft.tex
\section{Das Smartphone zur Messung von psychischen, physiologischen und motorischen Daten} 

% (fold)
\label{sec:das_smartphone_zur_messung_von_psychischen_physiologischen_und_motorischen_daten}

Als technisches Bindeglied zwischen den vier Messmethoden dient ein Smartphone. Das Smartphone ist das am schnellsten angenommene mobile Endgerät in der Geschichte der Menschheit. Menschen in Industrieländern benutzen das Smartphone alltäglich. Sie nutzen es z.~B. zum Informationsabruf, zur Unterhaltung, zur Terminplanung, für ihre Gesundheit und um ihre sozialen Kontakte zu pflegen. In der Bundesrepublik Deutschland besitzen 6 von 10 Bundesbürger ab 14 Jahren (63 Prozent) ein Smartphone; das sind 44 Millionen Menschen \citep{bitkom2015}.

Smartphones sind in der Lage programmierte Anwendungen (Apps) auszuführen und besitzen eine sehr hoch entwickelte interne Sensorik, eine hohe Speicherkapazität und eine integrierte Netzwerkanbindung. Das ermöglicht Apps zu programmieren, die die interne Sensorik auslesen und Daten auf den Smartphones oder über das Internet zu speichern. Die technischen Möglichkeiten und die allgegenwärtige Nutzung von Smartphones machen das Smartphone zu einem idealen wissenschaftlichen Werkzeug \citep{Raento2009}.

In der Mensch-Computer-Interaktion ist das Smartphone durch die notwendige Auseinandersetzung mit neuen Technologien als wissenschaftliches Werkzeug bereits etabliert \citep{Froehlich2007}. Apple's Veröffentlichung von ResearchKit 2015 zeigt, dass die Smartphone-basierende Forschung an Popularität gewinnt. ResearchKit ist ein Open Source Framework und dient als Baukasten für Forschungsanwendungen. Leider fehlt in vielen Forschungsdisziplinen außerhalb der Mensch-Computer-Interaktion häufig die Integration der technischen Möglichkeiten mangels notwendiger Programmierkenntnisse und Ressourcen.

Die Arbeitsgruppe um Gaggioli entwickelte mit Hilfe von Informatikern für mehrere interdisziplinäre Studien eine Plattform zur Erhebung von psycho-physiologischen Daten zur mentalen Gesundheitsforschung \citep{Gaggioli2013}. Ihre Plattform ermöglicht es, psychologische Daten mittels Smartphone-gestützter Experience Sampling Methode und kontinuierlich physiologische Daten mittels interner und externer Sensorik zu sammeln. Sie veröffentlichten ihre Plattform Open Source, stellten aber die Entwicklung ein oder nutzten fortan Closed Source. Damit steht ihre Plattform nicht für die meistverbreiteten Betriebssysteme wie Android OS oder iOS zur Verfügung.

% section das_smartphone_zur_messung_von_psychischen_physiologischen_und_motorischen_daten (end)



%!TEX root = /Users/sbogutzky/Entwicklung/projects/bogutzky/repositories/2939413/final-draft.tex
Im Kontext des \acs{BMBF}-Projekts untersuchten wir die Tätigkeit des Gehens (Abschnitt~\ref{sec:kontext_der_arbeit}). Das Gehen besitzt nachgewiesene positive Einflüsse auf das menschliche Wohlergehen. Gehen hat ein großes Potenzial zur Verbesserung und bei der Erhaltung von geistiger und körperlicher Gesundheit \citep{Lee2008, Morris1997}. Neben den allgemeinen Vorteilen für die körperliche Fitness, unterstützt das Gehen die kognitive Leistung und verlangsamt den Abfall kognitiver Fähigkeiten, was für die Vorbeugung von Demenz ein wichtiges Thema ist \citep{Weuve2004}. Auf psychologischer Ebene ist erwiesen, dass Gehen Depression reduziert \citep{Robertson2012}. Auf physiologischer Ebene minimiert das Gehen das Risiko an Herz-Kreislauf-Erkrankungen zu erkranken. Gewöhnliches Gehen kann in der Regel von jedermann durchgeführt werden und ist leicht in den Alltag integrierbar. In der Flow-Forschung ist das Gehen als zu untersuchende Tätigkeit noch nicht in Erscheinung getreten. In unserer Arbeit im Kontext des \acs{BMBF}-Projekts vermuten wir beim Gehen ein \emph{Flow-Format des Wohlergehens} (eher micro Flow). 

Aufgrund des in Kapitel~\ref{cha:technologie_beim_laufen} beschriebenen Anwendungsszenarios in der mobilen Mensch-Computer-Interaktion und um im Gegensatz zum Gehen einen \emph{leistungsorientierteren Flow} (eher deep Flow) zu untersuchen, beschäftige ich mich zusätzlich in der vorliegenden Arbeit mit der Tätigkeit des Laufens. 

Beide Tätigkeiten, Gehen und Laufen, lassen sich von Untersuchungspersonen in einem natürlichen Handlungsumfeld und unter geringen Einflüssen von Störfaktoren durchführen.  Beide Tätigkeiten können isoliert von anderen Menschen durchgeführt werden. Untersuchungspersonen müssen den Bewegungsablauf nicht explizit trainieren. Die Dauer der beiden Tätigkeiten ist variabel und ermöglichen eine \emph{langandauernde Durchführung} für \emph{genügend Anlaufzeit}, damit sich Flow-Erleben einstellt \citep[S.~109]{Henk2014}.

Um weitere Anforderungen für einen Versuchsaufbau, der es ermöglicht Flow-Erleben \emph{explizit}, \emph{implizit}, \emph{prozessorientiert}, \emph{mobil} und \emph{computergestützt} in natürlichen Handlungsumgebungen beim Gehen und Laufen zu erfassen, zu identifizieren, beschäftige ich mich im nachfolgenden Abschnitt mit Untersuchungen zum Laufen und Flow-Erleben. 

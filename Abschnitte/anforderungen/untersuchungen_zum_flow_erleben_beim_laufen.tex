

%!TEX root = /Users/sbogutzky/Entwicklung/projects/bogutzky/repositories/2939413/final-draft.tex
\section{Untersuchungen zum Flow-Erleben beim Laufen} 

% (fold)
\label{sec:untersuchungen_zum_flow_erleben_beim_laufen}

In der Literatur gehört Laufen zu den bevorzugten sportlichen Tätigkeiten, um Annahmen zum Flow-Erleben und verwandten Zuständen zu prüfen. \citet{Dietrich2004a} führten u. a. Laufexperimente durch, um sich ihrer These der trassierten Hypofrontalität empirisch zu nähern. 

\citet{Stoll2005} überprüften in zwei Studien mit insgesamt 234 Teilnehmern, ob eine \ac{AFP} für das Erleben für Flow beim Marathonlauf erforderlich ist und ob Flow-Erleben mit der Laufleistung beim Marathonlauf in einem Zusammenhang steht. Sie nutzen die \ac{FKS} zur Befragung von Teilnehmern nach zwei Marathonläufen. Ihre erste Studie ergab keinen Zusammenhang zwischen \ac{AFP} und dem Generalfaktor der \ac{FKS}. Ihre zweite Studie widerlegte das Ergebnis. \emph{Das lässt \citet{Stoll2005} darauf schließen, dass eine \ac{AFP} nicht zwingend erforderlich ist, um Flow beim Marathonlaufen zu erleben.} In beiden Studien fanden sie keinen Zusammenhang von Flow-Erleben und der Laufleistung. 

\citet{Schuler2009} untersuchten in drei Studien mit 109, 112 und 65 Teilnehmern, ob Flow-Erleben beim Marathonlaufen Einfluss auf die Laufleistung und auf die zukünftige Laufmotivation besitzt. Ihre Ergebnisse zeigen, dass Flow-Erleben nicht direkt mit der Laufleistung in einem Zusammenhang steht. Allerdings stieg die zukünftige Motivation zu Laufen bei denen, die Flow beim Laufen erlebten. Durch die Motivation zum Lauftraining verbessert sich indirekt die Laufleistung. In den ersten beiden Studien versetzten sich die Teilnehmer nach dem Marathonlauf in die zurückliegende Situation an den Kilometermarken 10, 20, 30 und 40 und füllten auf dessen Grundlage jeweils eine \ac{FKS} aus. In der dritten Studie befragten \citet{Schuler2009} die Läufer beim Training während des Laufens an den Kilometermarken 10, 20, 30 und 40 mit der Hilfe eines mitlaufenden Assistenten verbal. Als Fragebogen diente die \ac{FKS}. \emph{Alle drei Studien zeigten, dass das Flow-Erleben über die gelaufene Distanz hinweg abnimmt.} 

\citet{Jimenez-Torres2013} überprüften die neun Dimensionen des Flow-Erlebens nach \citet{Csikszentmihalyi1992} vor, nach dem Marathonlauf und während des Laufs bei 153 Teilnehmern. Sie nutzten ein Item jeder Dimension der \ac{FSS}-2 zur Befragung \citep{Jackson2002}. Sie stellten fest, dass der größte Anteil der neun Flow-Dimensionen während des Marathonlaufs auftrat. Nur eine geringe Anzahl der Flow-Dimensionen traten vor oder nach dem Lauf auf. 

In der Studie von \citet{Reinhardt2006} liefen 30 Teilnehmer für 40 Minuten auf einem Laufband in einer individuellen Laufgeschwindigkeit. Als Grundlage für die Laufbandgeschwindigkeit diente die \ac{HR}. Die Teilnehmer unterbrachen alle zehn Minuten ihren Lauf, um eine \ac{FKS} auszufüllen. \emph{Die Teilnehmer berichteten von einem stabilen und tiefen Flow-Erleben.} \emph{Allerdings zeigen die Ergebnisse keinen Zusammenhang von \ac{AFP} und Flow-Erleben.}

Wichtige Erkenntnisse, die ich aus den vorgestellten Untersuchungen mitnehme sind: 
\begin{itemize}
	
	\item das Flow-Erleben nimmt mit der Dauer oder der Distanz ab
	
	\item eine kurze Unterbrechung des Laufprozesses scheint keinen Einfluss auf die explizite Bewertung gehabt zu haben
	
	\item eine \ac{AFP} ist nicht zwingend für das Flow-Erleben beim Laufen erforderlich 
\end{itemize}

Alle vorgestellten Untersuchungen nutzten explizite Merkmale zur Identifikation von Flow-Erleben. Im nächsten Abschnitt beschäftige ich mich mit den Problemen bei der Messung von physiologischen Eigenschaften unter physischer Belastung, um im Anschluss Kandidaten für ein implizites Messverfahren auszuwählen, die ich einer expliziten Messung gegenüberstellen kann (Abschnitt~\ref{sec:herangehensweise}, Schritt 2). 

% section untersuchungen_zum_flow_erleben_beim_laufen (end)

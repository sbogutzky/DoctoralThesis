

%!TEX root = /Users/sbogutzky/Entwicklung/projects/bogutzky/repositories/2939413/final-draft.tex
\section{Zusammenfassung} 

% (fold)
\label{sec:zusammenfassung_4}

In diesem Kapitel beleuchte ich die Auswahl der Tätigkeiten Gehen und Laufen. Ich stelle Untersuchungen vor, die Zusammenhänge von Flow-Erleben beim Laufen untersuchten. Wichtige Erkenntnisse der Untersuchungen für die vorliegende Arbeit sind: 
\begin{itemize}
	
	\item das Flow-Erleben nimmt mit der Dauer oder der Distanz ab
	
	\item eine kurze Unterbrechung des Laufprozesses scheint keinen Einfluss auf die explizite Bewertung gehabt zu haben
	
	\item eine \ac{AFP} ist nicht zwingend für das Flow-Erleben beim Laufen erforderlich 
\end{itemize}

Im Anschluss zeige ich die Probleme der \ac{EEG}-, der \ac{EDA}- und der \ac{HRV}-Analyse unter physischer Belastung auf. In Folge dessen begründe ich die Auswahl der nachfolgenden expliziten und impliziten Messverfahren: 
\begin{enumerate}
	
	\item intervall-kontingente Erfassung und/oder Erfassung im Nachhinein durch die \ac{FKS} zur expliziten Bewertung des Flow-Erlebens
	
	\item kontinierliche Erfassung von kinematischen Daten der Bewegung zur Berechnung des Bewegungsaufwands
	
	\item kontinierliche Erfassung von physiologischen Daten des Herzen zur Berechnung der \ac{HR} und der \ac{HRV}
	
	\item 2. und 3. zur Berechnung kardio-lokomotorische Phasensynchronisation 
\end{enumerate}

Ich veranschauliche, wie ich das Smartphone in den nachfolgenden Studien als wissenschaftliches Werkzeug einsetze. Abschließend gehe ich kurz auf die technischen Arbeitsarbeitschritte ein, die für die Entwicklung einer mobilen Messanwendung für Smartphones für die nachfolgenden Studien notwendig sind.

% section zusammenfassung (end)

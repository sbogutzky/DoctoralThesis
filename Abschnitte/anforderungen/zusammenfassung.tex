

%!TEX root = /Users/sbogutzky/Entwicklung/projects/bogutzky/repositories/2939413/final-draft.tex
\section{Zusammenfassung} 

% (fold)
\label{sec:zusammenfassung_4}

In diesem Kapitel beleuchte ich die Auswahl der Tätigkeiten Gehen und Laufen. Ich stelle Untersuchungen vor, die Zusammenhänge von Flow-Erleben beim Laufen untersuchten. Wichtige Erkenntnisse der Untersuchungen für die vorliegende Arbeit sind: 
\begin{itemize}

	\item Das Flow-Erleben nimmt mit der Dauer oder der Distanz ab.

	\item Eine kurze Unterbrechung des Laufprozesses scheint keinen messbaren Einfluss auf die explizite Bewertung zu haben. 

	\item Eine \ac{AFP} ist nicht zwingend für das Flow-Erleben beim Laufen erforderlich.
	
\end{itemize}

	Im Anschluss zeige ich die Probleme der \ac{EEG}-, der \ac{EDA}- und der \ac{HRV}-Analyse unter physischer Belastung auf. Infolgedessen begründe ich die Auswahl der nachfolgenden expliziten und impliziten Messverfahren:
	
\begin{enumerate}

	\item die intervall-kontingente Erfassung und/oder Erfassung im Nachhinein durch die \ac{FKS} zur expliziten Bewertung des Flow-Erlebens, 

	\item die kontinuierliche Erfassung von kinematischen Daten der Bewegung zur Berechnung des Bewegungsaufwands, 

	\item die kontinuierliche Erfassung von physiologischen Daten des Herzens zur Berechnung der \ac{HR} und der \ac{HRV} und 

	\item eine Kombination aus 2. und 3. zur Berechnung der kardio-lokomotorischen Phasensynchronisation.
\end{enumerate}

Ich veranschauliche den Einsatz des Smartphones als wissenschaftliches Werkzeug und gehe abschließend kurz auf die technischen Arbeitsarbeitschritte ein, die für die Entwicklung einer mobilen Messanwendung für Smartphones für die nachfolgenden Studien notwendig sind.

% section zusammenfassung (end)

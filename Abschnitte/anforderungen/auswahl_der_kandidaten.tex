

%!TEX root = /Users/sbogutzky/Entwicklung/projects/bogutzky/repositories/2939413/final-draft.tex
\section{Auswahl der Kandidaten für die explizite und implizite Messung des Flow-Erlebens beim Gehen und Laufen} 

% (fold)
\label{sec:auswahl_der_kandidaten_fur_die_explizite_und_implizite_messung_des_flow_erlebens_beim_gehen_und_laufen}

Schauen wir uns die Probleme aus dem vorherigen Abschnitt an, ist der Einsatz von \ac{EEG}- und \ac{EDA}-Messungen wegen ihrer Störungsanfälligkeit fraglich. Die \ac{EMG}-Messung wäre theoretisch möglich, aber nicht sinnvoll, da aber eine trennscharfe Zuordnung zum Flow-Konstrukt fehlt. Die Aussagekraft von \ac{HRV}-Parametern und der Methodeneinsatz (zeitbezogen oder frequenzbezogen) ist von der Intensität der Beanspruchung der Tätigkeit abhängig.

Aus diesem Grund bleiben, die von uns vorgeschlagenen impliziten Messverfahren des Bewegungsaufwands (Abschnitt~\ref{ssub:der_bewegungsfluss}) und die kardio-lokomotorische Phasensynchronisation (Abschnitt~\ref{ssub:die_kardio_lokomotorische_phasensynchronisation}). Trotz mangelnder empirische Fundierung bei physischen Tätigkeiten überprüfe ich auch kardiovaskuläre Messungen mittels \ac{EKG}. 

Ein \ac{EKG} hat gegenüber den Geräten, die für eine \ac{EEG}, eine \ac{EDA} oder eine \ac{EMG} notwendig sind, den Vorteil, dass es effizient im Freien einzusetzen ist und dass dessen Messungen weniger störungsanfällig gegenüber äußeren Einflüssen wie thermoregulatorisches Schwitzen oder elektrischen Leitungen sind. Aus den Mitteln des \acs{BMBF}-Projekts finanzierten wir ein tragbares \ac{EKG}-Modul mit drei \ac{EKG}-Ableitungen für tragbare \acp{IMU} mit dem Namen Shimmer des Unternehmens Realtime Technologies Ltd. Zusätzlich finanzierten wir ein Sensor-Kit mit drei tragbaren Shimmer \acp{IMU} zur Erfassung der Beschleunigung und der Winkelgeschwindigkeit.

Diese sind notwendig, um die zyklische Bewegung der Beine zu erfassen, um Gangereignisse wie z.~B. \ac{MS} zu identifizieren und die Ableitung der Beschleunigung (Bewegungsaufwand) zu berechnen. Die präzise Erkennung von Gangereignissen und Herzschlägen ist zusätzlich die Grundlage der kardio-lokomotorischen Phasensynchronisation. 

Zur prozessorientierten expliziten Erfassung des Flow-Erlebens wurde von mir ein computergestütztes Experience Sampling Verfahren mit intervall-kontingentem Erfassungsprotokoll und Selbstauskünften im Nachhinein ausgewählt. Wegen ihrer effizienten Auswertung favorisiere ich eine psychometrische Skala. Ich nutze die im deutschsprachigen Raum in vielen Untersuchungen eingesetzte \ac{FKS} von \citet{Rheinberg2003}. Die gewählte Flow-Definition von \citet{Henk2014} ist in Anlehnung an die \ac{FKS} entstanden. Die beiden Kernmerkmale der Definition \emph{gänzliches Aufgehen in der Tätigkeit} und \emph{glatter Handlungsverlauf} entsprechen den zwei Dimensionen der \ac{FKS} \emph{Absorbiertheit} und \emph{glatter Verlauf}. Alle weiteren Merkmale aus Tabelle~\ref{tab:merkmale_eines_flow_zustandes} und Voraussetzungen aus Tabelle~\ref{tab:voraussetzungen_fuer_einen_flow_zustand} fragt die \ac{FKS} mit mindestens einem Item ab. 

Damit kommen die nachfolgenden expliziten und impliziten Messverfahren im weiteren Verlauf der Arbeit zum Einsatz:
\begin{enumerate}
	
	\item die intervall-kontingente Erfassung und/oder Erfassung im Nachhinein durch die \ac{FKS} zur expliziten Bewertung des Flow-Erlebens,

	\item die kontinuierliche Erfassung von kinematischen Daten der Bewegung zur Berechnung des Bewegungsaufwands, 

	\item die kontinuierliche Erfassung von physiologischen Daten des Herzens zur Berechnung der \ac{HR} und der \ac{HRV} und

	\item eine Kombination aus 2. und 3. zur Berechnung der kardio-lokomotorischen Phasensynchronisation.
\end{enumerate}

In Abbildung~\ref{fig:prozessorientiertes_flow_messmodell} stelle ich das reduzierte prozessorientierte \emph{Flow-Messmodell} für den weiteren Verlauf der Arbeit dar. In der Darstellung habe ich die Auswirkungen durch das jeweilige explizite oder implizite Messverfahren ersetzt. 
\begin{sidewaysfigure}
	\includegraphics[width=1.00 
	\textwidth]{prozessorientiertes_flow_messmodell.pdf} \caption[Das prozessorientierte Flow-Messmodell für die Entwicklung eines impliziten Messverfahrens des Flow-Erlebens]{Das prozessorientierte Flow-Messmodell für die Entwicklung eines impliziten Messverfahrens des Flow-Erlebens -- die Auswirkungen wurden durch das jeweilige explizite oder implizite Messverfahren ersetzt. Die explizite Befragung durch die \ac{FKS} findet nicht mehr im Prozess statt.} \label{fig:prozessorientiertes_flow_messmodell} 
\end{sidewaysfigure}

% section auswahl_der_kandidaten_fur_die_explizite_und_implizite_messung_des_flow_erlebens_beim_gehen_und_laufen (end)

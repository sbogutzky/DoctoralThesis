

%!TEX root = /Users/sbogutzky/Entwicklung/projects/bogutzky/repositories/2939413/final-draft.tex
\section{Auswahl der Kandidaten für die explizite und implizite Messung des Flow-Erlebens beim Gehen und Laufen} 

% (fold)
\label{sec:auswahl_der_kandidaten_fur_die_explizite_und_implizite_messung_des_flow_erlebens_beim_gehen_und_laufen}

Schauen wir uns die Probleme aus dem vorherigen Abschnitt an, ist der Einsatz von \ac{EEG}- und \ac{EDA}-Messung wegen ihrer Störungsanfälligkeit fraglich. Die \ac{EMG}-Messung wäre theoretisch möglich, da aber eine trennscharfe Zuordnung zum Flow-Konstrukt fehlt, nicht sinnvoll. Die Aussagekraft von \ac{HRV}-Parametern und die Methodeneinsatz (zeitbezogen oder frequenzbezogen) ist von der Intensität der Beanspruchung der Tätigkeit abhängig.

Aus diesem Grund bleiben, die von uns vorgeschlagenen impliziten Messverfahren des Bewegungsaufwand (Abschnitt~\ref{ssub:der_bewegungsfluss}) und die kardio-lokomotorischen Phasensynchronisation (Abschnitt~\ref{ssub:die_kardio_lokomotorische_phasensynchronisation}). Trotz mangelnder empirische Fundierung bei physischen Tätigkeiten überprüfe ich auch kardiovaskuläre Messungen mittels \ac{EKG}. 

Ein \ac{EKG} hat gegenüber den Geräten, die für eine \ac{EEG}, eine \ac{EDA} oder eine \ac{EMG} notwendig sind, den Vorteil, dass es effizient im Freien einzusetzen ist und dass dessen Messungen weniger störungsanfällig gegenüber äußeren Einflüssen wie thermoregulatorisches Schwitzen oder elektrischen Leitungen sind. Aus den Mitteln des \acs{BMBF}-Projekts finanzierten wir ein tragbares EKG-Modul mit drei \ac{EKG}-Ableitungen für tragbare \acp{IMU} mit dem Namen Shimmer des Unternehmens Realtime Technologies Ltd.. Zusätzlich finanzierten wir ein Sensor-Kit mit drei tragbaren Shimmer \acp{IMU} zur Erfassung der Beschleunigung und der Winkelgeschwindigkeit.

Diese sind nötig um die zyklischen Bewegung der Beine zu erfassen, um Gangergeignisse wie z.~B. \ac{MS} zu identifizieren und die Ableitung der Beschleunigung (Bewegungsaufwand) zuberechnen. Die prezise Erkennung von Gangereignissen und Herzschlägen ist zusätzlich die Grundlage der kardio-lokomotorische Phasensynchronisation. 

Zur prozessorientierten expliziten Erfassung des Flow-Erlebens wurde von mir ein computergestütztes Experience Sampling Verfahren mit intervall-kontingentem Erfassungsprotokoll und Selbstauskünften im Nachhinein ausgewählt. Wegen ihrer effizienten Auswertung favorisiere ich eine psychometrische Skala. Ich nutze die im deutschsprachigen Raum in vielen Untersuchungen eingesetzte \ac{FKS} von \citet{Rheinberg2003}. Ihre zwei Dimension ... 

% section auswahl_der_kandidaten_fur_die_explizite_und_implizite_messung_des_flow_erlebens_beim_gehen_und_laufen (end)

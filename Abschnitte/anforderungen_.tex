\section{Das Smartphone zur Messung von psychischen, physiologischen und motorischen Daten}
\label{sec:das_smartphone_zur_messung}
Als technisches Bindeglied zwischen den vier Messmethoden dient ein Smartphone. Das Smartphone ist die am schnellsten angenommene Technologie in der Geschichte der Menschheit. Menschen in Industrieländern benutzen das Smartphone alltäglich. Sie nutzen es z.~B. zum Informationsabruf, zur Unterhaltung, zur Terminplanung, für ihre Gesundheit und um ihre sozialen Kontakte zu pflegen. In der Bundesrepublik Deutschland besitzen 6 von 10 Bundesbürger ab 14 Jahren (63 Prozent) ein Smartphone; das sind 44 Millionen Menschen \citep[vgl.][]{bitkom2015}.

Smartphones sind in der Lage programmierte Anwendungen (Apps) auszuführen und besitzen eine sehr hoch entwickelte interne Sensorik, eine hohe Speicherkapazität und eine integrierte Netzwerkanbindung. Das ermöglicht, Apps zu programmieren, die die interne Sensorik auslesen und Daten auf den Smartphones oder über das Internet speichern. Die technischen Möglichkeiten und die allgegenwärtige Nutzung von Smartphones machen das Smartphone zu einem idealen wissenschaftlichen Werkzeug \citep[vgl.][]{Raento2009}.

In der Mensch-Computer-Interaktion ist das Smartphone durch die notwendige Auseinandersetzung mit neuen Technologien als wissenschaftliches Werkzeug bereits etabliert \citep{Froehlich2007}. \emph{Apple}'s Veröffentlichung von \emph{ResearchKit} 2015 zeigt, dass die Smartphone-basierende Forschung an Popularität gewinnt. \emph{ResearchKit} ist ein \emph{Open Source Framework} und dient als Baukasten für Forschungsanwendungen. Leider fehlt in vielen Forschungsdisziplinen außerhalb der Mensch-Computer-Interaktion häufig die Integration der technischen Möglichkeiten mangels notwendiger Programmierkenntnisse und Ressourcen.

Die Arbeitsgruppe um Gaggioli entwickelte mit Hilfe von Informatikern für mehrere interdisziplinäre Studien eine Plattform zur Erhebung von psycho-physiologischen Daten zur mentalen Gesundheitsforschung \citep{Gaggioli2013}. Ihre Plattform ermöglicht es, psychologische Daten mittels Smartphone-gestützter \emph{Experience Sampling} Methode und kontinuierlich physiologische Daten mittels interner und externer Sensorik zu sammeln. Sie veröffentlichten ihre Plattform \emph{Open Source}, stellten aber die Entwicklung ein oder nutzen fortan \emph{Closed Source}. Damit steht ihre Plattform nicht für die meistverbreiteten Betriebssysteme wie \emph{Android OS} oder \emph{iOS} zur Verfügung.

\section{Arbeitsschritte der technischen Arbeit}
\label{sec:arbeitsschritte_der_technischen_arbeit}
Der erste technische Schritt der vorliegenden Arbeit ist die Entwicklung einer Smartphone App zur Erfassung von physiologischen und kinematischen Daten. Meine Mitarbeiter und ich entwickelten eine App mit dem Namen \emph{\ac{PPC}} für das \emph{Android OS}. Der \ac{PPC} ermöglicht es, kontinuierlich \ac{EKG}-Daten mittels des tragbaren Sensors, kontinuierlich Bewegungsdaten mittels der internen und mehreren externen tragbaren Sensoren, kontinuierlich Positionsdaten mittels \ac{GPS} und intervall-kontingente psychometrische Skalen von Untersuchungspersonen zu erheben.

Zur Identifikation eines impliziten Messverfahrens des Flow-Erlebens ist der zweite Schritt die offline Verarbeitung und Analyse. Hierzu entwickelte ich eine Verarbeitungs- und Analysepipeline mit dem Namen \emph{\ac{PPP}}. Die \ac{PPP} besteht aus Programmen, die ich der Entwicklungsumgebung \emph{R} realisierte, und \emph{Open Source Software}. Eine detaillierte Darstellung des \ac{PPC}s und der \ac{PPP} folgt in den Abschnitten des Kapitels~\ref{cha:studien_zur_mobilen_und_prozessorientierten_messung}.

Die abschließenden Schritte zu einer App, die die Voraussetzungen während des Laufens verbessern, Flow zu erleben, sind die Übertragung der offline Berechnungen der \ac{PPP} in den \ac{PPC} und die Entwicklung eines geeigneten Konzepts zur Rückmeldung. Zu berücksichtigen sind Latenzzeiten bei der prozessorientierten Verarbeitung für Echtzeit-Rückmeldungen z.~B. in einer Lauf-App wie in Kapitel~\ref{cha:technologie_beim_laufen} beschrieben. Ich berichte von einer beispielhaften Realisierung der beiden abschließenden Schritte in der vorliegenden Arbeit im Abschnitt~\ref{sec:demonstrator}.

\section{Zusammenfassung}
\label{sec:zusammenfassung_4}
Im gegenwärtigen Kapitel beleuchte ich die Tätigkeitsauswahl in der vorliegenden Arbeit und in den Studien, die die Zusammenhänge von Flow und Laufen untersuchten. Im Anschluss zeige ich die Probleme der \ac{EEG}-, der \ac{EDA}- und der \ac{HRV}-Analyse unter physischer Belastung auf. In Folge dessen begründe ich die Auswahl der in Abbildung~\ref{fig:4_1_datenraum} dargestellten Kandidaten für ein implizites Messverfahren des Flow-Erlebens. Ich veranschauliche, wie ich das Smartphone in den nachfolgenden Studien als wissenschaftliches Werkzeug einsetze. Abschließend gehe ich auf die technischen und konzeptionellen Arbeiten ein, die für die nachfolgenden Studien und für eine App notwendig sind, die die Voraussetzungen während des Gehens und Laufens verbessert, Flow zu erleben.
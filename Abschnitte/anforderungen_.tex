%!TEX root = /Users/sbogutzky/Entwicklung/projects/bogutzky/repositories/2939413/final-draft.tex
\chapter{Flow-Erleben beim Gehen und Laufen messen: Anforderungen}
\label{cha:flow_erleben_beim_gehen_und_laufen_messen_anforderungen}
Im gegenwärtigen Kapitel gebe ich einen Überblick über die Anforderungen an ein implizites Messverfahren, das eine mobile und computergestützte Echtzeitverarbeitung beim Gehen und Laufen zulässt. Zu Beginn diskutiere ich die Tätigkeitsauswahl, Abschnitt~\ref{sec:taetigkeitsauswahl}. In Abschnitt~\ref{sec:probleme_bei_messungen} setze ich mich mit den Problemen der Messung von physiologischen Eigenschaften unter physischer Beanspruchung auseinander. Anschließend begründe ich in Abschnitt~\ref{sec:auswahl_der_kandidaten} meine Auswahl der Kandidaten für eine implizite Messung des Flow-Erlebens beim Gehen und Laufen. Im Anschluss beleuchte in Abschnitt~\ref{sec:das_smartphone_zur_messung} das Smartphone als wissenschaftliches Werkzeug. Vom vorherigen Abschnitt ausgehend stelle ich die Arbeitschritte der technischen Arbeit in Abschnitt~\ref{sec:arbeitsschritte_der_technischen_arbeit} vor. In Abschnitt~\ref{sec:zusammenfassung_4} fasse ich das Kapitel zusammen.

\section{Tätigkeitsauswahl}
\label{sec:taetigkeitsauswahl}
Infolge meiner Tätigkeit im Projekt "`Flow-Maschinen: Körperbewegung und Klang"' untersuche ich das Gehen. Aufgrund meiner Annahme, dass alltägliches Gehen wenig bis gar keine Voraussetzungen des Flow-Erlebens erfüllt und aufgrund des in Kapitel~\ref{cha:technologie_beim_laufen} beschriebenen Anwendungsszenarios in der mobilen Mensch-Computer-Interaktion untersuche ich zusätzlich das Laufen. Ich gehe davon aus, dass der höhere Komplexitätsgrad des Laufens durch die schnellere Bewegung die Anforderungen an die Untersuchungspersonen vergrößert und in Folge dessen die Wahrscheinlichkeit erhöht, Flow-Erleben zu messen.

In der Literatur gehört Laufen zu den bevorzugten sportlichen Tätigkeiten, um Annahmen zum Flow-Erleben und verwandten Zuständen zu prüfen. \citet{Dietrich2004a} führten u. a. Laufexperimente durch, um sich ihrer These der trassierten Hypofrontalität empirisch zu nähern. \citet{Stoll2005} überprüften in zwei Studien mit insgesamt 234 Teilnehmern, ob eine \ac{AFP} für das Erleben für Flow beim Marathonlauf erforderlich ist und ob Flow-Erleben mit der Laufleistung beim Marathonlauf in einem Zusammenhang steht. Sie nutzen die \ac{FKS} zur Befragung von Teilnehmern nach zwei Marathonläufen. Ihre erste Studie ergab keinen Zusammenhang zwischen \ac{AFP} und dem Generalfaktor der \ac{FKS}. Ihre zweite Studie widerlegte das Ergebnis. Das lässt \citet{Stoll2005} darauf schließen, dass eine \ac{AFP} nicht zwingend erforderlich ist, um Flow beim Marathonlaufen zu erleben. In beiden Studien fanden sie keinen Zusammenhang von Flow-Erleben und der Laufleistung. \citet{Schuler2009} untersuchten in drei Studien mit 109, 112 und 65 Teilnehmern, ob Flow-Erleben beim Marathonlaufen Einfluss auf die Laufleistung und auf die zukünftige Laufmotivation besitzt. Ihre Ergebnisse zeigen, dass Flow-Erleben nicht direkt mit der Laufleistung in einem Zusammenhang steht. Allerdings stieg die zukünftige Motivation zu Laufen bei denen, die Flow beim Laufen erlebten. Durch die Motivation zum Lauftraining verbessert sich indirekt die Laufleistung. In den ersten beiden Studien versetzten sich die Teilnehmer nach dem Marathonlauf in die zurückliegende Situation an den Kilometermarken 10, 20, 30 und 40 und füllten auf dessen Grundlage jeweils eine \ac{FKS} aus. In der dritten Studie befragten \citet{Schuler2009} die Läufer beim Training während des Laufens an den Kilometermarken 10, 20, 30 und 40 mit der Hilfe eines mitlaufenden Assistenten verbal. Als Fragebogen diente die \ac{FKS}. Alle drei Studien zeigten, dass das Flow-Erleben über die gelaufene Distanz hinweg abnimmt. \citet{Jimenez-Torres2013} überprüften die neun Dimensionen des Flow-Erlebens nach \citet{Csikszentmihalyi1992} vor, nach dem Marathonlauf und während des Laufs bei 153 Teilnehmern. Sie nutzten ein Item jeder Dimension der FSS-2 zur Befragung \citet{Jackson2002}. Sie stellten fest, dass der größte Anteil der neun Flow-Dimensionen während des Marathonlaufs auftrat. Nur eine geringe Anzahl der Flow-Dimensionen traten vor oder nach dem Lauf auf. In der Studie von \citet{Reinhardt2006} liefen 30 Teilnehmer für 40 Minuten auf einem Laufband in einer individuellen Laufgeschwindigkeit. Als Grundlage für die Laufbandgeschwindigkeit diente die \ac{HR}. Die Teilnehmer unterbrachen alle zehn Minuten ihren Lauf, um eine \ac{FKS} auszufüllen. Die Teilnehmer berichteten von einem stabilen und tiefen Flow-Erleben. Allerdings zeigen die Ergebnisse keinen Zusammenhang von \ac{AFP} und Flow-Erleben.

Zu der vorhandenen Forschung zu Flow-Erleben beim Laufen, sprechen zusätzliche Faktoren für die Tätigkeiten des Gehens und Laufens. Beide Tätigkeiten lassen sich von Untersuchungspersonen in einem natürlichen Handlungsumfeld und unter geringen Einfluss von Störfaktoren durchführen. Sie ermöglichen beide eine Durchführung isoliert von anderen Menschen. Untersuchungspersonen müssen den Bewegungsablauf nicht explizit trainieren. Die Dauer der beiden Tätigkeiten ist variabel. Ich verspreche mir von einer langandauernden Durchführung genügend Anlaufzeit für das Erleben von Flow. Eine Anlaufzeit von einigen Minuten ist nach \citet[S.~109]{Henk2014} nötig, damit sich Flow-Erleben einstellt. Die langandauernde Durchführung setzt zumindest für das Laufen durch die physisch höhere Belastung eine gewisse Grundausdauer voraus.

\section{Probleme bei Messungen von physiologischen Eigenschaften unter physischer Belastung}
\label{sec:probleme_bei_messungen}
Die Eignung eines Kandidaten zur impliziten Flow-Messung im Verlauf des Gehens oder des Laufens hängt von unterschiedlichen Faktoren ab. Im Nachfolgenden diskutiere ich die Faktoren Störungsanfälligkeit, Methodeneinsatz und Zuordnung. Mit Zuordnung ist die Zuordnung der Ergebnisse einer Messmethode zum Flow-Phänomen und zu einer konkreten Reaktion des menschlichen Körpers gemeint. Die drei genannten Faktoren zielen darauf ab, eine implizite Messung zu ermöglichen, die eine Echtzeitverarbeitung und -rückmeldung beim Gehen und Laufen gewährleistet.

\subsection{Störungsanfälligkeit}
Die Störungsanfälligkeit durch z.~B. elektromagnetische Felder in einem natürlichen Umfeld behindert die derzeitige Messung der Gehirnaktivität, bspw. über eine \ac{EEG} \citep[vgl.][S.~56]{Henk2014}. Dementsprechend beschränkt sich die Studie von \citet{Hugentobler2011} zum Flow-Erleben mit einem Mehrkanalsystem, auf eine künstliche Umgebung innerhalb eines Computerspiels unter Laborbedingungen in einem faradayschen Käfig. Bis qualitativ hochwertige \acp{EEG}, die Messung der Gehirnaktivität im Freien ermöglichen, sind wir nur in der Lage theoretisch begründete Annahmen anzustellen und (über Umwege) zu überprüfen \citep[vgl.][S.~56]{Henk2014}.

Kein externer, sondern ein menschlicher Faktor stört die Analyse der \ac{EDA} zur Identifikation von emotionalen Zuständen unter physischer Belastung. Das thermoregulatorische Schwitzen unter Dauerbelastung bereitet Probleme, da das thermoregulatorische Schwitzen nicht vom "`emotionalen Schwitzen"' zu unterscheiden ist \citep[vgl.][]{Baumeister2008}. Demzufolge ist von der Analyse der \ac{EDA} zur Identifikation von emotionalen Zuständen bei Tätigkeiten mit starker physischer Belastung abzusehen. Bei einem Einsatz ohne bzw. mit geringer physischer Beanspruchung, wie bei \citet{Kivikangas2006} und \citet{Nacke2008}, dürfen wir von unverfälschten Ergebnissen ausgehen.

Bei Aufnahmen von Signalen, wie z.~B. beim \ac{EKG} unter sehr hohen Belastungsintensitäten gilt das Signal-Rausch-Verhältnis zu beachten. Das Signal-Rausch-Verhältnis ist ein Maß für die technische Qualität eines Signals. Ist eine zu geringe Abtastrate z.~B. beim \ac{EKG} gewählt, führt das zu einem niedrigen Signal-Rausch-Verhältnis. Es resultieren Messfehler, die z.~B. die \ac{EKG}-Messung für eine Analyse der \ac{HRV} unbrauchbar macht \citep[vgl.][]{Hoos2010}.

\subsection{Methodeneinsatz}
Ein Grundproblem der frequenzbezogenen \ac{HRV}-Analyse unter physischer Belastung ist die Nichtstationarität der Zeitreihe der aufeinanderfolgenden RR-Intervalle \citep[vgl.][]{Hottenrott2006}. Für eine frequenzbezogene \ac{HRV}-Analyse sind zwei mathematische Voraussetzungen zu erfüllen: gleiche zeitliche Messabstände (Äquidistanz) und ein (quasi-)stationärer Signalcharakter der RR-Intervalle \citep[vgl.][]{Hoos2006}. (Quasi-)Stationarität bedeutet im dargestellten Zusammenhang, dass zumindest innerhalb des zu analysierenden Zeitintervalls die wesentlichen Verteilungsgrößen des Signals (Mittelwert und Varianz) zeitunabhängig sind \citep[vgl.][]{Hoos2006}.

Aus dem Grund der in der Regel nicht gegebenen Stationarität der Zeitreihe der RR-Intervalle unter physischer Belastung empfehlen \citet[S.~113]{Sarmiento2013} traditionelle Methoden der frequenzbezogenen Analyse wie \acs{FFT} und autoregressives (AR-)Modell nicht zu verwenden. Alternative Methoden der frequenzbezogenen Analyse, die das Problem der Nichtstationarität teilweise lösen, sind z.~B. die \emph{Coarse Graining Spektralanalyse (CGSA)}, die \emph{Kurzzeitfourier-Analyse (STFT)} oder die \emph{kontinuierliche Wavelet Transformation (CWT)} \citep[vgl.][S.~61f.]{Hoos2010}. Nicht lineare Verfahren zur Analyse der \ac{HRV} unter physischer Belastung (insbesondere \emph{\ac{DFA}}, \emph{Sample Entropie (SampEn)} sind nach \citet[vgl.][S.~61f.]{Hoos2010} auch potenzielle Lösungen.

\subsection{Zuordnung}
\label{sub:zuordnung}
Die Zuordnung der Frequenzbereiche \ac{LF} und \ac{HF} unter physischer Belastung ist problematisch, da sich ein mechanisch bedingtes Resonanz- und Kopplungsphänomen durch die Atmung und der motorischen Aktivität im erweiterten \ac{HF}-Frequenzbereich manifestiert \citep[vgl.][S.~62]{Hoos2010}. Demzufolge ist eine Zugehörigkeitsbestimmung der \ac{HF} zur parasympathischen Aktivität des \ac{VNS}, wie bei ruhenden bzw. sitzenden Tätigkeiten, fehlerhaft. Die drastische Reduktion der \ac{HRV} erschwert zusätzlich die Zuordnung von zeit-, frequenzbezogenen und nichtlinearen \ac{HRV}-Parametern.

Einen positiven Affekt im Gesicht während des Laufens zu messen ist auf der einen Seite problematisch, da sich die physische Belastung meiner Meinung nach gleichermaßen im Gesicht ausdrückt. Auf der anderen Seite müssen wir klären, inwiefern die \ac{EMG}-Messungen eine trennscharfe Bestimmung für Flow gewährleistet \citep[vgl.][]{Peifer2012}.

\section{Auswahl der Kandidaten für eine implizite Messung des Flow-Erlebens beim Gehen und Laufen}
\label{sec:auswahl_der_kandidaten}
Wenn wir uns die Gesamtheit der Probleme unserer Kandidaten zur impliziten Messung von Flow-Erleben beim Gehen und Laufen anschauen, sehe ich die Kandidaten, die auf der Biomechanik beruhen und die zeitbezogenen \ac{HRV}-Parameter im Vorteil. In Abbildung~\ref{fig:4_1_datenraum} stelle ich die Kandidaten dar, die ich für ein Messgerät für das Gehen und Laufen in Betracht ziehe.

\begin{figure}[t]
	\centering
		\includegraphics[width=1.00\textwidth]{4-1-datenraum-2}
	\caption[Der Datenraum für die Entwicklung eines impliziten Messverfahrens des Flow-Erlebens]{Der Datenraum für die Entwicklung eines impliziten Messverfahrens des Flow-Erlebens. Quelle: Eigene Darstellung}
	\label{fig:4_1_datenraum}
\end{figure}

Die Entscheidung fiel, trotz mangelnder empirische Fundierung bei physischen Tätigkeiten, auf kardiovaskuläre Messung mittels \ac{EKG} und der \ac{HRV}-Analyse als physiologische Messung. Gegenüber den Geräten, die für eine \ac{EEG}, eine \ac{EDA} oder eine \ac{EMG} notwendig sind, ist ein tragbares \ac{EKG} effizient im Freien einzusetzen und dessen Messungen sind weniger störungsanfällig gegenüber äußeren Einflüssen wie thermoregulatorisches Schwitzen oder elektrischen Leitungen. Die Miniaturisierung von Sensoren und Mikroprozessoren ermöglicht es, leicht am Körper zu tragende Geräte wie \acp{EKG} oder Bewegungssensoren zu entwickeln. Aus den Mitteln des \acs{BMBF}-Projekts finanzierten wir ein tragbares \emph{EKG-Modul} mit drei \ac{EKG}-Ableitungen für tragbare \acp{IMU} mit dem Namen \emph{Shimmer} des Unternehmens \emph{Realtime Technologies Ltd.}.

Auf der motorischen Ebene nutze ich den Bewegungsaufwand. Der Bewegungsaufwand lässt sich an verschiedenen Körperpositionen messen. Für das Gehen und Laufen erfasse ich ihn anhand der zyklischen Bewegung der Beine. Der Bewegungsaufwand ist die Ableitung der Beschleunigung. Zur Erfassung der Beschleunigung und der Winkelgeschwindigkeit finanzierten wir ein \emph{Sensor-Kit} mit drei tragbaren \emph{Shimmer} \acp{IMU} von den Mitteln des \acs{BMBF}-Projekts.

Zusätzlich kommen \ac{EKG}- und Bewegungsdaten bei der Bestimmung der kardio-lokomotorischen Phasensynchronisation zum Einsatz. Durch die Identifikation von Gangereignissen und Herzschlägen ist die kardio-lokomotorische Phasensynchronisation neben der \ac{HRV} und dem Bewegungsfluss der dritte Kandidat, den ich in der vorliegenden Arbeit betrachte.

Zur expliziten Erfassung des Flow-Erlebens wähle ich ein computergestütztes \emph{Experience Sampling} Verfahren mit intervall-kontingentem Erfassungsprotokoll und Selbstauskünften im Nachhinein aus. Wegen ihrer effizienten Auswertung favorisiere ich eine psychometrische Skala. Ich nutze die im deutschsprachigen Raum in vielen Fällen eingesetzte \ac{FKS} von \citet{Rheinberg2003}.

\section{Das Smartphone zur Messung von psychischen, physiologischen und motorischen Daten}
\label{sec:das_smartphone_zur_messung}
Als technisches Bindeglied zwischen den vier Messmethoden dient ein Smartphone. Das Smartphone ist die am schnellsten angenommene Technologie in der Geschichte der Menschheit. Menschen in Industrieländern benutzen das Smartphone alltäglich. Sie nutzen es z.~B. zum Informationsabruf, zur Unterhaltung, zur Terminplanung, für ihre Gesundheit und um ihre sozialen Kontakte zu pflegen. In der Bundesrepublik Deutschland besitzen 6 von 10 Bundesbürger ab 14 Jahren (63 Prozent) ein Smartphone; das sind 44 Millionen Menschen \citep[vgl.][]{bitkom2015}.

Smartphones sind in der Lage programmierte Anwendungen (Apps) auszuführen und besitzen eine sehr hoch entwickelte interne Sensorik, eine hohe Speicherkapazität und eine integrierte Netzwerkanbindung. Das ermöglicht, Apps zu programmieren, die die interne Sensorik auslesen und Daten auf den Smartphones oder über das Internet speichern. Die technischen Möglichkeiten und die allgegenwärtige Nutzung von Smartphones machen das Smartphone zu einem idealen wissenschaftlichen Werkzeug \citep[vgl.][]{Raento2009}.

In der Mensch-Computer-Interaktion ist das Smartphone durch die notwendige Auseinandersetzung mit neuen Technologien als wissenschaftliches Werkzeug bereits etabliert \citep{Froehlich2007}. \emph{Apple}'s Veröffentlichung von \emph{ResearchKit} 2015 zeigt, dass die Smartphone-basierende Forschung an Popularität gewinnt. \emph{ResearchKit} ist ein \emph{Open Source Framework} und dient als Baukasten für Forschungsanwendungen. Leider fehlt in vielen Forschungsdisziplinen außerhalb der Mensch-Computer-Interaktion häufig die Integration der technischen Möglichkeiten mangels notwendiger Programmierkenntnisse und Ressourcen.

Die Arbeitsgruppe um Gaggioli entwickelte mit Hilfe von Informatikern für mehrere interdisziplinäre Studien eine Plattform zur Erhebung von psycho-physiologischen Daten zur mentalen Gesundheitsforschung \citep{Gaggioli2013}. Ihre Plattform ermöglicht es, psychologische Daten mittels Smartphone-gestützter \emph{Experience Sampling} Methode und kontinuierlich physiologische Daten mittels interner und externer Sensorik zu sammeln. Sie veröffentlichten ihre Plattform \emph{Open Source}, stellten aber die Entwicklung ein oder nutzen fortan \emph{Closed Source}. Damit steht ihre Plattform nicht für die meistverbreiteten Betriebssysteme wie \emph{Android OS} oder \emph{iOS} zur Verfügung.

\section{Arbeitsschritte der technischen Arbeit}
\label{sec:arbeitsschritte_der_technischen_arbeit}
Der erste technische Schritt der vorliegenden Arbeit ist die Entwicklung einer Smartphone App zur Erfassung von physiologischen und kinematischen Daten. Meine Mitarbeiter und ich entwickelten eine App mit dem Namen \emph{\ac{PPC}} für das \emph{Android OS}. Der \ac{PPC} ermöglicht es, kontinuierlich \ac{EKG}-Daten mittels des tragbaren Sensors, kontinuierlich Bewegungsdaten mittels der internen und mehreren externen tragbaren Sensoren, kontinuierlich Positionsdaten mittels \ac{GPS} und intervall-kontingente psychometrische Skalen von Untersuchungspersonen zu erheben.

Zur Identifikation eines impliziten Messverfahrens des Flow-Erlebens ist der zweite Schritt die offline Verarbeitung und Analyse. Hierzu entwickelte ich eine Verarbeitungs- und Analysepipeline mit dem Namen \emph{\ac{PPP}}. Die \ac{PPP} besteht aus Programmen, die ich der Entwicklungsumgebung \emph{R} realisierte, und \emph{Open Source Software}. Eine detaillierte Darstellung des \ac{PPC}s und der \ac{PPP} folgt in den Abschnitten des Kapitels~\ref{cha:studien_zur_mobilen_und_prozessorientierten_messung}.

Die abschließenden Schritte zu einer App, die die Voraussetzungen während des Laufens verbessern, Flow zu erleben, sind die Übertragung der offline Berechnungen der \ac{PPP} in den \ac{PPC} und die Entwicklung eines geeigneten Konzepts zur Rückmeldung. Zu berücksichtigen sind Latenzzeiten bei der prozessorientierten Verarbeitung für Echtzeit-Rückmeldungen z.~B. in einer Lauf-App wie in Kapitel~\ref{cha:technologie_beim_laufen} beschrieben. Ich berichte von einer beispielhaften Realisierung der beiden abschließenden Schritte in der vorliegenden Arbeit im Abschnitt~\ref{sec:demonstrator}.

\section{Zusammenfassung}
\label{sec:zusammenfassung_4}
Im gegenwärtigen Kapitel beleuchte ich die Tätigkeitsauswahl in der vorliegenden Arbeit und in den Studien, die die Zusammenhänge von Flow und Laufen untersuchten. Im Anschluss zeige ich die Probleme der \ac{EEG}-, der \ac{EDA}- und der \ac{HRV}-Analyse unter physischer Belastung auf. In Folge dessen begründe ich die Auswahl der in Abbildung~\ref{fig:4_1_datenraum} dargestellten Kandidaten für ein implizites Messverfahren des Flow-Erlebens. Ich veranschauliche, wie ich das Smartphone in den nachfolgenden Studien als wissenschaftliches Werkzeug einsetze. Abschließend gehe ich auf die technischen und konzeptionellen Arbeiten ein, die für die nachfolgenden Studien und für eine App notwendig sind, die die Voraussetzungen während des Gehens und Laufens verbessert, Flow zu erleben.
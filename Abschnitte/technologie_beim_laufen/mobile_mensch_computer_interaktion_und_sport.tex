 

%!TEX root = /Users/sbogutzky/Entwicklung/projects/bogutzky/repositories/2939413/final-draft.tex
\section{Mobile Mensch-Computer-Interaktion und Sport} 

% (fold)
\label{sec:mobile_mensch_computer_interaktion_und_sport}

Die Miniaturisierung von elektronischen Komponenten ermöglichte die Entwicklung von mobilen Geräten. Mobile Geräte bezeichnen solche Geräte, die ihre Benutzer in die Lage versetzen, Computer in mobile Situationen zu nutzen. Mobile Situationen kennzeichnen Situationen, in denen wir, die Benutzer, aber auch von uns überwachte Gegenstände, mobil sind. \citet[][S.~24ff.]{Cooper2002} bezeichnet Benutzermobilität als physische Bewegung des Benutzers im Raum und in der Zeit. Nach \citet[][S.~7]{Roth2005} muss sich ein Benutzer nicht zwangsläufig bewegen, um mobil zu sein. 
\begin{itemize}
	
	\item Für die vorliegende Arbeit sind mobile Situationen von Bedeutung, in denen der Benutzer geht oder läuft und nur kurzfristig an einem Ort verweilt, bevor er die physische Bewegung wieder aufnimmt.
\end{itemize}

\citet[][S.~5]{Roth2005} zählt tragbare Standardcomputer wie Notebooks oder Bordcomputer zu den mobilen Geräten. 
\begin{itemize}
	
	\item In der vorliegenden Arbeit lege ich das Augenmerk auf mobile Geräte, die der Benutzer in der Regel in einer Hand hält (Handhelds) oder am Körper trägt (Wearables).
\end{itemize}

Darüber hinaus sind Handhelds wie z.~B. Smartphones in der Lage mobile Applikationen ablaufen zulassen. 
\begin{itemize}
	
	\item Als mobile Applikationen (Apps) bezeichne ich in der vorliegenden Arbeit Applikationen, die auf mobilen Geräten laufen. Obwohl sich App als Abkürzung für den englischen Begriff \emph{application software} auf jegliche Anwendungssoftware beziehen lässt, setze ich ihn in der vorliegenden Arbeit mit der Anwendungssoftware eines mobilen Gerätes gleich.
\end{itemize}

Typische Vertreter mobiler Geräte im Sport sind tragbare Datenlogger (Activity Tracker), tragbare Trägheitsmesseinheiten (\ac{IMU}), Smartphones und Computeruhren (Smartwatches). Sie besitzen alle eine Kombination von mehreren Sensoren, wie z.~B. Beschleunigungsmesser, Drehratensensor und \ac{GPS}-Empfänger, zur Messung von kinematischen Daten. 
\begin{itemize}
	
	\item In der vorliegenden Arbeit bezeichne ich mit kinematischen Merkmalen, Merkmale die auf Berechnungen beruhen, die die Gesetze der Kinematik als Grundlage heranziehen.
\end{itemize}

„Die Kinematik beschreibt die Bewegung von Punkten und Körpern im Raum. Zur vollständigen kinematischen Beschreibung einer Bewegung genügen die Größen Lage, Geschwindigkeit und Beschleunigung. Die Ursachen der Bewegung (z.~B. Kräfte) bleiben bei der kinematischen Beschreibung unberücksichtigt“ \citep[][S.~57]{Disselhorst-Klug2015}.

Die Gesetzmäßigkeiten der Kinematik eröffnen der mobilen Mensch-Computer-Interaktion im Sport nachfolgende Möglichkeiten: 
\begin{itemize}
	\item sie ermöglicht durch die vom \ac{GPS} gesendete Position und Zeit, die Berechnung der durchschnittlichen Geschwindigkeit eines Benutzers eines mobilen Geräts und der Entfernung, die er zurücklegte. 
	\item sie ermöglicht durch ihre Analyse (z.~B. Zeitverläufe der Gelenkbewegung) das Resultat einer Bewegung des menschlichen Körpers unter Beachtung der Biomechanik in unterschiedlichen Tätigkeiten zu quantifizieren. 
\end{itemize}

Die Biomechanik beschreibt die tätigkeitsspezifische Bewegung unter Verwendung von Methoden und Gesetzmäßigkeiten der Mechanik und Anatomie des menschlichen Körpers \citep[][S.~2ff.]{Winter2009}.

Mit Bezug auf die Modalitäten bei der Benutzung von mobilen Geräten lassen sich visuelle, akustische und taktile Rückmeldungen unterscheiden. Bei visueller Rückmeldung bekommt der Sportler Informationen bildlich oder als Text auf dem Display des mobilen Gerätes ausgegeben. Bei akustischer Rückmeldung hört er Klänge oder gesprochenen Text (auch verbale Rückmeldung genannt). Bei taktiler Rückmeldung überträgt das mobile Gerät die Information über die Berührung, z.~B. durch dessen Vibration.

Rückmeldungen im Sport erfolgen nach der physischen Belastung oder während der physischen Beanspruchung. Letzteres bezeichne ich in der vorliegenden Arbeit als Echtzeit (real time). Nach ISO/IEC Norm 2382 ist Echtzeitverarbeitung eine Verarbeitungsart, bei der die Programme für die Datenverarbeitung ständig betriebsbereit sind, sodass die Ausgangsdaten innerhalb einer bestimmten Zeitspanne zur Verfügung stehen.

Demzufolge verspricht Echtzeit eine verzögerungsarme Rückmeldung auf ein Ereignis durch das mobile Gerät. Die Verzögerung hängt von der Verarbeitungsdauer der Daten ab, die eine festgelegte Information ermöglicht. Auch wenn die Verarbeitungsdauer eine sofortige Rückmeldung zulässt, ist abhängig vom Anwendungsfall ein vorbestimmter Rückmeldungszeitpunkt vorstellbar. Eine Herausforderung besteht darin, Rückmeldungen in Echtzeit an die physische Beanspruchung des Sportlers und die mobile Situation anzupassen, damit sie den Fluss und die Erfahrung des Sportlers nicht unterbrechen \citep{Nylander2014}.
\begin{itemize}
	
	\item In Anbetracht der Tatsache, dass ein technisches System (mobiles Gerät), Daten eines biologischen Systems (Sportler) wie z.~B. Herz- und Bewegungsdaten verzögerungsarm verarbeitet und als Information an das gleiche biologische System zurückgibt, verwende ich in dieser Arbeit für die vorliegende Art von Rückmeldung den Begriff \emph{Realtime Biofeedback}.
\end{itemize}

„Biofeedback (altgr. \emph{bios} Leben und engl. \emph{feedback} Rückmeldung) ist eine Möglichkeit, Veränderungen von körperlichen Zuständen, die der unmittelbaren Sinneswahrnehmung nicht zugänglich sind, mit technischen Hilfsmitteln bewusst zu machen“ \citep[][S.~483]{Riemer2015}.

Biofeedback dient in den existierenden Technologien und Konzepten aus der Forschung zur Mensch-Computer-Interaktion im Laufsport überwiegend zur Leistungsdiagnostik und zur Trainingssteuerung. Die Trainingssteuerung dient dem Sportler, z.~B. zur Planung von Regenerationsphasen oder zur Anpassung der physischen Beanspruchung während des Trainings \citep[][S.~81-107]{Marquardt2011}. Jüngste Fortschritte auf dem Gebiet der mobilen Technologien und der Echtzeitverarbeitung ermöglichten erst Konzepte, die mit Hilfe von \emph{Realtime Biofeedback} das Körperbewusstsein und die Körperkontrolle während des Laufens verbessern \citep{Strohrmann2013, Strohrmann2013a, Strohrmann2014}. Diese Konzepte basieren auf implizit gemessenen Merkmalen, die wiederum ein Verständnis über die biomechanischen Eigenschaften des Gehens und des Laufens voraussetzen. 

% section mobile_mensch_computer_interaktion_und_sport (end)

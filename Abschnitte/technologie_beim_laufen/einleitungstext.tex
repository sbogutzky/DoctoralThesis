

%!TEX root = /Users/sbogutzky/Entwicklung/projects/bogutzky/repositories/2939413/final-draft.tex
Heutzutage sind Computer allgegenwärtig. Mark Weiser prägte schon 1988 den Ausdruck der allgewärtigen Computernutzung (ubiquitous computing) und beschrieb eine Zeit nach der Mensch-Computer-Interaktion am Schreibtisch, in der Computer in Alltagsgegenständen integriert sind. Gegenstände wie z.~B. Laufuhren, mit denen wir Sport treiben, machen da keine Ausnahme. Demzufolge erlangte auch die Mensch-Computer-Interaktion in den vergangenen Jahren in Bezug auf den Sport an Bedeutung \citep[][]{Nylander2014}. Die Motivation der Computernutzung im Sport besteht darin, positive Impulse für das menschliche Wohlbefinden durch Steigerung der Leistungsfähigkeit, Instandhaltung der Gesundheit, Verbesserung der Rehabilitation oder Überwachung von Krankheitsbildern, zu geben \citep[][]{DigitalSportsGroupatthePatternRecognitionLab}.

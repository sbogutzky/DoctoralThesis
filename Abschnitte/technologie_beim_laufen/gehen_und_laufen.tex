

%!TEX root = /Users/sbogutzky/Entwicklung/projects/bogutzky/repositories/2939413/final-draft.tex
\section{Gehen und Laufen} 

% (fold)
\label{sec:gehen_und_laufen}

Gehen und Laufen sind zyklische Tätigkeiten und folgen einem wiederholenden Muster, indem ein Schritt des einen Beins dem Schritt des anderen Beins folgt. \citet[][S.~9]{Bartlett2007} definiert einen Doppelschritt (stride) beim Gehen von der Bodenberührung des einen Fußes bis zur nächsten Bodenberührung des selben Fußes, von dem Schüsselereignis Abheben der Zehen (\ac{TO}) zum nächsten Abheben. 
\begin{figure}
	[ht] \centering 
	\includegraphics[width=1.00 
	\textwidth]{2_1_bewegungsablauf_gehen} \caption[Phasen des menschlichen Bewegungsablaufs beim Gehen]{Phasen des menschlichen Bewegungsablaufs beim Gehen.}\label{fig:2_1_bewegungsablauf_gehen} 
\end{figure}

Beim Gehen besitzt der Bewegungsablauf (Abbildung~\ref{fig:2_1_bewegungsablauf_gehen}) eine Einzelstützphase (single-support phase), wenn ein Fuß den Boden berührt und eine Doppelstützphase (double-support phase), wenn beide Füße den Boden berühren. Die Einzelstützphase beginnt mit dem Abheben des Fußes und die Doppelstützphase beginnt mit dem Aufsetzen des selben Fußes, mit dem Schlüsselereignis Aufsetzen der Ferse (\ac{HS}). Die Dauer der Einzelstützphase ist in der Regel viermal länger als die der Doppelstützphase. 

Betrachten wir alternativ jedes Bein beim Gehen einzeln, besitzt es im Bewegungsablauf eine Stützphase (stance phase) und eine Schwungphase (swing phase) \citep[][]{Bartlett2007}. 
\begin{itemize}
	
	\item In der vorliegenden Arbeit benutze ich das Adjektiv normal, um auszudrücken, dass es sich um eine Bewegung handelt, die Menschen in der Regel ohne Einflüsse wie z.~B. Verletzungen ähnlich ausführen.
\end{itemize}

Beim normalen Gehen in einer selbstbestimmten und angenehmen Geschwindigkeit beträgt die Stützphase durchschnittlich 60~\% und die Schwungphase durchschnittlich 40~\% des Bewegungsablaufs. Für die vorliegende Arbeit besitzt die Schwungphase ein bedeutendes Schlüsselereignis, den mittleren Schwung (\ac{MS}). Der mittlere Schwung ist gekennzeichnet durch eine hohe kinematische Energie beim Gehen und Laufen \citep[][]{Novacheck1998}.
\begin{figure}
	[ht] \centering 
	\includegraphics[width=1.00 
	\textwidth]{2_2_bewegungsablauf_laufen} \caption[Phasen des menschlichen Bewegungsablaufs beim Laufen]{Phasen des menschlichen Bewegungsablaufs beim Laufen.}\label{fig:2_2_bewegungsablauf_laufen} 
\end{figure}

Eigenschaften der menschlichen Bewegung sind \emph{individuell} und unterscheiden sich von Mensch zu Mensch. Beim normalen Gehen setzt der Fuß mit der Ferse bzw.\ Rückfuß \ac{HS} auf \citep[][S.~33]{Marquardt2011}. Anders verhält es sich beim Laufen, bei dem der Läufer mit dem Vorder-, Mittel- oder Rückfuß aufsetzt. In der Literatur finden wir die Ausdrücke \ac{IC} und \ac{IS}, um die Berührung des Bodes durch den Fuß und das Lösen des Fußes vom Boden zu verallgemeinern.

Laufen unterscheidet sich vom Gehen, da keine Doppelstützphase vorhanden ist, in der beide Füße gleichzeitig den Boden berühren \citep[][S.~15~f.]{Bartlett2007}. Der Bewegungsablauf beim Laufen (Abbildung~\ref{fig:2_2_bewegungsablauf_laufen}) unterteilt sich in eine Stützphase (support phase), in der ein Fuß den Boden berührt und in eine Schwungphase, in der beide Füße keinen Bodenkontakt besitzen. Nur die Stützphase, vom Aufsetzen des Fußes (\ac{IC}) bis zum Abstoßen des Fußes (\ac{IS}), ermöglicht es dem Läufer, Kraft für die Fortbewegung auszuüben.

\citet[][S.~17]{Bartlett2007} betrachtet die Schwungphase als Vorbereitung für das nächste Aufsetzen des Fußes und bezeichnet sie als recovery phase (Aufschwung, aber auch Wiederherstellung und Erholung). Beim Laufen oder Joggen ist die Dauer der Stützphase und der Schwungphase in der Regel gleich lang \citep[][S.~32~f.]{Marquardt2011}. Die Schwungphase verlängert sich mit höherer Laufgeschwindigkeit. Tabelle~\ref{tab:unterschiede_zwischen_laufen_und_gehen} fasst die Unterschiede zwischen Gehen und Laufen nochmals zusammen.
\begin{table}
	[ht] \caption[Unterschiede zwischen Laufen und Gehen]{Unterschiede zwischen Laufen und Gehen nach \citet{Marquardt2011}}\label{tab:unterschiede_zwischen_laufen_und_gehen} 
	\begin{tabularx}
		{ 
		\textwidth}{*{3}{>{\RaggedRight\arraybackslash}X}} \toprule & Gehen & Laufen \\
		\midrule Phasen & Stütz- und Schwungphase & Stütz- und Schwungphase mit Flugphase \\
		Verhältnis Stütz-/ Schwungphase & 60/40 & 50/50 (und kleiner) \\
		Spurbreite & breiter & schmaler \\
		Stoßkräfte (Landung) & 1- bis 1,5-Faches des Körpergewichts & 2- bis 3-Faches des Körpergewichts \\
		Stoßkräfte (Abdruck) & 1- bis 1,5-Faches des Körpergewicht & 3,5- bis 5-Faches des Körpergewichts \\
		Fußaufsatz & immer mit der Ferse (Rückfuß) & mit dem Vorder-, Mittel-, oder Rückfuß \\
		\bottomrule 
	\end{tabularx}
\end{table}

% section gehen_und_laufen (end)

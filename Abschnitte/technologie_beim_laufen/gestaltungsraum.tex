

%!TEX root = /Users/sbogutzky/Entwicklung/projects/bogutzky/repositories/2939413/final-draft.tex
\section{Gestaltungsraum für Technologien für das Laufen} 

% (fold)
\label{sec:gestaltungsraum_fur_technologien_fur_das_laufen}

Die vorgestellten Technologien ergeben nach \citet[][]{Jensen2014} den in Abbildung~\ref{fig:gestaltungsraum_entwicklungen} dargestellten zwei-dimensionalen Gestaltungsraum. Auf der Achse Rückmeldung liegen auf der einen Hälfte die darstellenden Rückmeldungen und auf der anderen Hälfte die assistierenden Rückmeldungen. Die Achse Fokus repräsentiert die leistungsbezogenen Technologien auf der einen Seite und die technikbezogenen Technologien auf der anderen Seite.
\begin{itemize}
	
	\item Das Ziel der vorliegenden Arbeit ist es in erster Linie nicht, die Laufleistung oder die Laufökonomie zu optimieren, sondern die Voraussetzungen für das Erleben von Flow beim Laufen zu verbessern.
\end{itemize}

Dabei besteht ein innerer Zusammenhang von Laufleistung, Laufökonomie und Flow-Erleben bzw.\ Wohlbefinden. Hinsichtlich des Flow-Konstrukts (Kapitel~\ref{cha:flow_erleben_messen}) sind Laufleistung und Laufökonomie keine Gegensätze. Aus diesem Grund ist eine rein leistungs- und technikbezogene Einteilung des Gestaltungsraums für Lauftechnologie für die vorliegende Arbeit nicht ausreichend.
\begin{figure}
	[!htb] \centering 
	\includegraphics[width=1.00 
	\textwidth]{gestaltungsraum_entwicklungen} \caption[Gestaltungsraum für Technologien für das Laufen]{Gestaltungsraum für Technologien für das Laufen. Quelle: \citet[][]{Jensen2014}. Leicht modifiziert und übersetzt}\label{fig:gestaltungsraum_entwicklungen} 
\end{figure}

Die reine Quantifizierung gemessener Tätigkeiten besitzt einen negativen Einfluss auf die intrinsische Motivation \citep[][]{Etkin2016}. Eine ausschließlich durch Messwerte ausgedrückte Sicht auf die Tätigkeit trägt zu einer verminderten Hingabe für die Tätigkeit und verminderten Freude bzw.\ zu subjektiven Wohlbefinden bei der Ausführung der Tätigkeit bei. Aus diesem Grund konzentrieren sich z.~B. \citet{Hajinejad2015} in ihrem Gestaltungsansatz darauf, die Erlebensqualität der Gehaktivität in den Vordergrund zu stellen.
\begin{itemize}
	
	\item In der vorliegenden Arbeit schlage ich vor, den vorgestellten Gestaltungsraum um den Faktor Wohlbefinden auf der Fokus-Achse zu erweitern.
\end{itemize}

Die beiden Faktoren auf der Achse Rückmeldung lassen auf die beiden Möglichkeit der Objektivierung des Flow-Erlebens beziehen. Darstellende Rückmeldung beziehen sich überwiegend auf zustandorientierte Messungen, wohingehen assistierenden Rückmeldungen den Prozess unterstützen. 
\begin{itemize}
	
	\item Auf dieser Grundlage motiviere ich assistierenden Echtzeit-Benutzerschnittstellen, die das Flow-Erleben im Prozess des Laufens und Gehens implizit messen. Damit soll die vorliegenden Arbeit zur Bildung des Fundaments beizutragen, das die Voraussetzungen des Läufers durch Echtzeitrückmeldungen verbessert, Flow zu erleben --- also Lauf-Apps zu entwickeln, deren Ziel es ist, Läufer in einen individuellen optimalen Zustand zu bringen und das Wohlbefinden während des Laufens zu fördern.
\end{itemize}

% section gestaltungsraum_fur_technologien_fur_das_laufen (end)

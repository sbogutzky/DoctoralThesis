

%!TEX root = /Users/sbogutzky/Entwicklung/projects/bogutzky/repositories/2939413/final-draft.tex
\section{Existierende Technologien beim Lauftraining} 

% (fold)
\label{sec:existierende_technologien_beim_lauftraining}

Die Erkenntnisse über die biomechanischen Eigenschaften des Gehens und des Laufens lassen sich in Konzepte für Benutzerschnittstellen (user interfaces) für das Geh- und Lauftraining einsetzen. Eine Vielzahl wissenschaftlicher Arbeiten aus der Mensch-Computer-Interaktion beschäftigt sich mit der Hilfe der vorgestellten Erkenntnisse mit der Identifikation von Personen und der Fallerkennung von Menschen. Die beiden genannten Anwendungsfälle sind nicht Bestandteil der vorliegenden Arbeit. In dieser Arbeit beschäftige ich mich im Bereich der mobilen Mensch-Computer-Interaktion mit Technologien für das Lauftraining. \citet{Jensen2014} unterteilen die existierende Technologie beim Lauftraining in die nachfolgenden Gruppen: 
\begin{itemize}
	\item traditionelle Benutzerschnittstellen (Abschnitt~\ref{sub:traditionelle_benutzerschnittstellen}) 
	\item Lauftechnik-Detektionstechnologie (Abschnitt~\ref{sub:lauftechnik_detektionstechnologie}) 
	\item assistierende Echtzeit-Benutzerschnittstellen (Abschnitt~\ref{sub:assistierende_echtzeit_benutzerschnittstellen}) 
\end{itemize}

\subsection{Traditionelle Benutzerschnittstellen} 

% (fold)
\label{sub:traditionelle_benutzerschnittstellen}

Eine Benutzerschnittstelle, die leistungsbezogene Merkmale des Lauftrainings wie Geschwindigkeit, Laufzeit und -distanz sowie Herzfrequenz darstellt, zählen \citet{Jensen2014} zu den traditionellen Benutzerschnittstellen. Als Beispiele sind Läuferuhren und Apps, die dem Läufer zur Anzeige leistungsbezogener Merkmale dienen, zu nennen.

Die ersten Läuferuhren erschienen im Jahr 2000. Zum Beispiel diente die Pellor 3D Sportuhr Running dem Zählen von Schritten und der Pulsmessung beim Laufen. Heute benutzen Laufuhren unterschiedliche Technologien wie \acs{GPS}-Empfänger, Herzfrequenzmesser und Beschleunigungsmesser. Die Läufer lesen Informationen entweder direkt von der Anzeige der Uhr während der Läufe ab oder sie lassen sich eine detaillierte Übersicht an einem PC zur Verfügung stellen. Zu den heutigen Laufuhren erhalten die Läufer Software bzw.\ Apps zur Übertragung und Darstellung der Informationen. Manche Laufuhren arbeiten mit akustischen und taktilen Signalen, um Läufer zu informieren, ob sie z.~B. ihren angestrebten Herzfrequenzbereich über- oder unterschreiten.

Entwickler von Lauf-Apps für Smartphones wie Runtastic, Runkeeper und Nike Running übernahmen die Konzepte der Laufuhren. Unter Verwendung der internen \acs{GPS}-Einheit und des Beschleunigungsmessers der Smartphones sowie von externen Herzfrequenzmessern stellen Lauf-Apps die gesammelten Informationen auf dem Smartphone-Bildschirm dar. Es entstanden spezielle Sportarmbänder zur Befestigung der Smartphones am Oberarm der Läufer. Sie ermöglichen dem Läufer während des Laufens auf den Bildschirm zu schauen, ohne dass sie das Smartphone z.~B. aus einer Tasche herausholen müssen. Viele Lauf-Apps arbeiten zusätzlich mit verbalen Rückmeldungen, die den Läufern z.~B. die aktuelle Laufzeit und -distanz ansagen. Die Entwickler der Lauf-Apps bieten in der Regel den Läufern zusätzlich an, die gesammelten Informationen auf ihre Server zu übertragen. Sie sind damit in der Lage, den Läufern eine detaillierte Übersicht ihrer Läufe auf einer Webseite darzustellen. Darüber hinaus bieten manche Laufuhren und Lauf-Apps Funktionen zur Trainingssteuerung an. Die Funktionen beinhalten die Gestaltung von Trainingsplänen und -zielen.

% subsection traditionelle_benutzerschnittstellen (end)
\subsection{Lauftechnik-Detektionstechnologie} 

% (fold)
\label{sub:lauftechnik_detektionstechnologie}

Lauftechnik-Detektionstechnologie besteht für \citet{Jensen2014} aus Handhelds und tragbaren Trägheitsmesseinheiten oder kamerabasierenden Bewegungserkennungssystemen. In der vorliegenden Arbeit liegt der Fokus auf Handhelds und tragbaren Trägheitsmesseinheiten, da wir in der Lage sind, sie nahezu uneingeschränkt außerhalb von Laboratorien einzusetzen. Abhängig von Energiebedarf und Energiequelle lassen sich beide kabellos über längere Zeiträume betreiben. Sie sind beide in der Lage kinematische Daten während des Laufens zu messen und zu sammeln. Kinematische Daten bilden die Grundlage zur Berechnung kinematischer Merkmale des Laufens. Zu den Merkmalen gehören z.~B. die Schrittfrequenz, die Bodenkontaktzeit (ground contact time), die vertikale Bewegung des Läufers und dessen Fußaufsatztyp (foot strike type).

\citet{Harms2010} entwickelten mit ETHOS einen 2,5 $cm^{2}$ großen tragbaren Prototypen, der einen Beschleunigungsmesser, ein Kreiselinstrument und ein Magnetometer besitzt. Eine ETHOS-Einheit ist in der Lage über einen längeren Zeitraum kinematische Daten zu sammeln und für eine nachträgliche Analyse bereitzustellen.

\citet{Strohrmann2011} nutzten gleichzeitig 12 ETHOS-Einheiten an verschiedenen Körperpositionen bei einem standardisierten Testlauf mit 12 Teilnehmern, die ein unterschiedliches Laufleistungsniveau besaßen. Das Leistungsniveau wurde anhand der wöchentlich gelaufenen Kilometer definiert. Die Autoren berechneten aus den kinematischen Daten von den Füßen die Bodenkontaktzeit und aus den kinematischen Daten von der Hüfte die vertikale Bewegung des Läufers. Die beiden Merkmale ermöglichten den Autoren, die 12 Teilnehmer zu unterscheiden und ihrem Laufleistungsniveau zu zuordnen. Die Autoren zeigten, dass der Einsatz der tragbaren Trägheitsmesseinheiten ohne Beeinträchtigung des Laufens außerhalb von Laboratorien praktikabel ist \citep{Strohrmann2011a}.

In einer nachfolgenden Studie berechneten \citet{Strohrmann2012} gleich zehn kinematische Merkmale, um Ermüdung beim Laufen zu identifizieren. Zum Einsatz kamen jeweils 12 ETHOS-Einheiten bei 21 Läufern. In der ersten Studie wurden 45 Minuten auf einem Laufband gelaufen. In der zweiten Studie liefen die Läufer im Freien auf einem Rundkurs. Die Autoren kamen aufgrund der beiden Studien zu dem Schluss, dass durch die \emph{Individualität} jedes Läufers kein \emph{generalisierbarer Laufprototyp} anhand der kinematischen Merkmale abzuleiten ist. Sie sehen den deutlichen Nutzen von tragbaren Technologien beim Laufen in wiederholenden quantitativen und objektiven Messungen der Lauftechnik. Die Messungen ermöglichen Einblicke in den Zusammenhang von Laufkinematik, Verletzungsrisiko, Ermüdung und Laufökonomie. Mit Blick auf die Vorbereitungszeit für jeden Lauf argumentierten die Autoren für den Einsatz von einer geringeren Anzahl von Messeinheiten. Darum benannten sie Fuß und Rumpf zu den bedeutsamsten Körperpositionen zur Berechnung von kinematischen Merkmalen.

Das von \citet{Eskofier2013} vorgestellte Klassifizierungssystem hatte das Ziel, Menschen den geeigneten Laufschuh für den jeweiligen Fußaufsatztypen zu empfehlen. Hierzu ermittelte das System anhand der Klassifizierung der Beschleunigungsdaten einer Trägheitsmesseinheit am Fuß den Fußaufsatztyp nach einem Lauf. Ein Beispiel der gleichen Forschungsgruppe zeigte die Nutzung von zwei Trägheitsmesseinheiten zur Bestimmung der kinematischen Merkmale Kniestreckwinkel und Kniebeugewinkel \citep{Jakob2013}. Beide Winkel berechneten die Autoren anhand von kinematischen Daten vom Unter- und Oberschenkel. Sie verglichen die vom Algorithmus berechneten Winkel mit einem Kamerasystem. Durch vergleichbare Ergebnisse und die Nutzungsmöglichkeit außerhalb von Laboratorien sehen die Autoren den Einsatz ihrer Technologie im Training und im Wettkampf zur Verbesserung der Leistungsfähigkeit durch objektive Rückmeldung.

% subsection lauftechnik_detektionstechnologie (end)
\subsection{Assistierende Echtzeit-Benutzerschnittstellen} 

% (fold)
\label{sub:assistierende_echtzeit_benutzerschnittstellen}

Die Beispiele aus Abschnitt~\ref{sub:lauftechnik_detektionstechnologie} haben gegenüber den traditionellen Benutzerschnittstellen das überwiegende Ziel, die Laufökonomie des Läufers zu steigern. Die Bestimmung und Rückmeldung der kinematischen Merkmale erfolgt nach dem Laufen. Assistierende Echtzeit-Benutzerschnittstellen nach \citet{Jensen2014} wie z.~B. die Laufuhr Garmin Forerunner 620 geben eine Rückmeldung auf die Lauftechnik während des Laufens. Die Laufuhr Garmin Forerunner 620 stellt z.~B. dem Läufer die Bodenkontaktzeit, die Kadenz und die vertikale Bewegung als numerischen Wert auf dem Bildschirm dar. Es ist schwer, die Lauftechnik anhand von visuell dargestellten Informationen zu ändern. Die kontinuierliche Informationsaufnahme, verbunden mit dem Blick auf die Uhr, trägt im ungünstigsten Fall zu einem unökologischen Laufstil bei \citep{Jensen2014}. Dieses Beispiel zeigt, dass der visuelle Kanal nicht in jedem Anwendungsfall die geeignetste Darstellungsform für Informationen ist. 

\citet{Zhao2007} merken dazu an, dass es Situationen gibt, in denen Benutzer ihre Aufmerksamkeit auf die Umgebung richten müssen und dass das Betrachten einer visuellen Schnittstelle ablenkend oder in manchen Fällen gefährlich ist. Das Laufen ist eine der mobilen Situationen, die die Aufmerksamkeit des Läufers in einem bestimmten Maße erfordert. \citet{Jensen2014} sehen in Bezug auf die Lauftechnologie einen Bedarf an alternativen Methoden der Rückmeldungen, die sich von den konventionellen bildschirmbasierenden Schnittstellen unterscheiden. Die nachfolgenden Arbeiten stellen unterschiedliche Lösungen für alternative Methoden der Rückmeldung beim Lauftraining vor. 

\citet{Wijnalda2005} präsentierten ein System, das anhand der Auswahl von Musikstücken oder deren Modifikation eine Echtzeit-Rückmeldung auf die Schrittfrequenz gibt. Die Läufer wählen aus drei verschiedenen Trainingsmodi aus. Der Pace-fixing-Modus dient dem Ausdauertraining und spielt Musikstücke mit gleichem Tempo ab. Der Pace-matching-Modus passt die Musik der Schrittfrequenz an und im Pace-influencing-Modus besitzt das Tempo der Musik die Aufgabe, die Schrittfrequenz zu beeinflussen. Das System mit dem Namen IM4Sports besteht aus einem tragbaren Musikspieler, einem Brustgurt zur Herzfrequenzmessung und einem Pedometer. Die Herzfrequenzmessung besitzt keinen Einfluss auf die Auswahl der Musikstücke. Die Autoren kamen zu dem Schluss, dass Läufer eine Adaptionszeit von mindestens 20 Sekunden benötigen, um eine Synchronisation von Schrittfrequenz und Musiktempo im Pace-influencing-Modus herbeizuführen.

Ein ähnliches Konzept wie bei IM4Sports wurde von \citet{Oliver2006} mit MP Train verfolgt. Das mobiltelefon-basierende System wählt anhand der Schrittfrequenz geeignete Musikstücke aus. Es passt die Auswahl mit Hilfe eines Brustgurts zur Herzfrequenzmessung an eine vordefinierte Trainingsintensität an. Unterschreitet die Herzfrequenz des Läufers die vordefinierte Herzfrequenz, wählt das System Musikstücke mit einem höheren Tempo aus. Andersherum wählt das System Musikstücke eines langsameren Tempos aus, wenn die Herzfrequenz des Läufers die vordefinierte Herzfrequenz überschreitet. MP Train wurde in einer zweiten Iteration durch \citet{DeOliveira2008} zu Triplebeat und um Funktionen wie z.~B. einer Herausforderungsfunktion erweitert. Hinweise aus der Literatur \citep{Bood2013} bestätigen den Einfluss eines Musikstücks oder eines Metronoms auf die Schrittfrequenz. \citet{Bood2013} schließen aus ihrer Studie, dass ein motivierendes Musikstück mit einem angepassten Tempo zur Schrittfrequenz einen positiven Effekt auf die auditorisch-motorische Synchronisation besitzt und die Laufökonomie verbessert.

\citet{Takata2007} bilden den Laufprozess in einem Zustandsübergangsdiagramm mit den Zuständen Aufwärmen, Haupttraining und Abwärmen ab. Ihr tragbares System ist in der Lage, anhand der Korrelation von voreingestellten Werten für jeden Zustand und den aktuellen Messwerten der Herzfrequenz oder der Schrittfrequenz den aktuellen Zustand des Läufers während des Laufens zu identifizieren. Die Autoren geben zu bedenken, dass sich die voreingestellten Werte aufgrund von verschiedenen Faktoren wie dem Alter oder der körperlichen Verfassung des Läufers \emph{individuell} unterscheiden. Das System ermöglicht, die Voreinstellung von vorherigen Trainingseinheiten abzuleiten. Im gleichen System integrierten die Autoren eine Funktion, um den Laufkurs zu erstellen. Anhand des gewünschten Kalorienverbrauchs und der Eigenschaften des Läufers wie z.~B. Leistungslevel, Alter, Gewicht, erstellt das System einen individuellen Laufkurs. Das System verlängert den Laufkurs in Echtzeit, wenn es merkt, dass der Läufer das Ziel wegen zu geringen Kalorienverbrauchs durch z.~B. langsames Laufen nicht erreicht. Anders herum verkürzt es den Laufkurs, wenn es einen zu hohen Kalorienverbrauch feststellt.

Das von \citet{Eriksson2010} vorgestellte System, bestehend aus einem mobilen Telefon und einem am Rücken befestigten Beschleunigungsmesser, gibt akustische Rückmeldungen auf die Schrittfrequenz und die vertikale Bewegung des Läufers. Die vertikale Bewegung bezeichnet die Auf- und Abbewegung des Läufers. Eine geringe vertikale Bewegung erhöht die Laufökonomie. Anhand der Vorgabe des Läufers passt das System ein getaktetes Audiosignal an, damit der Läufer die Schrittfrequenz erhöht oder verlangsamt. Das System gibt ein akustisches Warnsignal aus, wenn der Läufer eine vordefinierte vertikale Bewegung überschreitet. Das System ist in der Lage nach vier Sekunden einen Durchschnitt der vertikalen Bewegung oder nach jedem Schritt die vertikale Bewegung zu berechnen und eine Rückmeldung zu geben, falls der Läufer den vordefinierten Wert überschreitet. Ein Versuch mit einer Testperson zeigte, dass sie in beiden Modalitäten ihre vertikale Bewegung reduzierte und sich die maximale vertikale Bewegung der Vorgabe näherte.

Die symmetrische Linie des Körpers zur Überprüfung der Armbewegung beim Laufen nutzten \citet{Strohrmann2013, Strohrmann2014} in einer Smartphone App. Die App signalisiert dem Läufer über Vibration, wenn die Armbewegung nicht parallel zum Körper verläuft. In einer Vorstudie mit zehn Teilnehmern trainierten \citet{Strohrmann2013} einen Algorithmus zur Klassifizierung der drei Haltungsfälle: Arme laufen parallel zum Körper (ökonomischste Haltung), Arme zielen zur symmetrischen Linie und Arme kreuzen die symmetrische Linie. Die Autoren identifizierten den Oberarm als geeignetste Körperposition für die Klassifizierung. Der Vorteil des Oberarms ist zusätzlich, dass Läufer ihn im Regelfall nutzen, um ein Smartphone in einem Sportarmband zu tragen. In der Vorstudie nutzten die Autoren ETHOS-Einheiten. In der Hauptstudie mit 20 Teilnehmern kam das Smartphone mit dessen internen Beschleunigungsmesser und Kreiselinstrument zum Einsatz. Die Studie diente dem Vergleich der App-Rückmeldung und einer verbalen Rückmeldung durch einen Übungsleiter. In beiden Formen der Rückmeldung zeigten die Ergebnisse eine vergleichbare signifikante Verbesserung der Armhaltung. Die Autoren sehen den Vorteil der App darin, dass Läufer ohne Zugang zu einem persönlichen Trainer in der Lage sind, ihre Lauftechnik zu verbessern. Anhand eines Fragebogens schließen die Autoren zusätzlich auf eine hohe Akzeptanz ihres Konzepts.

% subsection assistierende_echtzeit_benutzerschnittstellen (end)
% section existierende_technologien_beim_lauftraining (end)

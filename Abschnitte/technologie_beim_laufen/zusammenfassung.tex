

%!TEX root = /Users/sbogutzky/Entwicklung/projects/bogutzky/repositories/2939413/final-draft.tex
\section{Zusammenfassung} 

% (fold)
\label{sec:zusammenfassung_2}

In diesem Kapitel führe ich in die Forschung im Bereich der Lauftechnologie ein. Zu Beginn stelle ich die Motivation der Computernutzung im Sport vor und betone damit die Allgegenwärtigkeit von Computern. In der vorliegenden Arbeit lege ich das Augenmerk auf Apps, die auf mobilen Geräten laufen, die der Benutzer in der Regel in einer Hand hält (Handhelds) oder am Körper trägt (Wearables).

Ihre Sensorik ermöglicht es, kinematische Daten zu sammeln und implizite Merkmalen beim Gehen und Laufen zu messen. Diese impliziten Merkmale basieren auf dem Wissen der menschlichen Biomechanik beim Gehen und Laufen mit ihren Schlüsselereignissen \ac{HS} bzw.\ \ac{IC}, \ac{TO} bzw.\ \ac{IS} und \ac{MS}. Eine zeitnahe Verarbeitung von impliziten Merkmalen während des Prozesses der Tätigkeit versetzt uns in die Lage, individuelles Echtzeit Biofeedback zu geben (\emph{Realtime Biofeedback}). 

Die Verallgemeinerung von impliziten Merkmalen ist eine zusätzliche Herausforderung. Die Untersuchung von \citet{Strohrmann2012} zeigt, dass z.~B. kinematische Merkmale mit einer hohen Individualität des Läufers verbunden sind. Diese Erkenntnis unterstreicht wie beim Flow-Erleben den Wert von \emph{intraindividuellen Untersuchungen}.

Laufleistungs- und lauftechnikbezogene Merkmale spielen in diversen Forschungsarbeiten der Mensch-Computer-Interaktion eine bedeutende Rolle beim Laufen. Der vorgestellte Gestaltungsraum (Abbildung~\ref{fig:gestaltungsraum_entwicklungen}) für die Lauftrainingstechnologie zeigt zusammenfassend Anwendungsfälle computerbasierender Systeme. Flow-Erleben und die im Zusammenhang stehenden Gemüts- und Gefühlszustände Freude und Wohlbefinden finden in dem Gestaltungsraum keine Berücksichtigung. Aus dem Grund schlage ich in diesem Fall eine Erweiterung des Gestaltungsraums um den Faktor Wohlbefinden vor.

In erster Linie sehe ich die vorgestellten assistierenden Echtzeit-Benutzerschnittstellen als Anknüpfungspunkt für Applikationen, die die Bedingungen verbessern, Flow während des Laufens zu erleben. Es fehlt uns für die Realisierung solcher Applikationen ein implizites Messverfahren, um einen Bewusstseinszustand wie Flow-Erleben im Prozess des Laufens zu erfassen. Demzufolge betrachte ich im nachfolgenden Kapitel das Flow-Erleben, bestehende explizite Messverfahren, Lösungsansätze zur impliziten Messung des Flow-Erlebens und prozessorientierte Ansätze mit expliziten Messverfahren.

% section zusammenfassung (end)

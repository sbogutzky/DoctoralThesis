

%!TEX root = /Users/sbogutzky/Entwicklung/projects/bogutzky/repositories/2939413/final-draft.tex
\chapter{Technologie beim Laufen} 

% (fold)
\label{cha:technologie_beim_laufen}

Heutzutage sind Computer allgegenwärtig. Mark Weiser prägte schon 1988 den Ausdruck der allgewärtigen Computernutzung (ubiquitous computing) und beschrieb eine Zeit nach der Mensch-Computer-Interaktion am Schreibtisch, in der Computer in Alltagsgegenständen integriert sind. Die Miniaturisierung von elektronischen Komponenten ermöglichte die Entwicklung von mobilen Geräten. Mobile Geräte bezeichnen solche Geräte, die ihre Benutzer in die Lage versetzen, Computer in mobile Situationen zu nutzen.

Mobile Situationen kennzeichnen Situationen, in denen wir, die Benutzer, aber auch von uns überwachte Gegenstände, mobil sind. \citet[][S.~24~ff.]{Cooper2002} bezeichnet Benutzermobilität als physische Bewegung des Benutzers im Raum und in der Zeit. Nach \citet[][S.~7]{Roth2005} muss sich ein Benutzer nicht zwangsläufig bewegen, um mobil zu sein. Für die vorliegende Arbeit sind mobile Situationen von Bedeutung, in denen der Benutzer geht oder läuft und nur kurzfristig an einem Ort verweilt, bevor er die physische Bewegung wieder aufnimmt.

\citet[][S.~5]{Roth2005} zählt tragbare Standardcomputer wie Notebooks oder Bordcomputer zu den mobilen Geräten. In der vorliegenden Arbeit lege ich das Augenmerk auf mobile Geräte, die der Benutzer in der Regel in einer Hand hält (Handhelds) oder am Körper trägt (Wearables).

Als mobile Applikationen (Apps) bezeichne ich in der vorliegenden Arbeit Applikationen, die auf mobilen Geräten laufen. Obwohl sich App als Abkürzung für den englischen Begriff \emph{application software} auf jegliche Anwendungssoftware beziehen lässt, setze ich ihn in der vorliegenden Arbeit mit der Anwendungssoftware eines mobilen Gerätes gleich.

Mobile Geräte, Apps und allgemein die Mensch-Computer-Interaktion erlangten in den vergangenen Jahren in Bezug auf den Sport an Bedeutung \citep[][]{Nylander2014}. Die Motivation der Computernutzung im Sport besteht darin, positive Impulse für das menschliche Wohlbefinden durch Steigerung der Leistungsfähigkeit, Instandhaltung der Gesundheit, Verbesserung der Rehabilitation oder Überwachung von Krankheitsbildern, zu geben \citep[][]{DigitalSportsGroupatthePatternRecognitionLab}.

Typische Vertreter mobiler Geräte im Sport sind tragbare Datenlogger (Activity Tracker), tragbare Trägheitsmesseinheiten (\ac{IMU}), Smartphones und Computeruhren (Smartwatches). Sie besitzen alle eine Kombination von mehreren Sensoren, wie z.~B. Beschleunigungsmesser, Drehratensensor und \ac{GPS}-Empfänger, zur Messung von kinematischen Daten. In der vorliegenden Arbeit bezeichne ich mit kinematischen Merkmalen, Merkmale die auf Berechnungen beruhen, die die Gesetze der Kinematik als Grundlage heranziehen.

„Die Kinematik beschreibt die Bewegung von Punkten und Körpern im Raum. Zur vollständigen kinematischen Beschreibung einer Bewegung genügen die Größen Lage, Geschwindigkeit und Beschleunigung. Die Ursachen der Bewegung (z.~B. Kräfte) bleiben bei der kinematischen Beschreibung unberücksichtigt“ \citep[][S.~57]{Disselhorst-Klug2015}.

Die Gesetzmäßigkeiten der Kinematik eröffnen der mobilen Mensch-Computer-Interaktion im Sport nachfolgende Möglichkeiten:
\begin{itemize}
	\item sie ermöglicht durch die vom \ac{GPS} gesendete Position und Zeit, die Berechnung der durchschnittlichen Geschwindigkeit eines Benutzers eines mobilen Geräts und der Entfernung, die er zurücklegte. 
	\item sie ermöglicht durch ihre Analyse (z.~B. Zeitverläufe der Gelenkbewegung) das Resultat einer Bewegung des menschlichen Körpers unter Beachtung der Biomechanik in unterschiedlichen Tätigkeiten zu quantifizieren. 
\end{itemize}

Die Biomechanik beschreibt die tätigkeitsspezifische Bewegung unter Verwendung von Methoden und Gesetzmäßigkeiten der Mechanik und Anatomie des menschlichen Körpers \citep[][ S.~2~ff.]{Winter2009}.

Mit Bezug auf die Modalitäten bei der Benutzung von mobilen Geräten lassen sich visuelle, akustische und taktile Rückmeldungen unterscheiden. Bei visueller Rückmeldung bekommt der Sportler Informationen bildlich oder als Text auf dem Display des mobilen Gerätes ausgegeben. Bei akustischer Rückmeldung hört er Klänge oder gesprochenen Text (auch verbale Rückmeldung genannt). Bei taktiler Rückmeldung überträgt das mobile Gerät die Information über die Berührung, z.~B. durch dessen Vibration.

Rückmeldungen im Sport erfolgen nach der physischen Belastung oder während der physischen Beanspruchung. Letzteres bezeichne ich in der vorliegenden Arbeit als Echtzeit (real time). Nach ISO/IEC Norm 2382 ist Echtzeitverarbeitung eine Verarbeitungsart, bei der die Programme für die Datenverarbeitung ständig betriebsbereit sind, sodass die Ausgangsdaten innerhalb einer bestimmten Zeitspanne zur Verfügung stehen.

Demzufolge verspricht Echtzeit eine verzögerungsarme Rückmeldung auf ein Ereignis durch das mobile Gerät. Die Verzögerung hängt von der Verarbeitungsdauer der Daten ab, die eine festgelegte Information ermöglicht. Auch wenn die Verarbeitungsdauer eine sofortige Rückmeldung zulässt, ist abhängig vom Anwendungsfall ein vorbestimmter Rückmeldungszeitpunkt vorstellbar. Eine Herausforderung besteht darin, Rückmeldungen in Echtzeit an die physische Beanspruchung des Sportlers und die mobile Situation anzupassen, damit sie den Fluss und die Erfahrung des Sportlers nicht unterbrechen \citep[][]{Nylander2014}.

In Anbetracht der Tatsache, dass ein technisches System (mobiles Gerät), Daten eines biologischen Systems (Sportler) wie z.~B. Herz- und Bewegungsdaten verzögerungsarm verarbeitet und als Information an das gleiche biologische System zurückgibt, verwende ich in dieser Arbeit für die vorliegende Art von Rückmeldung den Begriff \emph{Realtime Biofeedback}.

„Biofeedback (altgr. \emph{bios} „Leben“ und engl. \emph{feedback} „Rückmeldung“) ist eine Möglichkeit, Veränderungen von körperlichen Zuständen, die der unmittelbaren Sinneswahrnehmung nicht zugänglich sind, mit technischen Hilfsmitteln bewusst zu machen“ \citep[][S.~483]{Riemer2015}.

Biofeedback dient in den existierenden Technologien und Konzepten aus der Forschung zur Mensch-Computer-Interaktion im Laufsport überwiegend zur Leistungsdiagnostik und zur Trainingssteuerung. Die Trainingssteuerung dient dem Sportler, z.~B. zur Planung von Regenerationsphasen oder zur Anpassung der physischen Beanspruchung während des Trainings \citep[][S.~81-107]{Marquardt2011}. Jüngste Fortschritte auf dem Gebiet der mobilen Technologien und der Echtzeitverarbeitung ermöglichten erst Konzepte, die mit Hilfe von \emph{Realtime Biofeedback} das Körperbewusstsein und die Körperkontrolle während des Laufens verbessern \citep[][]{Strohrmann2013, Strohrmann2013a, Strohrmann2014}.

\section{Gehen und Laufen} 

% (fold)
\label{sec:gehen_und_laufen}

Gehen und Laufen sind zyklische Tätigkeiten und folgen einem wiederholenden Muster, indem ein Schritt des einen Beins dem Schritt des anderen Beins folgt. \citet[][S.~9]{Bartlett2007} definiert einen Doppelschritt (stride) beim Gehen von der Bodenberührung des einen Fußes bis zur nächsten Bodenberührung des selben Fußes, von dem Schüsselereignis Abheben der Zehen (\ac{TO}) zum nächsten Abheben. 
\begin{figure}
	[t] \centering 
	\includegraphics[width=1.00
	\textwidth]{2_1_bewegungsablauf_gehen.pdf} \caption[Phasen des menschlichen Bewegungsablaufs beim Gehen]{Phasen des menschlichen Bewegungsablaufs beim Gehen.}\label{fig:2_1_bewegungsablauf_gehen} 
\end{figure}

Beim Gehen besitzt der Bewegungsablauf (Abbildung~\ref{fig:2_1_bewegungsablauf_gehen}) eine Einzelstützphase (single-support phase), wenn ein Fuß den Boden berührt und eine Doppelstützphase (double-support phase), wenn beide Füße den Boden berühren. Die Einzelstützphase beginnt mit dem Abheben des Fußes und die Doppelstützphase beginnt mit dem Aufsetzen des selben Fußes, mit dem Schlüsselereignis Aufsetzen der Ferse (\ac{HS}). Die Dauer der Einzelstützphase ist in der Regel viermal länger als die der Doppelstützphase. 

Betrachten wir alternativ jedes Bein beim Gehen einzeln, besitzt es im Bewegungsablauf eine Stützphase (stance phase) und eine Schwungphase (swing phase) \citep[][]{Bartlett2007}. Beim normalen Gehen in einer selbstbestimmten und angenehmen Geschwindigkeit beträgt die Stützphase 60~\% und die Schwungphase 40~\% des Bewegungsablaufs. Für die vorliegende Arbeit besitzt die Schwungphase ein bedeutendes Schlüsselereignis, den mittleren Schwung (\ac{MS}). Der mittlere Schwung ist gekennzeichnet durch eine hohe kinematische Energie beim Gehen und Laufen \citep[][]{Novacheck1998}.
\begin{figure}
	[t] \centering 
	\includegraphics[width=1.00
	\textwidth]{2_2_bewegungsablauf_laufen.pdf} \caption[Phasen des menschlichen Bewegungsablaufs beim Laufen]{Phasen des menschlichen Bewegungsablaufs beim Laufen.}\label{fig:2_2_bewegungsablauf_laufen} 
\end{figure}

Eigenschaften der menschlichen Bewegung sind individuell und unterscheiden sich von Mensch zu Mensch. In der vorliegenden Arbeit benutze ich das Adjektiv normal, um auszudrücken, dass es sich um eine Bewegung handelt, die Menschen in der Regel ohne Einflüsse wie z.~B. Verletzungen ähnlich ausführen. Beim normalen Gehen setzt der Fuß mit der Ferse bzw.\ Rückfuß \ac{HS} auf \citep[][S.~33]{Marquardt2011}. Anders verhält es sich beim Laufen, bei dem der Läufer mit dem Vorder-, Mittel- oder Rückfuß aufsetzt. In der Literatur finden wir die Ausdrücke \ac{IC} und \ac{IS}, um die Berührung des Bodes durch den Fuß und das Lösen des Fußes vom Boden zu verallgemeinern.

Laufen unterscheidet sich vom Gehen, da keine Doppelstützphase vorhanden ist, in der beide Füße gleichzeitig den Boden berühren \citep[][S.~15~f.]{Bartlett2007}. Der Bewegungsablauf beim Laufen (Abbildung~\ref{fig:2_2_bewegungsablauf_laufen}) unterteilt sich in eine Stützphase (support phase), in der ein Fuß den Boden berührt und in eine Schwungphase, in der beide Füße keinen Bodenkontakt besitzen. Nur die Stützphase, vom Aufsetzen des Fußes (\ac{IC}) bis zum Abstoßen des Fußes (\ac{IS}), ermöglicht es dem Läufer, Kraft für die Fortbewegung auszuüben. \citet[][S.~17]{Bartlett2007} betrachtet die Schwungphase als Vorbereitung für das nächste Aufsetzen des Fußes und bezeichnet sie als recovery phase (Aufschwung, aber auch Wiederherstellung und Erholung). Beim Laufen oder Joggen ist die Dauer der Stützphase und der Schwungphase in der Regel gleich lang \citep[][S.~32~f.]{Marquardt2011}. Die Schwungphase verlängert sich mit höherer Laufgeschwindigkeit. Tabelle~\ref{tab:unterschiede_zwischen_laufen_und_gehen} fasst die Unterschiede zwischen Gehen und Laufen nochmals zusammen.
\begin{table}
	[t] \caption[Unterschiede zwischen Laufen und Gehen]{Unterschiede zwischen Laufen und Gehen nach \citet{Marquardt2011}}\label{tab:unterschiede_zwischen_laufen_und_gehen} 
	\begin{tabularx}
		{
		\textwidth}{*{3}{>{\RaggedRight\arraybackslash}X}} \toprule & Gehen & Laufen \\
		\midrule Phasen & Stütz- und Schwungphase & Stütz- und Schwungphase mit Flugphase \\
		Verhältnis Stütz-/ Schwungphase & 60/40 & 50/50 (und kleiner) \\
		Spurbreite & breiter & schmaler \\
		Stoßkräfte (Landung) & 1- bis 1,5-Faches des Körpergewichts & 2- bis 3-Faches des Körpergewichts \\
		Stoßkräfte (Abdruck) & 1- bis 1,5-Faches des Körpergewicht & 3,5- bis 5-Faches des Körpergewichts \\
		Fußaufsatz & immer mit der Ferse (Rückfuß) & mit dem Vorder-, Mittel-, oder Rückfuß \\
		\bottomrule 
	\end{tabularx}
\end{table}

% section gehen_und_laufen (end)
\section{Existierende Technologien beim Lauftraining} 

% (fold)
\label{sec:existierende_technologien_beim_lauftraining}

Die in Abschnitt~\ref{sec:gehen_und_laufen} angeführten Erkenntnisse über die biomechanischen Eigenschaften des Gehens und des Laufens lassen sich in Konzepte für Benutzerschnittstellen (user interfaces) für das Geh- und Lauftraining einsetzen. Eine Vielzahl wissenschaftlicher Arbeiten aus der Mensch-Computer-Interaktion beschäftigt sich mit der Hilfe der vorgestellten Erkenntnisse mit der Identifikation von Personen und der Fallerkennung von Menschen. Die beiden genannten Anwendungsfälle sind nicht Bestandteil der vorliegenden Arbeit. In dieser Arbeit beschäftige ich mich im Bereich der mobilen Mensch-Computer-Interaktion mit Technologien für das Lauftraining. \citet{Jensen2014} unterteilen die existierende Technologie beim Lauftraining in die nachfolgenden Gruppen:
\begin{itemize}
	\item traditionelle Benutzerschnittstellen (Abschnitt~\ref{sub:traditionelle_benutzerschnittstellen}) 
	\item Lauftechnik-Detektionstechnologie (Abschnitt~\ref{sub:lauftechnik_detektionstechnologie}) 
	\item assistierende Echtzeit-Benutzerschnittstellen (Abschnitt~\ref{sub:assistierende_echtzeit_benutzerschnittstellen}) 
\end{itemize}

\subsection{Traditionelle Benutzerschnittstellen} 

% (fold)
\label{sub:traditionelle_benutzerschnittstellen}

Eine Benutzerschnittstelle, die leistungsbezogene Merkmale des Lauftrainings wie Geschwindigkeit, Laufzeit und -distanz sowie Herzfrequenz darstellt, zählen \citet{Jensen2014} zu den traditionellen Benutzerschnittstellen. Als Beispiele sind Läuferuhren und Apps, die dem Läufer zur Anzeige leistungsbezogener Merkmale dienen, zu nennen.

Die ersten Läuferuhren erschienen im Jahr 2000. Zum Beispiel diente die Pellor 3D Sportuhr Running dem Zählen von Schritten und der Pulsmessung beim Laufen. Heute benutzen Laufuhren unterschiedliche Technologien wie \ac{GPS}-Empfänger, Herzfrequenzmesser und Beschleunigungsmesser. Die Informationen lesen Läufer entweder direkt von der Anzeige der Uhr während der Läufe ab oder sie lassen sich eine detaillierte Übersicht an einem PC zur Verfügung stellen. Zu den heutigen Laufuhren erhalten die Läufer Software bzw.\ Apps zur Übertragung und Darstellung der Informationen. Manche Laufuhren arbeiten mit akustischen und taktilen Signalen, um Läufer zu informieren, z.~B., dass sie ihren angestrebten Herzfrequenzbereich über- oder unterschreiten.

Entwickler von Lauf-Apps für Smartphones wie Runtastic, Runkeeper und Nike Running übernahmen die Konzepte der Laufuhren. Unter Verwendung der internen \ac{GPS}-Einheit und des Beschleunigungsmessers der Smartphones sowie von externen Herzfrequenzmessern stellen Lauf-Apps die gesammelten Informationen auf dem Smartphone-Bildschirm dar. Es entstanden spezielle Sportarmbänder zur Befestigung der Smartphones am Oberarm der Läufer. Sie ermöglichen dem Läufer während des Laufens auf den Bildschirm zu schauen, ohne das Smartphone z.~B. aus einer Tasche herausholen zu müssen. Viele Lauf-Apps arbeiten zusätzlich mit verbalen Rückmeldungen, die den Läufern z.~B. die aktuelle Laufzeit und -distanz ansagen. Die Entwickler der Lauf-Apps bieten in der Regel den Läufern zusätzlich an, die gesammelten Informationen auf ihre Server übertragen. Sie sind damit in der Lage, den Läufern eine detaillierte Übersicht ihrer Läufe auf einer Webseite darzustellen. Darüber hinaus bieten manche Laufuhren und Lauf-Apps Funktionen zur Trainingssteuerung an. Die Funktionen beinhalten die Gestaltung von Trainingsplänen und -zielen.

% subsection traditionelle_benutzerschnittstellen (end)
\subsection{Lauftechnik-Detektionstechnologie} 

% (fold)
\label{sub:lauftechnik_detektionstechnologie}

Lauftechnik-Detektionstechnologie besteht für \citet{Jensen2014} aus Handhelds und tragbaren Trägheitsmesseinheiten oder kamerabasierenden Bewegungserkennungssystemen. In der vorliegenden Arbeit liegt der Fokus auf Handhelds und tragbaren Trägheitsmesseinheiten, da wir in der Lage sind sie nahezu uneingeschränkt außerhalb von Laboratorien einzusetzen. Abhängig von Energiebedarf und Energiequelle lassen sich beide kabellos über längere Zeiträume betreiben. Sie sind beide in der Lage kinematische Daten während des Laufens zu messen und zu sammeln. Kinematische Daten bilden die Grundlage zur Berechnung kinematischer Merkmale des Laufens. Zu den Merkmalen gehören z.~B. die Schrittfrequenz, die Bodenkontaktzeit (ground contact time), die vertikale Bewegung des Läufers und dessen Fußaufsatztyp (foot strike type).

\citet{Harms2010} entwickelten mit ETHOS einen 2,5 $cm^{2}$ großen tragbaren Prototypen, der einen Beschleunigungsmesser, ein Kreiselinstrument und ein Magnetometer besitzt. Eine ETHOS Einheit ist in der Lage über einen längeren Zeitraum kinematische Daten zu sammeln und für eine nachträgliche Analyse bereitzustellen.

\citet{Strohrmann2011} nutzten gleichzeitig 12 ETHOS Einheiten an verschiedenen Körperpositionen bei einem standardisierten Testlauf mit 12 Teilnehmern, die ein unterschiedliches Laufleistungsniveau besaßen. Das Leistungsniveau wurde anhand der wöchentlich gelaufenen Kilometer definiert. Die Autoren berechneten aus den kinematischen Daten von den Füßen die Bodenkontaktzeit und aus den kinematischen Daten von der Hüfte die vertikale Bewegung des Läufers. Die beiden Merkmale ermöglichten den Autoren, die 12 Teilnehmer zu unterscheiden und ihrem Laufleistungsniveau zu zuordnen. Die Autoren zeigten, dass der Einsatz der tragbaren Trägheitsmesseinheiten ohne Beeinträchtigung des Laufens außerhalb von Laboratorien praktikabel ist \citep[][]{Strohrmann2011a}.

In einer nachfolgenden Studie berechneten \citet{Strohrmann2012} gleich zehn kinematische Merkmale, um Ermüdung beim Laufen zu identifizieren. Zum Einsatz kamen jeweils 12 ETHOS Einheiten bei 21 Läufern. Gelaufen wurde in der ersten Studie 45 Minuten auf einem Laufband. In der zweiten Studie liefen die Läufer im Freien auf einem Rundkurs. Die Autoren kommen aufgrund der beiden Studien zu dem Schluss, dass durch die \emph{Individualität} jedes Läufers kein \emph{generalisierbarer Laufprototyp} anhand der kinematischen Merkmale abzuleiten ist. Sie sehen den deutlichen Nutzen von tragbaren Technologien beim Laufen in wiederholenden quantitativen und objektiven Messungen der Lauftechnik. Die Messungen ermöglichen Einblicke in den Zusammenhang von Laufkinematik, Verletzungsrisiko, Ermüdung und Laufökonomie. Mit Blick auf die Vorbereitungszeit für jeden Lauf argumentieren die Autoren für den Einsatz von einer geringeren Anzahl von Messeinheiten. Darum benennen sie Fuß und Rumpf zu den bedeutsamsten Körperpositionen zur Berechnung von kinematischen Merkmalen.

Das von \citet{Eskofier2013} vorgestellte Klassifizierungssystem hat das Ziel, Menschen den geeigneten Laufschuh für den jeweiligen Fußaufsatztypen zu empfehlen. Hierzu ermittelte das System anhand der Klassifizierung der Beschleunigungsdaten einer Trägheitsmesseinheit am Fuß den Fußaufsatztyp nach einem Lauf. Ein Beispiel der gleichen Forschungsgruppe zeigt die Nutzung von zwei Trägheitsmesseinheiten zur Bestimmung der kinematischen Merkmale Kniestreckwinkel und Kniebeugewinkel \citep[][]{Jakob2013}. Beide Winkel berechnen die Autoren anhand von kinematischen Daten vom Unter- und Oberschenkel. Sie vergleichen die vom Algorithmus berechneten Winkel mit einem Kamerasystem. Durch vergleichbare Ergebnisse und die Nutzungsmöglichkeit außerhalb von Laboratorien sehen die Autoren den Einsatz ihrer Technologie im Training und in Wettkampf zur Verbesserung der Leistungsfähigkeit durch objektive Rückmeldung.

% subsection lauftechnik_detektionstechnologie (end)
\subsection{Assistierende Echtzeit-Benutzerschnittstellen} 

% (fold)
\label{sub:assistierende_echtzeit_benutzerschnittstellen}

Die Beispiele aus Abschnitt~\ref{sub:lauftechnik_detektionstechnologie} haben gegenüber den traditionellen Benutzerschnittstellen das überwiegende Ziel, die Laufökonomie des Läufers zu steigern. Die Bestimmung und Rückmeldung der kinematischen Merkmale erfolgt nach dem Laufen. Assistierende Echtzeit-Benutzerschnittstellen nach \citet{Jensen2014} wie z.~B. die Laufuhr Garmin Forerunner 620 geben Rückmeldung auf die Lauftechnik. Die Laufuhr Garmin Forerunner 620 stellt z.~B. dem Läufer während des Laufens die Bodenkontaktzeit, die Kadenz und die vertikale Bewegung als nummerischen Wert auf dem Bildschirm dar. Es ist schwer, die Lauftechnik anhand von visuell dargestellten Informationen zu ändern. Die kontinuierliche Informationsaufnahme, verbunden mit dem Blick auf die Uhr, trägt im ungünstigsten Fall zu einem unökologischen Laufstil bei \citep[][]{Jensen2014}. Dieses Beispiel zeigt, dass der visuelle Kanal nicht in jedem Anwendungsfall die geeignetste Darstellungsform für Informationen ist. 

\citet[][]{Zhao2007} merken dazu an, dass es Situationen gibt, in denen Benutzer ihre Aufmerksamkeit auf die Umgebung richten müssen und dass das Betrachten einer visuellen Schnittstelle ablenkend oder in manchen Fällen gefährlich ist. Das Laufen ist eine der mobilen Situationen, die die Aufmerksamkeit des Läufers in einem bestimmten Maße erfordert. \citet[][]{Jensen2014} sehen in Bezug auf Lauftechnologie einen Bedarf an alternativen Methoden der Rückmeldungen, die sich von den konventionellen bildschirmbasierenden Schnittstellen unterscheiden. Die nachfolgenden Arbeiten stellen unterschittliche Lösungen für alternative Methoden der Rückmeldung beim Lauftraining vor. 

\citet[][]{Wijnalda2005} präsentierten ein System, das anhand der Auswahl von Musikstücken oder deren Modifikation Echtzeit-Rückmeldung auf die Schrittfrequenz gibt. Die Läufer wählen aus drei verschiedenen Trainingsmodi aus. Der Pace-fixing-Modus dient dem Ausdauertraining und spielt Musikstücke mit gleichem Tempo ab. Der Pace-matching-Modus passt die Musik der Schrittfrequenz an und im Pace-influencing-Modus besitzt das Tempo der Musik die Aufgabe, die Schrittfrequenz zu beeinflussen. Das System mit dem Namen IM4Sports besteht aus einem tragbaren Musikspieler, einem Brustgurt zur Herzfrequenzmessung und einem Pedometer. Die Herzfrequenzmessung besitzt keinen Einfluss auf die Auswahl der Musikstücke. Die Autoren kamen zu dem Schluss, dass Läufer eine Adaptionszeit von mindestens 20 Sekunden benötigen, um eine Synchronisation von Schrittfrequenz und Musiktempo im Pace-influencing-Modus herbeizuführen.

Ein ähnliches Konzept wie bei IM4Sports wurde von \citet[][]{Oliver2006} mit MP Train verfolgt. Das mobiltelefon-basierende System wählt anhand der Schrittfrequenz geeignete Musikstücke aus. Es passt die Auswahl mit Hilfe eines Brustgurts zur Herzfrequenzmessung an eine vordefinierte Trainingsintensität an. Unterschreitet die Herzfrequenz des Läufers der vordefinierten Herzfrequenz, wählt das System Musikstücke mit einem höheren Tempo aus. Andersherum wählt das System Musikstücke eines langsameren Tempos aus, wenn die Herzfrequenz des Läufers die vordefinierte Herzfrequenz überschreitet. MP Train wurde in einer zweiten Iteration durch \citet[][]{DeOliveira2008} zu Triplebeat und um Funktionen wie z.~B. einer Herausforderungsfunktion erweitert. Hinweise aus der Literatur \citep[][]{Bood2013} bestätigen den Einfluss eines Musikstücks oder eines Metronoms auf die Schrittfrequenz. \citet[][]{Bood2013} schließen aus ihrer Studie, dass ein motivierendes Musikstück mit einem angepassten Tempo zur Schrittfrequenz einen positiven Effekt auf die auditorisch-motorische Synchronisation besitzt und die Laufökonomie verbessert.

\citet[][]{Takata2007} bilden den Laufprozess in einem Zustandsübergangsdiagramm mit den Zuständen Aufwärmen, Haupttraining und Abwärmen ab. Ihr tragbares System ist in der Lage, anhand der Korrelation von voreingestellten Werten für jeden Zustand und den aktuellen Messwerten der Herzfrequenz oder der Schrittfrequenz den aktuellen Zustand des Läufers während des Laufens zu identifizieren. Die Autoren geben zu bedenken, dass sich die voreingestellten Werte, aufgrund von verschieden Faktoren wie dem Alter oder der körperlichen Verfassung des Läufers, individuell unterscheiden. Das System ermöglicht, die Voreinstellung von vorherigen Trainingseinheiten abzuleiten. Im gleichen System integrierten die Autoren eine Funktion, um den Laufkurs zu erstellen. Anhand des gewünschten Kalorienverbrauchs und Eigenschaften des Läufers wie z.~B. Leistungslevel, Alter, Gewicht, erstellt das System einen individuellen Laufkurs. Das System verlängert den Laufkurs in Echtzeit, wenn es merkt, dass der Läufer das Ziel wegen zu geringen Kalorienverbrauchs durch z.~B. langsames Laufen, nicht erreicht. Anders herum verkürzt es den Laufkurs, wenn es einen zu hohen Kalorienverbrauch feststellt.

Das von \citet[][]{Eriksson2010} vorgestellte System, bestehend aus einem mobilen Telefon und einem am Rücken befestigten Beschleunigungsmesser, gibt akustische Rückmeldungen auf die Schrittfrequenz und die vertikale Bewegung des Läufers. Die vertikale Bewegung bezeichnet die Auf- und Abbewegung des Läufers. Eine geringe vertikale Bewegung erhöht die Laufökonomie. Anhand der Vorgabe des Läufers passt das System ein getaktetes Audiosignal an, damit der Läufer die Schrittfrequenz erhöht oder verlangsamt. Das System gibt ein akustisches Warnsignal aus, wenn der Läufer eine vordefinierte vertikale Bewegung überschreitet. Das System ist in der Lage nach vier Sekunden einen Durchschnitt der vertikalen Bewegung oder nach jedem Schritt die vertikale Bewegung zu berechnen und eine Rückmeldung zu geben, falls der Läufer den vordefinierten Wert überschreitet. Ein Versuch mit einer Testperson zeigte, dass sie in beiden Modalitäten ihre vertikale Bewegung reduzierte und sich die maximale vertikale Bewegung der Vorgabe näherte.

Die symmetrische Linie des Körpers zur Überprüfung der Armbewegung beim Laufen nutzen \citet{Strohrmann2013, Strohrmann2014} in einer Smartphone App. Die App signalisiert dem Läufer über Vibration, wenn die Armbewegung nicht parallel zum Körper verläuft. In einer Vorstudie mit zehn Teilnehmern trainierten \citet{Strohrmann2013} einen Algorithmus zur Klassifizierung der drei Haltungsfälle: Arme laufen parallel zum Körper (ökonomischste Haltung), Arme zielen zur symmetrischen Linie und Arme kreuzen die symmetrische Linie. Die Autoren identifizierten den Oberarm als geeignetste Körperposition für die Klassifizierung. Der Vorteil des Oberarms ist zusätzlich, dass Läufer ihn im Regelfall nutzen, um ein Smartphone in einem Sportarmband zu tragen. In der Vorstudie nutzen die Autoren ETHOS Einheiten. In der Hauptstudie mit 20 Teilnehmern kam das Smartphone mit dessen internen Beschleunigungsmesser und Kreiselinstrument zum Einsatz. Die Studie diente dem Vergleich der App Rückmeldung und einer verbalen Rückmeldung durch einen Übungsleiter. In beiden Formen der Rückmeldung zeigen die Ergebnisse eine vergleichbare signifikante Verbesserung der Armhaltung. Die Autoren sehen den Vorteil der App darin, dass Läufer ohne Zugang zu einem persönlichen Trainer in der Lage sind, mit der App ihre Lauftechnik zu verbessern. Anhand eines Fragebogens schließen die Autoren zusätzlich auf eine hohe Akzeptanz ihres Konzepts.

% subsection assistierende_echtzeit_benutzerschnittstellen (end)
% section existierende_technologien_beim_lauftraining (end)
\section{Gestaltungsraum für Technologien für das Laufen} 

% (fold)
\label{sec:gestaltungsraum_fur_technologien_fur_das_laufen}

Die in Abschnitt~\ref{sec:existierende_technologien_beim_lauftraining} vorgestellten Technologien ergeben nach \citet[][]{Jensen2014} den in Abbildung~\ref{fig:2_3_gestaltungsraum_und_entwicklungen} dargestellten zwei-dimensionalen Gestaltungsraum. Auf der Achse Rückmeldung liegen auf der einen Hälfte die darstellenden Rückmeldungen und auf der anderen Hälfte die assistierenden Rückmeldungen. Die Achse Fokus repräsentiert die leistungsbezogenen Technologien auf der einen Seite und die technikbezogenen Technologien auf der anderen Seite.

Das Ziel der vorliegenden Arbeit ist es in erster Linie nicht, die Laufleistung oder die Laufökonomie zu optimieren, sondern die Voraussetzungen für das Erleben von Flow beim Laufen zu verbessern. Dabei besteht ein innerer Zusammenhang von Laufleistung, Laufökonomie und Flow-Erleben bzw.\ Wohlbefinden. Hinsichtlich des Flow-Konstrukts (Kapitel~\ref{cha:flow_erleben_messen}) sind Laufleistung und Laufökonomie keine Gegensätze. Aus diesem Grund ist eine rein leistungs- und technikbezogene Einteilung des Gestaltungsraums für Lauftechnologie für die vorliegende Arbeit nicht ausreichend.

\begin{figure}
	[t] \centering 
	\includegraphics[width=1.00
	\textwidth]{2_3_gestaltungsraum_und_entwicklungen.pdf} \caption[Gestaltungsraum für Technologien für das Laufen]{Gestaltungsraum für Technologien für das Laufen. Quelle: \citet[][]{Jensen2014}. Leicht modifiziert und übersetzt}\label{fig:2_3_gestaltungsraum_und_entwicklungen} 
\end{figure}

Die reine Quantifizierung gemessener Tätigkeiten besitzt einen negativen Einfluss auf die intrinsische Motivation \citep[][]{Etkin2016}. Eine ausschließlich durch Messwerte ausgedrückte Sicht auf die Tätigkeit trägt zu einer verminderten Hingabe für die Tätigkeit und verminderten Freude bzw.\ zu subjektiven Wohlbefinden bei der Ausführung der Tätigkeit bei. Aus diesem Grund konzentrieren sich z.~B. \citet{Hajinejad2015} in ihrem Gestaltungsansatz darauf, die Erlebensqualität der Gehaktivität in den Vordergrund zu stellen.

In der vorliegenden Arbeit schlage ich vor, den vorgestellten Gestaltungsraum um den Faktor Wohlbefinden auf der Fokus-Achse zu erweitern. Die beiden Faktoren auf der Achse Rückmeldung lassen auf die beiden Möglichkeit der Objektivierung des Flow-Erlebens beziehen. Darstellende Rückmeldung beziehen sich überwiegend auf zustandorientierte Messungen, wohingehen assistierenden Rückmeldungen den Prozess unterstützen. Auf dieser Grundlage motiviere ich assistierenden Echtzeit-Benutzerschnittstellen, die das Flow-Erleben im Prozess des Laufens und Gehens implizit messen. Damit soll die vorliegenden Arbeit zur Bildung eines Fundaments beizutragen, das die Voraussetzungen des Läufers durch Echtzeitrückmeldungen verbessert, Flow zu erleben --- also Lauf-Apps zu entwickeln, deren Ziel es ist, Läufer in einen individuellen optimalen Zustand zu bringen und das Wohlbefinden während des Laufens zu fördern.

% section gestaltungsraum_fur_technologien_fur_das_laufen (end)
\section{Zusammenfassung} 

% (fold)
\label{sec:zusammenfassung}

In diesem Kapitel führe ich in die Forschung im Bereich der Lauftechnologie ein. Zu Beginn stelle ich die Motivation der Computernutzung im Sport vor und betone damit die Allgegenwärtigkeit von Computern. In der vorliegenden Arbeit lege ich das Augenmerk auf Apps die auf mobilen Geräten laufen, die der Benutzer in der Regel in einer Hand hält (Handhelds) oder am Körper trägt (Wearables).

Ihre Sensorik ermöglicht es kinematische Daten zu sammeln und Schlüsselereignisse beim Gehen und Laufen zu identifizieren. Aus diesem Grund erörtere ich die Schlüsselereignisse \ac{HS} bzw.\ \ac{IC}, \ac{TO} bzw.\ \ac{IS} und \ac{MS} der Biomechanik des menschlichen Gangs beim Gehen und Laufen. Das wichtiges Schlüsselereignis für die vorliegende Arbeit ist der \acf{MS}.

Die Untersuchung von \citet{Strohrmann2012} zeigt, dass kinematischen Merkmale mit einer hohen Individualität des Läufers verbunden sind. Diese Erkenntnis unterstreicht, wie beim Flow-Erleben, den Wert von intraindividuellen Untersuchungen.

Laufleistungs- und lauftechnikbezogene Merkmale spielen eine bedeutende Rolle in diversen Forschungsarbeiten der Mensch-Computer-Interaktion beim Laufen. Der vorgestellte Gestaltungsraum (Abbildung~\ref{fig:2_3_gestaltungsraum_und_entwicklungen}) für Lauftrainingstechnologie zeigt zusammenfassend, für welche Anwendungsfälle computerbasierende Systeme vorhanden sind. Flow-Erleben und die in Zusammenhang stehenden Gemüts- und Gefühlszustände Freude und Wohlbefinden finden in dem Gestaltungsraum keine Berücksichtigung. Aus dem Grund schlage ich in diesem eine Erweiterung des Gestaltungsraums um den Faktor Wohlbefinden vor.

In erster Linie sehe ich die vorgestellten assistierenden Echtzeit-Benutzerschnittstellen als Anknüpfungspunkt für Applikationen, die die Voraussetzungen verbessern, Flow während Laufen zu erleben. Es fehlt uns für die Realisierung solcher Applikationen ein implizites Messverfahren, um einen Bewusstseinszustand wie Flow-Erleben im Prozess des Laufens zu erfassen. Demzufolge betrachte ich im nachfolgenden Kapitel das Flow-Erleben, bestehende explizite Messverfahren, prozessorientierte Ansätze mit expliziten Messverfahren und Lösungsansätze zur impliziten Messung des Flow-Erlebens.

% section zusammenfassung (end)
% chapter technologie_beim_laufen (end)

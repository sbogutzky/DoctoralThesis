

%!TEX root = /Users/sbogutzky/Entwicklung/projects/bogutzky/repositories/2939413/final-draft.tex
\section{Explizite Messverfahren des Flow-Erlebens} 

% (fold)
\label{sec:explizite_messverfahren_des_flow_erlebens}

Subjektive, nachträgliche Auskünfte über das Erleben durch strukturierte Interviews oder durch standardisierte psychometrische Skalen gehören bislang zu den erprobtesten und zuverlässigen Flow-Messverfahren. Als Grundlage der expliziten Messverfahren dienen die in Tabelle~\ref{tab:merkmale_eines_flow_zustandes} und Tabelle~\ref{tab:voraussetzungen_fuer_einen_flow_zustand} vorgestellten Merkmale und Voraussetzungen.

\subsection{Strukturierte Interviews und Fragebögen} 

% (fold)
\label{sub:strukturierte_interviews_und_fragebogen}

Strukturierte Interviews und Fragebögen fragen das Vorhandensein von Flow-Voraussetzungen und -Merkmalen ab. Die Erhebung mit Hilfe von strukturierten Interviews oder Fragebögen findet in der Regel im Anschluss an die Tätigkeit (Anforderungssituation) statt. Die Untersuchenden sind in der Lage mit einer solchen Erhebung festzustellen, ob die befragte Person bei der Durchführung der jeweiligen Tätigkeit Flow erlebte.

Für ein Interview und für die Beantwortung eines Fragebogens gilt in der Regel: Umso länger die Zeitspanne zwischen der jeweiligen Tätigkeit und der Befragungsmethode ist, desto größer ist die Gefahr, dass die befragte Person Einzelheiten vergisst oder sich Ungenauigkeiten bei der Reflexion einschleichen \citep[][S.~87]{Henk2014}. Die Aufzeichnung der jeweiligen Tätigkeit durch eine Videokamera und das Durchgehen der Aufzeichnung mit der befragten Person erhöht die Präzision der Reflexion \citep[Video-Recall, ][S.~566]{Leuchter2006}.

Fragebögen mit offenen Fragen stellen die eingängigste Form der Fragebögen dar. Sie führen zu einer qualitativen Datenerhebung. Die Auswertung qualitativer Daten ist in der Regel schwierig und zeitaufwendig. Ein Mittel, die Auswertung zu vereinfachen und die Ergebnisse zu vereinheitlichen, ist die Strukturierung der Fragen und die Einführung von psychometrischen Skalen.

% subsection strukturierte_interviews_und_fragebogen (end)
\subsection{Psychometrische Skalen} 

% (fold)
\label{sub:psychometrische_skalen}

Eine psychometrische Skala besteht aus mehreren Items, die die Befragten auf einem ordinalskalierten Maß (z.~B. \emph{trifft nicht zu} = 1 bis \emph{trifft zu} = 5) bewerten. Anhand der bewerteten Items lassen sich die persönlicher Einstellungen oder das Befinden der Befragten messen. In den nachfolgenden Abschnitten stelle ich die \ac{FKS} \citep{Rheinberg2003}, die \ac{FSS} \citep{Jackson1996} und den Fragebogen von \citet{Keller2008} vor. Die genannten drei psychometrischen Skalen kamen überwiegend in Forschungsarbeiten zu Lösungsansätzen zur Identifikation von impliziten Merkmalen des Flow-Erlebens zum Einsatz (Abschnitt~\ref{sec:losungsansatze_zur_impliziten_messung_von_flow_erleben}).

% TODO: GEQ aufnehmen

\subsubsection{Flow-Kurzskala} 

% (fold)
\label{ssub:flow_kurzskala}

Die \ac{FKS} besteht aus insgesamt 16 Items (Anhang~\ref{sec:flow_kurzskala}). Die ersten zehn Items bilden anhand einer 7-Punkte-Likert-Skala (\emph{trifft nicht zu} = 1 bis \emph{trifft zu} = 7) die Komponenten des Flow-Erlebens ab (Tabelle~\ref{tab:komponenten_des_flow_erlebens}). Diese zehn Items fassen \citet{Rheinberg2003} zum Generalfaktor des Flow-Erlebens zusammen. Zur Differenzierung des Flow-Konstrukts ist der Generalfaktor der \ac{FKS} in zwei Faktoren (Unterdimensionen) unterteilt. Faktor I umfasst sechs Items, die Aussagen zum \emph{glatten automatisierten Verlauf} einer Tätigkeit zusammenfassen. Faktor II beinhaltet vier Items, die mit \emph{Absorbiertheit} in Zusammenhang stehen. Der Reliabilitätskoeffizient der zehn Items des Generalfaktors (Cronbachs~$\alpha$) lag nach Angaben von \citet[S.~9]{Rheinberg2003} bei einer \ac{ESM}-Studie (Abschnitt~\ref{sub:experience_sampling_method}) mit knapp 900 Messungen, einer Statistik-Stichprobe ($N = 123$; Engeser, in Vorb.) und einer Vorlesungsstichprobe ($N = 63$) im Bereich um $\alpha = 0{,}90$. \citet{Rheinberg2003} erweiterten die \ac{FKS} um eine Besorgniskomponente, da sie davon ausgehen, dass bei der Durchführung einer Tätigkeit bzw. in einer Anforderungssituation nicht ausschließlich Flow entsteht. Die Besorgniskomponente besteht aus drei Items (Nr. 11 bis Nr. 13, Cronbachs $\alpha = 0{,}80$ bis $\alpha = 0{,}90$). Das Ende der \ac{FKS} fragt die persönlich erlebte Anforderung der Tätigkeit und dessen Gleichgewicht mit den eigenen Fähigkeiten ab. Dieser Teil der \ac{FKS} besteht aus drei Items mit jeweils einer 9-Punkte-Likert-Skala. Das Item 14 fokussiert sich auf einen Vergleich der Schwierigkeit der jetzigen Tätigkeit mit allen anderen Tätigkeiten (leicht vs. schwer) und das Item 15 auf die eigene Leistungsfähigkeit (niedrig vs. hoch). Das Item 16 fragt direkt, auf die aktuelle Tätigkeit bezogen, nach der subjektiv wahrgenommenen \ac{AFP} (zu gering vs. zu hoch). 
\begin{itemize}
	
	\item In der vorliegenden Arbeit ist die \ac{AFP} eine Bewertung des Gleichgewichts zwischen den Anforderungen der Tätigkeit und den eigenen Fähigkeiten.
\end{itemize}

% subsubsection flow_kurzskala (end)
\subsubsection{Flow-State-Scale} 

% (fold)
\label{ssub:flow_state_scale}

Die \ac{FSS} dient der Ermittlung des Flow-Zustand bei sportlichen Aktivitäten. Sie enthält neun Dimensionen mit jeweils vier Items auf einer 5-Punkte-Likert-Skala (\emph{strongly disagree} = 1 bis \emph{strongly agree} = 5, Anhang~\ref{sec:flow_state_scale}). Die Dimensionen der Skala repräsentieren die acht von \citet[S. 73-101]{Csikszentmihalyi1992} diskutierten Komponenten der Freude: \emph{Herausfordernde Aktivität, für die man besondere Geschicklichkeit braucht}, \emph{Der Zusammenfluss von Handeln und Bewusstsein}, \emph{Klare Ziele und Rückmeldung}, \emph{Konzentration auf die anstehende Aufgabe}, \emph{Das Paradox der Kontrolle}, \emph{Der Verlust des Selbstgefühls}, \emph{Die Veränderung der Zeit} und \emph{Die autotelische Erfahrung}. Sie fungieren im weiteren Verlauf als neun Dimensionen des Flow-Erlebens, da klare Ziele und eindeutige Rückmeldungen zwei von einander getrennte Dimension darstellen. In einer Studie mit 394 Athleten, die sich im Nachhinein in einen optimalen Moment während ihrer sportlichen Tätigkeit hineinversetzen sollten, besaß die \ac{FSS} in ihrer originalen Form einen Cronbachs~$\alpha$ von 0,83. Auf ihrer Grundlage entstanden eine revidierte Version (FSS-2, 36 Items, 9 Dimensionen) und eine gekürzte Version \citep[short FSS-2, 9 Items,][]{Jackson2002, Jackson2008}. Zusätzlich gibt es zu jeder Version, eine Dispositional-Version (\acs{DFS} und DFS-2, jeweils 36 Items und short DFS 9 Items). Die Dispositional-Versionen sind im Wortlaut und Zeitform geändert und dienen zur Ermittlung einer generellen Tendenz der Befragten Flow in ihrem Sport zu erleben \citep[][S.~356]{Jackson1998}.

% subsubsection flow_state_scale (end)
\subsubsection{Keller und Bless Fragebogen} 

% (fold)
\label{ssub:keller_und_bless_fragebogen}

Der Fragebogen von Keller und Bless fragt mit mehreren Items auf einer 7-Punkte-Likert-Skala (\emph{trifft nicht zu} = 1 bis \emph{trifft zu} = 7) die nachfolgenden Dimensionen ab: \emph{Gefühl der Kontrolle}, \emph{Involvierung und Vergnügen}, \emph{wahrgenommene Passung von Fähigkeiten und Aufgabenanforderungen}. Zusätzlich misst der Fragebogen das Zeitgefühl auf einer 10~cm langen Linie. Die erste Dimension \emph{Gefühl der Kontrolle} umfasst zehn Items, die Aussagen zur Kontrolle über die Ergebnisse einer Tätigkeit zusammenfassen. Die Reliabilität der zehn Items lag bei einer Befragung mit 72 Teilnehmern unmittelbar im Anschluss eines Computerspiels bei einem Cronbachs~$\alpha$ von 0,93. Die 14 Items der Dimension \emph{Involvierung und Vergnügen} fassen die Involvierung und das Vergnügen in und an einer Tätigkeit zusammen. Deren Cronbachs~$\alpha$ lag bei 0,95. Dimension 3 beinhaltet ein Item, das das Gleichgewicht von Anforderungen der Tätigkeit und eigenen Fähigkeiten abfragt. Keller und Bless setzten ihren Fragebogen in zwei Studien ein \citep{Keller2008}. In der zweiten Studie ersetzten sie die Items der Dimension \emph{Gefühl der Kontrolle} durch Items, die das Ausmaß einer deutlichen Affektreaktion (z.~B. Unruhe) der befragten Person in Erfahrung bringen. Diese neun Items besaßen in der zweiten Studie bei der Befragung mit 149 Teilnehmern unmittelbar im Anschluss eines Computerspiels einen Reliabilitätskoeffizienten Cronbachs~$\alpha$ von 0,88. Der Cronbachs~$\alpha$ für die Dimension \emph{Involvierung und Vergnügen} lag wie bei der ersten Studie bei 0,95.

% subsubsection keller_und_bless_fragebogen (end)
% subsection psychometrische_skalen (end)
% section explizite_messverfahren_des_flow_erlebens (end)

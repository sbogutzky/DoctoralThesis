

%!TEX root = /Users/sbogutzky/Entwicklung/projects/bogutzky/repositories/2939413/final-draft.tex
\section{Was ist Flow?} 

% (fold)
\label{sec:was_ist_flow}

\citet[S.~58f.]{Csikszentmihalyi2010} definiert das Flow-Leben als das holistische Gefühl bei völligem Aufgehen in einer Tätigkeit. Seither sind unterschiedliche Definitionen des Flow-Erlebens entstanden, die sich in ihren Merkmalen nicht wesentlich unterscheiden. Zum Beispiel definieren \citet[][S.~263]{Rheinberg2003} Flow als den Zustand des reflexionsfreien, gänzlichen Aufgehens in einer glatt laufenden Tätigkeit. \citep[S.~153ff.]{Rheinberg2008} beschreibt das Flow-Erleben mit sechs Komponenten, die die acht Merkmale des Flow-Erlebens nach \citet[S.~108ff.]{Csikszentmihalyi2010} zusammenfassen. In der vorliegenden Arbeit stelle ich die deutschsprachige Beschreibung nach \citep[]{Rheinberg2008} beispielhaft dar (Tabelle~\ref{tab:komponenten_des_flow_erlebens}). 
\begin{table}
	[!htb] \caption[Komponenten des Flow-Erlebens.]{Komponenten des Flow-Erlebens (zusammengefasst nach \citet{Csikszentmihalyi2010}; \citep[S.~153ff]{Rheinberg2008}).} \label{tab:komponenten_des_flow_erlebens} 
	\begin{tabularx}
		{ 
		\textwidth}{X} \midrule 1. Passung zwischen Fähigkeit und Anforderung. Man fühlt sich optimal beansprucht und hat trotz hoher Anforderungen das sichere Gefühl, das Geschehen noch unter Kontrolle zu haben. \\
		2. Handlungsanforderungen und Rückmeldungen werden als klar und interpretationsfrei erlebt, so dass man jederzeit und ohne nachzudenken weiß, was jetzt richtig zu tun ist. \\
		3. Der Handlungsablauf wird als glatt erlebt. Ein Schritt geht flüssig in den nächsten über, als liefe das Geschehen gleitend wie aus einer inneren Logik. (Aus dieser Komponente rührt wohl die Bezeichnung »Flow«.) \\
		4. Man muss sich nicht willentlich konzentrieren, vielmehr kommt die Konzentration wie von selbst, ganz so wie die Atmung. Es kommt zum Ausblenden aller Kognitionen, die nicht unmittelbar auf die jetzige Ausführungsregulation gerichtet sind. \\
		5. Das Zeiterleben ist stark beeinträchtigt; man vergisst die Zeit und weiß nicht, wie lange man schon dabei ist. Stunden vergehen wie Minuten. \\
		6. Man erlebt sich selbst nicht mehr abgehoben von der Tätigkeit, man geht vielmehr gänzlich in der eigenen Aktivität auf (sog. »Verschmelzen« von Selbst und Tätigkeit). Es kommt zum Verlust von Reflexivität und Selbstbewusstheit. \\
		\bottomrule 
	\end{tabularx}
\end{table}
\begin{itemize}
	
	\item Ich nutze die Definition von \citet{Henk2014}, da er Merkmale (Tabelle~\ref{tab:merkmale_eines_flow_zustandes}) und Voraussetzungen (Tabelle~\ref{tab:voraussetzungen_fuer_einen_flow_zustand}) nicht miteinander vermischt. Er definiert Flow-Erleben als ein Verschmelzen von Handlung und Bewusstsein, das sich durch das gleichzeitige Erleben des Handlungsverlaufs als glatt und fließend und das gänzliche Aufgehen in der Tätigkeit auszeichnet. 
\end{itemize}

Diese Trennung macht für die vorliegende Arbeit Sinn, da ich davon ausgehe, dass die beiden Kernmerkmale: das \emph{gänzliche Aufgehen in der Tätigkeit} und den \emph{glatten Handlungsverlauf} implizit messen kann. Die dargestellten Voraussetzung sind hingehen herzustellen und ihr vorhanden sein, kann ich in der Regel nur explixit nachweisen. 
\begin{table}
	[!htb] \caption[Merkmale eines Flow-Zustands]{Merkmale eines Flow-Zustands nach \citet{Henk2014}.} \label{tab:merkmale_eines_flow_zustandes} 
	\begin{tabularx}
		{ 
		\textwidth}{*{2}{>{\RaggedRight\arraybackslash}X}} \toprule Kernmerkmale & weitere Merkmale \\
		\midrule Verschmelzen von Handlung und Bewusstsein: & Zentrierung der Aufmerksamkeit \\
		- gänzliches Aufgehen in der Tätigkeit & Selbstvergessenheit \\
		- glatter Handlungsverlauf & Verlust des Zeitgefühls \\
		& keine Besorgtheit über Misserfolg \\
		\bottomrule 
	\end{tabularx}
\end{table}
\begin{table}
	[!htb] \caption[Voraussetzungen für einen Flow-Zustand]{Voraussetzungen für einen Flow-Zustand nach \citet{Henk2014}.} \label{tab:voraussetzungen_fuer_einen_flow_zustand} 
	\begin{tabularx}
		{ 
		\textwidth}{*{1}{>{\RaggedRight\arraybackslash}X}} \toprule Voraussetzungen \\
		\midrule Gleichgewicht zwischen Anforderungen der Tätigkeit und eigenen Fähigkeiten \\
		klare Handlungsschritte und -ziele \\
		unmittelbare, eindeutige Rückmeldungen \\
		\bottomrule 
	\end{tabularx}
\end{table}

Nach \citet[]{Csikszentmihalyi2010} ist das Gleichgewicht zwischen Anforderungen der Tätigkeit und den Fähigkeiten des Handelnden die wichtigste Voraussetzung für das Eintreten eines Flow-Zustands. Aus diesem Grund stützen sich mehrere Modelle auf diesen Zusammenhang. \citet[S.~75]{Csikszentmihalyi2010} entwickelte auf der Grundlage von Befragungen nach Tätigkeiten wie Schach spielen, Klettern im Felsen, Rock-Tanzen und der Chirurgie das erste und bisher eingängigste Modell der Flow-Theorie. 
\begin{figure}
	[!htb] \centering 
	\includegraphics[width=1.00 
	\textwidth]{kanalmodell.pdf} \caption[Das Kanalmodell des Flow-Erlebens]{Das Kanalmodell des Flow-Erlebens nach \citet[S.~75]{Csikszentmihalyi2010}} \label{fig:kanalmodell} 
\end{figure}

Das Kanalmodell beschreibt das Flow-Erleben anhand des Gleichgewichts zwischen Anforderungen der Tätigkeit und den Fähigkeiten des Handelnden. Es besagt, dass sobald die Anforderungen und Fähigkeiten auf demselben Niveau liegen, der Handelnde in der Lage ist, Flow zu erleben (siehe Abbildung~\ref{fig:kanalmodell}). 

Im Laufe der Flow-Forschung entwickelten unterschiedliche Wissenschaftler verfeinerte Modelle wie das Quadrantenmodell \citep[S.~286]{Csikszentmihalyi1995} oder das Oktantenmodell \citep[S.~296]{Massimini1995}. Die genannten Modelle betrachten das Erleben von Flow als einen konkreten Zustand, der nur Eintritt, wenn ein Gleichgewicht zwischen Anforderungen und Fähigkeiten sowie ein konkretes Niveau an Anforderung vorhanden ist. Damit betrachten diese Modelle nur einen Teilaspekt des Flow-Erlebens, der zumindest für das Marathonlaufen \citep{Stoll2005} und das Laufen auf einem Laufband \citep{Reinhardt2006} nicht bestätigt werden konnte. 

Darüber hinaus hat diese eingeschränkte Modellierung Grenzen, wenn wir davon ausgehen, dass sich die Einflussfaktoren bzw. die Bedingungen um Flow zu erleben während der Tatigkeit kontinuierlich ändern \citep{Grueter2016b}.

Das Flow-Erleben selber kann sich auch von Flow-Zustand zu Flow-Zustand unterscheiden. \citet[][S.~222]{Csikszentmihalyi2010} spricht z.~B. von Gewohnheiten strukturierter Alltagserfahrungen mit leichten intrinsischen motivierenden Charakter und bezeichnet diese Gewohnheiten als micro Flow. Micro Flow entspricht dem Gefühl, das die Alltagtätigkeiten angenehm macht. Auf der anderen Seite eines Kontinuums beschreibt Deep Flow eine sehr intensive Erfahrung, vom Verschmelzen von Handlung und Bewusstsein. Deep Flow verbinden wir mit dem gänzlichen Aufgehen in der Tätigkeit, Selbstvergessenheit und Verlust des Zeitgefühls. Auf unterschiedliche Flow-Formate verweist auch \citet{Moneta2012} auf der Grundlage des Gleichgewichts zwischen Anforderungen der Tätigkeit und Fähigkeiten den Fähigkeiten des Handelnen. Je nachdem, ob Anforderungen der Tätigkeit oder Fähigkeit des Handelnden dominieren könnte man von einem \emph{Flow des Wohlergehens} (Fähigkeiten des Handelnen dominieren, eher mirco Flow) und einem \emph{leistungsorientieren Flow} (Anforderungen der Tätigkeit dominieren, eher deep Flow) sprechen \citep{Grueter2016b}.

% section was_ist_flow (end)

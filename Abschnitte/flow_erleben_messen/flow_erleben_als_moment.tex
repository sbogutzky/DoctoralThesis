

%!TEX root = /Users/sbogutzky/Entwicklung/projects/bogutzky/repositories/2939413/final-draft.tex
\section{Flow-Erleben als Moment der Tätigkeit} 

% (fold)
\label{sec:flow_erleben_als_moment_der_tatigkeit}

Aus dem Grund, dass die dargestellten Modelle des Flow-Erlebens Grenzen aufweisen, verfolge ich in der vorliegenden Arbeit einen prozessorientierten Ansatz, der das Flow-Erleben als Moment während einer Tätigkeit (Prozess) definiert \citep[][S.~2]{Grueter2006}. Ein Prozess besteht aus mehreren einzelnen Momenten (Zuständen und Übergängen zwischen Zuständen; Abbildung~\ref{fig:prozessorientiertes_flow_modell_1}). Für den Übergang in den Zustand \emph{Flow} versuchen wir vor der Tätigkeit die Voraussetzungen aus der  Tabelle~\ref{tab:voraussetzungen_fuer_einen_flow_zustand} herzustellen, um Flow zu Erleben. Wir können aber nie sicherstellen, dass der Übergang in den Zustand \emph{Flow} gelingt (auch nicht, wenn alle aufgeführten Voraussetzungen hergestellt sind). Erlebt der Handelnde Flow, folgt im Anschluss nach der Definition von \citet{Henk2014} das Verschmelzen von Handlung und Bewusstsein, das in der nächsten Iteration erhalten bleibt, sich verändert oder wieder auflöst. Die Bedingungen verändern sich von Iteration zu Iteration, da sie vom Tätigkeitsprozess abhängig sind. Ist z.~B. das Handlungsziel erreicht, erfolgt in der Regel das Auflösen des Flow-Zustands. Das Auflösen mündet wie alle Erlebensmomente, die nicht das Flow-Erleben ausdrücken, im Zustand \emph{nicht Flow}. 
\begin{sidewaysfigure}
	\includegraphics[width=1.00 
	\textwidth]{prozessorientiertes_flow_modell_1.pdf} \caption[Prozessorientiertes Modell des Flow-Erlebens (Einführung)]{Prozessorientiertes Modell mit Auslassungen der implizit messbaren Auswirkungen des Flow-Erlebens} \label{fig:prozessorientiertes_flow_modell_1} 
\end{sidewaysfigure}

Flow-Erleben ist bildlich vergleichbar mit dem Schlafen. Zunächst schaffen wir allgemeine Voraussetzungen. Wir ziehen uns den Schlafanzug an und decken uns zu. Im Anschluss schließen wir die Augen, beenden unsere Gedankengänge und fallen ggf. in den Schlaf. Wir sind nicht in der Lage selbstständig zu bestimmen, wann wir einschlafen. Schlafen wir, regeneriert unser Organismus und die Bedingungen für das Schlafen verändern sich. Am Morgen haben wir uns soweit erholt, dass wir ohne externe Helfer wie z.~B. dem Wecker aufwachen. 
\begin{itemize}
	
	\item In der vorliegenden Arbeit gehe ich davon aus, dass ich die Zustände und die Übergänge zwischen den Zuständen eines Prozesses, die zum Zustand \emph{Flow} führen, messen kann (Abschnitt~\ref{sec:herangehensweise}, Schritt 3). 
\end{itemize}

Der erste Schritt dient der Sammlung von Daten und der zweite Schritt beschäftigt sich mit der Identifizierung von impliziten Merkmalen des Flow-Erlebens beim Gehen und Laufen. Implizite Merkmale kategorisiere ich in der vorliegenden Arbeit in Merkmale des Gehirns, in physiologische und in motorische Merkmale. Physiologische Eigenschaften wie die Herzfrequenz, die Atemfrequenz, das Schwitzen oder das Anspannen konkreter Muskelpartien steuert das \ac{VNS} ohne bewusste menschliche Kontrolle. Im Gegensatz sind wir in der Lage, die Motorik mehr oder weniger willkürlich in einem Bewegungsablauf zu steuern. 

Durch die Trennung von Physiologie und Motorik erhalte ich für jeden Moment im Prozess die drei Ebenen: \emph{Gehirnaktivität}, \emph{Physiologisch} und \emph{Motorisch} \citep[][S.~15]{Grueter2016a}. Alle drei Ebenen sind auch im Zustand \emph{nicht Flow} vorhanden. Abbildung~\ref{fig:prozessorientiertes_flow_modell_1} skizziert das prozessorientierte Modell des Flow-Erlebens. Aus diesem Grund wurden die drei Ebenen im Zustand \emph{nicht Flow} von mir weggelassen. Die drei Punkte in den Ebenen stellen Auslassungen dar, die ich im Verlauf dieses Kapitels spezifiziere. Ich gehe davon aus, dass sich das \emph{gänzliche Aufgehen in der Tätigkeit} auf die \emph{Gehirnaktivität} und \emph{physiologische Ebene} auswirkt. Gleichzeitig nehme ich an, dass ein \emph{glatter Handlungsverlauf} Auswirkungen auf die \emph{physiologische} und \emph{motorische} Ebene hat. Alle Veränderungen auf den drei Ebenen lassen sich implizit messen (Abschnitt~\ref{sec:losungsansatze_zur_impliziten_messung_von_flow_erleben}). Zwischen den drei Ebenen besteht eine Verbindung, die ich aber in der vorliegenden Arbeit nicht weiter spezifiziere. 

Die \emph{Funktionen des Bewusstseins} bzw. die Merkmale des Flow-Erlebens nach \citet{Henk2014} stehen in Verbindung mit der \emph{Gehirnaktivität} und werden in der Regel mit expliziten Messverfahren in Erfahrung gebracht. Im nachfolgenden Abschnitt gebe ich einen kurzen Überblick über ausgewählte Befragungsmethoden zur expliziten Erfassung des Flow-Erlebens auf der Bewusstseinsebene. 

% section flow_erleben_als_moment_der_tatigkeit (end)

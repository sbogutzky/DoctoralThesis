\section{Zusammenfassung} % (fold)
\label{sec:zusammenfassung_3}

Auf der Grundlage des gewärtigen Kapitels gehe ich davon aus, dass die einzelnen Ebenen des Modellraums in Verbindung stehen. Abbildung~\ref{fig:3_3_modellraum_2} stellt die Annahmen beruhend auf der bisherigen Forschung auf den einzelnen Ebenen und eventuelle Beziehungen zwischen den Ebenen mit Pfeilen dar.

\begin{sidewaysfigure}
	\includegraphics[width=1.00\textwidth]{3-3-modellraum-2}
	\caption[Ein Modellraum des Flow-Erlebens (Ergänzung)]{Ein Modellraum des Flow-Erlebens (Ergänzung). Quelle: Eigene Darstellung}
	\label{fig:3_3_modellraum_2}
\end{sidewaysfigure}

Bei allen impliziten Ansätzen mangelt es an der empirischen Fundierung. Der größte Teil der Studien befasst sich mit dem Zusammenhang von \ac{HRV} und Flow-Erleben bei sitzenden Tätigkeiten. Sie weisen unterschiedliche Ergebnissen auf (siehe Tabelle~\ref{tab:studienubersicht_zu_flow_erleben}). Zudem lassen sich ihre Ergebnisse nicht bedenkenlos auf physische Tätigkeiten im Freien übertragen. Bei den anderen impliziten Ansätzen fehlt es entweder am eindeutigen Bezug zum Bewusstseinszustand Flow (z.~B. \ac{EMG}) oder an den Voraussetzungen für eine effiziente Erfassung bei physischen Tätigkeiten im Freien (z.~B. \ac{EDA} und \ac{EEG}). Die genannten Probleme erörtere ich detailliert in Abschnitt~\ref{sec:probleme_bei_messungen}.

In Abbildung~\ref{fig:3_4_datenraum_1} zeige ich den Datenraum mit den potenziellen Kandidaten für ein implizites Messverfahren des Flow-Erlebens. Bei dem Bewegungsfluss und der kardio-lokomotorischen Phasensynchronisation handelt es sich um zwei eigene Ansätze, um Flow-Erleben implizit zu messen. Der Bewegungsfluss, qualifiziert durch den Bewegungsaufwand, ist ein kinematisches Merkmal und gehört zur motorischen Ebene des Datenraums. Die kardio-lokomotorische Phasensynchronisation stellt einen Sonderfall dar, da ihre Analyse auf zwei Datenreihen aus zwei unterschiedlichen Ebenen basiert. Für beide Messverfahren fehlt bisher eine empirische Fundierung für einen Zusammenhang mit dem Flow-Erleben.

\begin{sidewaysfigure}
	\includegraphics[width=1.00\textwidth]{3-4-datenraum-1}
	\caption[Ein Datenraum für die Entwicklung eines impliziten Messverfahrens des Flow-Erlebens]{Ein Datenraum für die Entwicklung eines impliziten Messverfahrens des Flow-Erlebens. \\ \hspace{\textwidth}Quelle: Eigene Darstellung}
	\label{fig:3_4_datenraum_1}
\end{sidewaysfigure}

% section zusammenfassung (end)
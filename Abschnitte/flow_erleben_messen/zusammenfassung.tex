

%!TEX root = /Users/sbogutzky/Entwicklung/projects/bogutzky/repositories/2939413/final-draft.tex
\section{Zusammenfassung} 

% (fold)
\label{sec:zusammenfassung_3}

In diesem Kapitel stelle ich das Flow"=Erleben als messbare Größe dar und veranschauliche die Merkmale, die das Messen eines Bewusstseinszustands wie Flow"=Erleben ermöglichen. Ich präsentiere ein prozessorientiertes Modell des Flow"=Erlebens (Abbildung~\ref{fig:prozessorientiertes_flow_modell_1}) in dem ich die angenommenen Auswirkungen auf das Gehirn, auf die Physiologie und auf die Motorik bisher weggelassen habe. 

Die Funktionen des Bewusstseins mit subjektiven, nachträglichen Auskünften über das Erleben durch strukturierte Interviews oder durch standardisierte psychometrische Skalen abzufragen, gehört bislang zu den erprobtesten und zuverlässigen Flow"=Messverfahren. Nichtsdestotrotz gibt es die Kritik, dass Aussagen über Flow als nicht bewusst bzw. nicht reflektiert erlebten Zustand nicht verlässlich sind \citep{Henk2014}. Aus diesem Grund ist die Entwicklung direkter bzw. impliziter Messverfahren von Flow"=Erleben auch für die Flow"=Forschung mittelfristig unumgänglich. Lösungsansätze für ein implizites Messverfahren des Flow"=Erlebens basieren auf den Annahmen 
\begin{itemize}
	
	\item einer Beanspruchung der gesamten Informationsverarbeitungskapazität im Gehirn, die zu einer impliziten Ausführung der Tätigkeit einher geht \citep{Dietrich2004},
	
	\item einer Anpassung des \ac{VNS}s, die entweder zu einem optimalen Zustand der Erregung \citep{deManzano2010, Keller2011} oder sich in einer Abnahme der mentalen und körperlichen Anstrengung \citep{Peifer2014, Tozman2015, Harmat2015} ausdrückt,
	
	\item einer automatischen und effizienteren Durchführung der einzelnen Handlungsschritte, die sich durch eine Verringerung des Bewegungsaufwands zeigt (Abschnitt~\ref{ssub:der_bewegungsfluss}) oder 
	
	\item eines Zusammenhangs zwischen zweier optimalen Zustände (Abschnitt~\ref{ssub:die_kardio_lokomotorische_phasensynchronisation}). 
\end{itemize}

In Abbildung~\ref{fig:prozessorientiertes_flow_modell_2} habe ich die Auslassungen im vorläufigem prozessorientierten Modells des Flow"=Erlebens durch die aufgeführten Annahmen ergänzt. Aus der Sicht des prozessorientierten Modells des Flow"=Erlebens hat der standardmäßige Einsatz von expliziten Messverfahren und die Objektivierung von impliziten Daten in \emph{akkumulierter Form} den Nachteil, dass wir nicht in der Lage sind, eine präzise Verortung eines Flow"=Moments im Prozess der Tätigkeit vorzunehmen. 

Einen ersten Schritt, der dem prozessorientierten Modell des Flow"=Erlebens gerecht wird, ist ein computergestütztes Experience Sampling Verfahren bei gleichzeitiger Erhebung von physiologischen, kinematischen und kontextbezogenen Daten (Abschnitt~\ref{sec:herangehensweise}, Schritt~1). Demzufolge betrachte ich im nachfolgenden Kapitel die Anforderungen an einen Versuchsaufbau, der es ermöglicht Flow"=Erleben \emph{explizit}, \emph{implizit}, \emph{prozessorientiert}, \emph{mobil} und \emph{computergestützt} in natürlichen Handlungsumgebungen beim Gehen und Laufen zu erfassen.
\begin{sidewaysfigure}
	\includegraphics[width=1.00 
	\textwidth]{prozessorientiertes_flow_modell_2.pdf} \caption[Prozessorientiertes Modell des Flow-Erlebens (Erweiterung)]{Prozessorientiertes Modell mit Annahmen zu implizit messbaren Auswirkungen des Flow-Erlebens} \label{fig:prozessorientiertes_flow_modell_2} 
\end{sidewaysfigure}

% section zusammenfassung (end)

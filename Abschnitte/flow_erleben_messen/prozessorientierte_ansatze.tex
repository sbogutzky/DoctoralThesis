

%!TEX root = /Users/sbogutzky/Entwicklung/projects/bogutzky/repositories/2939413/final-draft.tex
\section{Prozessorientierte Ansätze mit expliziten Messverfahren} 

% (fold)
\label{sec:prozessorientierte_ansatze_mit_expliziten_messverfahren}

Aus der Sicht des vorgestellten prozessorientierten Modells des Flow-Erlebens (Abbildung~\ref{fig:prozessorientiertes_flow_modell_1}) hat der standardmäßige Einsatz von strukturierten Interviews, Fragebögen und psychometrischen Skalen, aber auch die Objektivierung von impliziten Daten in \emph{akkumulierter Form} beim Gehen und Laufen in natürlichen Handlungsumgebungen im Nachhinein einen grundlegenden Nachteil,
\begin{itemize}
	\item die unspräzise bis mangelde Verortung eines Flow-Moments im Prozess der Tätigkeit. 
\end{itemize}

Einen ersten Schritt zu einem prozessorientierten Ansatz mit expliziten Messverfahren machen \citet{Larson1983} mit der \ac{ESM}.

\subsection{Experience Sampling Method} 

% (fold)
\label{sub:experience_sampling_method}

\citet{Larson1983} entwickelten die \ac{ESM}, um das Verhalten und das Erleben in natürlichen Handlungsumgebungen zu erheben. Die originale \ac{ESM} von Larson und Csikszentmihalyi wurde von mehreren Wissenschaftlern \citep[z.~B.][]{Schallberger2001, Rheinberg2003} durch verbesserte Fragebögen bzw. psychometrische Skalen optimiert. Heutzutage fassen wir unter \ac{ESM} eine Reihe von ambulanten Assessment-Strategien zusammen. Ambulantes Assessment ist die Erhebung von Daten im Alltag der zu Untersuchenden.

In einer typischen \ac{ESM}-Studie erhalten die Teilnehmer für den gewählten Testzeitraum einen akustischen Signalgeber sowie Formulare oder Fragebögen in gehefteter Form. Ein akustisches Signal fordert den Teilnehmer auf, eine Selbstauskunft abzugeben. Zu Variablen, die die Wahrnehmung, Motivation und den Gemütszustand des Teilnehmers widerspiegeln, fragen Wissenschaftler in einer üblichen \ac{ESM}-Studie Variablen zur Situation, wie Ort, Aktivität und sozialer Kontext, ab. 

Durch die Mehrfachabfrage lässt sich der Verlauf der untersuchten Variablen jedes Teilnehmers über den Testzeitraum abhängig von der zeitlichen Auflösung rekonstruieren. Es gibt drei Vorgehen bei der \ac{ESM}: (a) intervall-kontingente, (b) ereignis-kontingente und (c) signal-kontingente Protokolle \citep[][S.~198ff.]{Reis2000}. Bei intervall-kontingenten Messungen geben die Teilnehmer zu vorbestimmten Intervallen (z.~B. jede Stunde oder zur gleichen Tageszeit) eine Selbstauskunft ab. Teilnehmer einer ereignis-kontingenten Messstudie geben nur bei konkreten Ereignissen eine Selbstauskunft ab. Beim signal-kontingenten Vorgehen erhalten die Teilnehmer zu beliebigen Zeitpunkten ein Signal, das sie auffordert, eine Selbstauskunft abzugeben. Die Wahl des Vorgehens hängt von der Aufgabe der Studie ab. Der Vorteil der \ac{ESM} ist, dass wir die Messung direkt nach der unterbrochenen Tätigkeit durchführen. Ein Nachteil ist wiederum diese Unterbrechung der Tätigkeit bzw. des Erlebens durch den Signalgeber.

% subsection experience_sampling_method (end)
\subsection{Computergestützte Experience Sampling Verfahren} 

% (fold)
\label{sub:computergestutzte_experience_sampling_verfahren}

Computergestützte Experience Sampling Verfahren nutzen heutzutage die technischen Möglichkeiten von mobilen Geräten, um eine Datenerhebung automatisch zu initiieren. 2003 entwickelten \citet{Intille2003} eine Version der \ac{ESM} mit Hilfe eines \acs{PDA}s. Sie beabsichtigten ein kontextsensitives Experience Sampling, das die beschriebenen Erhebungsprotokolle ergänzt. Ein Beispiel für einen Kontextbezug ist die Ortsinformation. Mit der Hilfe eines \acs{GPS}-Empfängers fordert z.~B. der \acs{PDA} den Teilnehmer automatisch auf, eine Selbstauskunft zu geben, wenn sich der jeweilige Teilnehmer an einem konkreten Ort aufhält.

Die Plattform MyExperience läuft auf einem Smartphone mit einem Windows Betriebssystem und ermöglicht kontextbezogene Erhebungsprotokolle \citep{Froehlich2007}. Sie benutzt die interne Sensorik des Smartphones, um z.~B. eine Lokalisierung über \acs{GPS} oder \acs{GSM} vorzunehmen und um Informationen über die Smartphone-Nutzung zu sammeln. Sensorereignisse ermöglichen dem Untersuchenden, benutzerdefinierte Aktionen wie die Synchronisation von Daten über das Internet, das Senden von \acs{SMS}-Nachrichten an den Untersuchenden oder die Anzeige eines Fragebogens auszuführen. Mit MyExperience sind die Untersuchenden in der Lage, Selbstauskünfte nicht nur zeitlich, sondern auch bei vordefinierten Ereignissen abzufragen.

Eine kommerzielle computergestützte \ac{ESM}-Lösung ist Breakthrough Research von MetricWire. Breakthrough Research ist ein geschlossenes Server-Client-System und ermöglicht smartphone-gestütztes Experience Sampling. Die Erstellung einer Befragung erfolgt über eine Webplattform und die Darstellung in einer App für das Android OS oder das iOS. Die Lösung unterstützt die Lokalisierung durch \acs{GPS}.

\citet{Gaggioli2013} setzen in ihrer Studie zum Flow-Erleben anstelle von Fragebögen in gehefteter Form ein computergestütztes Experience Sampling Verfahren auf einem Smartphone ein. Die Forschungsgruppe um Gaggioli entwickelte eine Plattform mit dem Namen PsychLog für das mobile Windows Betriebssystem zur mobilen Erhebung von subjektiven und kardiovaskulären Daten \citep{Gaggioli2013a}.

Der jetzige Stand der Technik ermöglicht computergestützte Experience Sampling Verfahren mit komplexeren Erhebungsprotokollen bei gleichzeitiger Erhebung von physiologischen, kinematischen und kontextbezogenen Daten. Das versetzt die Untersuchenden in die Lage, eine präzisere und weniger aufwendige Kontrolle darüber zu erhalten, wann und wie oft ein Teilnehmer aufgefordert wird, eine digitale Selbstauskunft abzugeben. Zusätzlich reduziert die digitale Form der Erhebung menschliche Fehler bei der Verarbeitung der Daten. Die physiologischen, kinematischen und kontextbezogenen Daten geben ihnen Aufschluss darüber, welche Tätigkeit der Teilnehmer vor einer Selbstauskunft durchführte und wie der Organismus des Teilnehmers auf die jeweilige Tätigkeit reagierte.

% subsection computergestutzte_experience_sampling_verfahren (end)
% section prozessorientierte_ansatze_mit_expliziten_messverfahren (end)

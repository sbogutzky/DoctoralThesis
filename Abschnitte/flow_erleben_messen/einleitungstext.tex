

%!TEX root = /Users/sbogutzky/Entwicklung/projects/bogutzky/repositories/2939413/final-draft.tex
Csikszentmihalyi fand 1975 erstmals über qualitative Methoden (explizite Befragung nach der zu untersuchenden Tätigkeit) Zugang zum Flow-Phänomen. Bis heute gehören explizite Messverfahren zu den erprobtesten und zuverlässigsten Flow-Messverfahren. In der vorliegenden Arbeit dient ein explizites Messverfahren als Referenz, um ein implizites Messverfahren des Flow-Erlebens unter Belastung zu identifizieren (Abschnitt~\ref{sec:herangehensweise} Schritt 2). Im Gegensatz zu einem expliziten Messverfahren kommt ein implizites Messverfahren ohne die Unterbrechung der ausgeübten Tätigkeit aus und erfasst das Erleben beiläufig ohne mündliche Auskünfte. Kandidaten für ein implizites Flow-Messverfahren bei sitzenden Tätigkeiten wurden bereits von mehreren Wissenschaftlern der Flow-Forschung vorgeschlagen und untersucht (Abschnitt~\ref{sec:losungsansatze_zur_impliziten_messung_von_flow_erleben}). Auf dem Weg von einem \emph{impliziten zustandsorientierten} Messverfahren zu einem \emph{impliziten prozessorientierten} Messverfahren zieht die vorliegenden Arbeit Ansätze heran, die \emph{explizite} Merkmale des Flow-Erlebens mehrfach im Verlauf einer Tätigkeit (\emph{prozessorientiert}) messen. Um implizite Kandidaten dem Flow-Konstrukt zuordnen zu können, beschäftige ich mich im nachfolgenden Abschnitt mit dem Flow-Phänomen und mit den Komponenten des Flow-Erlebens, die sich messen lassen. 

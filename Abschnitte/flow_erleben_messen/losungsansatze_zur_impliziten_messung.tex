

%!TEX root = /Users/sbogutzky/Entwicklung/projects/bogutzky/repositories/2939413/final-draft.tex
\section{Lösungsansätze zur impliziten Messung von Flow-Erleben} 

% (fold)
\label{sec:losungsansatze_zur_impliziten_messung_von_flow_erleben}

Die vorgestellten expliziten Messverfahren haben introspektive Aussagen über das Erlebte als Ergebnis. Introspektive Aussagen sind für Flow als nicht bewusst bzw. nicht reflektiert erlebten Zustand nicht verlässlich, da eventuell keine oder nur unzureichend bewusst zugängliche Gedächtnisinhalte im Flow-Zustand angelegt werden \citep[][S.~82]{Henk2014}. Damit ist die Entwicklung direkter bzw. impliziter Messverfahren von Flow-Erleben auch für die Flow-Forschung mittelfristig unumgänglich \citep[][S.~86]{Henk2014}.

In der Praxis bewährte sich zu den in Abbildung~\ref{fig:prozessorientiertes_flow_modell_1} dargestellten impliziten Messverfahren die Messung des Stresshormons Cortisol \citep{Keller2011, Peifer2014, Peifer2015}. Für die vorliegende Arbeit ist das Verfahren nicht praktikabel, da der Cortisolgehalt erst 30 Minuten nach der Anforderungssituation im Speichel für die konkrete Anforderungssituation messbar ist. 

Potentielle Kandidaten für implizite Messverfahren, die ich in den nachfolgenden Abschnitten erläutere, sind: 
\begin{itemize}
	
	\item die Messung der Gehirnaktivität,
	
	\item die Messung von physiologischen Merkmalen und
	
	\item die Messung von motorischen Merkmalen. 
\end{itemize}

Die in den nachfolgenden Abschnitten vorgestellten Arbeiten nutzten in der Regel implizit gemessene Daten in \emph{akkumulierter Form} als Merkmale, um Zusammenhänge mit explizit gemessenen Merkmalen des Flow-Erlebens zu untersuchen.

\subsection{Flow im Gehirn} 

% (fold)
\label{sub:flow_im_gehirn}

\citet[S.~758f.]{Dietrich2004} beschäftigte sich damit Flow-Erleben anhand der Gehirnaktivität zu erklären und geht davon aus, dass ein Zustand der \emph{transienten Hypofrontalität} für das Flow-Erleben notwendig ist. Er beschreibt den Zustand der \emph{transienten Hypofrontalität} als eine Aktivitätsabnahme im präfrontalen Kortex, einem Areal des Gehirns, das wir für die explizite Kontrolle einer ausgeführten Tätigkeit benötigen \citep{Dietrich2003, Dietrich2004}. Diese Unterfunktion des präfrontalen Kortexes führt zu einem Zustand zwischen Wachheit und Schlaf \citep[][S.~241]{Dietrich2003} und zu einer impliziten Ausführung, die das Bewusstsein umgeht \citep[][S.~753]{Dietrich2004}.

\citet[][S.~757]{Dietrich2004} vergleicht in einer Gegenüberstellung die implizite Ausführung im Zustand der \emph{transienten Hypofrontalität} mit den neun Dimensionen des Flow-Erlebens nach \citet{Csikszentmihalyi1992}. Die Gegenüberstellung ist in Tabelle~\ref{tab:funktionen_des_praefrontalen_kortexes} zusammengefasst und verdeutlicht den Zusammenhang der beiden Zustände. 
\begin{table}
	[!htb] \caption[Funktionen des präfrontalen Kortexes und Merkmale eines Flow-Zustands]{Funktionen des präfrontalen Kortexes und Merkmale eines Flow-Zustands (\citet{Henk2014} nach \citet{Dietrich2004})} \label{tab:funktionen_des_praefrontalen_kortexes} 
	\begin{tabularx}
		{ 
		\textwidth}{*{2}{>{\RaggedRight\arraybackslash}X}} \toprule Funktionen des präfrontalen Kortexes & Merkmale eines Flow-Zustands \\
		\midrule Dauerhafte, gerichtete Aufmerksamkeit & Zentrierung der Aufmerksamkeit \\
		Selbstreflektierendes Bewusstsein, Vorstellungen über die eigene Person & Verlust der reflexiven Selbstbewusstheit \\
		Zeitliche Integration & Verlust des Zeitgefühls \\
		Handlungsplanung & keine Besorgtheit über Misserfolg \\
		\bottomrule 
	\end{tabularx}
\end{table}

Zur Messung der Gehirnaktivität setzen Wissenschaftler die \ac{EEG} ein. Hierzu platzieren sie Elektroden auf die Kopfhaut des zu Untersuchenden. Mehrkanalsysteme dienen der klinischen Untersuchung. Je mehr Kanäle ein Messsystem besitzt, umso besser lassen sich Hirnströme identifizieren und aktive Gehirnareale wie der präfrontale Kortex im Nachhinein lokalisieren. Eine große Anzahl von Elektroden, die mit Kabel verbunden sind, erhöhen wiederum die Störungsanfälligkeit eines Mehrkanalsystems. Das macht den Einsatz eines Mehrkanalsystems außerhalb von Laboratorien unmöglich, da ein faradayscher Käfig zur Abschirmung vor externen Störquellen erforderlich ist. Durch die Nutzung eines Zweikanalsystems sind Wissenschaftler in der Lage auch außerhalb von Laboratorien \ac{EEG}-Messungen durchzuführen.

\citet{Hugentobler2011} setzte beide Arten von Systemen ein. Er versuchte Flow-Erleben mit der Hilfe eines Computerspiels zu induzieren und führte gleichzeitig \ac{EEG}-Messungen durch. Als explizites Vergleichsmessverfahren nutzte er die \ac{FKS}. Nachdem er bedeutende Gehirnareale im Flow-Erleben mit Hilfe des Mehrkanalsystems identifiziert hatte, nutzte er ein Zweikanalsystem. Hugentoblers Ergebnisse lassen darauf schließen, dass Flow-Erleben mit höheren Theta-Wellen in Verbindung steht. „Theta-Wellen beschreiben im Wachzustand Intuition und Erinnerung, allenfalls auch tiefe Entspannung — ein Zustand zwischen Wachheit und Schlaf“ \citep[S.~149]{Hugentobler2011}. Damit stehen die Ergebnisse in einer Linie mit dem theoretischen Erklärungsversuch von Dietrich.

\citet{Harmat2015} nutzten in ihrer Studie zum Flow-Erleben beim Tetris-Spielen bei 35 der 77 Teilnehmern eine funktionelle Nahinfrarotspektroskopie, um Änderungen der Oxygenierung (Sauerstoffgenerierung) im präfrontalen Kortex als Indiz für eine mentale Mühelosigkeit im Flow-Erleben zu identifizieren. Sie fanden aber keinen Anhaltspunkt dafür, dass Flow-Erleben, gemessen mit einer Teilmenge der \ac{FSS}, und eine Verringerung in den vorderen Gehirnregionen im Zusammenhang stehen.

% subsection flow_im_gehirn (end)
\subsection{Physiologische Merkmale des Flow-Erlebens} 

% (fold)
\label{sub:physiologische_merkmale_des_flow_erlebens}

In der heutigen Flow-Forschung gehen die Wissenschaftler davon aus, dass die physiologischen Eigenschaften des Menschen im Flow-Erleben mit anderen Bewusstseinszuständen vergleichbar sind. Wie z.~B. in Abschnitt~\ref{sec:flow_erleben_als_moment_der_tatigkeit} beschrieben, lässt sich der Eintritt in das Flow-Erleben, der Erhalt und die Veränderung des Flow-Erlebens sowie der Austritt aus dem Flow-Erleben mit dem Einschlafen, der Regeneration während des Schlafens und dem Aufwachen vergleichen. Trotz Schaffung von allgemeinen Voraussetzungen für beide Prozesse sind wir aber nicht in der Lage die Übergänge in die jeweiligen Zustände (nicht schlafend und schlafend und \emph{nicht Flow} oder \emph{Flow}) selber zu bestimmen.

Für jeden Zustand passt das \ac{VNS} die Aktivität und die Stoffwechselprozesse unseres Organismus ohne eigenes Dazutun an. Darum wird es auch als autonomes Nervensystem bezeichnet. Maßgeblich tragen die beiden Kontrahenten des \ac{VNS}s, Sympathikus und Parasympathikus, die Verantwortung für die Anpassung. Der Sympathikus bereitet ihn auf körperliche und geistige Leistungen vor. Er sorgt unter anderem dafür, dass das Herz schneller und kräftiger schlägt, sich weniger Speichel bildet und sich die Schweißsekretion erhöht. Der Parasympathikus wiederum kümmert sich um die Körperfunktionen in Ruhe, um die Regeneration sowie um den Aufbau körpereigener Reserven. Er verlangsamt unter anderem den Herzschlag, erhöht die Speichelsekretion und vermindert die Schweißsekretion. Die gegensinnigen Aktionen des Sympathikus und Parasympathikus führen zu einem dynamischen Gleichgewicht, die die Aktivität und die Stoffwechselprozesse unseres Organismus für jeden Zustand optimal anpasst. 

Anhand von physiologischen Messungen und deren Analyse sind wir in der Lage Aussagen über den emotionalen Gemütszustand des zu Untersuchenden zu treffen und zu identifizieren, welcher der beiden Nervenstränge während eines Zustands aktiv ist. Die Intensität der jeweiligen Aktivierung ermöglicht herauszufinden, ob und in welcher Intensität ein Mensch einen Zustand erlebt. In den nachfolgenden Abschnitten gebe ich einen Überblick über physiologische Messungen, die Wissenschaftler in der Flow-Forschung einsetzten.

\subsubsection{Elektromyographie} 

% (fold)
\label{ssub:elektromyographie}

Emotionale Gemütszustände lassen sich bei vielen Menschen im Gesicht ablesen. Die \ac{EMG} misst die elektrische Muskelaktivität und lässt sich einsetzen, um emotionale Gemütszustände anhand von Muskelgruppen im Gesicht automatisch zu bestimmen. \citet{Kivikangas2006, Nacke2008, deManzano2010} nutzten die \ac{EMG}-Analyse in ihren Studien, da sie davon ausgehen, dass Flow-Erleben positive Gemütszustände, wie Freude, zur Folge hat. Die drei Untersuchungen kommen zu gegensätzlichen Ergebnissen, welche Muskelpartien im Gesicht geeignet sind, Flow-Erleben zu identifizieren. \citet[][S.~153]{Peifer2012} vermutete, dass das unter anderem an den unterschiedlichen Bedingungen der Untersuchungen lag. \citet{Kivikangas2006} und \citet{Nacke2008} untersuchten Spieler beim Spielen eines Ego-Shooters; \citet{deManzano2010} untersuchten hingegen Pianospieler, die ihr Lieblingsstück spielten. \citet{Kivikangas2006} rekrutierte studentische Freiwillige, wobei \citet{Nacke2008} männliche Hardcore-Computerspieler und \citet{deManzano2010} professionelle Pianospieler rekrutierten. Nach \citet[][S.~153]{Peifer2012} sind zusätzliche Studien nötig, um die gegensätzlichen Resultate zu klären, und um die \ac{EMG}-Messung zur Identifikation von Flow-Erleben zu nutzen.

% subsubsection elektromyographie (end)
\subsubsection{Elektrodermale Aktivität} 

% (fold)
\label{ssub:elektrodermale_aktivitat}

Die \ac{EDA} beschreibt den elektrischen Leitungswiderstand der Haut. Eine Änderung der \ac{EDA} kommt durch eine Anpassung im \ac{VNS} und die als Folge dessen angeregte oder abgeschwächte Schweißsekretion zustande. Den Anpassungsprozess des \ac{VNS}s wollten \citet{Kivikangas2006, Nacke2008} als Merkmal für Flow-Erleben nutzen. Sie begründen ihren Ansatz mit der Änderung des Gemütszustands zur Erregung, die durch eine Aktivierung des Sympathikus beim sitzenden Computerspieler gekennzeichnet ist. Sie folgern, dass die Erregung eine Folge der erhöhten mentalen Anstrengung ist, die auf eine Zentrierung der Aufmerksamkeit bzw. eine Fokussierung auf die Tätigkeit schließen lässt.

\citet{Kilpatrick1972} unterscheidet eine ereignisbasierte Hautleitfähigkeitsreaktion (\ac{SCR}) und einen Langzeit-Hautleitfähigkeitslevel (\ac{SCL}). \citet[][S.~158]{Peifer2012} vermutet, dass das \ac{SCL} gegenüber der \ac{SCR} ein vielversprechender Indikator für Flow-Erleben ist. \citet[][S.~158]{Peifer2012} begründete ihre Annahme damit, dass das Flow-Erleben kein diskreter Zustand ist, den wir nicht durch konkrete Stimuli oder Erfahrungen hervorzurufen können, sondern eher durch \emph{lang andauernde glatte Aktivitäten} begünstigt wird. \citet{Kilpatrick1972} fand keinen Zusammenhang zwischen \ac{SCL} und Flow-Erleben. 

\citet{Nacke2008} benutzten das \ac{SCL}, um unterschiedlich konzipierte Level des Computerspiels Half-Life 2 zu vergleichen. Sie fanden einen signifikanten Unterschied zwischen den Leveldesigns anhand der \ac{SCL}. Das Leveldesign, welches sie für das Flow-Erleben konzipierten, hatte im Mittel das höchste \ac{SCL}. Zur weiteren Überprüfung, ob das jeweilige Leveldesign den geforderten Zustand induzierte, nutzten \citet{Nacke2008} nach jedem Spielen den \ac{GEQ}.

Aussagen aus dem \ac{GEQ} und dem \ac{SAM} kamen auch in der Studie von \citet{Friedrichs2015} zum Einsatz. In dieser Studie konzipierten die Wissenschaftler ein Computerspiel, welches positive und negative Gemütszustände (Emotionen) induzieren sollte. Mit dem Spiel überprüften sie, ob es möglich ist, positive und negative Emotionen anhand von physiologischen Daten zu klassifizieren. Mit einer Wahrscheinlichkeit von 70~\% gelang es ihnen mit einer \ac{SVM} und einer Matrix von zehn physiologischen Merkmalen die richtige Emotion zu klassifizieren. Zu den physiologischen Merkmalen gehörten u. a. die elektrodermale Aktivität und kardiovaskuläre Messungen.

% subsubsection elektrodermale_aktivitat (end)
\subsubsection{Kardiovaskuläre Messungen} 

% (fold)
\label{ssub:kardiovaskulare_messungen}

Kardiovaskuläre Messungen betrachten die Herzaktivität. Das Herz besitzt einen sinusförmigen Impuls von 60 bis 80 Schlägen pro Minute. Abgetragene Schläge pro Minute (\ac{BPM}) messen die \ac{HR}. Das \ac{VNS} beeinflusst das Herz, um die Leistung des Herzens ständig wechselnden äußeren und inneren Bedingungen anzupassen. Allgemein erhöht der Sympathikus die \ac{HR} und vergrößert die Gefäße, damit der menschliche Organismus in der Lage ist den äußeren Anforderungen zu begegnen \citep[][S.~226]{Porges1995}. Der Parasympathikus bewirkt das Gegenteil im Zuge der Erholung \citep[][S.~226]{Porges1995}. Im Verlauf der Erholung dominiert der Parasympathikus und die \ac{HR} ist gering. 

Beide Nervenstränge beeinflussen sowohl die \ac{HR} als auch die \ac{HRV}. Die \ac{HRV} repräsentiert die Variabilität des zeitlichen Abstands von zwei aufeinander folgenden Herzschlägen. Wir messen die zeitlichen Abstände mit der Hilfe eines \ac{EKG}s oder eines Plethysmographen. Die darauffolgende Analyse der Variabilität der zeitlichen Abstände in ruhenden bzw. sitzenden Tätigkeiten erlaubt uns einen Einblick in den Steuermechanismus des \ac{VNS}s \citep{Jalife1983}. Zur Bestimmung der sympathischen und parasympathischen Aktivität analysierten wir die NN-Intervalle (bei R-Zacke zu R-Zacke, RR-Intervalle) zeit- oder frequenzbezogen. Bei ruhenden bzw. sitzenden Tätigkeiten ist die physiologische Interpretation der zeit- und frequenzbezogenen \ac{HRV}-Parameter weitestgehend manifestiert \citep[][S.~360]{TaskForce1996}. Tabelle~\ref{tab:beschreibung_der_hrv_parameter} gibt eine Übersicht über die in den nachfolgenden Studien genutzten \ac{HRV}-Parameter und ihre Bedeutung im Zusammenhang mit dem \ac{VNS}. 

\begin{table}
	[!htb] \caption[Beschreibung der \acs{HRV}-Parameter]{Beschreibung der \acs{HRV}-Parameter} \label{tab:beschreibung_der_hrv_parameter} 
	\begin{tabularx}
		{ 
		\textwidth}{p{.15 
		\textwidth} p{.15 
		\textwidth} p{.12 
		\textwidth} p{.12 
		\textwidth} X} \toprule Analyse & Methode & Maß & Einheit & Definition und Erklärung \\
		\midrule Zeitbe"-zogene & Statistisch & \acs{RMSSD} & \emph{ms} & Quadratwurzel des Mittelwerts der Summe aller quadrierten Differenzen zwischen benachbarten NN-Intervallen; Parameter der Kurzzeitvariabilität; zur Betrachtung des parasympathischen Einflusses. \\
		\hline
		
		Frequenz"-bezogene & \acs{FFT} und autoregressives (AR-) Modell & \acs{LF} & $ms^{2}$ & Leistungsdichtespektrum im Frequenzbereich von 0,04 bis 0,15 Hz, Periodendauer von 7 bis 25 s; daran ist sowohl der Sympathikus als auch der Parasympathikus beteiligt, wobei der Anteil des Sympathikus überwiegt. \\
		\cline{3-5}
		
		& & \acs{HF} & $ms^{2}$ & Leistungsdichtespektrum im Frequenzbereich von 0,15 bis 0,40 Hz, Periodendauer 2,5 bis 7 s; zeigt ausschließlich den parasympathischen Stimmungsanteil. \\
		\cline{3-5}
		
		& & \acs{LF}/\acs{HF} & \emph{k. E.} & Quotient der sympathovagalen Balance; als Wert des Zusammenspiels von Parasympathikus (\acs{HF}) und Sympathikus (\acs{LF}) \acs{LF}/\acs{HF} $\uparrow{}$ = Sympathikus $\uparrow{}$ (Sympathikusaktivität steigt an) \acs{LF}/\acs{HF} $\downarrow{}$ = Parasympathikus $\uparrow{}$. \\
		\bottomrule 
	\end{tabularx}
\end{table}

In diesen Studien dient die Analyse der \ac{HRV} als Hinweis für eine Änderung des Gemütszustands zur Erregung, was auf einen Anstieg der sympathischen Aktivität schließen lässt. Eine zweite Vermutung ist, dass durch den automatischen und effizienteren Ablauf der Handlungsschritte im Flow-Erleben eine \emph{Mühelosigkeit} entsteht und die mentale und körperliche Anstrengung abnimmt \citep[][S.~308]{deManzano2010}. Die Abnahme lässt auf eine Aktivierung des parasympathischen Zweigs des \ac{VNS}s und eine steigende \ac{HRV} schließen. 

\paragraph{\citet{deManzano2010}} 

% (fold)
\label{par:demanzano2010}

fanden im Gegensatz zu ihrer aufgestellten Annahme eine Abnahme der parasympathischen Aktivität während des Flow-Erlebens beim Pianospielen. In ihrer Studie mit 21 männlichen Pianospielern identifizierten sie vielmehr eine Aktivierung des sympathischen Zweigs anhand eines ansteigenden \acs{LF}/\acs{HF}-Verhältnisses und einer steigenden \ac{HR} während des Flow-Erlebens. Die rekrutierten Pianospieler spielten jeweils fünf Wiederholungen eines von ihnen ausgewählten Musikstückes (Länge: drei bis sieben Minuten). In den Unterbrechungen der Wiederholungen nutzen \citet{deManzano2010} eine Teilmenge der \ac{FSS}, um den erlebten Flow der Pianospieler explizit zu ermitteln. Zur Durchführung der \ac{HRV}-Analyse zeichneten sie kontinuierlich \ac{EKG}-Daten auf.

% paragraph demanzano2010 (end)
\paragraph{\citet{Keller2011}} 

% (fold)
\label{par:keller2011}

nutzten die Voraussetzung vom \emph{Gleichgewicht zwischen Anforderungen der Tätigkeit und eigenen Fähigkeiten} in ihrem Vorgehen. Sie stellten in einem computerbasierten Fragespiel eine Langeweilekondition, eine Vereinbarkeitskondition bzw. Flow-Kondition und eine Überlastungskondition her. Acht Teilnehmer beantworteten jeweils fünf Minuten Wissensfragen in den drei Konditionen während die Wissenschaftler \ac{EKG}-Daten aufzeichneten. Am Ende jeder Kondition füllten die Teilnehmer einen Fragebogen \citep{Keller2008} aus, um den erlebten Flow explizit zu ermitteln. Die Ergebnisse zeigen eine Verringerung des zeitbezogenen \ac{HRV}-Parameters \acs{RMSSD} in der Vereinbarkeitskondition im Gegensatz zu den anderen Konditionen. Das lässt auf eine fallende parasympathische Aktivität im Flow-Erleben schließen.

% paragraph keller2011 (end)
\paragraph{\citet{Gaggioli2013}} 

% (fold)
\label{par:gaggioli2013}

untersuchten den Zusammenhang zwischen Flow-Erleben und \ac{HRV}-Parametern im Laufe des Alltags von 15 Studenten über eine Dauer von sieben Tagen. Die Autoren sind bisher die Einzigen, die Messungen von psychologischen Eigenschaften zur Erforschung von Flow-Erleben in einer natürlichen Handlungsumgebung durchführten. Sie nutzten ein computergestütztes Experience Sampling Verfahren. Sie statteten die Studenten mit einem Smartphone und einem mobilen \ac{EKG}-Sensor aus. Im Laufe des Alltags nutzten sie PsychLog (Abschnitt~\ref{sub:computergestutzte_experience_sampling_verfahren}), um von den Studenten zu zufälligen Zeiten subjektive Daten und Informationen über ihre Tätigkeit zu erhalten. Flow-Erleben identifizierten sie anhand eines überdurchschnittlichen Gleichgewichts zwischen der wahrgenommenen Herausforderung, der Fähigkeitsangabe und den überdurchschnittlichen Werten von positiven Emotionen, wie Freunde, Zufriedenheit, usw. In ihren 561 Selbstberichten und 377 validen \ac{EKG}-Aufnahmen fanden sie damit 32 Flow-Erlebnisse. Ihre Aufnahmen der \ac{EKG}-Daten begannen jeweils 20 Minuten vor jeder Befragung und endeten 20 Minuten nach dem Start der Befragung. Sie fanden damit eine Beziehung zwischen steigendender \ac{HR}, ansteigendem \acs{LF}/\acs{HF}-Verhältnis und erlebten Flow. Beides lässt sie auf eine steigende sympathische Aktivierung im Verlauf des Erlebens von Flow schließen. Die jeweilige Handlungsumgebung und der Charakter der Tätigkeit, bei den die 32 Flow-Erlebnisse auftraten, können der Arbeit nicht entnommen werden. 

% paragraph gaggioli2013 (end)
\paragraph{\citet{Peifer2014}} 

% (fold)
\label{par:peifer2014}

belegten die Folgerung von \citet{deManzano2010, Peifer2012} insofern, dass nicht eine geringe und nicht eine hohe sympathische Aktivität, sondern eine optimale Aktivität dazwischen, ideal für das Flow-Erleben ist. In einer Studie untersuchten \citet{Peifer2014} die Beziehung zwischen Flow und Stress. Als Indikatoren für sympathische und parasympathische Aktivität dienten ihnen die frequenzbezogenen \ac{HRV}-Parameter \acs{LF} und \acs{HF}. 22 männliche Teilnehmer führten eine komplexe Aufgabe am Computer für 60 Minuten durch und bewerteten im Anschluss ihren Zustand mit Hilfe der \ac{FKS}. Zur \ac{HRV}-Analyse zeichneten \citet{Peifer2014} kontinuierlich die \ac{EKG}-Daten auf. Die Wissenschaftler werteten die Faktoren \emph{glatter automatisierter Verlauf} und \emph{Absorbiertheit} der \ac{FKS} aus. Zur Identifikation des explizit erlebten Flows nutzten sie nur die Absorbiertheit. Das begründeten sie wie folgt:

Ein glatter automatisierter Verlauf kann auch entstehen, wenn die Fähigkeiten die Handlungsmöglichkeiten übersteigen. Das führt eher zur Langeweile statt zum Flow-Erleben. Absorbiertheit wiederum entsteht nur, wenn die Anforderungen der Tätigkeit und die eigenen Fähigkeiten in einem Gleichgewicht stehen.

Ihre Ergebnisse zeigen eine umgekehrte u-förmige Beziehung zwischen erlebten Flow mit dem \ac{HRV}-Parameter \acs{LF}. Der \ac{HRV}-Parameter \acs{HF} verhielt sich linear und positiv zum erlebten Flow. Die Ergebnisse legen nahe, dass moderate sympathische Aktivität und eine Co-Aktivierung beider Zweige des \ac{VNS} das Erleben von Flow charakterisieren.

% paragraph peifer2014 (end)
\paragraph{\citet{Tozman2015}} 

% (fold)
\label{par:tozman2015}

entwickelten einen Fahrsimulator mit drei verschiedenen Schwierigkeitsgraden. Sie stellten wie bei \citet{Keller2011} eine Langeweilekondition, eine Vereinbarkeitskondition bzw. Flow-Kondition und eine Überlastungskondition her. 15 Teilnehmer fuhren die drei Simulationen jeweils für sechs Minuten in unterschiedlicher Reihenfolge. Nach jeder Simulation beantworteten die Teilnehmer eine \ac{FKS}. Die Wissenschaftler zeichneten während jeder Simulation \ac{EKG}-Daten des Teilnehmers auf. Anders als \citet{Peifer2014} nutzten \citet{Tozman2015} den Generalfaktor der \ac{FKS} als explizites Flow-Merkmal. Den Wert für \acs{LF} und \acs{HF} subtrahierten sie mit jeweils einem Wert aus einer Baseline-Messung. Ihre Ergebnisse zeigen eine negative Beziehung zwischen dem frequenzbezogenen \ac{HRV}-Parameter \acs{LF} und Flow-Erleben in der Flow-Kondition. Zusätzlich fanden sie jeweils eine umgekehrte u-förmige Beziehung zwischen dem Generalfaktor der \ac{FKS} und den frequenzbezogenen \ac{HRV}-Parametern \acs{LF} und \acs{HF} in der Überlastungskondition. Die Ergebnisse legen eine Abnahme der Aktivität oder eine moderate Aktivität des sympathischen Zweigs des \ac{VNS}s bei Überlastung nahe.

% paragraph tozman2015 (end)
\paragraph{\citet{Harmat2015}} 

% (fold)
\label{par:harmat2015}

untersuchten Flow-Erleben beim Tetris-Spielen in drei verschiedenen Schwierigkeitsgraden. Sie stellten wie bei \citet{Keller2011, Tozman2015} eine Langeweilekondition, eine Vereinbarkeitskondition bzw. Flow-Kondition und eine Überlastungskondition her. In einer Kalibrierungsphase spielten die Teilnehmer für sechs Minuten Tetris in einem adaptiven Modus, indem die fallenden Steine auf Grundlage der Leistung des Teilnehmers schneller bzw. langsamer wurden. Die am Ende der Kalibrierungsphase zu spielende Fallgeschwindigkeit diente als Fallgeschwindigkeit in der Vereinbarkeitskondition. Die Geschwindigkeiten für die Langeweilekondition und Überlastungskondition wurden dementsprechend angepasst. 77 Teilnehmer spielten die drei Konditionen jeweils für vier oder sechs Minuten in unterschiedlicher Reihenfolge. Nach jeder Kondition beantworteten die Teilnehmer eine Teilmenge der \ac{FSS}. Die Wissenschaftler zeichneten während jedes Spiels \ac{EKG}-Daten der Teilnehmer auf. Ihre Ergebnisse zeigen einen negativen Zusammenhang zwischen frequenzbezogenen \ac{HRV}-Parameter \acs{LF} und Flow-Erleben in der Flow-Kondition. Die Ergebnisse legen eine Abnahme der Aktivität des sympathischen Zweigs des \ac{VNS}s im Flow-Erleben nahe.

% paragraph harmat2015 (end)
\paragraph{Zusammenfassend} 
% (fold)
\label{par:zusammenfassung} 

legen alle vorgestellten Studien zum Flow-Erleben und der \ac{HRV} (Tabelle~\ref{tab:studienubersicht_zu_flow_erleben}), außer die Studie von \cite{Harmat2015}, einen Zusammenhang zwischen Flow-Zuständen und steigender Aktivität des Sympathikus, gemessen durch die \ac{HRV}-Parameter \acs{LF}/\acs{HF} oder \acs{LF} und steigender \ac{HR}, nahe. Die Ergebnisse von \citet{Peifer2014} und \citet{Tozman2015} lassen aber zusätzlich auf eine optimale Aktivierung des Sympathikus vermuten, d.~h. die Aktivität des Sympathikus muss moderat bleiben, um Flow zu erleben. In der konkreten Flow-Kondition deuten die Ergebnisse von \citet{Tozman2015} sowie die Ergebnisse von \cite{Harmat2015} auf eine Abnahme der sympathischen Aktivität hin.

Bei der parasympathischen Aktivität zeichnen sich keine einheitlichen Ergebnisse ab. Die Ergebnisse von \citet{deManzano2010} und \citet{Keller2011} legen eine Abnahme der Aktivität des Parasympathikus im Flow-Zustand, gemessen durch die \ac{HRV}-Parameter \acs{HF} und \acs{RMSSD}, nahe. Das Ergebnis von \citet{Peifer2014} lässt auf einen Anstieg der parasympathischen Aktivität, gemessen durch den \ac{HRV}-Parameter \acs{HF}, schließen. Das Ergebnis von \citet{Tozman2015} deutet auf eine moderate Aktivität des Parasympathikus (\acs{HF}) während eines Flow-Zustands hin.

\begin{table}
	[!htb] \caption[Studienübersicht zu Flow-Erleben und \acs{HRV}]{Studienübersicht zu Flow-Erleben und \acs{HRV}} \label{tab:studienubersicht_zu_flow_erleben} 
	\begin{tabularx}
		{ 
		\textwidth}{p{.17 
		\textwidth} p{.05 
		\textwidth} p{.07 
		\textwidth} p{.05 
		\textwidth} X} \toprule Studie & Teil"-neh"-mer & Sitz"-ung (p.P.) & Da"-ten"-paa"-re & Ergebnis \\
		\midrule \citet{deManzano2010} & 21 & 5 & 105 & Flow = \acs{LF}/\acs{HF} $\uparrow{}$ + \ac{HR} $\uparrow{}$ + \acs{HF} $\downarrow{}$ 
		\newline Flow = Sympathikus $\uparrow{}$ + Parasymp. $\downarrow{}$ \\
		\citet{Keller2011} & 8 & 3 & 24 & Flow = \acs{RMSSD} $\downarrow{}$ 
		\newline Flow = Parasympathikus $\downarrow{}$ \\
		\citet{Gaggioli2013} & 15 & & 377 & Flow = \acs{LF}/\acs{HF} $\uparrow{}$ + \ac{HR} $\uparrow{}$ 
		\newline Flow = Sympathikus $\uparrow{}$ \\
		\citet{Peifer2014} & 22 & 1 & 22 & Flow = \acs{LF} $\cap$ + \acs{HF} $\uparrow{}$ 
		\newline Flow = Sympathikus $\cap$ + Parasymp. $\uparrow{}$ \\
		\citet{Tozman2015} & 15 & 3 & 45 & Flow = \acs{LF} $\downarrow{}$ 
		\newline Flow = Sympathikus $\downarrow{}$ 
		\newline Überlastung = \acs{LF} $\cap$ + \acs{HF} $\cap$ 
		\newline Überlastung = Sympathikus $\cap$ + Parasymp. $\cap$ \\
		\citet{Harmat2015} & 77 & 3 & 241 & Flow = \acs{LF} $\downarrow{}$ 
		\newline Flow = Sympathikus $\downarrow{}$ \\
		\bottomrule & & & & $\uparrow{}$ steigt an; $\downarrow{}$ fällt ab; $\cap$ moderate 
	\end{tabularx}
\end{table}

Zusammengefasst legen die Ergebnisse eine Änderung des Gemütszustands zur Erregung beim Flow-Erleben nahe. Diese Erregung darf aber ein gewisses \emph{individuelles Maß nicht übersteigen}. Eine \emph{gemessene Mühelosigkeit} durch den automatischen und effizienteren Ablauf der Handlungsschritte im Flow-Erleben legen die Ergebnisse von \citet{Peifer2014} sowie teilweise von \citet{Tozman2015, Harmat2015} nahe.

% paragraph zusammenfassung (end)
% subsubsection kardiovaskulare_messungen (end)
\subsection{Der Bewegungsfluss} 

% (fold)
\label{ssub:der_bewegungsfluss}

Auf der Grundlage, dass Menschen explizit durchgeführte Tätigkeiten effizienter ausführen \citep[][S.~753]{Dietrich2004}, was wir für das Flow-Erleben annehmen, betrachte ich in diesem Abschnitt einen im Kontext des \acs{BMBF}-Projekts (Abschnitt~\ref{sec:kontext_der_arbeit}) entstandenen Lösungsansatz zur impliziten Flow-Messung durch ein kinematisches Merkmal beim Gehen und Laufen.

Beim Gehen und Laufen erhalten die Füße von uns über das \acs{ZNS} eine relativ hohe Aufmerksamkeit, um eine möglichst \emph{glatte} Bewegung durchzuführen \citep[][S.~193]{Brooks1986}. Dabei ist das \acs{ZNS} für die Auslösung der willkürlichen Motorik zuständig. Motorisch von zentraler Bedeutung für den Bewegungsablauf des Gehens und des Laufens ist die Kontrolle über die Fußflugbahn, um ein sicheres Abstoßen des Fußes (\ac{IS}) und eine sanfte Landung des Fußes (\ac{IC}) zu erreichen \citep[][S.~197]{Winter1989}. Die beiden genannten biomechanischen Schlüsselereignisse der Standphase sind nach \citet{Aminian2002, Lee2011} von zentraler Bedeutung für eine verzögerungsarme Erkennung zusätzlicher Eigenschaften. Anhand eines Schlüsselereignisses wie \ac{IS}, \ac{IC} oder \ac{MS} lässt sich der Bewegungsablauf in einzelne Zyklen trennen. Jeden Zyklus betrachten wir als Handlungsschritt der Tätigkeit Gehen und Laufen, den der Mensch mehr oder weniger glatt bzw. fließend oder eben effizient ausführt. Die Effizienz einer Bewegung drückt sich im Bewegungsfluss aus. 

Die Quantifizierung des Bewegungsflusses erfolgt über den Bewegungsaufwand (jerk cost) \citep{Nelson1983, Hogan1984, Flash1985}. Sie definieren \emph{jerk} als zeitliche Änderung der Beschleunigung. Der Bewegungsaufwand ist die Summe der im Verlauf einer Bewegung erfolgten Beschleunigungsänderungen \citep{Schneider1990}. \citet[][S.~1698]{Flash1985} führten an, dass Bewegungen um so fließender sind, je kleiner der Beschleunigungsaufwand ist. Sie überprüften ihre Hypothese für eindimensionale Armbewegungen und verglichen die durchgeführten Bewegungen des Experiments mit den berechneten Ergebnissen des geringst möglichen Bewegungsaufwands (minimum-jerk movement).

Den Bewegungsaufwand nutzte \citet{Hreljac2000}, um eine Gruppe ambitionierter Läufer mit einer Gruppe von Freizeit-Läufern beim Laufen und schnellem Gehen zu vergleichen. Er filmte die Läufer auf einem Laufband mit einer Videokamera und extrahierte 2D-Bewegungsdaten. Er berechnete den Beschleunigungsaufwand an der Ferse, da die Ferse als Teil des Fußes eine bedeutende Aufgabe im Bewegungsfluss des Gehens und Laufens besitzt. \citet{Hreljac2000} zeigte, dass sich der Bewegungsfluss der beiden Tätigkeiten objektiv durch Auswertung des Beschleunigungsaufwandes an der Ferse quantifizieren lässt und dass ambitionierte Läufer eine \emph{glattere} Bewegung beim Laufen und schnellem Gehen durchführen als Freizeit-Läufer.

Unsere These, dass Menschen im Flow-Erleben ihren Bewegungsablauf beim Gehen und Laufen fließender ausführen, was wir anhand des Bewegungsflusses implizit messen können, wird von der nachfolgenden Passage aus \citet[][S.~121]{Meinel2007} unterstützt:

„Als eine psychische Erscheinungsform sind Emotionen zu werten, die seit langem als »Flow«-Erlebnis bekannt sind. Gemeint ist damit das positive Befinden, das Erfolgserlebnis, das mit einem reibungslosen, »fließenden« Bewegungsablauf verbunden ist \textellipsis.“

% subsection der_bewegungsfluss (end)
\subsection{Die kardio-lokomotorische Phasensynchronisation} 

% (fold)
\label{ssub:die_kardio_lokomotorische_phasensynchronisation}

Im Kontext des \acs{BMBF}-Projekts entwickelte Barbara Grüter die Hypothese, dass die kardio-lokomotorischen Phasensynchronisation als implizites Merkmal des (Flow-) Erlebens beim Gehen zu verwenden ist. Wir überprüften die Hypothese in einer Machbarkeitsstudie im Rahmen des \acs{BMBF}-Projekts (Abschnitt~\ref{sec:flow_und_gehen_intraindividuell}) in Zusammenarbeit mit Licinio Roque und Rui Craveirinha (beide von der Universität Coimbra, Portugal). 
\begin{itemize}
	
	\item Die Eigenschaft der \emph{Zeitbezogenheit} und der \emph{unkomplizierten interindividuellen Vergleichbarkeit} der kardio-lokomotorischen Phasensynchronisation motivierten mich, die Hypothese im Rahmen dieser Dissertation auf das Laufen zu übertragen.
	
	\item Wir ordnen die kardio-lokomotorische Phasensynchronisation zwischen der physiologischen und der motorischen Ebene ein, da sie von beiden Ebenen abhängig ist \citep{Grueter2016a}.
\end{itemize}

Bei der kardio-lokomotorischen Phasensynchronisation handelt es sich um ein zeitbezogenes Phänomen, das das Zusammenwirken des Herzens und der Bewegung betrachtet. Wir berechnen sie mit Hilfe der Zeitpunkte von Herzschlägen und Schritten. Die Phasensynchronisation bzw. Kopplung stellt sich in unterschiedlichen Verhältnissen dar (z.~B. ein Herzschlag folgt auf zwei Schritte). Zur Quantifizierung der kardio-lokomotorischen Phasensynchronisation dienen verschiedene Indizes, z.~B. ein Index basierend auf der Intensität des Mittelwerts der Fourier Reihe der Verteilung der Herzschläge im Bewegungszyklus -- Phasenkohärenz Index \citep{Rosenblum2003} oder basierend auf der Shannon Entropie -- Shannon Entropie Index \citep{Tass1998, Niizeki2005}. Zur Darstellung der kardio-lokomotorischen Phasensynchronisation dient ein \emph{Phasen Stroboscope} \citep{Mrowka2000} oder auch die Synchrogramm-Technik \citep{Schafer1999}.

Ein ökonomisches Zusammenwirken der Systeme Herz und Bewegungsapparat betrachten \citet[S.~18]{Niizeki2014} als einen energetisch vorteilhaften Zustand des menschlichen Organismus. Die Ergebnisse von \citet{Phillips2013} legen z.~B. für das Laufen nahe, dass eine herbeigeführte kardio-lokomotorische Phasensynchronisation mit einer Verbesserung der Laufleistung in Verbindung steht.

Wir gehen davon aus, dass eine starke kardio-lokomotorische Phasensynchronisation der ideale Zustand zwischen beiden Systemen ist und etwas mit dem \emph{gänzlichen Aufgehen in der Tätigkeit} zu tun hat. Aus diesem Grund vermuten wir einen Zusammenhang zwischen den beiden Idealzuständen \emph{Flow-Erleben} und \emph{hoher kardio-lokomotorischer Phasensynchronisation} \citep{Grueter2016a}.

% subsubsection die_kardio_lokomotorische_phasensynchronisation (end)
% subsection physiologische_merkmale_des_flow_erlebens (end)
% section losungsansatze_zur_impliziten_messung_von_flow_erleben (end)

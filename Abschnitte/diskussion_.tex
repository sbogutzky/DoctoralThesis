%!TEX root = /Users/sbogutzky/Entwicklung/projects/bogutzky/repositories/2939413/final-draft.tex
\chapter{Diskussion}
\label{cha:diskussion}
Im vorherigen Kapiteln motivierte ich das Erleben von Flow in einer Echtzeit-Benutzerschnittstelle für Sport und Gesundheit zu nutzen. 

Ich stellte hierzu Kandidaten für ein implizites Messverfahren des Flow-Erlebens vor und fand direkten intraindividuellen Zusammenhang von Flow-Erleben und Doppelschrittfrequenz (Abschnitt x) und eien indirekten Zusammenhang zwischen Flow-Erleben und der kardio-lokomotorische Phasensynchronisation als eine implizit \emph{physiologisch messbare \ac{AFP}} (Abschnitt). Im diesem Kapitel resümiere ich die wesentlichen Erkenntnisse meiner Arbeit und diskutiere sie zusammenfassend. 

Die vorliegende Arbeit ist eine der ersten Arbeiten, die Zusammenhänge von Flow-Erleben und physiologischen Merkmalen unter einer physisch beanspruchenden Tätigkeit untersuchte. Ich versuchte in meiner Forschung an die Arbeiten von \citet{deManzano2010, Keller2011, Gaggioli2013, Peifer2014, Tozman2015}, die allesamt Zusammenhänge zwischen Flow-Erleben und der Herzfrequenzvariabilität nachweisen konnten, anzuknüpfen. Aber gerade die Interpretation von HRV-Merkmalen, die die sympathische und parasympathische Aktivität des \ac{VNS} nachweisen sollen, sind unter physischer Belastung nahezu unmöglich. Grund dafür ist die dramatische Reduktion der \ac{HRV} unter physischer Belastung \citep[][]{Hoos2010}. 

Aus diesem Grund suchte wir im Kontext des \emph{BMBF}-Projekt nach Kandidaten für ein implizites Messverfahren des Flow-Erlebens, die auf der Biomechanik des Gehens und Laufens beruhen. Dabei stieß Nassrin Hajinejad bei \citet[][S.~121]{Meinel2007} auf den Bewegungsfluss und ich knüpfte zur Quantifizierung an die Arbeit von \citet{Hreljac2000} über Bewegungsaufwand an (Abschnitt). Er nutzte den Bewegungsaufwand, um eine Gruppe ambitionierter Läufer mit einer Gruppe von Freizeitläufern beim Laufen und schnellem Gehen zu vergleichen. Ich konnte in dieser Arbeit keinen direkten Zusammenhang nachweisen. Ein Grund könnte die Filterung durch individuelle die Grenzfrequenzen darstellen, da eine zu hohe Filterung viele Merkmale entfernt. 
In meiner finalen Laufstudie habe ich den Bewegungsaufwand vernachlässigt, da der Nachweis von Flow-Erleben durch den Bewegungsaufwand personenindividuell wäre und bei variierenden Laufgeschwindigkeiten eine komplexere Aufgabe darstellt wird.

Für einen solchen Nachweis wäre eine Laufbandstudie nach dem Beispiel von \citet{Reinhardt2006} die geeignete Methode, um zumindest die Laufgeschwindigkeit konstant zu halten und für alle Untersuchungspersonen vergleichbar zu machen. Weiterhin würde eine Laufbandstudie externe Störfaktoren, die in meiner Studie vorhanden waren und auch Einfluss auf das Erleben besitzen, eliminieren. Es ist aber zu bedenken, dass die Ergebnisse nur bedingt auf die Realität übertragbar sind. \citet{Strohrmann2012} stellten zum Beispiel fest, dass das Laufen auf einem Laufband sich von dem Laufen in Außen-Umgebungen in Bezug auf kinematische Merkmale unterscheidet. Nichtsdestotrotz ist eine Laufbandstudie ein logischer Schritt, da ich zumindest den Einfluss auf die kardio-lokomotorische Phasensynchronisation eher gering einschätze. 

In der intraindividuellen Studie konnte ich einen Nachweis erbringen, das ein Zusammenhang zwischen Flow-Erleben und optimale Doppelschrittfrequenz vorhanden ist. In der finalen Studie konnte ich diesen Zusammenhang nicht nachweisen, da jeder der 31 Untersuchungsperson seine eigene individuelle Schrittfrequenz besitzt.  

Ein implizites Merkmal, welches sich unkompliziert interindividuell vergleichen lässt, stellt die kardio-lokomotorische Phasensynchronisation dar. Die kardio-lokomotorische Phasensynchronisation (Abschnitt). \citet[][S.~18]{Niizeki2014} sehen in hoher kardio-lokomotorischen Phasensynchronisation einen energetisch vorteilhaften Zustand des menschlichen Organismus und Barbara Grüter entwickelte im Kontext des \acs{BMBF}-Projekts die Hypothese, dass die kardio-lokomotorischen Phasensynchronisation als implizites Merkmal des (Flow-) Erlebens beim Gehen zu verwenden ist. Wegen den beiden positiv zu bewertenden Eigenschaften der kardio-lokomotorischen Phasensynchronisation: 
\begin{itemize}
	
	\item relative Prozessbezogenheit mit markanten Mustern und
	
	\item unkomplizierte interindividuelle Vergleichbarkeit
\end{itemize}

übernahm ich die kardio-lokomotorischen Phasensynchronisation konnte aber in meine Studien direkten Zusammenhänge zum Flow-Erleben feststellen. 

Hierfür mache ich teilweise die Operationalisierung des Flow-Erlebens durch die \ac{FKS} verantwortlich. Die Items des glatten Verlaufs wurden eher positiv bewertet. Das erschert eine Differzierung über diese Subdimension. Eine Bewertung des glatten Verlaufs (sechs Items) hat zusätzlich einen größeren Einfluß auf den Generalfaktor als die Bewertung der Absorbiertheit (vier Items). Eine bessere Deferenzierbarkeit bietet die Absorbiertheit, aber auch in dieser Skala gab es ein Problem mit dem Item \emph{Ich fühle mich optimal beansprucht}. In der beschriebenen finalen Studie wurde dieses Item wie die Items des glatten Verlaufs eher positiver bewertet, da optimal gegenüber den anderen Items der Skala unterschiedlich auffasst werden kann. Eine Untersuchungsperson ist z.~B. optimal beansprucht, wenn sie am Limit ihrer Leistungsfähigkeit ist und eine andere Untersuchungsperson ist optimal beansprucht, wenn sie beim Laufen sprechen kann. Die anderen Items der Skala Absorbiertheit sind sprachlich eindeutiger ausgeführt worden -- zeitvergessen, vertieft in die Tätigkeit und selbstvergessen. Damit ist dieses Item zumindestens beim Laufen nicht trennscharf der Absorbiertheit zuzuordnen. 

Trotz dieser Kritik an der eingesetzen expliziten Referenz Messmethode habe ich einen indirekten Zusammenhang zwischen Flow-Erleben und kardio-lokomotorischer Phasensynchronisation und als eine implizit \emph{physiologisch messbare \ac{AFP}} gefunden. Dieser Zusammensammhang besagt, dass 

die Untersuchungspersonen, die eine kardio-lokomotorische Phasensynchronisation herstellen konnten gegenüber den anderen Untersuchungspersonen vertiefter in die Tätigkeit des Laufens waren (gemessen an der Absorbiertheit).

Die Absorbiertheit stellt zwar nur ein Teilkonstrukt des Flow-Erlebens dar, ist

Die technische mobile Umsetzung in Echtzeit eines impliziten Messverfahrens einer der beiden favorisierten Kandidaten ist heutzutage durch die hohen Rechenleistungen von Smartphones bedenkenlos möglich. Das zusätzliche Equipment wäre aber für den Freizeitgebrauch noch nicht preiswert und komfortabel genug. Die Untersuchungsperson musste in meinen Studien mindestens eine \ac{IMU} am unteren Bein und einen hochwertigen Brustgurt zur Herzfrequenzmessung tragen, damit der \ac{PPC} die benötigten Daten erheben konnte. Mit dem \emph{BioHarness 3} nutzte ich einen Brustgurt, der eine hohe \ac{EKG}-Qualität gewährleistet. Das macht den \emph{BioHarness} mit einem Anschaffungspreis von über 500 EUR kostspielig für den Freizeitbedarf. 

Eine laufende Person müsste zusätzlich zum \emph{BioHarness 3} mindestens eine tragbare \ac{IMU} zusätzlich tragen. Eine Möglichkeit, auf ein zusätzliches technisches Gerät zu verzichten, ist, ein Konzept zu entwickeln, das das Smartphone selbst als Datenerhebungsgerät verwendet \citep[vgl.][]{Strohrmann2013, Strohrmann2014}. Wie ich in Abschnitt~\ref{sec:demonstrator} dokumentiert habe, ist ein solches Konzept beim Gehen realisierbar, in dem die Person das Smartphone in der Hosentasche trägt. Damit ist das Konzept aber auf Jeans mit eng anliegenden Taschen beschränkt und darum nicht für jeden Benutzer alltagstauglich. Beim Laufen sehe ich diese Möglichkeit, das Smartphone in der Hosentasche zu tragen, noch weniger, da viele Sporthosen gar keine oder weite Hosentaschen haben. Eine Alternative zur Bestimmung des Bewegungsablaufs am Bein ist die Bestimmung des Bewegungsablaufs am Oberarm, da die Armarbeit entsprechend den Gesetzen der Kreuzkoordination bei geübten Läufern die Beinarbeit bestimmt \citep[vgl.][S.~70]{Marquardt2011}.
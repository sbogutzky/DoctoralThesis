%!TEX root = /Users/sbogutzky/Entwicklung/projects/bogutzky/repositories/2939413/final-draft.tex
\chapter{Diskussion}
\label{cha:diskussion}
In den bisherigen Kapiteln motivierte ich das Erleben von Flow in einer Echtzeit-Benutzerschnittstelle für Sport und Gesundheit oder präziser in einer Lauf-App zu nutzen. Ich stellte hierzu Kandidaten für ein implizites Messverfahren des Flow-Erlebens vor und argumentierte für die kardio-lokomotorische Phasensynchronisation als eine Voraussetzung, um Flow beim Laufen zu erleben. Zusätzlich sehe ich im Bewegungsaufwand eine personenindividuelle Folge des Flow-Erlebens. Es gelang mir in meinen Studien nicht, eine statistisch signifikante Gültigkeit dieser Annahmen zu dokumentieren. Im gegenwärtigen Kapitel resümiere ich die wesentlichen Ergebnisse meiner Arbeit und diskutiere sie zusammenfassend. 

Die vorliegende Arbeit ist eine der ersten Arbeiten, die Zusammenhänge von Flow-Erleben und physiologischen Merkmalen unter einer physisch beanspruchenden Tätigkeit untersuchte. Ich versuchte in meiner Forschung an die Arbeiten von \citet{deManzano2010, Keller2011, Gaggioli2013, Peifer2014, Tozman2015}, die allesamt Zusammenhänge zwischen Flow-Erleben und der Herzfrequenzvariabilität nachweisen konnten, anzuknüpfen. Aber gerade die Interpretation von Merkmalen, die die sympathische und parasympathische Aktivität des \ac{VNS} nachweisen sollen, sind unter physischer Belastung nahezu unmöglich. Grund dafür ist die dramatische Reduktion der \ac{HRV} unter physischer Belastung \citep[vgl.][]{Hoos2010}. 

Ich suchte daher nach Kandidaten für ein implizites Messverfahren des Flow-Erlebens, die auf der Biomechanik des Gehens und Laufens beruhen. Dabei stieß ich bei \citet[][S.~121]{Meinel2007} auf den Bewegungsfluss und knüpfte zur Quantifizierung an die Arbeit von \citet{Hreljac2000} über Bewegungsaufwand an. Er nutzte den Bewegungsaufwand, um eine Gruppe ambitionierter Läufer mit einer Gruppe von Freizeitläufern beim Laufen und schnellem Gehen zu vergleichen. Ich konnte in dieser Arbeit teilweise nachweisen, dass der Bewegungsaufwand bei einem Läufer sinkt, wenn er Flow erlebte (siehe Abschnitt~\ref{par:zusammenhang_zwischen_glattem_verlauf_und_bewegungsaufwand}). Dieser Nachweis ist personenindividuell möglich gewesen, da die Untersuchungsperson ihre Laufgeschwindigkeit von Sitzung zu Sitzung nicht groß veränderte. Der Nachweis wird bei variierenden Laufgeschwindigkeiten komplex, da bei niedrigen oder höheren Lauf- bzw. Gehgeschwindigkeiten der Bewegungsaufwand schrumpft bzw. wächst. Ich konnte in meiner finalen Laufstudie mit mehreren unterschiedlichen Untersuchungspersonen, die unterschiedlich schnell liefen und unterschiedlich viel Kraft in ihre Laufbewegung investierten, keinen Zusammenhang zwischen Flow-Erleben und Bewegungsaufwand nachweisen. 

Für einen solchen Nachweis wäre eine Laufbandstudie nach dem Beispiel von \citet{Reinhardt2006} die geeignete Methode, um zumindest die Laufgeschwindigkeit konstant zu halten und für alle Untersuchungspersonen vergleichbar zu machen. Weiterhin würde eine Laufbandstudie externe Störfaktoren, die in meiner Studie vorhanden waren und auch Einfluss auf das Erleben besitzen, eliminieren. Es ist aber zu bedenken, dass die Ergebnisse nur bedingt auf die Realität übertragbar sind. \citet{Strohrmann2012} stellten zum Beispiel fest, dass das Laufen auf einem Laufband sich von dem Laufen in Außen-Umgebungen in Bezug auf kinematische Merkmale unterscheidet. Nichtsdestotrotz ist eine Laufbandstudie ein logischer Schritt, da ich zumindest den Einfluss auf die kardio-lokomotorische Phasensynchronisation eher gering einschätze. 

\citet[][S.~18]{Niizeki2014} sehen in hoher kardio-lokomotorischen Phasensynchronisation einen energetisch vorteilhaften Zustand des menschlichen Organismus. Ein Zusammenhang zum Flow-Erleben beim Gehen und Laufen wurde bisher in der Flow-Forschung nicht hergestellt. Ich konnte den statistischen signifikanten Nachweis für einen Zusammenhang in der vorliegenden Arbeit nicht liefern, aber die Ergebnisse der Studien deuten auf einen solchen Zusammenhang hin. 
 
Die kardio-lokomotorische Phasensynchronisation sehe ich als einen sehr geeigneten Kandidaten für ein implizites Messverfahren des Flow-Erlebens beim Laufen, da sie auf einen Anpassungsprozess von zwei Systemen des Körpers beruht. Dieses führt zu einer teilweisen Normalisierung und macht die kardio-lokomotorische Phasensynchronisation zwischen unterschiedlichen Untersuchungspersonen bequem vergleichbar. Ich konnte aber in meiner finalen Laufstudie den gewünschten positiven Zusammenhang zwischen Flow-Erleben und kardio-lokomotorischer Phasensynchronisation nicht finden. 

Hierfür mache ich zum einen die Auswahl der Untersuchungspersonen und zum anderen die Operationalisierung durch die \ac{FKS} verantwortlich. Die Untersuchungspersonen bewerteten in meiner finalen Laufstudie auch bei Überlastung die Absorbiertheit der \ac{FKS} sehr hoch (siehe Abschnitt~\ref{sub:diskussion_3}). Der Begriff optimal wurde von den Untersuchungspersonen sehr unterschiedlich interpretiert. Keine Beanspruchung der Vorbereitung auf den Lauf bewerteten vier Untersuchungspersonen optimaler als die Beanspruchung beim Laufen. Deswegen erhielt ich von diesen vier Untersuchungspersonen der finalen Laufstudie in der \emph{Baseline}-Messung höhere Werte für die Absorbiertheit als nach dem Lauf. Ich sehe, wie schon in den vorherigen Diskussionsabschnitten erläutert, die Operationalisierung des Flow-Erlebens durch die \ac{FKS} kritisch und würde bei einer nächsten Studie das Instrument des strukturierten Interviews vorziehen.

Die technische mobile Umsetzung in Echtzeit eines impliziten Messverfahrens einer der beiden favorisierten Kandidaten ist heutzutage durch die hohen Rechenleistungen von Smartphones bedenkenlos möglich. Das zusätzliche Equipment wäre aber für den Freizeitgebrauch noch nicht preiswert und komfortabel genug. Die Untersuchungsperson musste in meinen Studien mindestens eine \ac{IMU} am unteren Bein und einen hochwertigen Brustgurt zur Herzfrequenzmessung tragen, damit der \ac{PPC} die benötigten Daten erheben konnte. Mit dem \emph{BioHarness 3} nutzte ich einen Brustgurt, der eine hohe \ac{EKG}-Qualität gewährleistet. Das macht den \emph{BioHarness} mit einem Anschaffungspreis von über 500 EUR kostspielig für den Freizeitbedarf. 

Eine laufende Person müsste zusätzlich zum \emph{BioHarness 3} mindestens eine tragbare \ac{IMU} zusätzlich tragen. Eine Möglichkeit, auf ein zusätzliches technisches Gerät zu verzichten, ist, ein Konzept zu entwickeln, das das Smartphone selbst als Datenerhebungsgerät verwendet \citep[vgl.][]{Strohrmann2013, Strohrmann2014}. Wie ich in Abschnitt~\ref{sec:demonstrator} dokumentiert habe, ist ein solches Konzept beim Gehen realisierbar, in dem die Person das Smartphone in der Hosentasche trägt. Damit ist das Konzept aber auf Jeans mit eng anliegenden Taschen beschränkt und darum nicht für jeden Benutzer alltagstauglich. Beim Laufen sehe ich diese Möglichkeit, das Smartphone in der Hosentasche zu tragen, noch weniger, da viele Sporthosen gar keine oder weite Hosentaschen haben. Eine Alternative zur Bestimmung des Bewegungsablaufs am Bein ist die Bestimmung des Bewegungsablaufs am Oberarm, da die Armarbeit entsprechend den Gesetzen der Kreuzkoordination bei geübten Läufern die Beinarbeit bestimmt \citep[vgl.][S.~70]{Marquardt2011}.
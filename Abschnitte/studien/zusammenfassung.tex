

%!TEX root = /Users/sbogutzky/Entwicklung/projects/bogutzky/repositories/2939413/final-draft.tex
\section{Zusammenfassung} \label{sec:zusammenfassung_5}

In diesem Kapitel beschreibe ich drei Studien zum Flow-Erlben Die erste Studie (Abschnitt~\ref{sec:flow_und_laufen_intraindividuell}) galt der suche nach Zusammenhängen zwischen subjektiven durch Experience Sampling erhobenen expliziten Flow-Merkmalen und Kandidaten für ein implizites Messverfahren des Flow-Erlebens beim Laufen. In ihr fand ich die nachfolgenden intraindividullen Zusammenhänge: 
\begin{itemize}
	\item Flow-Erleben, gemessen durch den Generalfaktor der \ac{FKS}, und die Doppelschrittfrequenz stehen in einen quadratischen Zusammenhang in Form eines umgedrehten Us. Demzufolge begünstigt eine individuelle optimale Doppelschrittfrequenz das Flow-Erleben. Dieses Ergebnis steht in einer Linie mit der These von \citet[][S.~148]{Peifer2012}, in der das Flow-Erleben mit optimaler physiologischer Aktivierung (Optimized Physiological Activation) für die entsprechende Aktivität einhergeht, während alle anderen Prozesse runterreguliert werden.
	
	\item Der Bewegungsaufwand und die Doppelschrittfrequenz stehen in einem positiven linearen Zusammenhang, d. h. umso tiefer/höher die Doppelschrittfrequenz ist, umso tiefer/höher ist der Bewegungsaufwand.
	
	\item Der Bewegungsaufwand und die mittlere \ac{HR} stehen in einen quadratischen Zusammenhang in Form eines Us. Demzufolge geht eine optimale mittlere \ac{HR} mit einem geringeren Bewegungsaufwand einher.
	
	\item Ein ähnlicher Zusammenhang besteht zwischen der kardio-lokomotorischen Phasensynchronisation, gemessen durch den mittleren normalisierten Shannon Entropie Index, und der mittleren \ac{HR}. Dieser quadratische Zusammenhang hat die Form eines umgedrehten Us und bedeutet, dass eine optimale mittlere \ac{HR} eine hohe kardio-lokomotorische Phasensynchronisation begünstigt. 
	
	\item Ein positiver linearer Zusammenhang besteht zwischen der \ac{AFP} und der mittleren \ac{HR}, umso tiefer/höher die mittlere \ac{HR} umso geringer/höher bewertete der Läufer die \ac{AFP}. 
\end{itemize}

Aufgrund der Erkenntnisse der ersten Laufstudie stellte ich die Annahme auf, das die kardio-lokomotorische Phasensynchronisation eine \emph{physiologisch messbare \ac{AFP}} darstellt und damit eine Verbindung zum Flow-Kanalmodell (Abbildung~\ref{fig:kanalmodell}) bildet.

In der (zeitlich gesehen) zweite Studie (Abschnitt~\ref{sec:flow_und_gehen_intraindividuell}) untersuchten wir die Tätigkeit des Gehens intraindividuell mit dem Untersuchungsaufbau der ersten Laufstudie. Sie diente als Machbarkeitstudie des \acs{BMBF}-Projekts zur kardio-lokomotorischen Phasensynchronisation. 

Die beiden in dieser Studie identifizierten positiv zu bewertenden Eigenschaften der kardio-lokomotorischen Phasensynchronisation: 
\begin{itemize}
	
	\item relative Prozessbezogenheit mit markanten Mustern und
	
	\item unkomplizierte interindividuelle Vergleichbarkeit 
\end{itemize}

gaben mir den Anlass, die kardio-lokomotorische Phasensynchronisation im Nachhinein in der ersten Laufstudie zu überprüfen. 

In Abschnitt~\ref{sec:flow_und_laufen_interindividuell} dokumentiere ich meine finale Studie, in der erneut das Laufen im Fokus steht. In ihr überprüfte ich einen direkten und indirekten Zusammenhang von Flow-Erleben und kardio-lokomotorischer Phasensynchronisation. 
\begin{itemize}
	
	\item In ihr konnte bestätigen, dass die Untersuchungspersonen, die eine kardio-lokomotorische Phasensynchronisation herstellen konnten, vertiefter in die Tatigkeit des Laufens waren als die Untersuchungspersonen, die keine kardio-lokomotorische Phasensynchronisation herstellen konnten. 
\end{itemize}

Einige Ausnahme (Abschnitt~\ref{sub:diskussion_5_3}) erschweren den prozessorientierten Einsatz der kardio-lokomotorische Phasensynchronisation als \emph{implizit messbare \ac{AFP}}, machen sie aber nicht unmöglich. In Kapitel x veranschauliche ihre beispielhafte Integration in eine assistierende Echtzeit-Benutzerschnittstelle.

Eine genaue Bestimmung von Flow-Erleben im Prozess bleibt dieser Arbeit verwehrt, da ich keine eindeutigen Hinweise darauf habe, das der Eintritt in eine hohe kardio-lokomotorische Phasensynchronisation mit dem Eintritt in den Zustand \emph{flow} (Abbildung~\ref{fig:prozessorientiertes_flow_modell_2}) und die Rückkehr in eine niegrige kardio-lokomotorische Phasensynchronisation mit dem Eintritt in den Zustand \emph{nicht flow} Abbildung~\ref{fig:prozessorientiertes_flow_modell_2}) einhergeht. Hirzu sind zusätzliche Untersuchungen mit hinreichenden befriedigenden Ergebnissen notwendig.



%!TEX root = /Users/sbogutzky/Entwicklung/projects/bogutzky/repositories/2939413/final-draft.tex
\section{Flow und Laufen (intraindividuell)} 

% (fold)
\label{sec:flow_und_laufen_intraindividuell}

\section{Einleitung} % (fold)
\label{sec:einleitung_5_1}

Ziel der in diesem Abschnitt beschriebenen Studie war es, zu untersuchen, ob subjektive, durch Experience Sampling erhobene, Flow-Merkmale mit den nachfolgenden Kandidaten für ein implizites Messverfahren des Flow-Erlebens beim Laufen in einem Zusammenhang stehen (Abschnitt~\ref{sec:herangehensweise}, Schritt 2):

\begin{itemize}
	\item mittlere HR
	\item \acs{RMSSD} der zeitbezogenen statitischen \ac{HRV}-Analyse
	\item Bewegungsaufwand (Abschnitt~\ref{ssub:der_bewegungsfluss}) oder
	\item normalisierter Shannon Entropie Index der kardio-lokomotorischen Phasensynchronisation (Abschnitt~\ref{ssub:die_kardio_lokomotorische_phasensynchronisation})
\end{itemize}

Die Auswahl der Merkmale begründe ich durch:

\begin{enumerate}
	\item ihre nachgewiesenen Zusammenhänge zum Flow-Erleben bei Tätigkeiten mit geringer physischer Beansprung (Abschnitt~\ref{ssub:kardiovaskulare_messungen}) 
	\item die Praktikabilität bei der Ausführung einer physisch belastenden Tätigkeit
\end{enumerate}

Aufgrund der durch das \acs{BMBF}-Projekt bereitgestellten Shimmer \acp{IMU}, nutze ich ein Shimmer EKG-Modul, das auf die Hauptplatine einer Shimmer \ac{IMU} aufgesetzt wird. Das Shimmer EKG-Modul arbeitet mit der Ableitung nach Einthoven \citep[][S.~85ff.]{Behrends2002}. Es handelt sich um eine bipolare Extremitätenableitung, die man routinemäßig mit drei Elektroden plus einer Erdungselektrode erfasst. Ich platziere vier konventionelle Einwegelektroden auf der Körperoberfläche und verbinden sie mit vier Kabelverbindungen mit dem Shimmer EKG-Modul wie in Abbildung~\ref{fig:5_1_equipment_setup} dargestellt. Damit erfasst das Shimmer EKG-Modul mit zwei Kanälen die zwei Ableitungen (II-III). Ableitung I berechne ich, indem ich das Signal von LA-LL von dem Signal von RA-LL subtrahiere.

\begin{itemize}
	\item Ableitung I: zwischen rechtem und linkem Arm (RA-LA)
	\item Ableitung II: zwischen rechtem Arm und linkem Bein (RA-LL)
	\item Ableitung III: zwischen linkem Arm und linkem Bein (LA-LL)
\end{itemize}

Die Ableitung nach Einthoven dient zur Darstellung von Potenzialänderungen in der Frontalebene und ermöglicht u. a. die Identifikation von Herzschlägen, die wir zur Analyse der \ac{HRV} benötigen.

Wie in Abschnitt~\ref{ssub:kardiovaskulare_messungen} zusammengefasst, handelt es sich bei der \acs{RMSSD} um ein physiologisches Merkmal der zeitbezogenen statitischen \ac{HRV}-Analyse, mit dem \citet{Keller2011} bei einer sitzenden Tätigkeit Zusammenhänge mit Flow-Erleben feststellten. Auf Zusammenhanganalysen mit frequenzbezogenen \ac{HRV}-Merkmalen verzichtete ich aufgrund der in Abschnitt~\ref{sub:zuordnung} beschriebenen Probleme.

Kinematische Daten messe ich mit den bereitgestellten Shimmer \acp{IMU} und einem zusätzlichen Shimmer Gyro-Modul. Die kinematischen Daten benötige ich, um einzelne sich wiederholende Bewegungsabläufe der Bewegung zu erkennen und den Bewegungsaufwand zu berechnen. Zusätzliche kinematische Merkmale, die z.~B. Abschnitt~\ref{sub:lauftechnik_detektionstechnologie} beschreibt, berücksichtigte ich nicht.

Die Anordnung des Equipments und die Zuverlässigkeit der Datenaufzeichnung testete ich in Vorab-Tests mit drei Freiwilligen aus der Fakultät 4 der Hochschule Bremen. Für die Vorab-Tests verwendete ich denselben Aufbau des Systems, der bei dieser Studie zum Einsatz kam (siehe Abschnitt~\ref{sub:apparat}).

\begin{figure}[t]
	\centering
		\includegraphics[width=1.00\textwidth]{5_1_equipment_setup}
	\caption[Equipment der ersten Studie zum Flow-Erleben beim Laufen]{Equipment der ersten Studie zum Flow-Erleben beim Laufen}
	\label{fig:5_1_equipment_setup}
\end{figure}

% section einleitung (end)

\section{Methode} % (fold)
\label{sec:methode}

\citet[][S.~989]{Strohrmann2012} argumentieren, Untersuchungen in Außen-Umgebungen durchzuführen, da sich das Laufen auf einem Laufband von dem Laufen in Außen-Umgebungen in Bezug auf kinematische Merkmale unterscheidet. Aus dem genannten Grund entschied ich mich, die Läufe unter realen Bedingungen durchzuführen. Untersuchungen außerhalb von Laboratorien stellen bei ihrer Durchführung eine Herausforderung dar. Außen-Umgebungen sind in einem hohen Maße veränderlich und kontextsensitive Faktoren als Einflüsse auf Merkmale sind sind grundsätzlich als gegeben anzunehmen. Gleichwohl sind solche Untersuchungen unerlässlich, um eine realistischere Einschätzung der Nutzung in Außen-Umgebungen zu ermöglichen.

Für die in diesem Abschnitt dokumentierte Studie setzte ich ein Echtzeit-Datenerfassungsverfahren (real-time data capture) ein. Die Datenerfassung umfasst \ac{EKG}-Daten und kinematische Daten sowie Selbstauskünfte durch die \ac{FKS}. Die aufgezeichneten Datenströme eines einzelnen Bewegungsablaufs beim Laufen sind in Abbildung~\ref{fig:5_2_daten} dargestellt. Der Bewegungsablauf beginnt und endet mit dem mittleren Schwung des rechten Beines (negative Signalspitze der Winkelgeschwindigkeit um die X-Achse (grün)).

\begin{sidewaysfigure}
	\resizebox{1.00\textwidth}{!}{%
	    \input{./tikz/5_2_daten}
	}%
	\caption[Aufgenommene Datenströme beim Laufen]{Aufgenommene Datenströme eines Bewegungsaublaufs in der ersten Studie zum Flow-Erleben beim Laufen. Von links nach rechts: (schwarz) EKG-Ableitung RA-LL und EKG-Ableitung LA-LL; (grün) Beschleunigung in X-Richtung und Winkelgeschwindigkeit um die X Achse; (blau) Beschleunigung in Y-Richtung und Winkelgeschwindigkeit um die Y-Achse; (rot) Beschleunigung in Z-Richtung und Winkelgeschwindigkeit um die Z-Achse}
	\label{fig:5_2_daten}
\end{sidewaysfigure}

Ich orientiere mich am Vorgehen des Experiments von \citet{Reinhardt2006} und der Studie von \citet{Schuler2009}. Wie bei Reinhardt führe ich eine Befragung an vordefinierten Zeitmarken durch, an denen die Untersuchungsperson die Lauftätigkeit kurzzeitig unterbricht. Im Gegensatz zu \citet{Reinhardt2006} ist der zweite Schritt der beschriebenen Studie in der Korrelationsforschung anzusiedeln. Der erste Schritt ist das Sammeln von Daten und dritte Schritt ist die Untersuchung des zeitlichen Verhaltens (Prozess) der gesammelten Daten (Abschnitt~\ref{sec:herangehensweise}). Aufgrund der gegebenen logischen Abhängigkeit von Schritt 2 und 3 liegt das Augenmerk zunächst auf Schritt 2, der Suche nach signifikanten Korrelationen zwischen expliziten Merkmalen und impliziten Kandidaten des Flow-Erlebens. Das Ziel ist bei Kenntnis der Werte der impliziten Merkmale (Prädiktoren) der Werte der expliziten Flow-Merkmale (Kriterien) vorherzusagen. Als Kreterium diente unabhängig voneinander der Generalfaktor, der erste und der zweite Faktor der \ac{FKS} und als Prädiktor diente unabhängig voneinander die mittlere \ac{HR}, die \acs{RMSSD} der zeitbasierten \ac{HRV}-Analyse, der Bewegungsaufwand und der normalisierte Shannon Entropie Index der kardio-lokomotorischen Phasensynchronisation.

Mit Blick auf die Durchführung der Studie besitzt die Studie größere Gemeinsamkeiten mit der dritten Studie von \citet{Schuler2009}. \citet{Schuler2009} führen im Gegensatz zu dieser Studie mit intervall-kontingenten Untersuchungsprotokoll Befragungen nach vordefinieren Kilometermarken beim Marathonlaufen durch. 

Die gesamte Studie bestand aus einer initialen Sitzung und sechs Sitzungen mit einer Zeitdauer von etwa 1,5 Stunden. In der Zeit rüstete ich die Untersuchungsperson aus und führte vor dem Lauf eine Befragung mit der \ac{FKS} nach einer 15-minütigen Ruhephase durch. Ich gab keine Anweisung und keine Hilfestellung zur Beantwortung. Anschließend führte die Untersuchungsperson einen Lauf von etwa einer Stunde durch. Die Sitzungstermine verteilte ich auf sechs aufeinanderfolgende Donnerstage im Spätherbst 2013. Der Start der Sitzungen variierte zwischen 17:15 Uhr und 18:30 Uhr.

In der Studie lief ein gesunder Freizeitläufer im Alter von 29 Jahren als zu untersuchende Person. Er hatte Erfahrungen bei Amateurausdauerläufen gesammelt, bestätigte mir aber, im genannten Zeitraum für kein Ereignis zu trainieren.

Vor jedem Lauf rüstete ich ihn mit einem geladenen Smartphone, einem passenden Smartphone-Armband, einem geladenen Shimmer \ac{IMU} mit Shimmer Gyro-Modul, einem geladenen Shimmer \ac{IMU} mit Shimmer EKG-Modul und vier Einwegelektroden aus. Die Anordnung des Equipments ist Abbildung~\ref{fig:5_1_equipment_setup} zu entnehmen. Zudem wies ich ihn in der initialen Sitzung darauf hin, dass der \ac{PPC} \ac{EKG}-Daten, kinematische Daten und \ac{GPS}-Positionen im Verlauf der gesamten Sitzung protokolliert. Ich erklärte ihm zusätzlich die App und testete mit ihm das akustische Signal und die Vibration, die zur Aufforderung einer Selbstauskunft dient. Die Auswahl der Strecke überließ ich ihm bei der ersten Sitzung. Ich wies ihn darauf hin, dass er die gewählte Strecke in den nächsten sechs Sitzungen erneut laufen muss. Die gewählte Laufstrecke von 14~km bestand aus Hin- und Rückweg und ist in Abbildung~\ref{fig:5_3_laufen_1_karte} dargestellt.

\begin{figure}[t]
	\centering
		\includegraphics[width=1.00\textwidth]{5_3_laufen_1_karte}
	\caption[Laufstrecke -- Hin- und Rückweg]{Laufstrecke -- Hin- und Rückweg insgesamt 14~km (erstellt mit Google Maps)}
	\label{fig:5_3_laufen_1_karte}
\end{figure}

Während der Sitzung trug die Untersuchungsperson das Smartphone in dem passenden Smartphone-Armband am Oberarm, sofern er es nicht zur Beantwortung einer \ac{FKS} nutzte, zu der er alle 15 Minuten aufgefordert wurde. Ich teilte ihm in der initialen Sitzung mit, dass er sich in keiner Prüfungssituation befindet. Ich wies ihn an, die Strecke in einem für ihn optimalen Tempo zu laufen, das ihn nicht überfordert und nicht unterfordert. Der Gedanke, der hinter der gegebenen Anweisung steckt, ist das Gleichgewicht zwischen Anforderung der Tätigkeit und der eigenen Fähigkeiten herzustellen. Eine konkrete Zeit zu Laufen definiert klare Handlungsschritte und ein Ziel und der Läufer erhält durch das Vorankommen unmittelbare und eindeutige Rückmeldungen. Demzufolge sind die allgemeinen Voraussetzungen aus Tabelle~\ref{tab:voraussetzungen_fuer_einen_flow_zustand} gegeben, um Flow erleben zu können. 

\subsection{Apparat} % (fold)
\label{sub:apparat}

Zur Untersuchung von expliziten und impliziten Merkmalen des Flow-Erlebens entwickelte ich ein System zur Sammlung und Segmentierung von subjektiven, physiologischen und kinematischen Daten und zur Erkennung und Analyse von subjektiven, physiologischen und Merkmalen, die auf der Biomechanik des Gehens und des Laufens beruhen. Das System besteht aus dem \ac{PPC} für Smartphones zur Sammlung und der \ac{PPP}, bestehend aus R-Programmen für einen PC zur Segmentierung, Erkennung und Analyse. Im nachfolgenden kennzeichne ich die software-technischen Systemkomponenten.

\subsubsection{PsychoPhysioCollector} % (fold)
\label{ssub:psychophysiocollector}

Der \ac{PPC} besteht technisch gesehen aus drei Komponenten: \emph{Fragebögen}, \emph{Sensoren} und \emph{integrierten Datenmanagement}. Der \ac{PPC} läuft auf dem Android OS ab Version 4.4 und kommuniziert mit den Shimmer \acp{IMU} über Bluetooth. In der beschriebenen Studie kam ein Samsung Galaxy Nexus GT-I9250 zum Einsatz.

\paragraph{Die Fragebogenkomponente} % (fold)
\label{par:die_fragebogenkomponente}

ermöglicht, Befragungen intervall-kontingent durchzuführen. Wir ermöglichten die Erstellung der Fragebögen über die textbasierte JSON. In der beschriebenen Studie kam die original \ac{FKS} zum Einsatz und damit Items in Form von Likert-Skalen. Der \ac{PPC} gibt die Möglichkeit das Befragungsintervall in fünf Minuten Abständen einzustellen.

Nach Ablauf eines Befragungsintervalls signalisiert der \ac{PPC} der Untersuchungsperson mit einem akustischen Signal und mit einer Vibration, dass diese eine Selbstauskunft abgeben muss. Die Likert-Skalen realisierten wir mit einer Bewertungskomponente, die per Fingerberührung der Untersuchungsperson bewertet wird.

% paragraph die_fragebogenkomponente (end)

\paragraph{Die Sensorkomponente} % (fold)
\label{par:die_sensorkomponente}

Die Sensorkomponente verbindet den \ac{PPC} mit den Shimmer \acp{IMU} und liest die Sensordaten aus. Hierzu ist die Hauptplatine einer Shimmer \ac{IMU} mit einem Bluetooth Modem der Klasse 2 besetzt, das eine Übertragungsreichweite von ca. 10 Meter gewährleistet. Durch eine Firmware, die man in den \acs{ROM} der Shimmer \ac{IMU} schreibt und einer in Java geschriebenen \acs{API} wird die Kommunikation zwischen dem \ac{PPC} und den Shimmer \acp{IMU} realisiert. Die bereitgestellten Shimmer \acp{IMU} sind vom Typ R2. Sie besitzen einen Lithium-Ionen-Akkumulator (sog. Lithium-Iionen-Akku) mit einer Kapazität von 450 mAH und einen drei Achsen-Beschleunigungsmesser mit einem betriebssicheren Messbereich von \mbox{$\pm$1,5~g / $\pm$6 g}. Erweiterbar ist ein Shimmer \ac{IMU} mit unterschiedlichen Tochterplatinen. In der beschriebenen Studie kam eine Shimmer \ac{IMU} mit EKG-Modul (Abbildung~\ref{fig:5_4_shimmer_imu_im_einsatz}, links) und ein Shimmer \ac{IMU} mit Gyro-Modul (Abbildung~\ref{fig:5_4_shimmer_imu_im_einsatz}, rechts) zum Einsatz. Das Gyro-Modul besitzt ein drei Achsen-Kreiselinstrument mit einen betriebssicheren Messbereich von \mbox{$\pm$500 $deg \cdot s^{-1}$}. Die Größe einer Shimmer \ac{IMU} beträgt 53~mm $\times$ 32~mm $\times$ 15 mm. Der \ac{PPC} ermöglicht es, die Konfiguration einer Shimmer \ac{IMU} anzumelden und die Abtastrate und Messbereiche festzulegen. Die Konfiguration der ersten Laufstudie ist der Tabelle~\ref{tab:sensorkonfiguration_erste_studie_laufen} zu entnehmen. Mit elastischen Textilbefestigungen lassen sich die Shimmer \acp{IMU} nahtlos an den Körper der Untersuchungsperson befestigen.

\begin{table}[t]
	\caption[Sensorkonfiguration der ersten Studie zum Flow-Erleben beim Laufen]{Sensorkonfiguration der ersten Studie zum Flow-Erleben beim Laufen}
	\label{tab:sensorkonfiguration_erste_studie_laufen}
	\begin{tabularx}{\textwidth}{p{.30\textwidth} p{.20\textwidth} p{.50\textwidth}}
\toprule
& Abtastrate & Betriebssicherer Messbereich \\
\midrule
EKG & 204,8~Hz & \\
Beschleunigungsmesser & 85,3~Hz & 1,5~g \\
Kreiselinstrument & 85,3~Hz & 500 $deg \cdot s^{-1}$ \\
\bottomrule
\end{tabularx}
\end{table}

Zusätzlich liest die Sensorkomponente die \ac{GPS}-Einheit des Smartphones aus. Die Genauigkeit stellten wir auf die höchste Stufe, die das Android OS zu Verfügung stellt, ein.

\begin{figure}[t]
	\centering
		\includegraphics[width=1.00\textwidth]{5_4_shimmer_imu_im_einsatz}
	\caption[Shimmer IMUs mit Modulen]{Shimmer IMU des Typs R2 mit EKG-Modul am Oberkörper (links) und mit Gyro-Modul am Schienbein (rechts); nachgestellt}
	\label{fig:5_4_shimmer_imu_im_einsatz}
\end{figure}

% paragraph die_sensorkomponente (end)

% subsubsection psychophysiocollector (end)

% subsection apparat (end)

% section methode (end)

% section flow_und_laufen_intraindividuell (end)


%!TEX root = /Users/sbogutzky/Entwicklung/projects/bogutzky/repositories/2939413/final-draft.tex
\section{Flow und Laufen (intraindividuell)} 

% (fold)
\label{sec:flow_und_laufen_intraindividuell}

\subsection{Einleitung} 

% (fold)
\label{sub:einleitung_5_1}

Ziel der in diesem Abschnitt beschriebenen Studie war es, zu untersuchen, ob subjektive, durch Experience Sampling erhobene, explizite Flow-Merkmale mit den nachfolgenden Kandidaten für ein implizites Messverfahren des Flow-Erlebens beim Laufen in einem Zusammenhang stehen (Abschnitt~\ref{sec:herangehensweise}, Schritt 2): 
\begin{itemize}
	\item mittlere \ac{HR} 
	\item \acs{RMSSD} der zeitbezogenen statitischen \ac{HRV}-Analyse 
	\item mittlere Doppelschrittfrequenz 
	\item Bewegungsaufwand (Abschnitt~\ref{ssub:der_bewegungsfluss}) oder 
	\item normalisierter Shannon Entropie Index der kardio-lokomotorischen Phasensynchronisation (Abschnitt~\ref{ssub:die_kardio_lokomotorische_phasensynchronisation}) 
\end{itemize}

Die Auswahl der Merkmale begründe ich durch: 
\begin{enumerate}
	\item ihre nachgewiesenen Zusammenhänge zum Flow-Erleben bei Tätigkeiten mit geringer physischer Beansprung (Abschnitt~\ref{ssub:kardiovaskulare_messungen}) 
	\item die Praktikabilität bei der Ausführung einer physisch beanspruchenden Tätigkeit 
\end{enumerate}

Im Anschluss (Abschnitt~\ref{sec:herangehensweise}, Schritt 3) untersuche ich das zeitliche Verhalten (Prozess) der impliziten Daten, um Übergänge zwischen \emph{Flow} und \emph{nicht Flow} zu identifizieren.

Aufgrund der durch das \acs{BMBF}-Projekt bereitgestellten Shimmer \acp{IMU}, nutzte ich ein Shimmer EKG-Modul, das auf die Hauptplatine einer Shimmer \ac{IMU} aufgesetzt wird. Das Shimmer EKG-Modul arbeitet mit der Ableitung nach Einthoven \citep[][S.~85ff.]{Behrends2002}. Es handelt sich um eine bipolare Extremitätenableitung, die man routinemäßig mit drei Elektroden plus einer Erdungselektrode erfasst. Ich platzierte vier konventionelle Einwegelektroden auf der Körperoberfläche und verband sie mit vier Kabelverbindungen mit dem Shimmer EKG-Modul wie in Abbildung~\ref{fig:5_1_equipment_setup} dargestellt. Damit erfasst das Shimmer EKG-Modul mit zwei Kanälen die zwei Ableitungen (II-III). Ableitung I berechnete ich, indem ich das Signal von LA-LL von dem Signal von RA-LL subtrahierte. 
\begin{itemize}
	\item Ableitung I: zwischen rechtem und linkem Arm (RA-LA) 
	\item Ableitung II: zwischen rechtem Arm und linkem Bein (RA-LL) 
	\item Ableitung III: zwischen linkem Arm und linkem Bein (LA-LL) 
\end{itemize}

Die Ableitung nach Einthoven dient zur Darstellung von Potenzialänderungen in der Frontalebene und ermöglicht u. a. die Identifikation von Herzschlägen, die wir zur Analyse der \ac{HRV} benötigen.

Wie in Abschnitt~\ref{ssub:kardiovaskulare_messungen} zusammengefasst, handelt es sich bei der \acs{RMSSD} um ein physiologisches Merkmal der zeitbezogenen statitischen \ac{HRV}-Analyse, mit dem \citet{Keller2011} bei einer sitzenden Tätigkeit Zusammenhänge mit Flow-Erleben feststellten. Auf Zusammenhangsanalysen mit frequenzbezogenen \ac{HRV}-Merkmalen verzichtete ich aufgrund der in Abschnitt~\ref{sub:zuordnung} beschriebenen Probleme.

Kinematische Daten maß ich mit den bereitgestellten Shimmer \acp{IMU} und einem zusätzlichen Shimmer Gyro-Modul. Die kinematischen Daten benötigte ich, um einzelne sich wiederholende Bewegungsabläufe der Bewegung zu erkennen und den Bewegungsaufwand zu berechnen. Zusätzliche kinematische Merkmale, die z.~B. Abschnitt~\ref{sub:lauftechnik_detektionstechnologie} beschreibt, berücksichtigte ich nicht.

Die Anordnung des Equipments und die Zuverlässigkeit der Datenaufzeichnung testete ich in Vorab-Tests mit drei Freiwilligen aus der Fakultät 4 der Hochschule Bremen. Für die Vorab-Tests verwendete ich denselben Aufbau des Systems, der bei dieser Studie zum Einsatz kam (siehe Abschnitt~\ref{sub:apparat}). 
\begin{figure}
	[!htb] \centering 
	\includegraphics[width=1.00 
	\textwidth]{5_1_equipment_setup} \caption[Equipment der ersten Studie zum Flow-Erleben beim Laufen]{Equipment der ersten Studie zum Flow-Erleben beim Laufen} \label{fig:5_1_equipment_setup} 
\end{figure}

% subsection einleitung (end)
\subsection{Methode} 

% (fold)
\label{sub:methode}

\citet[][S.~989]{Strohrmann2012} argumentieren, Untersuchungen in Außen-Umgebungen durchzuführen, da sich das Laufen auf einem Laufband von dem Laufen in Außen-Umgebungen in Bezug auf kinematische Merkmale unterscheidet. Aus dem genannten Grund entschied ich mich, die Läufe unter realen Bedingungen durchzuführen. Untersuchungen außerhalb von Laboratorien stellen bei ihrer Durchführung eine Herausforderung dar. Außen-Umgebungen sind in einem hohen Maße veränderlich und kontextsensitive Faktoren als Einflüsse auf Merkmale sind grundsätzlich als gegeben anzunehmen. Gleichwohl sind solche Untersuchungen unerlässlich, um eine realistischere Einschätzung der Nutzung in Außen-Umgebungen zu ermöglichen.

Für die in diesem Abschnitt dokumentierte Studie setzte ich ein Echtzeit-Datenerfassungsverfahren (real-time data capture) ein. Die Datenerfassung umfasst \ac{EKG}-Daten und kinematische Daten sowie Selbstauskünfte durch die \ac{FKS}. Die aufgezeichneten Datenströme eines einzelnen Bewegungsablaufs beim Laufen sind in Abbildung~\ref{fig:5_2_daten} dargestellt. Der Bewegungsablauf beginnt und endet mit dem mittleren Schwung des rechten Beines (negative Signalspitze der Winkelgeschwindigkeit um die X-Achse (grün)). 
\begin{sidewaysfigure}
	\resizebox{1.00 
	\textwidth}{!}{
	
	%
	\input{./tikz/5_2_daten} }
	
	%
	\caption[Aufgenommene Datenströme beim Laufen]{Aufgenommene Datenströme eines Bewegungsaublaufs in der ersten Studie zum Flow-Erleben beim Laufen. Von links nach rechts: (schwarz) EKG-Ableitung RA-LL und EKG-Ableitung LA-LL; (grün) Beschleunigung in X-Richtung und Winkelgeschwindigkeit um die X Achse; (blau) Beschleunigung in Y-Richtung und Winkelgeschwindigkeit um die Y-Achse; (rot) Beschleunigung in Z-Richtung und Winkelgeschwindigkeit um die Z-Achse} \label{fig:5_2_daten} 
\end{sidewaysfigure}

Ich orientierte mich am Vorgehen des Experiments von \citet{Reinhardt2006} und der dritten Studie von \citet{Schuler2009}. Wie bei Reinhardt führte ich eine Befragung an vordefinierten Zeitmarken durch, an denen die Untersuchungsperson die Lauftätigkeit kurzzeitig unterbrach. Im Gegensatz zu \citet{Reinhardt2006} ist der zweite Schritt der beschriebenen Studie in der Korrelationsforschung anzusiedeln. Der erste Schritt ist das Sammeln von Daten und der dritte Schritt ist die Untersuchung des zeitlichen Verhaltens (Prozess) der gesammelten Daten (Abschnitt~\ref{sec:herangehensweise}). Aufgrund der gegebenen logischen Abhängigkeit von Schritt 2 und 3 lag das Augenmerk zunächst auf Schritt 2 --- der Suche nach signifikanten Korrelationen zwischen expliziten Merkmalen und impliziten Kandidaten des Flow-Erlebens. Das Ziel war bei Kenntnis der Werte der impliziten Merkmale (unabhängigen Variable) die Werte der expliziten Flow-Merkmale (abhängige Variable) vorherzusagen. Als abhängige Variable diente unabhängig voneinander der Generalfaktor, der erste und der zweite Faktor der \ac{FKS} und als unabhängige Variable diente unabhängig voneinander die mittlere \ac{HR}, die \acs{RMSSD} der zeitbasierten \ac{HRV}-Analyse, der Bewegungsaufwand und der normalisierte Shannon Entropie Index der kardio-lokomotorischen Phasensynchronisation.

Mit Blick auf die Durchführung der Studie besaß die Studie größere Gemeinsamkeiten mit der dritten Studie von \citet{Schuler2009}. \citet{Schuler2009} führten im Gegensatz zu dieser Studie mit intervall-kontingenten Untersuchungsprotokoll Befragungen nach vordefinieren Kilometermarken beim Marathonlaufen durch. 

Die gesamte Studie bestand aus einer initialen Sitzung und sechs weiteren Sitzungen mit einer Zeitdauer von etwa 1,5 Stunden. In der Zeit rüstete ich die Untersuchungsperson aus und führte vor dem Lauf eine Befragung mit der \ac{FKS} nach einer 15-minütigen Ruhephase durch. Ich gab keine Anweisung und keine Hilfestellung zur Beantwortung. Anschließend führte die Untersuchungsperson einen Lauf von etwa einer Stunde durch. Die Lauftermine verteilte ich auf sechs aufeinanderfolgende Donnerstage im Spätherbst 2013. Der Start der Läufe variierte zwischen 17:15 Uhr und 18:30 Uhr.

In der Studie lief ein gesunder Freizeitläufer im Alter von 29 Jahren als zu untersuchende Person. Er hatte Erfahrungen bei Amateurausdauerläufen gesammelt, bestätigte mir aber, im genannten Zeitraum für kein Ereignis zu trainieren.

Vor jedem Lauf rüstete ich ihn mit einem geladenen Smartphone, einem passenden Smartphone-Armband, einem geladenen Shimmer \ac{IMU} mit Shimmer Gyro-Modul, einem geladenen Shimmer \ac{IMU} mit Shimmer EKG-Modul und vier Einwegelektroden aus. Die Anordnung des Equipments ist Abbildung~\ref{fig:5_1_equipment_setup} zu entnehmen. Zudem wies ich ihn vor dem initialen Lauf darauf hin, dass der \ac{PPC} \ac{EKG}-Daten, kinematische Daten und \ac{GPS}-Positionen im Verlauf des gesamten Laufs protokolliert. Ich erklärte ihm zusätzlich die App und testete mit ihm das akustische Signal und die Vibration, die zur Aufforderung einer Selbstauskunft dient. Die Auswahl der Strecke überließ ich ihm beim ersten Lauf. Ich wies ihn darauf hin, dass er die gewählte Strecke in den nächsten sechs Läufen erneut laufen muss. Die gewählte Laufstrecke von 14~km bestand aus Hin- und Rückweg und ist in Abbildung~\ref{fig:5_3_laufen_1_karte} dargestellt. 
\begin{figure}
	[!htb] \centering 
	\includegraphics[height=0.50 
	\textheight]{5_3_laufen_1_karte} \caption[Laufstrecke -- Hin- und Rückweg]{Laufstrecke -- Hin- und Rückweg insgesamt 14~km (erstellt mit OpenStreetMap)} \label{fig:5_3_laufen_1_karte} 
\end{figure}

Während des Laufes trug die Untersuchungsperson das Smartphone in dem passenden Smartphone-Armband am Oberarm, sofern er es nicht zur Beantwortung einer \ac{FKS} nutzte, zu der er alle 15 Minuten aufgefordert wurde. Ich teilte ihm vor dem initialen Lauf mit, dass er sich in keiner Prüfungssituation befindet. Ich wies ihn an, die Strecke in einem für ihn optimalen Tempo zu laufen, das ihn nicht überfordert und nicht unterfordert. Der Gedanke, der hinter der gegebenen Anweisung steckt, ist das Gleichgewicht zwischen Anforderung der Tätigkeit und der eigenen Fähigkeiten herzustellen. Eine konkrete Zeit zu Laufen definiert klare Handlungsschritte und ein Ziel und der Läufer erhält durch das Vorankommen unmittelbare und eindeutige Rückmeldungen. Demzufolge sind die allgemeinen Voraussetzungen aus Tabelle~\ref{tab:voraussetzungen_fuer_einen_flow_zustand} gegeben, um Flow erleben zu können. 

% section methode (end)
\subsection{Apparat} 

% (fold)
\label{sub:apparat}

Zur Untersuchung von expliziten und impliziten Merkmalen des Flow-Erlebens konzipierte ich ein System zur Sammlung und Segmentierung von subjektiven, physiologischen und kinematischen Daten und zur Erkennung und Analyse von subjektiven, physiologischen Merkmalen und kinematischen Merkmalen, die auf der Biomechanik des Gehens und des Laufens beruhen. Das System besteht aus dem \ac{PPC} für Smartphones zur Sammlung und der \ac{PPP}, bestehend aus R-Programmen für einen PC zur Segmentierung, Erkennung und Analyse. Im nachfolgenden kennzeichne ich die software-technischen Systemkomponenten.

\subsubsection{PsychoPhysioCollector} 

% (fold)
\label{ssub:psychophysiocollector}

Der \ac{PPC} besteht technisch gesehen aus drei Komponenten: \emph{Fragebögen}, \emph{Sensoren} und \emph{integrierten Datenmanagement}. Der \ac{PPC} läuft auf dem Android OS ab Version 4.4 und kommuniziert mit den Shimmer \acp{IMU} über Bluetooth. In der beschriebenen Studie kam ein Samsung Galaxy Nexus GT-I9250 zum Einsatz.

\paragraph{Die Fragebogenkomponente} 

% (fold)
\label{par:die_fragebogenkomponente}

ermöglicht, Befragungen intervall-kontingent durchzuführen. Wir ermöglichten die Erstellung der Fragebögen über die textbasierte JSON. In der beschriebenen Studie kam die original \ac{FKS} zum Einsatz und damit Items in Form von Likert-Skalen. Der \ac{PPC} gibt die Möglichkeit das Befragungsintervall in fünf Minuten Abständen einzustellen.

Nach Ablauf eines Befragungsintervalls signalisiert der \ac{PPC} der Untersuchungsperson mit einem akustischen Signal und mit einer Vibration, dass diese eine Selbstauskunft abgeben muss. Die Likert-Skalen realisierten wir mit einer Bewertungskomponente, die per Fingerberührung der Untersuchungsperson bewertet wird.

% paragraph die_fragebogenkomponente (end)
\paragraph{Die Sensorkomponente} 

% (fold)
\label{par:die_sensorkomponente}

verbindet den \ac{PPC} mit den Shimmer \acp{IMU} und liest die Sensordaten aus. Hierzu ist die Hauptplatine einer Shimmer \ac{IMU} mit einem Bluetooth Modem der Klasse 2 besetzt, das eine Übertragungsreichweite von ca. 10 Meter gewährleistet. Durch eine Firmware, die man in den \acs{ROM} der Shimmer \ac{IMU} schreibt und einer in Java geschriebenen \acs{API} wird die Kommunikation zwischen dem \ac{PPC} und den Shimmer \acp{IMU} realisiert. Die bereitgestellten Shimmer \acp{IMU} sind vom Typ R2. Sie besitzen einen Lithium-Ionen-Akkumulator (Lithium-Iionen-Akku) mit einer Kapazität von 450 mAH und einen drei Achsen-Beschleunigungsmesser mit einem betriebssicheren Messbereich von \mbox{$\pm$1,5~g / $\pm$6 g}. Erweiterbar ist ein Shimmer \ac{IMU} mit unterschiedlichen Tochterplatinen. In der beschriebenen Studie kam eine Shimmer \ac{IMU} mit EKG-Modul (Abbildung~\ref{fig:5_4_shimmer_imu_im_einsatz}, links) und ein Shimmer \ac{IMU} mit Gyro-Modul (Abbildung~\ref{fig:5_4_shimmer_imu_im_einsatz}, rechts) zum Einsatz. Das Gyro-Modul besitzt ein drei Achsen-Kreiselinstrument mit einen betriebssicheren Messbereich von \mbox{$\pm$500 $deg \cdot s^{-1}$}. Die Größe einer Shimmer \ac{IMU} beträgt 53~mm $\times$ 32~mm $\times$ 15 mm. Der \ac{PPC} ermöglicht es, die Konfiguration einer Shimmer \ac{IMU} anzumelden und die Abtastrate und Messbereiche festzulegen. Die Konfiguration der ersten Laufstudie ist der Tabelle~\ref{tab:sensorkonfiguration_erste_studie_laufen} zu entnehmen. Mit elastischen Textilbefestigungen lassen sich die Shimmer \acp{IMU} nahtlos an den Körper der Untersuchungsperson befestigen. 
\begin{table}
	[!htb] \caption[Sensorkonfiguration der ersten Studie zum Flow-Erleben beim Laufen]{Sensorkonfiguration der ersten Studie zum Flow-Erleben beim Laufen} \label{tab:sensorkonfiguration_erste_studie_laufen} 
	\begin{tabularx}
		{ 
		\textwidth}{p{.30 
		\textwidth} p{.20 
		\textwidth} p{.50 
		\textwidth}} \toprule & Abtastrate & Betriebssicherer Messbereich \\
		\midrule EKG & 204,8~Hz & \\
		Beschleunigungsmesser & 85,3~Hz & 1,5~g \\
		Kreiselinstrument & 85,3~Hz & 500 $deg \cdot s^{-1}$ \\
		\bottomrule 
	\end{tabularx}
\end{table}

Zusätzlich liest die Sensorkomponente die \ac{GPS}-Einheit des Smartphones aus. Die Genauigkeit stellten wir auf die höchste Stufe, die das Android OS zu Verfügung stellt, ein. 
\begin{figure}
	[!htb] \centering 
	\includegraphics[width=1.00 
	\textwidth]{5_4_shimmer_imu_im_einsatz} \caption[Shimmer IMUs mit Modulen]{Shimmer IMU des Typs R2 mit EKG-Modul am Oberkörper (links) und mit Gyro-Modul am Schienbein (rechts); nachgestellt} \label{fig:5_4_shimmer_imu_im_einsatz} 
\end{figure}

% paragraph die_sensorkomponente (end)
\paragraph{Das Datenmanagement} 

% (fold)
\label{par:das_datenmanagement} des \ac{PPC}s speichert die Untersuchungsdaten in Ordner auf dem Android Dateisystem. Die Order erhalten ihren Namen vom Zeitstempel des Anfangs der Untersuchung. In einem Ordner speichert der \ac{PPC} alle Selbstauskünfte und alle Sensordaten einer Untersuchung in Textdateien des Formats \acs{csv}. Selbstauskünfte schreibt das System ohne Verzögerung in eine Textdatei namens \emph{self-report.csv}. Die Sensordaten von den Shimmer \acp{IMU} und der \ac{GPS}-Einheit sichert das System wegen ihrer Häufigkeit sequenziell. Das heißt, eine vorbestimmte Anzahl an Datensätzen wird im Arbeitsspeicher des Smartphones (abhängig von der jeweiligen eingestellten Abtastrate) gehalten. Erreicht das System die Anzahl eines vorbestimmten Wertes, schreibt es die Datensätze in die jeweilige Textdatei des zugehörigen Sensors. Um den Arbeitsspeicher nicht zu überfüllen, nutzten wir eine Double-buffering Strategie. Textdateien erhalten ihren Namen vom Bluetooth-Gerätenamen, worüber auch die Zuordnung der Datensätze beim Speichern erfolgt. \ac{GPS}-Positionen schreibt das System in eine Datei namens \emph{gps-position.csv}.

% paragraph das_datenmanagement (end)
% subsubsection psychophysiocollector (end)
\subsubsection{PsychoPhysioPipeline} 

% (fold)
\label{ssub:psychophysiopipeline}

Die \ac{PPP} realisierten wir in einzelnen kleinen R-Programmen zur Segmentierung der Untersuchungsdaten, zur Erkennung von Merkmalen und zu deren Analyse. R ist eine Sprache und Umgebung für statistische Berechnungen und Grafiken.

\paragraph{R-Programme zur Segmentierung.} 

% (fold)
\label{par:r_programme_zur_segmentierung}

Die R-Programme mit der geringsten Komplexität segmentieren die Untersuchungsdaten anhand der Start- und Endzeitpunkte der Selbstauskünfte. Sie speichern die Daten in mehrere kleine Textdateien im \acs{csv}-Format. Ich nutze interaktive Grafiken zur visuellen Kontrolle der Daten. Das R-Programm zur Segmentierung der \ac{GPS}-Positionen versah ich mit einer Funktion, die \acs{KML}-Dateien herausschreibt. \acs{KML}-Dateien ermöglichen mir, den zeitlichen Verlauf eines Laufes in Google Earth nachzuvollziehen.

% paragraph r_programme_zur_segmentierung (end)
\paragraph{Kubios HRV zur HRV-Analyse.} 

% (fold)
\label{par:kubios_hrv_zur_hrv_analyse}

Nach der Segmentierung der \ac{EKG}-Daten führe ich die Analyse der \ac{HRV} in Kubios HRV durch. Kubios HRV ist eine benutzerfreundliche Software für die Analyse der \ac{HRV} \citep{Tarvainen2014}. Kubios HRV ist freie Software und wurde mit MATLAB entwickelt. Mit Kubios HRV speichere ich ggf. zeitbezogene, frequenzbezogene und nicht lineare \ac{HRV}-Parameter und die Zeitreihe der R-Spitzen in Textdateien im txt-Format.

% paragraph kubios_hrv_zur_hrv_analyse (end)
\paragraph{R-Programme zur Erkennung.} 

% (fold)
\label{par:r_programme_zur_erkennung}

Mit einem R-Programm berechne ich die Faktoren der \ac{FKS} jeder Selbstauskunft und schreibe sie in eine Textdatei im \acs{csv}-Format.

Die Berechnung des Bewegungsaufwands erfolgt nach der Erkennung des mittleren Schwungs der Lauf- oder Gehbewegung. Das ausgeprägte Signalmuster der Winkelgeschwindigkeit um die X-Achse (Abbildung~\ref{fig:5_2_daten}, grünes Signal unten), ermöglicht die Lokalisierung der in Abschnitt~\ref{sec:gehen_und_laufen} beschriebenen biomechanischen Merkmale. Das Muster besteht aus zwei Maxima (positive Signalspitze) auf beiden Seiten eines Minimums (negatives Signalspitze). Die beiden positiven Signalspitzen ordnen wir dem Aufsetzen des Fußes (\ac{IC}) und dem Anheben des Fußes (\ac{IS}) zu \citep[][]{Aminian2002}. Die negative Signalspitze ist dem mittleren Schwung (\ac{MS}) zuzuordnen \citep[][]{Aminian2002}.

Durch eine zweifache Filterung des Signals mit personenindividuellen Grenzfrequenzen bestimmt das R-Programm nacheinander die negative Spitze des mittleren Schwungs in den gefilterten Signalen. Die Technik vermindert fehlerhafte Erkennungen bei Lokalisierungen im Rohsignal \citep[][]{Lee2011}. Ich entferne Erkennungen zwischen personenindividuell auswählbaren Grenzen (z.~B. -200 und 200 $deg \cdot s^{-1}$), da Erkennungen in diesen individuellen Grenzen keinem biomechanischen Merkmal zuzuordnen sind. Nach einer visuellen Kontrolle durch interaktive Grafiken speichert das R-Programm die Zeitstempel der \ac{MS}-Erkennungen in eine Textdatei im \acs{csv}-Format für weitere Arbeitschritte.

Zur Berechnung des Bewegungsaufwands glättet ein weiteres R-Programm die Beschleunigungsdaten mit einem Null-Phasen Butterworth Tiefpassfilter vierter Ordnung nach dem Beispiel von \citet{Hreljac2000}. Im Gegensatz zu \citet{Hreljac2000}, der nur zwei dimensionale Beschleunigungsdaten von Filmmaterial aus der sagittalen Ebene extrahierte, nutze ich die drei Dimension, die mir der Beschleunigungsmesser, der Shimmer \ac{IMU} liefert. Damit verspreche ich mir eine genauere Beschreibung des Bewegungsaufwands.

Die Beschleunigung ist die Summe von Gravitation und lineare Beschleunigung (Benutzerbeschleunigung). Zur Berechnung beider Beschleunigungen und zur einzelnen Betrachtung der Benutzerbeschleunigung ist die Schätzung der Orientierung der Shimmer \acp{IMU} im dreidimensionalen Raum notwendig. Unter Verwendung des Beschleunigungsmessers und des Kreiselinstruments, ist die Schätzung der Orientierung der Shimmer \ac{IMU} relativ zu ihrer anfänglichen Orientierung möglich. Für eine vollständige Lösung des Orientierungsproblems, die eine absolute Schätzung beinhaltet, ist ein Magnetometer erforderlich. Die Entwicklung eines Algorithmus zur Schätzung der Orientierung eines Objekts im dreidimensionalen Raum ist ein eigenes Forschungsthema. Viele Lösungen für das Orientierungsproblem sind in der Literatur zu finden. Ein Beispiel sowohl für einen relativen Algorithmus und einen absoluten Algorithmus geben \citet{Madgwick2011}. Aus dem Grund der Komplexität der vorhandenen Lösungen und da ich Vergleiche von Bewegungsaufwänden von verschiedenen Messungen auf gleicher Berechnungsgrundlage durchführe, nehme ich keine Schätzung der Orientierung vor und nutze zur Berechnung des Bewegungsaufwands die Summe von Gravitation und linearer Beschleunigung.

Die optimalen Grenzfrequenzen zur Glättung der Beschleunigungsdaten im dreidimensionalen Raum (x, y, z) bestimmt das Programm aus allen in Datenbasis enthaltenen Beschleunigungsdaten unter Verwendung der Restwertmethode von \citet{Wells1980}. Nach der Glättung trennt das R-Programm die Beschleunigungen in \mbox{x-, y-, z-Richtung} anhand der \ac{MS}-Erkennungen und berechnet den Bewegungsaufwand für jeden einzelnen Bewegungsablauf mit nachfolgender Formel: 
\begin{equation}
	Bewegungsaufwand = \int^T_0 \left\langle {Ruck^2_{x}(t) + Ruck^2_{y}(t) + Ruck^2_{z}(t)}\right\rangle dt 
\end{equation}

Bewegungsabläufe von einer Dauer über 1,3 Sekunden entfernt das R-Programm, um Unterbrechungen der Lauf- oder Gehtätigkeiten auf Grund ihres geringen Bewegungsaufwands nicht einzubeziehen. Nach einer visuellen Kontrolle durch interaktive Grafiken speichert das R-Programm die Zeitpunkte, die Dauer und den Bewegungsaufwand der einzelnen Bewegungsabläufe in eine Textdatei im \acs{csv}-Format. 
\begin{figure}
	[!htb] \input{./tikz/5_5_grundlage_der_klps} \caption[Grundlage der Berechnung der kardio-lokomotorischen Phasensynchronisation]{Grundlage der Berechnung der kardio-lokomotorischen Phasensynchronisation} \label{fig:5_5_grundlage_der_klps} 
\end{figure}

Die kardio-lokomotorischen Phasensynchronisation interpretieren \citet[][S.~12]{Niizeki2014} als ein konsistentes Auftreten eines Herzschlages in der gleichen relativen Phase aufeinanderfolgender Bewegungsablaufe. Das heißt, z.~B. bei einer hohen kardio-lokomotorischen Phasensynchronisation beim Laufen tritt der Herzschlag in der Regel auf, wenn das Bein einen relativen Weg zurückgelegt hat. Die relative Phase beschreibt die Beziehung zwischen Herzschlag und zurückgelegten Weg im Bewegungsablauf. Zu ihrer Berechnung benötigen wir die momentanen Phasen der beiden Oszillatoren (Herz und Bewegungsapparat). Es gibt zwei Vorgehen die momentanen Phasen zu berechnen: ein ereignis-bezogenes Vorgehen und ein signalanalytisches Vorgehen. In der vorliegenden Arbeit beschränke ich mich auf die Erklärung des ereignisbezogenen Vorgehens, da ich die Zeitpunkte unserer beiden Ereignisse (R-Spitze und \ac{MS}) in den vorherigen Verarbeitungsschritten bestimme. Abbildung~\ref{fig:5_5_grundlage_der_klps} stellt die aufeinanderfolgenden Ereignisse dar.

Innerhalb eines Oszillators berechnen wir die momentane Phase mit: 
\begin{equation}
	\phi(t) = 2 \pi \frac{t-t_{k}}{t_{k+1}-t_{k}} + 2 \pi k, 
\end{equation}

wobei $t_{k}$ der Zeit des k-ten Ereignisses entspricht. Die relative Phase für das Auftreten eines Herzschlages bezüglich des Bewegungsablauf berechnen wir demzufolge mit: 
\begin{equation}
	\Psi(t_{k}) = 1 \frac{\phi_{L}(t_{k}) \bmod 2 \pi}{2 \pi}, 
\end{equation}

wobei $t_{k}$ der Zeit des k-ten Auftretens eines Herzschlags und $\phi_{L}$ der momentanen Phase des Bewegungsablaufs entspricht. Stellen wir $\Psi(t_{k})$ über $t_{k}$ dar, erhalten wir das kardio-lokomotorische Synchrogramm. Mit Hilfe des Synchrogramms bin ich in der Lage nachfolgende Aussagen zu treffen:
\begin{itemize}
	
	\item Sind beiden rhythmischen Oszillatoren unabhängig voneinander, gibt es keine bevorzugte Phase. Die Verteilung von $\Psi(t_{k})$ ist zufällig.
	
	\item Tritt alternativ eine $n:m$ Phasensynchronisation auf, d. h. $\Psi(t_{k})$ tritt genau zu den gleichen $n$ Werten innerhalb der $m$ Bewegungsablauf auf, beobachten wir $n$ parallele horizontale Linien.
\end{itemize}

Die visuelle Bestimmung der Synchronität erfolgt über ein R-Programm. Das Programm berechnet zur Quantifizierung der kardio-lokomotorischen Phasensynchronisation den Phasenkohärenz Index \citep{Rosenblum2003} und den normalisierten Shannon Entropie Index \citep{Tass1998, Niizeki2005} und stellt sie dar. Indizes und Phasen sichert das Programm in einer Textdatei im \acs{csv}-Format.

% paragraph r_programme_zur_erkennung (end)
% subsubsection psychophysiopipeline (end)
\subsubsection{Programme zur Analyse von zeitbasierten multimodalen Daten} 

% (fold)
\label{ssub:programme_zur_analyse_von_zeitbasierten_multimodalen_daten}

Im \acs{BMBF}-Projekt probierten wir unterschiedliche Software zur Analyse von zeitbasierten multimodalen Daten wie den Observer XT von Noldus aus. Der Observer ist ein Werkzeug, das die parallele Untersuchung von mehreren Datenströmen in der Zeit und mit Bezug auf den geografischen Raum ermöglicht. Den Bezug zum geografischen Raum stellt der Observer über eine zweite Software namens Tracklab her. Der Observer ermöglicht die Annotation der Daten, um konkrete Zeiträume bzw. Phasen genauer durch verändern des Zoomfaktors zu betrachten. Leider ist der Import von Daten in den Observer und in Tracklab aufwendig. Für beide Programme braucht man teure Lizenzen und die Nutzbarkeit ist für unseren Verwendungszweck nicht befriedigend.

Ein freies Programm namens ChronoViz \citep{Fouse2010, Fouse2011}, welches die Visualisierung von mehreren Datenströmen in Bezug auf einen geografischen Raum ermöglicht, war nicht in der Lage die Menge an Daten zu verarbeiten. Daraus folgt, dass ein benutzerfreundliches Werkzeug, dass eine prozessorientierte Untersuchung mit Bezug auf einen geografischen Raum im Bereich der Mensch-Computer-Interaktion fehlt.

Als Konsequenz nutzte ich zur Visualisierung der Datenströme R und zur Visualisierung der geographischen Daten Google Earth, auch wenn ich das Zoomen in die Daten und die Synchronisation der Zeit zwischen Datenpunkt und geografischer Position per Hand vornehmen musste.

% subsubsection programme_zur_analyse_von_zeitbasierten_multimodalen_daten (end)
% subsubsection programme_zur_analyse_von (end)
% subsection apparat (end)
\subsection{Operationalisierung und gewonnene Daten} 

% (fold)
\label{sub:operationalisierung_und_gewonnene_daten}

Auf der Grundlage der sechs Läufe erhielt ich 24 Selbstauskünfte durch die Befragung mit der \ac{FKS}. Die \ac{FKS} besitzt nach \citet{Rheinberg2003} eine hohe Güte. Nichtsdestotrotz überprüfte ich die gewonnenen Daten nochmals auf ihre Eignung. Ich garantiere damit, dass die aus der Skala ermittelten Messwerte, die zu beschreibenden Faktoren mit hoher Reliabilität wiedergeben. Gleichzeitig prüfe ich, ob ich die \ac{FKS} im vorliegenden Einzelfall einer laufenden Person erfolgreich eingesetzt habe.

Zur Verifizierung betrachtete ich die 24 Selbstauskünfte und bestimmte den Mittelwert und die Standardabweichung jedes Items des Generalfaktors und der beiden Faktoren der \ac{FKS}. Ebenso berechnete ich für jeden Faktor und dessen Items die Item-Faktor-Korrelation. Die Item-Faktor-Korrelation dient häufig als Testgröße der Trennschärfe. Je höher und gleichmäßiger die Trennschärfe ist, desto höher ist die Konsistenz zwischen den Items und desto besser die verwendete Skala. Als Daumenregel gilt, eine Trennschärfe von größer als 0,5 ist passend. Solange die Items nur geringfügig unter 0,5 liegen, sieht man sie als vertretbar an. Wichtig ist in jedem Fall, dass keine der Trennschärfen signifikant von allen übrigen abweicht und dass die Trennschärfen niemals negativ sind \citep[][S.~219f.]{Bortz2006}.

Neben der Trennschärfe der einzelnen Items bestimmte ich für die Faktoren der \ac{FKS} die Maßzahl Cronbachs~$\alpha$, um festzustellen, inwieweit ich die Faktoren zur Messung der einzelnen latenten Wirkungsdimensionen heranziehen kann. Das Cronbachs~$\alpha$ ist eine in den Wirtschafts- und Sozialwissenschaften häufig verwendetes Reliabilitätskriterium in der Testkonstruktion und -evaluation. Vernünftige Werte für das Cronbachs~$\alpha$ sind größer 0,7 \citep[][S.~189f.]{Bortz2006}.

Die Reliabilität des Generalfaktors und der Absorbiertheit weisen mit 0,81 und 0,77 auf eine gute Eignung hin. Die Reliabilität des glatten Verlaufs ist mit 0,68 akzeptabel. Die Trennschärfen sind bis auf die Items \emph{Mein Kopf ist völlig klar} und \emph{Ich weiß bei jedem Schritt, was ich zu tun habe} für den Generalfaktor, bei denen die Trennschärfe unter 0,4 liegt, akzeptabel (Tabellen~\ref{tab:generalfaktor_erste_studie_laufen}, \ref{tab:glatter_verlauf_erste_studie_laufen} und \ref{tab:absorbiertheit_erste_studie_laufen}). In der Untersuchung hatte das Ergebnis für die Items keine Konsequenzen, da es sich bei \ac{FKS} um eine wissenschaftlich eingehend untersuchte Skala handelt und da ich mögliche inhaltliche Verfälschungen durch weglassen von Items vermeiden wollte.

Zu jedem Befragungszeitpunkt gehören ca. 15 Minuten an \ac{EKG}-Daten und kinematischen Daten, die der \ac{PPC} vor jeder Befragung aufzeichnete. Es sind ca. 15 Minuten, da der Signalgeber nach 15 Minuten die Untersuchungsperson aufforderte, eine Selbstauskunft abzugeben. Das gewährleistete nicht in allen Fällen das Stehenbleiben und das Ausfüllen. Aus diesem Grund verwendete ich die Daten bis zum letzten erkannten Laufschritt vor der Fertigstellung des Fragebogens. Den letzten Laufschritt identifizierte ich im Nachhinein. 

Zur R-Spitzen-Erkennung nutzte ich die Ableitung III, da sie die größten R-Spitzen aufwieß. Die R-Spitzen identifiziert Kubios HRV in aller Regel automatisch (abhängig von der Signalgüte), trotzdem ist eine manuelle Nachbearbeitung notwendig. Nicht erkannte R-Spitzen fügte ich hinzu und zu viel erkannte R-Spitzen entfernte ich. 

Zur Berechung des Bewegungsaufwands (Abschnitt~\ref{ssub:der_bewegungsfluss}) für jeden Bewegungsablauf (Doppelschritt) filterte das R-Programm die gemessene Beschleunigung entlang des koronaren Schnittes (von rechts nach links, x-Achse), entlang des sagittalen Schnittes (von unten nach oben, y-Achse) und entlang des axialen Schnittes (von hinten nach vorne, z-Achse) mit den optimalen Grenzfrequenzen von 1,5~Hz, 4,0~Hz und 5,5~Hz.

Die kardio-lokomotorische Phasensynchronisation (Abschnitt~\ref{ssub:die_kardio_lokomotorische_phasensynchronisation}) quantifizierte ich durch den normalisierten Shannon Entropie Index. Zur Berechnung der Shannon Entropie nutzte ich ein 15 Sekunden Fenster der relativen Phase und verschob dieses jeweils um eine Sekunde, um aufeinanderfolgende Werte zu erhalten. Zur Berechnung der Shannon Entropie teilte ich den Messbereich der relativen Phase (0 bis 1) in eine optimale Anzahl von Abschnitten mit $N = exp(0{,}626 + 0{,}4 \cdot ln(M-1))$, wobei $M$ die Anzahl der Werte der relativen Phase im Zeitfenster darstellt \citep[][S.~20]{Rosenblum2003}.

\subsubsection{Korrelationanalyse} 

% (fold)
\label{subs:korrelationanalyse}

Um Zusammenhänge zwischen den expliziten Flow-Merkmalen der \ac{FKS} und impliziten Merkmalen zu untersuchen (Anschnitt~\ref{sec:herangehensweise}, Schritt 2), berechnete ich akkumulierte implizite Merkmale der Kurzzeit-\ac{HRV} (mittlere \ac{HR} und \acs{RMSSD}), des Bewegungsaufwand (mittlerer Bewegungsaufwand und mittlere Doppelschrittfrequenz) und der kardio-lokomotorische Phasensynchronisation (mittlerer normalisierter Shannon Entropie Index). Als Zeitspanne nutzte ich die fünf Minuten direkt vor der Befragung mit der \ac{FKS}, da die \ac{FKS} den gegenwärtigen Flow-Zustand mit Items (Aussagen) wie \emph{Ich fühle mich optimal beansprucht} abfragt. Dabei halte ich die Länge von fünf Minuten für einen guten Kompromiss, da ich damit die empfohlene Länge für die Berechungen der Kurzzeit-\ac{HRV} \citep[][S.~360]{TaskForce1996} einhalte und genügend Anlaufzeit nach dem Start des Laufes und nach jeder Unterbrechung durch die Befragung mit der \ac{FKS} bleibt (Kapitel~\ref{cha:flow_erleben_beim_gehen_und_laufen_messen_anforderungen}). Zusätzlich musste ich bei der Länge von fünf Minuten keine \ac{EKG}-Messung aufgrund von vielen Artefakten aus der Datenbasis entfernen, was bei längeren Zeitspannen der Fall gewesen wäre und was zu einer ungleichen Anzahl von expliziten und impliziten Merkmalen geführt hätte.

% subsubsection korrelationanalyse (end)
% subsection operationalisierung_und_gewonnene_daten (end)
\subsection{Ergebnisse} 

% (fold)
\label{sub:ergebnisse}

\subsubsection{Beobachtungen} 

% (fold)
\label{ssub:beobachtungen} 
\begin{figure}
	[!htb] \input{./tikz/5_6_ubersicht_nach_laufen_erste_studie} \caption[Übersicht der expliziten und impliziten Merkmale nach Läufen der ersten Studie]{Übersicht der expliziten und impliziten Merkmale nach Läufen der ersten Studie [$N = 4$]} \label{fig:5_6_ubersicht_nach_laufen_erste_studie} 
\end{figure}
\begin{figure}
	[!htb] \input{./tikz/5_7_ubersicht_nach_messzeitpunkten_erste_studie} \caption[Übersicht der expliziten und impliziten Merkmale nach Messzeitpunkten der ersten Studie]{Übersicht der expliziten und impliziten Merkmale nach Messzeitpunkten der ersten Studie [$N = 6$]} \label{fig:5_7_ubersicht_nach_messzeitpunkten_erste_studie} 
\end{figure}

Betrachten wir in der Laufübersicht die einzelnen Merkmale können wir erkennen, dass die Bewertungen der expliziten Flow-Merkmale erhoben durch die \ac{FKS} in den letzten drei Läufen im Durchschnitt höher waren als die Bewertungen in den ersten drei Läufen (Abbildung~\ref{fig:5_6_ubersicht_nach_laufen_erste_studie}, Reihe 1). Zudem stieg die \ac{AFP} im Durchschnitt, um 0,75 Punkte (Abbildung~\ref{fig:5_6_ubersicht_nach_laufen_erste_studie}, Reihe 2, Spalte 1). Zusätzlich war die Varianz der mittleren \ac{HR} in den letzten drei Läufen größer als in den ersten drei Läufen. 

Im dritten Lauf (17.10) bewertete der Befragte im Gegensatz zu allen anderen Läufen die expliziten Flow-Merkmalen am niedrigsten. Zudem weist der \ac{HRV}-Parameter \acs{RMSSD} eine hohe Varianz im Gegensatz zu den anderen Läufen auf (Abbildung~\ref{fig:5_6_ubersicht_nach_laufen_erste_studie}, Reihe 2, Spalte 2). 

Beim letzten Lauf erhöhte der Läufer deutlich das Tempo gemessen an der Doppelschrittfrequenz (Abbildung~, Reihe 3, Spalte 1). Es erhöhte sich zusätzlich der Bewegungsaufwand in diesem Lauf (Abbildung~\ref{fig:5_6_ubersicht_nach_laufen_erste_studie}, Reihe 2, Spalte 3). Im ersten Lauf gab es am wenigsten kardio-lokomotorische Phasensynchronisation gemessen am mittleren normalisierten Shannon Entropie Index (Abbildung~\ref{fig:5_6_ubersicht_nach_laufen_erste_studie}, Reihe 3, Spalte 3).

Betrachten wir die einzelnen Merkmale nach Messzeitpunkten können wir ein gleichbleibendes Verhalten der explizit erfagten Flow-Mermale erkennen (Abbildung~\ref{fig:5_7_ubersicht_nach_messzeitpunkten_erste_studie}, Reihe 1). Die \ac{AFP} wurde vom Befragten nach ersten Befragung (15') am niedrigsten bewertet (Abbildung~\ref{fig:5_7_ubersicht_nach_messzeitpunkten_erste_studie}, Reihe 2, Spalte 1). Die mittlere \ac{HR} weist nach der zweiten Befragung (30') im Gegensatz zu den anderen Messzeitpunkten eine niedrige Varianz auf (Abbildung~\ref{fig:5_7_ubersicht_nach_messzeitpunkten_erste_studie}, Reihe 2, Spalte 2) und die kardio-lokomotorische Phasensynchronisation ist beim gleichen Messzeitpunkt im Durchschnitt am höchsten (Abbildung~\ref{fig:5_7_ubersicht_nach_messzeitpunkten_erste_studie}, Reihe 3, Spalte 3). 

Ich überprüfte Effekte des Messzeitpunkts auf die einzelnen Merkmale mit dem Friedman-Ranksummen-Test. Der Friedman-Ranksummen-Test testete die Gleichheit des Lageparameters der Merkmale auf der Grundlage der Messzeitpunkte. Ich wählte den Friedman-Ranksummen-Test, da er anders als z.~B. die Varianzanalyse (ANOVA) keine Normalverteilung der einzelnen Stichproben voraussetzt. Damit ist der Friedman-Ranksummen-Test parameterfreie Alternative zur ANOVA mit wiederholten Messungen. Die Ergebnisse des Friedman-Rank-Summen-Tests zeigen, dass außer für den mittleren Bewegungsaufwand, die mittlere Doppelschrittfrequenz und der \ac{AFP} kein Effekt des Messzeitpunkts besteht. Zur genaueren Überprüfung der Unterschiede setzte ich zusätzlich den Wilcoxon-Vorzeichen-Rang-Tests ein. Die Ergebnisse des Wilcoxon-Vorzeichen-Rang-Tests bestätigen Unterschiede beim mittleren Bewegungsaufwand zwischen den Messzeitpunkten 15‘ ($Mdn = 16{,}9 \times 10^3 \: m^2 \cdot s^{-5}$) und 30‘ ($Mdn = 16{,}06 \times 10^3 \: m^2 \cdot s^{-5}$), $Z = 2{,}2; p < 0{,}05$ und zwischen den Messzeitpunkten 15‘ ($Mdn = 16{,}9 \times 10^3 \: m^2 \cdot s^{-5}$) und 45‘ ($Mdn = 15{,}79 \times 10^3 \: m^2 \cdot s^{-5}$), $Z = 2{,}2; p < 0{,}05$. Weitere Unterschiede bestehen zwischen Messzeitpunkten 15‘ ($Mdn = 87{,}65 \: 1/min$) und 45‘ ($Mdn = 87{,}28 \: 1/min$), $Z = -2{,}2; p < 0{,}05$ bei der Doppelschrittfrequenz. Bei der \ac{AFP} konnte der Test keine signifikanten Unterschiede feststellen. 

% subsubsection beobachtungen (end)
\subsubsection{Korrelationsanalyse} 

% (fold)
\label{subs:korrelationsanalyse}

Im den nächsten Analyseschritten vernachlässigte ich die Bedingung der Unabhängigkeit der Stichproben. Ich führte eine bivariate Korrelationsanalyse durch, um lineare Zusammenhänge zwischen den einzelnen Merkmalen zu untersuchen. Die Korrelationmatrix (Tabelle~\ref{tab:korrelationen_erste_studie_laufen}) zeigt die signifikante Zusammenhänge zwischen Generalfaktor und seiner beiden Dimension \emph{glatter Verlauf} und \emph{Absorbiertheit}. Zusätzlich korrelieren beide Dimension positiv. Zwischen den impliziten Merkmalen (mittlere \ac{HR}, \acs{RMSSD}, mittlerer normalisierter Shannon Entropie Index, mittlerer Doppelschrittfrequenz und Bewegungsaufwand) und den explizit erhobenen Flow-Merkmalen konnte ich keine lineare Zusammenhänge feststellen. Weitere signifikante positive Zusammenhänge zeigt die Korrelationsmatrix zwischen \ac{AFP} und mittlerer \ac{HR} und zwischen mittlerer Doppelschrittfrequenz und mittlerem Bewegungsaufwand. 
\begin{sidewaystable}
	\centering \caption[Korrelationsmatrix (Erste Studie: Laufen)]{Korrelationsmatrix der ersten Studie zum Flow-Erleben beim Laufen: Arithmetisches Mittel, Standardabweichung und Korrelationen [$N = 24$]\\
	\hspace{ 
	\textwidth} \emph{Anmerkung}: Bew. = Bewegungsaufwand \\
	\hspace{ 
	\textwidth}* Korrelation ist auf dem Niveau von 0,05 (zweiseitig) signifikant \\
	\hspace{ 
	\textwidth}** Korrelation ist auf dem Niveau von 0,01 (zweiseitig) signifikant} \label{tab:korrelationen_erste_studie_laufen} 
	\begin{tabular}
		{lxxxxxxxxxx} \toprule & M & SD & 1 & 2 & 3 & 4 & 5 & 6 & 7 & 8 \\
		\midrule 1. Generalfaktor & 4,74 & 0,39 & & & & & & & & \\
		2. Glatter Verlauf & 5,01 & 0,41 & 0,93^{**} & & & & & & & \\
		3. Absorbiertheit & 4,34 & 0,47 & 0,87^{**} & 0,61^{**} & & & & & & \\
		4. AFP & 3,38 & 0,65 & 0,12 & -0,04 & 0,31 & & & & & \\
		5. Herzfrequenz ($1/min$) & 174,40 & 4,35 & 0,14 & 0,15 & 0,09 & 0,56^{**} & & & & \\
		6. RMSSD ($ms$) & 10,14 & 3,14 & -0,30 & -0,25 & -0,30 & -0,13 & -0,02 & & & \\
		7. Norm. Shan. Entr. Index & 0,18 & 0,15 & -0,32 & -0,35 & -0,20 & 0,06 & 0,11 & 0,10 & & \\
		8. Doppelschrittfr. ($1/min$) & 87,69 & 0,59 & 0,13 & 0,28 & -0,11 & -0,26 & -0,02 & -0,16 & -0,38 & \\
		9. Bew. ($\times 10^3 \: m^2 \cdot s^{-5}$) & 17,09 & 1,95 & 0,16 & 0,23 & 0,02 & -0,32 & -0,25 & -0,24 & -0,34 & 0,46^{*} \\
		\bottomrule 
	\end{tabular}
\end{sidewaystable}

% subsubsection korrelationsanalyse (end)
\subsubsection{Regressionsanalyse} 

% (fold)
\label{ssub:regressionsanalyse} 
\begin{figure}
	[!htb] \input{./tikz/5_8_regression_analyse} \caption[Quadratische Zusammenhänge zwischen expliziten und impliziten Merkmalen beim Laufen]{Quadratische Zusammenhänge zwischen expliziten und impliziten Merkmalen beim Laufen. (A) Generalfaktor und mittlere Doppelschrittfrequenz; (B) mittlerer Bewegungsaufwand und mittlere HR; (C) Mittlerer normalisierter Shannon Entropie Index und mittlere HR\\
	\hspace{ 
	\textwidth}\emph{Anmerkung}: Gestrichelte Linie stellt das bestmögliche quardratische Modell dar.} \label{fig:5_8_regression_analyse} 
\end{figure}

Ich führte Regressionsanalysen durch, um quadratische Zusammenhänge wie z.~B. bei \citet{Peifer2014} zu finden ($y=\beta_{0}+\beta_{1}x+\beta_{2}x^{2}+e$). Dabei testete ich quardratische Ausdrücke auf der Seite der unabhängigen Variable $x$. Wie bei der Korrelationanalyse wird eine lineare Regression durchgeführt. Zur Bestimmung der Modellparameter $\beta_i$ wird die Methode der kleinsten Quadrate eingesetzt. 

Ich fand einen signifikanten Zusammenhang zwischen einem expliziten erhobene Merkmal und einem implizit gemessenen Merkmal. Der Zusammenhang von Flow-Erleben, gemessen durch den Generalfaktor der \ac{FKS}, und der mittleren Doppelschrittfrequenz drückt sich durch ein umgedrehtes U aus, $R^2 = 0{,}3; F(2, 21) = 4{,}41; p < 0{,}05$ (Abbildung~\ref{fig:5_8_regression_analyse}, A). Alle weiteren signifikanten Zusammenhänge bestehen zwischen zwei implizit gemessenen Merkmalen. Die Ergebnisse zeigen einen Zusammenhang in Form eines Us zwischen dem mittleren Bewegungsaufwand und der mittleren HR , $R^2 = 0{,}33; F(2, 21) = 5{,}11; p < 0{,}05$ (Abbildung~\ref{fig:5_8_regression_analyse}, B) und einen Zusammenhang in Form eines umgedrehten Us zwischen dem mittleren normalisierten Shannon Entropie Index der kardio-lokomotorischen Phasensynchronisation und der mittleren HR , $R^2 = 0{,}64; F(2, 21) = 18{,}85; p < 0{,}001$ (Abbildung~\ref{fig:5_8_regression_analyse}, C).

% subsubsection regressionsanalyse (end)
\subsubsection{Prozessorientierter Ansatz} 

% (fold)
\label{ssub:prozessorientierter_ansatz_1} 
\begin{sidewaysfigure}
	\resizebox{1.00 
	\textwidth}{!}{
	
	%
	\input{./tikz/5_9_prozessorientierter_ansatz} }
	
	%
	\caption[Beispielhafte Prozessdarstellung des letzten Laufabschnittes vom 07. November 2013 (Erste Studie: Laufen)]{Beispielhafte Prozessdarstellung des letzten Laufabschnittes vom 07. November 2013 \\
	\hspace{ 
	\textwidth} \emph{Anmerkung}: Rel. Phase = Relative Phase} \label{fig:5_9_prozessorientierter_ansatz} 
\end{sidewaysfigure}

Im dritten Schritt des Herangehens (Abschnitt~\ref{sec:herangehensweise}) begutachtete ich die vollständigen 15 Minuten der Zeitreihen der impliziten Messungen in R und Google Earth. Abbildung~\ref{fig:5_9_prozessorientierter_ansatz} stellt als Beispiel die Datenreihen des letzten Messezeitraumes des letzten Laufes vom 07.~November 2013 dar. Die mittlere Doppelschrittfrequenz und der Bewegungsaufwand besitzen den Doppelschritt als zeitliche Einheit. Die zeitliche Einheit der mittleren \ac{HR} sowie eines Punktes in der relativen Phase der kardio-lokomotorischen Phasensynchronisation ist ein Herzschlag. Den normalisierten Shannon Entropie Index berechnete ich jede Sekunde über ein 15 Sekunden Fenster der relativen Phase der kardio-lokomotorischen Phasensynchronisation. 

In allen 24 15-minütigen Zeitreihen der Doppelschrittfrequenz konnte ich keine markanten Muster erkennen, die ich für die Unterscheidung von \emph{nicht Flow} und \emph{Flow} in Betracht ziehen würde. Markante Muster treten relativen Phase der kardio-lokomotorischen Synchronisation auf, können aber zum jetzigen Zeitpunkt nicht von mir \emph{nicht Flow} oder \emph{Flow} zugeordnet werden. Die drei Muster sind: 
\begin{itemize}
	
	\item Aufsteigende Punkte beschreiben ein Ungleichgewicht, indem die Doppelschrittfrequenz höher ist als die \ac{HR} (Abbildung~\ref{fig:5_9_prozessorientierter_ansatz}, bis Sekunde 3000, zwischen Sekunde 3200 und Sekunde 3400)
	
	\item Wagerechte Linien beschreiben ein Gleichgewicht zwischen Doppelschrittfrequenz und \ac{HR} (Abbildung~\ref{fig:5_9_prozessorientierter_ansatz}, zwischen Sekunde 3000 und Sekunde 3200, zwischen Sekunde 3400 und Sekunde 3475)
	
	\item Absteigende Punkte beschreiben ein Ungleichgewicht, indem die Doppelschrittfrequenz niedriger ist als die \ac{HR} (Abbildung~\ref{fig:5_9_prozessorientierter_ansatz}, ab Sekunde 3475) 
\end{itemize}

Der normalisierte Shannon Entropie Index zeigt an, wo das Gleichgewicht am ausgeprägtesten ist. Eine Unterscheidung zwischen aufsteigenden und absteigenden Punkten macht der Algorithmus nicht. 

% subsubsection prozessorientierter_ansatz (end)
% subsection ergebnisse (end)
\subsection{Diskussion} 

% (fold)
\label{sub:diskussion}

Ziel der in diesem Abschnitt beschriebenen Studie war es, zu untersuchen, ob subjektive, durch Experience Sampling erhobene, explizite Flow-Merkmale mit impliziten Merkmalen beim Laufen in einem Zusammenhang stehen und ob ich im zeitlichen Verhalten (Prozess) der impliziten Daten Übergänge zwischen \emph{Flow} und \emph{nicht Flow} identifizieren kann. 

Im ersten Schritt sammelte ich explizite und implizite Daten während des Laufens und untersuchte im zweiten Schritt lineare und quadratische Zusammenhänge zwischen den berechneten expliziten und impliziten Merkmalen. 

Im Gegensatz zu den Studie von \citet{Stoll2005} und \citet{Schuler2009} hatte die Dauer des Laufes keinen Effekt auf die expliziten Flow-Merkmale. Ein Grund dafür könnte die geringe Dauer von einer Stunde gegenüber der Dauer für die Bewältigung eines Marathons gewesen sein.

In der Zusammenhangsanalyse fand ich die nachfolgenden signifikanten Zusammenhänge: 
\begin{itemize}
	
	\item Flow-Erleben, gemessen durch den Generalfaktor der \ac{FKS}, und die Doppelschrittfrequenz stehen in einen quadratischen Zusammenhang in Form eines umgedrehten Us. Demzufolge begünstigt eine optimale Doppelschrittfrequenz das Flow-Erleben. Beim Läufer lag dieser optimale Wert zwischen 87,5 $1/min$ und 88 $1/min$. Dieses Ergebnis steht ein einer Linie mit der These von \citet[][S.~148]{Peifer2012}, in der das Flow-Erleben mit optimaler physiologischer Aktivierung (Optimized Physiological Activation) für die entsprechende Aktivität einhergeht, während alle anderen Prozesse runterreguliert werden.
	
	\item Der Bewegungsaufwand und die Doppelschrittfrequenz stehen in einem positiven linearen Zusammenhang, d. h. umso tiefer/höher die Doppelschrittfrequenz ist, umso tiefer/höher ist der Bewegungsaufwand.
	
	\item Der Bewegungsaufwand und die mittlere \ac{HR} stehen in einen quadratischen Zusammenhang in Form eines Us. Demzufolge geht eine optimale mittlere \ac{HR} mit einem geringeren Bewegungsaufwand einher.
	
	\item Ein ähnlicher Zusammenhang besteht zwischen der kardio-lokomotorischen Phasensynchronisation, gemessen durch den mittleren normalisierten Shannon Entropie Index, und der mittleren \ac{HR}. Dieser quadratische Zusammenhang hat die Form eines umgedrehten Us und bedeutet, dass eine optimale mittlere \ac{HR} eine hohe kardio-lokomotorische Phasensynchronisation begünstigt. Beim Läufer lag dieser optimale Wert bei ca. 175 $1/min$.
	
	\item Ein positiver linearer Zusammenhang besteht zwischen der \ac{AFP} und der mittleren \ac{HR}, umso tiefer/höher die mittlere \ac{HR} umso geringer/höher bewertete der Läufer die \ac{AFP}. 
\end{itemize}

Diese fünf signifikanten Zusammenhänge zeigen logische Beziehungen zwischen Flow-Erleben, \ac{AFP} und den Ebenen \emph{Physiologisch} und \emph{Motorisch} des prozessorientierten Flow-Modells, welches ich in Abschnitt~\ref{sec:flow_erleben_als_moment_der_tatigkeit} eingeführt habe. Nichtsdestotrotz fehlt eine aussagekräftige Verbindung von der mittlere \ac{HR}, der kardio-lokomotorischer Phasensynchronisation und des Bewegungsaufwands zu explizit erhobenen Flow-Erleben. Denn eine optimale Doppelschrittfrequenz für Flow-Erleben hat z.~B. keine optimale mittlere \ac{HR} zur Folge. In dieser Studie habe ich optimale Zustände entdeckt, kann diese aber nicht direkt mit dem Flow-Konstrukt empirisch in Verbindung bringen. 

Zusammenhänge mit dem \acs{RMSSD} als Merkmal der Kurzzeit-\ac{HRV}, die z.~B. durch eine Abnahme der mentalen und körperlichen Anstrengung resultieren, konnte ich keine finde. Dies kann ich mir mit der starke Reduktion der \ac{HRV} unter physischer Belastung erklären.

Die Doppelschrittfrequenz ist bisher das implizite Merkmal, welches eine direkte Verbindung zum explizit erhobenen Flow-Erleben gewährleistet. Eine optimale Doppelschrittfrequenz müsste für jede Untersuchungsperson oder Benutzer einer App in mehreren Läufen berechnet werden. Demzufolge ist die Verallgemeinerung dieses impliziten Merkmals eine komplexe Aufgabe. Zudem war ich im dritten Schritt nicht in der Lage markante Muster im zeitlichen Verlauf der Doppelschrittfrequenz wahrzunehmen, anhand ich zwischen \emph{nicht Flow} oder \emph{Flow} unterscheiden konnte.

Ein Merkmal, welches sich aufgrund seiner Eigenschaft unkompliziert verallgemeinern lässt und zudem markante Muster im zeitlichen Verlauf aufweist, ist die kardio-lokomotorische Phasensynchronisation Abschnitt~\ref{ssub:die_kardio_lokomotorische_phasensynchronisation}. Leider konnte ich in dieser Studie keine Zusammenhänge zwischen hoher kardio-lokomotorischen Phasensynchronisation als optimalen Zustand und Flow-Erleben feststellen. 

Meine Annahme ist, dass die kardio-lokomotorischen Phasensynchronisation mit ihren drei Zuständen (Abschnitt~\ref{ssub:prozessorientierter_ansatz_1}) mit der \ac{AFP} im Zusammenhang steht. Eine direkten Verbindung konnte ich nicht herstellen, da sich die drei Muster nicht 1 zu 1 auf eine Unterforderung, eine Überforderung oder einen optimalen Zustand abbilden lassen. Die kardio-lokomotorische Phasensynchronisation drückt die Regulierung des Herzens auf Grundlage der Schrittfrequenz aus. Zum Beispiel kann ein Läufer fünf Minuten in einer hohen Intensität laufen und es kann eine hohe Phasensynchronisation zwischen Bewegungsapparat und Herz entstehen. Im Anschluss der fünf Minuten läuft der Läufer ein moderates Tempo. Es startet ein Regulierungsprozess, der sich in der relativen Phase in absteigende Punkte ausdrückt (Abbildung~\ref{fig:5_9_prozessorientierter_ansatz}, ab Sekunde 3475). Dieser Prozess ist nicht mit einer Überforderung gleichzusetzen. In einem Zeitabschnitt mit gleichbleibend erhöhter mittlere \ac{HR} hingegen, können wir von einer Überforderung ausgehen. Ebenfalls können wir in der andere Richtung bei aufsteigenden Punkten nicht immer von Unterforderung ausgehen, auch wenn diese Regulierung schneller vom menschlichen Organismus bewerkstelligt wird.

Abgesehen von den dargestellten Ausnahmen macht ein Zusammenhang zwischen kardio-lokomotorischer Phasensynchronisation und \ac{AFP} sinn und bildet eine Verbindung zu Flow-Kanalmodell (Abbildung~\ref{fig:3_1_kanalmodell}). Auch wenn ich wie \citet{Stoll2005} und \citet{Reinhardt2006} keinen signifikanten Zusammenhang von Flow-Erleben und \ac{AFP} feststellen konnte, berichte ich in Abschnitt~\ref{sec:flow_und_laufen_interindividuell} von einer Studie, die diese Verbindungen ein weiteres Mal interindividuell mit einer Gruppe von Läufern prüft.

% Abnahme der mentalen und körperlichen Anstrengung
% subsection diskussion (end)
% section flow_und_laufen_intraindividuell (end)

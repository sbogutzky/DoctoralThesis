

%!TEX root = /Users/sbogutzky/Entwicklung/projects/bogutzky/repositories/2939413/final-draft.tex
Befassen wir uns eingehend mit den in Abschnitt~\ref{sec:losungsansatze_zur_impliziten_messung_von_flow_erleben} genannten Forschungsarbeiten zu verschiedenartigen impliziten Merkmalen des Flow-Erlebens und dessen Problemen unter physischer Belastung in Abschnitt~\ref{sec:probleme_bei_messungen}, fällt uns auf, dass uns Erkenntnisse fehlen, um Flow-Erleben mobil und prozessorientiert beim Gehen und Laufen zu messen.

Ein Großteil der Studien zur Messung von physiologischen Merkmalen des Flow-Erlebens wurde \emph{unter kontrollierten Laborbedingungen} und bei \emph{geringer physischer Belastung} durchgeführt. 

Ihre Ergebnisse basieren überwiegend auf experimentellen Untersuchungsplänen mit \emph{mehreren Untersuchungspersonen} und \emph{interindividuellen Vergleichen}. Die Individualität jeder einzelnen Untersuchungsperson wurde nicht berücksichtig.

Mir als Autor sind keine veröffentlichten Ergebnisse von Studien bekannt, in denen untersucht wurde, implizite Merkmale, die auf der \emph{Biomechanik der Tätigkeit beruhen}, zur Messung des Flow-Erlebens heranzuziehen. Mit Blick auf Aussagen, die von einer effizienteren Durchführung der Tätigkeit im Flow-Erleben berichten, ist die Untersuchung dieser impliziten Merkmale von hohem Interesse.

Alle vorgestellten Forschungsarbeiten vergleichen \emph{Flow-Zustände} mit \emph{akkumulierten physiologischen Merkmalen}. Es fehlt an prozessorientierten Ansätzen, die den zeitlichen Verlauf des Flow-Erlebens anhand von impliziten Merkmalen beschreiben.

Angesichts dieser vier Forschungslücken suche und prüfe ich in den nachfolgenden beschriebenen Studien potentielle Kandidaten für ein implizites Flow-Messverfahren, das eine prozessorientierte Objektivierung für den Einsatz in Apps gewährleistet. Ich beschreibe insgesamt drei Studien und deren Ergebnisse.

Der Fokus der in Abschnitt~\ref{sec:flow_und_laufen_intraindividuell} beschriebenen Studie liegt auf dem Laufen einer Untersuchungsperson. In ihr suchte ich Zusammenhänge zwischen subjektiven durch Experience Sampling erhobenen expliziten Flow-Merkmalen und Kandidaten für ein implizites Messverfahren des Flow-Erlebens beim Laufen. Zusätzlich diente sie als technische Machbarkeitsstudie. 

Mit der in Abschnitt~\ref{sec:flow_und_gehen_intraindividuell} beschriebenen Studie untersuchte wir die Tätigkeit des Gehens intraindividuell mit dem Untersuchungsaufbau der ersten Studie. Sie diente als Machbarkeitstudie des \acs{BMBF}-Projekts zur kardio-lokomotorischen Phasensynchronisation. Die kardio-lokomotorischen Phasensynchronisation wurde von mir nach der Gehstudie in die erste Laufstudie übernommen. 

In Abschnitt~\ref{sec:flow_und_laufen_interindividuell} dokumentiere ich meine finale Studie, in der erneut das Laufen im Fokus steht. In ihr überprüfe ich einen direkten und indirekten Zusammenhang von Flow-Erleben und kardio-lokomotorischer Phasensynchronisation. Bei dieser Studie handelt es sich um eine Studie mit interindividuellen Vergleichen auf der Grundlage von 31 Untersuchungsperson beim Laufen. 

Ich weise darauf hin, dass in den beschriebenen Studien unterschiedliche mobile Geräte und Programme bzw. Versionen von Programmen zum Einsatz gekommen sind. Ich und die beiden im Kontext des \acs{BMBF}-Projekts eingestellten und genannten studentischen Hilfskräfte (Abschnitt~\ref{sec:arbeitsschritte_der_technischen_arbeit}) führten mehrfach Verbesserungen im Laufe der Forschungsarbeit durch, um die Programme an den jeweiligen Anforderungen der Studien anzupassen. Erst die Erfahrung, die wir durch die Verwendung verschiedener Iterationen mobiler Geräte und Programme erlangten, versetzten uns in die Lage, die nötigen Änderungen zu erkennen und zu realisieren. Die in einer Studie verwendeten mobilen Geräte und Programme beschreibe ich jeweils im zugehörigen Abschnitt. Dort nenne und erläutere ich Änderungen und Verbesserungen, die wir gegenüber früheren Versionen vorgenommen haben. Ich verzichte auf ein gesondertes Kapitel zur verwendeten Gerätetechnik.


%!TEX root = /Users/sbogutzky/Entwicklung/projects/bogutzky/repositories/2939413/final-draft.tex
\section{Flow und Laufen (interindividuell)} 

% (fold)
\label{sec:flow_und_laufen_interindividuell}

\subsection{Einleitung} 

% (fold)
\label{sub:einleitung_5_3}

In Anbetracht der Erkenntnisse der ersten Laufstudie (Abschnitt~\ref{sub:diskussion_5_1}) gehen wir davon aus, das die Ebenen \emph{Physiologisch} und \emph{Motorisch} aus dem prozessorientierten Modell (Abbildung~\ref{fig:prozessorientiertes_flow_modell_2}) bei Flow-Zuständen in Zusammenhang stehen~\citep{Grueter2016a}. Dieser Zusammenhang drückt sich in Flow-Zuständen in einer individuellen Schrittfrequenz aus. Gleichzeitig steht die Schrittfrequenz mit der \ac{HR} und dem Bewegungsaufwand in Verbindung. Deshalb gehe ich davon aus, dass eine individuelle optimale \ac{HR} und eine individuelle optimale niedriger Bewegungsaufwand mit dem Flow-Erleben einhergehen, auch wenn ich dies in Abschnitt~\ref{sec:flow_und_laufen_intraindividuell} nicht belegen konnte. 

Alle drei genannten impliziten Merkmale konnten in der vorherigen Laufstudie keine prozessbasierten Erkenntnisse liefern. Die kardio-lokomotorische Phasensychronisation hingegen (Abschnitt~\ref{ssub:die_kardio_lokomotorische_phasensynchronisation}) kann diese Lücke schließen und stellt gleichzeitig die Verbindung der beiden Ebenen \emph{Physiologisch} und \emph{Motorisch} des prozessorientierten Modell (Abbildung~\ref{fig:prozessorientiertes_flow_modell_2} da. Zudem ist sie das einzige emplizite Merkmale, welches unkompliziert bei interindividuellen Vergleichen einsetzbar ist. 

\citet{Peifer2014} fanden einen quadratischen Zusammenhang zwischen dem Flow-Erleben gemessen mit der Absorbiertheit der \ac{FKS} und dem physiologischen Merkmal \ac{LF} der \ac{HRV}. Ich schließe auf einen ähnlichen Zusammenhang zwischen Flow-Erleben und der mittleren \ac{HR}. \citet{deManzano2010, Gaggioli2013} fanden einen positiven linearen Zusammenhang zwischen Flow-Erleben und mittlerer \ac{HR}. Ihre Ergebnisse stehen nicht im Gegensatz zu meiner Annahme, da in ihren Studien die zweite Hälfte der umgedrehten U-Kurve unberücksichtigt blieb.

Aus einem quadratischen Zusammenhang zwischen Flow-Erleben und mittlerer \ac{HR} geht ein positiver linearer Zusammenhang zwischen Flow-Erleben und der kardio-lokomotorischen Phasensychronisation hervor. Diesen Zusammenhang bin ich in der Lage mit nachfolgenen beschriebenen Studie zu überprüfen mit mehreren Untersuchungsperson, die nur einmal Laufen zu überprüfen. Die Herangehensweise gleich dem in Abschnitt~\ref{sec:herangehensweise} herangehen: (1) Daten sammeln und Merkmale extrahieren, (2) Zusammenhänge zwischen expliziten und impliziten Merkmalen suchen und ggf. (3) den Prozess der Tätigkeit bzw. der Daten der Tätigkeit untersuchen. 

% subsection einleitung (end)
\subsection{Hypothesen} 

% (fold)
\label{sub:hypothesen}

Aus den in der Einleitung dargestellten Annahmen fasse ich die zwei nachfolgenden Forschungshypothesen für die beschriebene Studie zusammen: 
\begin{description}
	\item[$H^1$] Zwischen Flow-Erleben gemessen durch den Generalfaktor der \ac{FKS} und dem normalisierten Shannon Entropie Index der kardio-lokomotorischen Phasensynchronisation besteht ein positiver linearer Zusammenhang beim Laufen. 
	\item[$H^2$] Zwischen Flow-Erleben gemessen durch den Generalfaktor der \ac{FKS} und dem Bewegungsaufwand besteht ein negativer linearer Zusammenhang beim Laufen. 
\end{description}

% subsection hypothesen (end)
\subsection{Methode} 

% (fold)
\label{sub:methode_5_3}

An der Studie nahmen 36 gesunde männliche Untersuchungspersonen im Alter von 18 bis 40 Jahre ($M = 29; SD = 5{,}74$) teil. Zwanzig Untersuchungsteilnehmer gehörten der ersten Herrenmannschaft Fußball und zehn der Altherrenmannschaft Fußball des TSV Dörverden an. Zusätzlich liefen sechs Freizeitläufer in der Studie mit. Alle waren es gewohnt, mindestens eine halbe Stunde in der Woche zu laufen. Von den Untersuchungspersonen hatten vier schon an Laufereignissen wie Halbmarathon oder Marathon teilgenommen. Alle Untersuchungspersonen nahmen freiwillig ohne eine Vergütung an der Studie teil. 

Mit jeder Untersuchungsperson führte ich die Untersuchung einzeln durch. Die Untersuchung fand auf dem Sportplatzgelände, in den Kabinen des TSV Dörverden und auf einer naheliegenden Laufstrecke statt. Die einzelnen Untersuchungen führte ich im Zeitraum von Mitte März bis Ende Mai 2016 durch. Der Beginn jeder Sitzung variierte zwischen 16 Uhr und 19 Uhr und dauerte etwa eine Stunde. Jede Sitzung begann mit der persönlichen Begrüßung der Untersuchungsperson und mit dem Hinweis, dass ich als Untersuchungsleiter die Kommunikation vor dem Lauf möglichst gering halten möchte. Ich wies die Untersuchungspersonen daraufhin, dass sie sich im gesamten Verlauf der Sitzung in keiner Prüfungssituation befinden. Im Anschluss zog sich die Untersuchungsperson um und bereitete sich für etwa fünf Minuten auf den Lauf vor. Anschließend erfolgte eine Befragung mit der \ac{FKS} auf Grundlage der vergangenen 10 Minuten, die das Umziehen und die kurze Vorbereitungszeit einbeziehen sollte. 

Danach rüstete ich die Untersuchungsperson mit zwei Shimmer \acp{IMU} mit Gyro-Modulen am linken und rechten Unterschenkel unterhalb der Wade aus. Ich korrigierte die Trageposition der \acp{IMU} basiert auf übliche Tragepositionen auf Hinweisen aus der Literatur \citep[][]{Hreljac1993}. Das Smartphone blieb wie in den Fallstudien am Oberarm befestigt, diente aber nur als Datenverarbeitungsgerät. Es kam ein leistungsstarkes Sony Xperia Z1 mit Android Version 5.1.1 zum Einsatz.

Ich nutzte für die kardiovaskulären Messungen ein physiologisches Monitoring Modul namens BioHarness 3 des Unternehmens Zephyr. Das Modul gewährleistet durch eine interne Vorverarbeitung und eine Zwischenspeicherung gleichwertige kardiovaskuläre Messungen wie das Shimmer EKG-Modul. Vorteile des BioHarness 3 Moduls auf einem Brustgurt sind die einfachere Handhabung beim Anlegen und die Vorverarbeitung des \ac{EKG}-Signals durch das Modul selbst. Nach Herstellerangaben taste das Modul mit 1024~Hz ab. Das gesamte angelegte Equipment ist in Abbildung~\ref{fig:equipment_3} dargestellt. Die Untersuchungsperson wurde von mir angewiesen, jegliches technische Gerät und ggf. ihre Uhr während des Laufes abzulegen.
\begin{figure}
	[!htb] \centering 
	\includegraphics[width=1.00
	\textwidth]{equipment_3} \caption[Equipment der finalen Studie zum Flow-Erleben beim Laufen]{Equipment der finalen Studie zum Flow-Erleben beim Laufen.} \label{fig:equipment_3} 
\end{figure}

Die Laufstrecke wurde von mir ausgewählt (Abbildung~\ref{fig:landkarte_3}), da sie von allen Fußballern des TSV Dörverden des Öfteren gelaufen wurde. Alle Freizeitläufer kannten sich in Dörverden so weit aus, dass es für sie kein Problem war, die Laufstrecke ohne eine Karte zu laufen. Die Strecke besteht aus einem Rundkurs, den die Untersuchungsperson dreimal laufen musste. Die Gesamtstrecke betrug 4,5 Kilometer, war eben und die Untersuchungsperson lief auf Asphalt. Die Laufstrecke führt durch ein Waldgebiet mit geringem Verkehrsaufkommen. Sie besitzt eine merkbare Senkung, in der Regel sich ein Gefühl der Leichtigkeit einstellt. 
\begin{figure}
	[!htb] \centering 
	\includegraphics[height=0.50
	\textheight]{landkarte_3} \caption[Laufstrecke -- Rundkurs]{Laufstrecke -- Rundkurs drei mal -- insgesamt 4,5~km (erstellt aus GPS-Positionen mit OpenStreetMap)} \label{fig:landkarte_3} 
\end{figure}

Ich erläuterte der Untersuchungsperson nach Zeigen der Laufstrecke, dass es in der Untersuchung nicht um die individuelle Laufleistung geht. Ich wies die Untersuchungsperson an, die Laufstrecke in einem Lauftempo zu laufen, das sich für die Untersuchungsperson optimal anfühlt, diese nicht überfordert oder unterfordert. Trotz dieser Anweisung ging ich von Ausreißern nach unten und nach oben aus, da bei jüngeren Fußballern öfters der direkte Vergleich mit anderen im Vordergrund steht und bei einigen der Gedanke vorherrscht, mit wenig Leistung das Optimum zu erreichen. 

Danach begleitete ich die Untersuchungsperson aus der Kabine und startete die Datenaufnahme am Smartphone über den \ac{PPC}. Direkt im Anschluss führte die Untersuchungsperson den Lauf durch. Die Untersuchungsteilnehmer benötigten für die Laufstrecke im Durchschnitt 25 Minuten und 26 Sekunden. 

Ich erwartete den Läufer am Sportgelände zurück und stoppte bei Ankunft die Datenaufnahme. Die Untersuchungsperson beantwortete im Anschluss an eine kurze Erholungspause die \ac{FKS} auf Grundlage des Laufes. Danach erfragte ich verbal, wie die Untersuchungsperson ihr Laufleistungsvermögen auf einer Skala von 1 bis 10 einschätzte, wobei 1 das geringste (unterfordert) und 10 das höchste Leistungsniveau (überfordert) darstellt. Zusätzlich gab mir der Läufer an, ob und wo er Personen auf dem Rundkurs gesehen hat. Außerdem fragte ich ihn, ob er an einer Stelle des Laufes das Gefühl hatte, endlos weiterlaufen zu können. 

Nach dem Kurzinterview füllte die Untersuchungsperson ein Konsensformular aus. Ich beendete die Sitzung und begleitete die Untersuchungsperson aus der Kabine. Ich teilte keinem Läufer vor Abschluss der Studie Ergebnisse wie z.~B. Laufzeit mit. 

% Ich bewertete auf Grundlage der Fragen und der körperlich ersichtlichen Verfassung geheim und subjektiv die physische Beanspruchung der einzelnen Teilnehmer in den drei Kategorien unterfordert, optimal und überfordert.
% subsection methode (end)
\subsection{Apparat} 

% (fold)
\label{sub:apparat}

In der beschriebenen Laufstudie kam eine erweiterte Version des \ac{PPC}s zum Einsatz. Wir erweiterten den \ac{PPC} dahin, dass er Daten des BioHarness 3 über Bluetooth empfangen und im Android Dateisystem speichern kann. Für alle Sensoren, ob interne \ac{GPS}-Einheit, Shimmer \ac{IMU} oder BioHarness erhält der \ac{PPC} die Daten über einen Android Service. Ein Service kapselt Programmlogik, die das Android OS unabhängig von der App ausführen kann. Android OS ist in der Lage, Services automatisch neu zu starten. Es ist möglich, ihnen eine höhere Ausführungspriorität als der sichtbaren Activity (Teil der App mit dem der Benutzer interagiert) zuzuordnen. Der Service ermöglicht, unterschiedlichen sichtbaren Activities die ankommenden Daten zu visualisieren. Wir hielten dieses gerade für die \ac{EKG}-Daten für notwendig, da wir als Untersuchende damit die Anbringung der Elektroden am Oberkörper und die zuverlässige Übertragung der \ac{EKG}-Daten im Vorlauf der Untersuchung überprüfen können. 

Der BioHarness 3 ist ein kompaktes physiologisches Monitoring Modul, das die Erhebung von physiologischen Daten ermöglicht. Er nimmt im Gegensatz zu den Shimmer \acp{IMU} mit Basis-Firmware einige Datenverarbeitungsschritte vor. Damit erhält der \ac{PPC} vom BioHarness 3 vorverarbeitete Daten wie die mittlere \ac{HR}, RR-Intervalle, Atemfrequenz, Haltungswinkel, Beschleunigung und Körpertemperatur des Untersuchenden. Das BioHarness-Modul kann die Untersuchungsperson unterschiedlich tragen: mit einem Brustgurt, mit klebbaren Elektroden oder in einem Kompressionsshirt. Ich nutzte in der beschriebenen Studie den mitgelieferten Brustgurt. Bei der Messung der kardiovaskulären Daten macht sich der BioHarness die Eigenschaft des Aktionspotenzials zunutze, indem er mit der Hilfe des Brustgurts die Veränderung des Potenzials über Äquipotenziallinien misst. Der Brustgurt misst mittels befeuchtbaren Elektrodenfeldern, die mit der Körperoberfläche an zwei verschiedenen Stellen in Kontakt sind, die Potenzialdifferenz. Erreicht die Differenz ein Maximum, geht das Modul von einem Herzschlag aus. 

Wir veränderten zusätzlich die Interaktion mit dem Fragebogen. In der genutzten Version des \ac{PPC} startet der Fragebogen erst durch die Interaktion der Untersuchungsperson, damit der \ac{PPC} die genaue Startzeit, die Dauer und die Endzeit der Selbstauskunft erfassen kann. Dadurch ist eine nachträgliche Identifikation, die in den Fallstudien durchgeführt wurde, nicht mehr nötig. 

Die Sensorkonfiguration der beschriebenen Studie ist der Tabelle~\ref{tab:sensorkonfiguration_3} zu entnehmen. Ich musste keine Veränderungen an der \ac{PPP} durchführen. 
\begin{table}
	[!htb] \caption[Sensorkonfiguration der finalen Studie zum Flow-Erleben beim Laufen]{Sensorkonfiguration der finalen Studie zum Flow-Erleben beim Laufen} \label{tab:sensorkonfiguration_3} 
	\begin{tabularx}
		{
		\textwidth}{p{.30
		\textwidth} p{.20
		\textwidth} p{.50
		\textwidth}} \toprule & Abtastrate & Betriebssicherer Messbereich \\
		\midrule BioHarness EKG & 1024~Hz & \\
		Beschleunigungsmesser & 56,7~Hz & 1,5~g \\
		Kreiselinstrument & 56,7~Hz & 500 $deg \cdot s^{-1}$ \\
		\bottomrule 
	\end{tabularx}
\end{table}

% subsection apparat (end)
% section flow_und_laufen_interindividuell (end)



%!TEX root = /Users/sbogutzky/Entwicklung/projects/bogutzky/repositories/2939413/final-draft.tex
\section{Gegenstand, Problemstellung und Ziele} 

% (fold)
\label{sec:gegenstand_problemstellung_und_ziele}

Um Flow-Erleben durch die Anwendungen der Mensch-Computer-Interaktion (human-computer interaction) zu unterstützen, ist ein implizites Messverfahren zwingend notwendig. Ein implizites Messverfahren beschreibt ein Messverfahren, das ohne die Unterbrechung der ausgeübten Tätigkeit auskommt und das Erleben beiläufig ohne mündliche Auskünfte erfasst.

Beim Laufen sind psychologische Merkmale und Bedingungen für das Flow-Erleben in mehreren Studien dokumentiert und untersucht worden \citep{Stoll2005, Reinhardt2006, Schuler2009, Jimenez-Torres2013}. Ihre Erkenntnisse basieren allesamt auf explizit erfragte Merkmale des Flow-Erlebens durch Selbstauskünfte. Es fehlen jedoch Erkenntnisse über psycho-physiologischen Zusammenhänge des Flow-Erlebens (implizite Merkmale) unter \emph{physischer Belastung}. 

Studien zu impliziten Merkmalen des Flow-Erlebens betrachten überwiegend sitzende Tätigkeiten \citep{deManzano2010, Keller2011, Peifer2014, Tozman2015, Harmat2015}. Ihre Ergebnisse basieren überwiegend auf experimentellen Untersuchungsplänen mit mehreren Untersuchungspersonen und interindividuellen Vergleichen. Nach \citet[][S.77]{Henk2014} wären \emph{intraindividuelle Untersuchungen} für aussagekräftige Schlussfolgerungen notwendig: Auf individueller Ebene kann beispielsweise ein Zusammenhang zwischen einem erlebten Flow-Zustand und hoher Effizienz der Bewegung beim Gehen bestehen, auch wenn die Effizienz interindividuell gesehen als gering einzuordnen ist.

Im Flow erlebt der Handelnde „den Prozess als ein einheitliches ‚Fließen‘ von einem Augenblick zum nächsten, wobei er Meister seines Handelns ist“ \citep[][S.~59]{Csikszentmihalyi2010}. Flow ist das „reflexionsfreie, gänzliche Aufgehen in einer glatt laufenden Tätigkeit, die man trotz hoher Beanspruchung noch unter Kontrolle hat“ \citep[][S.~156]{Rheinberg2003}. Im Flow verschmelzen Handlung und Bewusstsein und der Handelnde erlebt den Handlungsverlauf „als glatt und fließend“ \citep[][S.~13]{Henk2014}.

Betrachten wir die vorherigen Beschreibungen des Flow-Erlebens, deuten die Abschnitte, die das Erleben als fließend oder glatt beschreiben, auf \emph{psycho-motorische} Zusammenhänge im Flow-Erleben hin. Ungeachtet dessen gibt es keine empirischen Untersuchungen.

Flow-Erleben auf der Basis von explizit erfragten Merkmalen im Prozess der Tätigkeit Laufen dokumentierten \citet{Reinhardt2006, Schuler2009} und rekonstruierten den Verlauf des Flow-Erlebens über die Zeit. \emph{Prozessorientierte Messungen von physiologischen und motorischen Merkmalen} zum Vergleich mit mehrfachen Selbstauskünften über das Flow-Erleben wurde hingehen noch nicht durchgeführt.

Angesichts der vier Forschungslücken 
\begin{itemize}
	
	\item fehlende implizite Merkmale des Flow-Erlebens unter physischer Belastung,
	
	\item fehlende Berücksichtigung der Individualität bei der Ausführung der Tätigkeit und beim Erleben von Flow,
	
	\item fehlende Empirie zu psycho-motorische Zusammenhänge des Flow-Erlebens und
	
	\item fehlender Vergleich von physiologischen und motorischen Merkmalen mit explizit erfragten Merkmalen des Flow-Erlebens im Prozess 
\end{itemize}
suche und prüfe ich potentiellen Kandidaten für ein implizites Flow-Messverfahren, das eine prozessorientierte Objektivierung für den Einsatz in Apps gewährleistet. 

Lauf-Apps dienen als Anwendungsbeispiel, in denen Flow-Erleben z.~B. als Maß des Wohlbefindens, die überwiegend leistungsbezogenen Kennzahlen wie z.~B. zurückgelegte Strecke, Kalorienverbrauch, usw.\ ergänzt. Damit möchte ich zur Bildung des Fundaments beitragen, Lauf-Apps zu entwickeln, die die Bedingungen verbessern, Flow beim Laufen zu erleben. Der Gegenstand der vorliegenden Arbeit mit seinen Problemen und dem von mir verfassten Ziel entstand im Kontext des \acs{BMBF}-Projekts \emph{Flow-Maschinen: Körperbewegung und Klang} (10/2012 bis 10/2015) an der Hochschule Bremen.

% section gegenstand_problemstellung_und_ziele (end)



%!TEX root = /Users/sbogutzky/Entwicklung/projects/bogutzky/repositories/2939413/final-draft.tex
\section{Herangehensweise} 

% (fold)
\label{sec:herangehensweise}

Zurzeit dominieren explizite zustandsorientierte Messverfahren die Untersuchungen des Flow-Erlebens. Sie sind charakterisiert durch ihre indirekte Beziehung zum Flow-Erleben und ihre Erfassung nach der zu untersuchenden Tätigkeit. Diese Messverfahren erfragen die Komponenten des Flow-Konstrukts, welche ich in Kapitel~\ref{cha:flow_erleben_messen} vorstelle. 

In drei Schritten nähere ich mich einem impliziten Flow-Messverfahren, das eine prozessorientierte Objektivierung für den Einsatz in Apps beim Gehen und Laufen gewährleistet: 
\begin{enumerate}
	\item Um den statischen Charakter der expliziten zustandsorientierten Messverfahren aufzulösen, messe ich wie \citet{Reinhardt2006, Schuler2009} die expliziten Flow-Merkmale durch mehrfache Selbstauskünfte in einer fortlaufenden Tätigkeit. Je nach zeitlichem Abstand kann ich damit den erlebte Flow über die Zeit rekonstruieren. Gleichzeitig messe ich implizite physiologische und kinematische Daten. Zu diesem Zweck entstand eine mobile Messanwendung für Smartphones, auf die ich in Kapitel~\ref{cha:studien_zur_mobilen_und_prozessorientierten_messung} eingehe.
	
	\item Die Daten dienen in akkumulierter Form als implizite Merkmale (Kandidaten), die ich den Komponenten des Flow-Konstrukts zuordne. Ausgehend von den expliziten Merkmalen (abhängige Variable) untersuche ich die Korrelation zu diesen Kandidaten für ein implizites Messverfahren (unabhängige Variable).
	
	\item Ich untersuche das zeitliche Verhalten (Prozess) der impliziten Daten, um Übergänge zwischen \emph{Flow} und \emph{nicht Flow} beschreiben zu können. 
\end{enumerate}

Aufgrund der gegebenen logischen Abhängigkeit von Schritt 2 und 3 liegt das Augenmerk zunächst auf Schritt 2, der Suche nach signifikanten Korrelationen zwischen expliziten Merkmalen und impliziten Kandidaten des Flow-Erlebens. 

% section herangehensweise (end)



%!TEX root = /Users/sbogutzky/Entwicklung/projects/bogutzky/repositories/2939413/final-draft.tex
\section{Aufbau der Arbeit} 

% (fold)
\label{sec:aufbau_der_arbeit}

In Kapitel~\ref{cha:technologie_beim_laufen} schaffe ich ein Verständnis für die zahlreichen Aspekte, mit denen ich mich in der vorliegenden Arbeit im Bereich der mobilen Mensch-Computer-Interaktion in Bezug auf Sport und Laufen beschäftige. Es zeigt im vorgestellten Bereich der mobilen Mensch-Computer-Interaktion technologische Anknüpfungspunkte für ein implizites Messverfahren des Flow-Erlebens auf.

Ich stelle in Kapitel~\ref{cha:flow_erleben_messen} Flow-Erleben als messbare Größe dar und veranschauliche die Merkmale, die das Messen eines Bewusstseinszustands wie Flow-Erleben ermöglichen. Ich beleuchte die expliziten Messverfahren, bespreche prozessorientierte Ansätze mit expliziten Messverfahren und stelle Lösungsansätze zur impliziten Messung von Flow-Erleben, deren Datenerhebung eine Echtzeitverarbeitung ermöglichen, vor.

In Kapitel~\ref{cha:flow_erleben_beim_gehen_und_laufen_messen_anforderungen} gebe ich einen Überblick über die Anforderungen an ein implizites Messverfahren, das eine mobile und computergestützte Echtzeitverarbeitung beim Gehen und Laufen zulässt. Ich veranschauliche die technischen Arbeitsschritte, die zu der Entwicklung einer App notwendig sind, die die Voraussetzungen verbessert, Flow beim Gehen und Laufen zu erleben.

Ich stelle in Kapitel~\ref{cha:studien_zur_mobilen_und_prozessorientierten_messung} drei Studien vor und diskutiere ihre Methode und Ergebnisse. Bei den ersten beiden Studien (Laufen und Gehen) nutze ich einen intraindividuellen Versuchsaufbau und gehe nach den drei beschriebenen Schritten in Abschnitt~\ref{sec:herangehensweise} vor. In der dritten Studie stelle auf der Grundlage der Ergebnisse der ersten beiden Studien Forschungshypothesen auf und prüfe diese mit einer Gruppe von Läufern (interindividuell).

In Kapitel~\ref{cha:integration_in_eine_assistierende_echtzeit_benutzerschnittstelle} veranschauliche ich die mögliche Integration einer impliziten Messung des Flow-Erlebens in eine Echtzeit-Benutzerschnittstelle. Dabei stelle ich den Demonstrator der Flow-Maschine vor.

Eine zusammenfassende Diskussion mit Bewertung und Einordnung der Ergebnisse der vorliegenden Arbeit liefere ich in Kapitel~\ref{cha:diskussion}. Abschließend fasse ich in Kapitel~\ref{cha:wissenschaftlicher_beitrag_und_ausblick} die mit der vorliegenden Dissertation geleisteten wissenschaftlichen Beiträge zusammen und gebe einen Ausblick auf Anknüpfungspunkte für fortführende Forschungsarbeiten.

% section aufbau_der_arbeit (end)

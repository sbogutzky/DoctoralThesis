

%!TEX root = /Users/sbogutzky/Entwicklung/projects/bogutzky/repositories/2939413/final-draft.tex
\section{Aufbau der Arbeit} 

% (fold)
\label{sec:aufbau_der_arbeit}

In Kapitel~\ref{cha:technologie_beim_laufen} schaffe ich ein Verständnis für die zahlreichen Aspekte, mit denen ich mich in der vorliegenden Arbeit im Bereich der mobilen Mensch-Computer-Interaktion in Bezug auf Sport und Laufen beschäftige. Es zeigt im vorgestellten Bereich der mobilen Mensch-Computer-Interaktion technologische Anknüpfungspunkte für ein implizites Messverfahren des Flow-Erlebens auf.

Ich stelle in Kapitel~\ref{cha:flow_erleben_messen} Flow-Erleben als messbare Größe dar und veranschauliche die Merkmale, die das Messen eines Bewusstseinszustands wie Flow-Erleben ermöglichen. Ich beleuchte die expliziten Messverfahren, stelle Lösungsansätze zur impliziten Messung von Flow-Erleben, deren Datenerhebung eine Echtzeitverarbeitung ermöglichen, vor und bespreche prozessorientierte Ansätze mit expliziten Messverfahren.

In Kapitel~\ref{cha:flow_erleben_beim_gehen_und_laufen_messen_anforderungen} beleuchte ich die Auswahl der Tätigkeiten Gehen und Laufen. Ich stelle Untersuchungen vor, die Zusammenhänge von Flow-Erleben beim Laufen untersuchten. Ich begründe die Auswahl der expliziten und impliziten Messverfahren für diese Arbeit und veranschauliche die Nutzung des Smartphones als wissenschaftliches Werkzeug.

Ich stelle in Kapitel~\ref{cha:studien_zur_mobilen_und_prozessorientierten_messung} drei Studien vor und diskutiere ihre Methode und Ergebnisse. Die erste Studie galt der Suche nach Zusammenhängen zwischen subjektiven durch Experience Sampling erhobenen expliziten Flow-Merkmalen und Kandidaten für ein implizites Messverfahren des Flow-Erlebens beim Laufen (intraindividuell). In der (zeitlich gesehen) zweiten Studie untersuchten wir die Tätigkeit des Gehens. Diese Studie diente als Machbarkeitsstudie des \acs{BMBF}-Projekts zur kardio-lokomotorischen Phasensynchronisation. In der finalen Studie überprüfte ich einen direkten und indirekten Zusammenhang zwischen Flow-Erleben und kardio-lokomotorischer Phasensynchronisation beim Laufen (interindividuell).

In Kapitel~\ref{cha:integration_in_eine_assistierende_echtzeit_benutzerschnittstelle} veranschauliche ich die mögliche Integration einer impliziten Messung des Flow-Erlebens in eine Echtzeit-Benutzerschnittstelle. Eine zusammenfassende Diskussion mit Bewertung und Einordnung der Ergebnisse der vorliegenden Arbeit liefere ich in Kapitel~\ref{cha:diskussion}. Abschließend fasse ich in Kapitel~\ref{cha:wissenschaftlicher_beitrag_und_ausblick} die mit der vorliegenden Dissertation geleisteten wissenschaftlichen Beiträge zusammen und gebe einen Ausblick auf Anknüpfungspunkte für fortführende Forschungsarbeiten.

% section aufbau_der_arbeit (end)

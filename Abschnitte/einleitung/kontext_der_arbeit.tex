

%!TEX root = /Users/sbogutzky/Entwicklung/projects/bogutzky/repositories/2939413/final-draft.tex
\section{Kontext der Arbeit} 

% (fold)
\label{sec:kontext_der_arbeit}

Körperliche Bewegung wie z.~B. Gehen und Laufen ist eine der Voraussetzungen von Gesundheit. Die Weltgesundheitsorganisation WHO definiert Gesundheit als „vollständiges körperliches, geistiges und soziales Wohlergehen [\textellipsis]“ \citep[S.~100]{WorldHealthOrganization1948}. Mangelnde körperliche Bewegung im Alltag und ungesunde Ernährung gefährden den Zustand und verursachen Erkrankungen wie Fettleibigkeit, Haltungsschäden und Herzerkrankungen. Programme zur Prävention und Rehabilitation der Erkrankungen setzen darauf, ihre Teilnehmer zu regelmäßiger körperlicher Bewegung zu motivieren. Nicht selten scheitern die Teilnehmer daran, regelmäßige körperliche Bewegung in ihren Alltag zu integrieren.

Die Zielsetzung des \acs{BMBF}-Projekts \emph{Flow-Maschinen: Körperbewegung und Klang} war, Menschen im Alltag bei ihren gesundheitsförderlichen Absichten aktiv zu unterstützen. Der Ausgangspunkt des Projektvorhabens war die Annahme, dass das Erleben von Flow während der körperlichen Bewegung uns Menschen intrinsisch motiviert, sich mehr zu bewegen. Zu diesem Zweck ist der Gegenstand des Projekts die Gestaltung, Realisierung und Erprobung von Flow-Maschinen. Flow-Maschinen sind mobile Applikationen (Apps), die das Gehen von Kindern und Jugendlichen, Erwachsenen und älteren Menschen durch Klang unterstützen, sich unkompliziert in den Alltag ihrer Benutzer integrieren lassen und das Erleben von Flow fördern.

Das Projektvorhaben war in drei relativ eigenständige Teilvorhaben organisiert, die bei der Entwicklung eines vorläufigen computergestützten Modells des Erlebens beim Gehen (Protomodell) und bei der iterativen Entwicklung der Flow-Maschinen integriert zusammenwirkten: 
\begin{itemize}
	
	\item Die Theoriebildung und die prozessorientierte Modellierung von Erleben beim Gehen wurde durch Prof.\ Barbara Grüter, Diplompsychologin und Professorin der Mensch-Computer-Interaktion, vorangetrieben.
	
	\item Die Möglichkeiten zur Unterstützung des Erlebens beim Gehen mit dem Smartphone durch Klang wurde von Nassrin Hajinejad entwickelt und untersucht \citep{Hajinejad2013, Hajinejad2015}. Dabei verfolgte sie einen daten- und erfahrungsorientierten Bottom-up-Ansatz.
	
	\item Die Suche nach potentiellen Kandidaten für ein implizites Flow-Messverfahren, das eine prozessorientierte Objektivierung für den Einsatz in Apps gewährleistet, wurde von mir durchgeführt und stellt den Hintergrund der vorliegenden Arbeit dar. Im Gegensatz zu Hajinejad verfolgte ich einen analytischen theoriegeleiteten Top-down-Ansatz, dessen Herangehensweise ich im nachfolgenden Abschnitt vorstelle. 
\end{itemize}

% section kontext_der_arbeit (end)



%!TEX root = /Users/sbogutzky/Entwicklung/projects/bogutzky/repositories/2939413/final-draft.tex
\section{Motivation} 

% (fold)
\label{sec:motivation}

Warum schnüren Millionen von Menschen in der Welt tagtäglich ihre Sportschuhe und laufen? Weil sie Freude an der Ausübung der Tätigkeit empfinden.

Den Gemütszustand der Freude verbinden wir bei vielen Tätigkeiten bzw. Anforderungssituationen mit dem Erleben von Flow. Flow"=Erleben ist ein Bewusstseinszustand, in dem wir auf unsere Tätigkeit fokussiert sind. Sportler verknüpfen Flow"=Erleben in der Regel mit hervorragenden Momenten, in denen ihr Körper und ihr Geist mühelos zusammen arbeiten \citep[S.~5]{Jackson1999}. Das Flow"=Konzept ist untrennbar mit dem Namen Mihaly Csikszentmihalyi verbunden. Csikszentmihalyi untersuchte in den 1970er"=Jahren, warum Menschen intrinsisch (ihrer selbst willen) motiviert sind, zu spielen. Csikszentmihalyis Untersuchungen mündeten in der Entdeckung von Flow. Er fand ihn z.~B. beim Schach spielen, Klettern im Felsen, Rock"=Tanzen und Arbeiten.

\citet[][S.~58f.]{Csikszentmihalyi2010} beschreibt das Erleben von Flow als völliges Aufgehen in einer Tätigkeit. Florian \citet[][S.~13]{Henk2014} definiert ihn als ein Verschmelzen von Handlung und Bewusstsein, das sich durch das gleichzeitige Erleben des Handlungsverlaufs als glatt und fließend und des gänzlichen Aufgehens in der Tätigkeit auszeichnet. Für \citet[][S.~602]{Csikszentmihalyi2005} ist Flow der ausschlaggebende Grund, warum Menschen eine Tätigkeit als intrinsisch belohnend empfinden. Die Ergebnisse der Studie von \citet[][S.~174]{Schuler2009} lassen darauf schließen, dass Flow durch die intrinsische Belohnung zur langfristigen Erhaltung der Ausübung einer sportlichen Aktivität, wie Laufen, beiträgt.

% section motivation (end)

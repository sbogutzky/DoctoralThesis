
\section{Zusammenfassung}
\label{sec:zusammenfassung_5}

In diesem Kapitel habe ich drei Studien beschrieben. 

In der ersten Studie (Abschnitt~\ref{sec:flow_und_laufen_intraindividuell}) suchte ich nach Zusammenhänge zwischen subjektiven durch Experience Sampling erhobenen expliziten Flow-Merkmalen und Kandidaten für ein implizites Messverfahren des Flow-Erlebens beim Laufen. Ihr fand ich folgende intraindividulle Zusammenhänge:

	\item Flow-Erleben, gemessen durch den Generalfaktor der \ac{FKS}, und die Doppelschrittfrequenz stehen in einen quadratischen Zusammenhang in Form eines umgedrehten Us. Demzufolge begünstigt eine optimale Doppelschrittfrequenz das Flow-Erleben. Dieses Ergebnis steht ein einer Linie mit der These von \citet[][S.~148]{Peifer2012}, in der das Flow-Erleben mit optimaler physiologischer Aktivierung (Optimized Physiological Activation) für die entsprechende Aktivität einhergeht, während alle anderen Prozesse runterreguliert werden.
	
	\item Der Bewegungsaufwand und die Doppelschrittfrequenz stehen in einem positiven linearen Zusammenhang, d. h. umso tiefer/höher die Doppelschrittfrequenz ist, umso tiefer/höher ist der Bewegungsaufwand.
	
	\item Der Bewegungsaufwand und die mittlere \ac{HR} stehen in einen quadratischen Zusammenhang in Form eines Us. Demzufolge geht eine optimale mittlere \ac{HR} mit einem geringeren Bewegungsaufwand einher.
	
	\item Ein ähnlicher Zusammenhang besteht zwischen der kardio-lokomotorischen Phasensynchronisation, gemessen durch den mittleren normalisierten Shannon Entropie Index, und der mittleren \ac{HR}. Dieser quadratische Zusammenhang hat die Form eines umgedrehten Us und bedeutet, dass eine optimale mittlere \ac{HR} eine hohe kardio-lokomotorische Phasensynchronisation begünstigt. 
	
	\item Ein positiver linearer Zusammenhang besteht zwischen der \ac{AFP} und der mittleren \ac{HR}, umso tiefer/höher die mittlere \ac{HR} umso geringer/höher bewertete der Läufer die \ac{AFP}. 
\end{itemize}

Aufgrund der Erkenntnis stellte ich die Annahme auf, das die kardio-lokomotorische Phasensynchronisation eine \emph{physiologisch messbare \ac{AFP}} darstellt und damit eine Verbindung zum Flow-Kanalmodell (Abbildung~\ref{fig:kanalmodell}) bildet.

In der zeitlich gesehen zweite Studie (Abschnitt~\ref{sec:flow_und_gehen_intraindividuell}) untersuchten wir die Tätigkeit des Gehens intraindividuell mit dem Untersuchungsaufbau der ersten Studie. Sie diente als Machbarkeitstudie des \acs{BMBF}-Projekts zur kardio-lokomotorischen Phasensynchronisation. 

Die beiden in dieser Studie identifizierten positiv zu bewertenden Eigenschaften der kardio-lokomotorischen Phasensynchronisation: 
\begin{itemize}
	
	\item relative Prozessbezogenheit mit markanten Mustern und
	
	\item unkomplizierte interindividuelle Vergleichbarkeit
\end{itemize}

ließen mich die kardio-lokomotorische Phasensynchronisation im Nachhinein in der ersten Laufstudie einsetzen.   

In Abschnitt~\ref{sec:flow_und_laufen_interindividuell} dokumentiere ich meine finale Studie, in der erneut das Laufen im Fokus steht. In ihr überprüfe ich einen direkten und indirekten Zusammenhang von Flow-Erleben und kardio-lokomotorischer Phasensynchronisation. Bei dieser Studie handelt es sich um eine Studie mit interindividuellen Vergleichen auf der Grundlage von 31 Untersuchungsperson beim Laufen. 

Alle Zusammenhänge benötigen zusätzliche Untersuchungen mit hinreichenden befriedigenden Ergebnissen, bevor einer der Kandidaten als implizites Messverfahren des Flow-Erlebens automatisiert und in Echtzeit mobil von jemandem einzusetzen ist. 


%!TEX root = /Users/sbogutzky/Entwicklung/projects/bogutzky/repositories/2939413/final-draft.tex
\chapter{Diskussion} \label{cha:diskussion} Im vorherigen Kapiteln motivierte ich das Erleben von Flow in einer Echtzeit-Benutzerschnittstelle für Sport und Gesundheit zu nutzen. Ich stellte hierzu Kandidaten für ein implizites Messverfahren des Flow-Erlebens vor und fand direkten intraindividuellen Zusammenhang von Flow-Erleben und Doppelschrittfrequenz (Abschnitt~\ref{ssub:effekt_der_physiologisch_gemessenen_afp}) und einen indirekten Zusammenhang zwischen Flow-Erleben und der kardio-lokomotorische Phasensynchronisation als eine implizit \emph{physiologisch messbare \ac{AFP}} (Abschnitt~\ref{ssub:regressionsanalyse}). Im diesem Kapitel resümiere ich die wesentlichen Erkenntnisse meiner Arbeit und diskutiere sie zusammenfassend auf der Grundlage der Herangehensweise (Abschnitt~\ref{sec:herangehensweise}). 

\section{Datensammlung} 

% (fold)
\label{sec:datensammlung}

Die Datensammlung mit dem \ac{PPC} ist als sehr effizient anzusehen. Nichtsdestotrotz gibt es bei meiner Methode Daten zu erheben einige Diskussionpunkte. Ein Punkt ist die intervall-kontingente Erhebung durch die \ac{FKS}. Bei der Erfassung der erlebten Flows bei kurzen Abständen der Befragungen kann es zu monotonen stereotypen Antworten kommen. Allgemein ist das größte Problem der \ac{FKS} und anderen psychometrischen Skalen, dass unklar bleibt, wie die Antwort in Form eines Kreuzes an einer Stelle der Skala zustande kommt. Ein weiterer Punkt ist die verstrichende Zeit zwischen dem Erleben und der Beanwortung des Fragebogens. Im Fall der psychometrische Skala erhalten wir überhaupt keinen Anhangspunkt, in welchen zeitlichen Raum die Untersuchungsperson das Erleben bewertet hat. Die Aufzeichnung der jeweiligen Tätigkeit durch eine Videokamera und das Durchgehen der Aufzeichnung mit der befragten Person könnte beide erstgenannten Punkte eine geignetere Methode darstellen \citep[Video-Recall][S.~566]{Leuchter2006}. Die Video-Recall löst gleichzeitig ein weiteres Problem, die Auswahl des Zeitraums der Daten zur Analyse, die von mir in den Studien für jede Untersuchungsperson gleich vorgenommen wurde (fünf und 15 Minuten vor der Befragung). In unseren Studien könnte es sein, dass die jeweilige Untersuchungsperson in diesen Zeitraum der Tätigkeit garnicht ihr Erleben konkret bewertet hat, sondern die Zeitraum davor. 

Die vorliegende Arbeit ist eine der ersten Arbeiten, die Zusammenhänge von Flow-Erleben und physiologischen Merkmalen unter einer physisch beanspruchenden Tätigkeit untersuchte. Ich versuchte in meiner Forschung an die Arbeiten von \citet{deManzano2010, Keller2011, Gaggioli2013, Peifer2014, Tozman2015}, die allesamt Zusammenhänge zwischen Flow-Erleben und der Herzfrequenzvariabilität nachweisen konnten, anzuknüpfen. Gerade die Interpretation von \ac{HRV}-Merkmalen, die die sympathische und parasympathische Aktivität des \acs{VNS} nachweisen sollen, sind unter physischer Belastung nahezu unmöglich. Grund dafür ist die dramatische Reduktion der \ac{HRV} unter physischer Belastung \citep{Hoos2010}. Ein Möglichkeit trotzdem die \ac{HRV} bei physisch beanspruchenden Tätigkeit einzusetzen sind Vorher-Nachher-Messungen im Zustand der Ruhe. In Fall einer Vorher-Nachher-Messung bleibt uns aber ein Einblick in die Prozesse des Herzens während der Tätigkeit verwehrt und die Ergebnisse würde nicht zu einer prozessorientierten Lösung führen. 

Ein weiterer Punkt ist die Untersuchung selber. In unseren Studien haben wir versucht mit Flow-Erleben einen Bewusstheitszustand zu erfassen. Allgemein wird dieser Bewusstheitszustand oder das Erleben durch das tragen des Equipments, durch die Untersuchungssituation selber und die Anweisungen durch den Untersuchungsleiter gegenüber der eigentlichen Tätigkeit verändert, was Einfluss auf die expliziten Bewertungen, aber auch auf die physiologischen Daten haben kann.

Die Auswahl meiner Kandidaten für die implizite Erfassung des Flow-Erlebens sind theoritisch begründet. Einen Kandidaten habe ich dabei unberücksichtig gelassen. Dieser Kandidat ist die Atmung. Die Atmung ist ein wesentlicher Faktor bei sportlicher Ertüchtigung und \citet{Harmat2015} fanden bei ihrer Untersuchung einen Zusammenhang zwischen Flow-Erleben langsamer tiefer Atmung beim Tetris spielen. Beim Laufen kann man, wie es der Bioharness 3 macht, die Erhebung des Brustkorbs messen. Diese Messung ist aber bisher sehr unspezise. 

% section datensammlung (end)
\section{Zusammenhänge} 

% (fold)
\label{sec:zusammenhange}

Im Kontext des \acs{BMBF}-Projekt nach Kandidaten für ein implizites Messverfahren des Flow-Erlebens, die auf der Biomechanik des Gehens und Laufens beruhen. Dabei stieß Nassrin Hajinejad bei \citet[][S.~121]{Meinel2007} auf den Bewegungsfluss und ich knüpfte zur Quantifizierung an die Arbeit von \citet{Hreljac2000} über Bewegungsaufwand an (Abschnitt~\ref{ssub:der_bewegungsfluss}). Er nutzte den Bewegungsaufwand, um eine Gruppe ambitionierter Läufer mit einer Gruppe von Freizeitläufern beim Laufen und schnellem Gehen zu vergleichen. 

Ich konnte in dieser Arbeit keinen direkten Zusammenhang nachweisen. Ein Grund könnte die Filterung durch individuelle die Grenzfrequenzen darstellen, da eine zu hohe Filterung viele markante Merkmale der Beschleunigung entfernt. In meiner finalen Laufstudie habe ich den Bewegungsaufwand vernachlässigt, da der Nachweis von Flow-Erleben durch den Bewegungsaufwand personenindividuell wäre und bei variierenden Laufgeschwindigkeiten eine komplexere Aufgabe darstellt.

In der intraindividuellen Studie konnte ich einen Nachweis erbringen, das ein Zusammenhang zwischen Flow-Erleben und optimale Doppelschrittfrequenz vorhanden ist. In der finalen Studie konnte ich diesen Zusammenhang nicht nachweisen, da jeder der 31 Untersuchungsperson seine eigene individuelle Schrittfrequenz besitzt. 

Für einen solchen Nachweis wäre eine Laufbandstudie nach dem Beispiel von \citet{Reinhardt2006} die geeignete Methode, um zumindest die Laufgeschwindigkeit konstant zu halten und für alle Untersuchungspersonen vergleichbar zu machen. Weiterhin würde eine Laufbandstudie externe Störfaktoren, die in einer Studie in einem natürlichen Handlungsumfeld immer vorhanden sind und auch Einfluss auf das Erleben besitzen, eliminieren. Es ist aber zu bedenken, dass die Ergebnisse nur bedingt auf die Realität übertragbar sind. \citet{Strohrmann2012} stellten zum Beispiel fest, dass das Laufen auf einem Laufband sich von dem Laufen in Außen-Umgebungen in Bezug auf kinematische Merkmale unterscheidet. 

Ein implizites Merkmal, welches sich unkompliziert interindividuell vergleichen lässt, stellt die kardio-lokomotorische Phasensynchronisation dar. Die kardio-lokomotorische Phasensynchronisation (Abschnitt~\ref{ssub:die_kardio_lokomotorische_phasensynchronisation}). \citet[][S.~18]{Niizeki2014} sehen in hoher kardio-lokomotorischen Phasensynchronisation einen energetisch vorteilhaften Zustand des menschlichen Organismus und Barbara Grüter entwickelte im Kontext des \acs{BMBF}-Projekts die Hypothese, dass die kardio-lokomotorischen Phasensynchronisation als implizites Merkmal des (Flow-) Erlebens beim Gehen zu verwenden ist. Wegen den beiden positiv zu bewertenden Eigenschaften der kardio-lokomotorischen Phasensynchronisation: 
\begin{itemize}
	
	\item relative Prozessbezogenheit mit markanten Mustern und
	
	\item unkomplizierte interindividuelle Vergleichbarkeit 
\end{itemize}

übernahm ich die kardio-lokomotorischen Phasensynchronisation konnte aber in den vorgestellten keinen Studien direkten Zusammenhang zum Flow-Erleben nachweisen.

Hierfür mache ich teilweise die Operationalisierung des Flow-Erlebens durch die \ac{FKS} verantwortlich. Die Items des glatten Verlaufs wurden eher positiv bewertet. Das erschert eine Differzierung über diese Subdimension. Eine Bewertung des glatten Verlaufs (sechs Items) hat zusätzlich einen größeren Einfluß auf den Generalfaktor als die Bewertung der Absorbiertheit (vier Items). Eine bessere Deferenzierbarkeit bietet die Absorbiertheit, aber auch in dieser Skala gab es ein Problem mit dem Item \emph{Ich fühle mich optimal beansprucht}. In der beschriebenen finalen Studie wurde dieses Item wie die Items des glatten Verlaufs eher positiver bewertet, da optimal gegenüber den anderen Items der Skala unterschiedlich auffasst werden kann. Eine Untersuchungsperson ist z.~B. optimal beansprucht, wenn sie am Limit ihrer Leistungsfähigkeit ist und eine andere Untersuchungsperson ist optimal beansprucht, wenn sie beim Laufen sprechen kann. Die anderen Items der Skala Absorbiertheit sind sprachlich eindeutiger ausgeführt worden -- zeitvergessen, vertieft in die Tätigkeit und selbstvergessen. Damit ist dieses Item zumindestens beim Laufen nicht trennscharf der Absorbiertheit zuzuordnen. Die Verwendung der \ac{FKS} beim Gehen und Laufen müsste man bei weiteren Studien überdenken.

Ein weitere Grund zumindest in der finalen Laufstudie könnte das routinemäßige Training ausmachen. In der individuellen Studie konnte der Läufer zum Ende hin immer häufiger von Flow-Zuständen berichten und auch öfter eine kardio-lokomotorische Phasensynchronisation herstellen. Auch wenn ich keinen Zusammenhang feststellen konnte, weißt das auf eine sogenannte Laufroutine hin, die einen Experten und einen Novizen unterscheiden könnten. In der finalen Studie liefen umwiegenden Fußballer, die ich nicht als Experten einstufen würde. Eine Studie mit Amateur- bis Profiläufern könnte zumindest was die kardio-lokomotorische Phasensynchronisation betritt, eindeutigere Ergebnisse liefern.

Trotz dieser beiden Kritikpunkte habe ich einen indirekten Zusammenhang zwischen Flow-Erleben und kardio-lokomotorischer Phasensynchronisation und als eine implizit \emph{physiologisch messbare \ac{AFP}} gefunden. Dieser Zusammensammhang besagt, dass 
\begin{itemize}
	
	\item die Untersuchungspersonen, die eine kardio-lokomotorische Phasensynchronisation herstellen konnten gegenüber den anderen Untersuchungspersonen vertiefter in die Tätigkeit des Laufens waren (gemessen an der Absorbiertheit).
\end{itemize}

Die Absorbiertheit stellt zwar nur ein Teilkonstrukt des Flow-Erlebens dar, wird aber z.~B. von \citet{Peifer2015} als das eindeutigere Flow-Merkmal bezeichnet und in ihren Studie als explizite Flow-Referenz genutzt. Zusätzlich spiegeln die Werte des Generalfaktors in meiner finalen Laufstudie die gleichen Unterschiede wieder. 

% section zusammenhange (end)
\section{Prozess} 

% (fold)
\label{sec:prozess}

In der vorliegenden Arbeit bin ich davon ausgegangen, dass ich die Zustände und die Übergänge zwischen den Zuständen eines Prozesses, die zum Zustand Flow führen, messen kann. Gucke ich mir die impliziten Merkmale der mittleren HR, der mitteren Doppelschrittfrequenz und des Bewegungsaufwand fällt es mir schwer in diesen Signalen mit vielen Datenpunkten markanter Muster (Übergänge) zu erkennen, die etwas mit dem Erleben zu tun haben könnten und sich auch wiederholen. Ein Grund dafür ist die hohe Auflösung der Daten, doch auch eine Reduzierung durch z.~B. einen gleitenden Mittelwert führte zu keiner objektiven Verbesserung des Problems. Die kardio-lokomotorische Phasensynchronisation enthält trotz in der Regel gleicher Anzahl an Datenpunkten markante Muster, die wir mit bloßen Augen erkennen können: 
\begin{itemize}
	
	\item Aufsteigende Punkte beschreiben ein Ungleichgewicht, indem die mittlere Schrittfrequenz höher ist als die mittlere HR (Abbildung~\ref{fig:prozessorientierte_ansicht_2}, links) -- \emph{Schritt dominiert}
	
	\item Wagerechte Linien beschreiben ein Gleichgewicht zwischen mittlerer Schrittfrequenz und mittlerer HR (Abbildung~\ref{fig:prozessorientierte_ansicht_2}, mitte) -- \emph{Gleichgewicht}
	
	\item Absteigende Punkte beschreiben ein Ungleichgewicht, indem die mittlere Schrittfrequenz niedriger ist als die mittlere HR (Abbildung~\ref{fig:prozessorientierte_ansicht_2}, rechts) -- \emph{Herz dominiert} 
\end{itemize}

Nichtdestrotz bleibt mir die genaue Bestimmung von Übergängen beim Flow-Erleben im Prozess verwehrt, da ich keine eindeutigen Hinweise darauf habe, das der Eintritt in eine hohe kardio-lokomotorische Phasensynchronisation (\emph{Gleichgewicht}) mit dem Eintritt in den Zustand \emph{flow} (Abbildung~\ref{fig:prozessorientiertes_flow_modell_2}) und die Rückkehr in eine niegrige kardio-lokomotorische Phasensynchronisation (\emph{Schritt dominiert} und \emph{Herz dominiert}) mit dem Eintritt in den Zustand \emph{nicht flow} (Abbildung~\ref{fig:prozessorientiertes_flow_modell_2}) einhergeht. Eine Präzision der Bestimmung könnte auch hier die Einführung des Video-Recalls bringen.

% section prozess (end)
\section{Integration} 

% (fold)
\label{sec:integration}

Die technische mobile Umsetzung in Echtzeit der kardio-lokomotorische Phasensynchronisation als eine implizit \emph{physiologisch messbare \ac{AFP}} ist heutzutage durch die hohen Rechenleistungen von Smartphones bedenkenlos möglich. Das zusätzliche Equipment wäre aber für den Freizeitgebrauch noch nicht preiswert und komfortabel genug. Die Untersuchungsperson musste in meinen Studien mindestens eine \ac{IMU} am unteren Bein und einen hochwertigen Brustgurt zur Herzfrequenzmessung tragen, damit der \ac{PPC} die benötigten Daten erheben konnte. Mit dem \emph{BioHarness 3} nutzte ich einen Brustgurt, der eine hohe \ac{EKG}-Qualität gewährleistet. Das macht den BioHarness mit einem Anschaffungspreis von über 500 EUR kostspielig für den Freizeitbedarf. 

Eine laufende Person müsste zusätzlich zum \emph{BioHarness 3} mindestens eine tragbare \ac{IMU} zusätzlich tragen. Eine Möglichkeit, auf ein zusätzliches technisches Gerät zu verzichten, ist, ein Konzept zu entwickeln, das das Smartphone selbst als Datenerhebungsgerät verwendet \citep{Strohrmann2013, Strohrmann2014}. Wie ich in Abschnitt~\ref{sec:closed_loop_ansatz} dokumentiert habe, ist ein solches Konzept beim Gehen realisierbar, in dem die Person das Smartphone in der Hosentasche trägt. Damit ist das Konzept aber auf Jeans mit eng anliegenden Taschen beschränkt und darum nicht für jeden Benutzer alltagstauglich. Beim Laufen sehe ich diese Möglichkeit, das Smartphone in der Hosentasche zu tragen, noch weniger, da viele Sporthosen gar keine oder weite Hosentaschen haben. Eine Alternative zur Bestimmung des Bewegungsablaufs am Bein ist die Bestimmung des Bewegungsablaufs am Oberarm, da die Armarbeit entsprechend den Gesetzen der Kreuzkoordination bei geübten Läufern die Beinarbeit bestimmt \citep[][S.~70]{Marquardt2011}.

% section integration (end)

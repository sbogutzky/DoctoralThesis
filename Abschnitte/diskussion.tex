

%!TEX root = /Users/sbogutzky/Entwicklung/projects/bogutzky/repositories/2939413/final-draft.tex
\chapter{Diskussion} \label{cha:diskussion} In den vorherigen Kapiteln regte ich an das Erleben von Flow in einer Echtzeit-Benutzerschnittstelle für Sport und Gesundheit zu nutzen. Ich stellte hierzu Kandidaten für ein implizites Messverfahren des Flow-Erlebens vor und fand einen direkten intraindividuellen Zusammenhang zwischen Flow-Erleben und Doppelschrittfrequenz (Abschnitt~\ref{ssub:effekt_der_physiologisch_gemessenen_afp}) sowie einen Zusammenhang zwischen Flow-Erleben und der kardio-lokomotorischen Phasensynchronisation als eine implizit \emph{physiologisch messbare \ac{AFP}} (Abschnitt~\ref{ssub:regressionsanalyse}). In diesem Kapitel resümiere ich die wesentlichen Erkenntnisse meiner Arbeit und diskutiere sie zusammenfassend auf der Grundlage der Herangehensweise (Abschnitt~\ref{sec:herangehensweise}). 

\section{Datensammlung} 

% (fold)
\label{sec:datensammlung}

Die Datensammlung mit dem \ac{PPC} ist als sehr effizient anzusehen. Dennoch gibt es bei meiner Methode Daten zu erheben einige Diskussionspunkte. Anzuführen ist die intervall-kontingente Erhebung durch die \ac{FKS}. Bei der Erfassung der erlebten Flows -- bei kurzen Abständen der Befragungen -- kann es zu monotonen stereotypen Antworten kommen. Allgemein ist das größte Problem der \ac{FKS} und anderen psychometrischen Skalen, dass unklar bleibt, wie die Antwort in Form eines Kreuzes an einer Stelle der Skala zustande kommt. Ein weiterer Punkt ist die verstrichene Zeit zwischen dem Erleben und der Beantwortung des Fragebogens. Im Fall der psychometrischen Skala erhalten wir überhaupt keinen Anhaltspunkt, in welchen zeitlichen Raum die Untersuchungsperson das Erleben bewertet hat. Die Aufzeichnung der jeweiligen Tätigkeit durch eine Videokamera und das Anschauen der Aufzeichnung mit der befragten Person könnte bei den erstgenannten Punkten eine geeignetere Methode darstellen \citep[Video-Recall,][S.~566]{Leuchter2006}. Das Video-Recall löst gleichzeitig ein weiteres Problem, und zwar die Auswahl des Zeitraums der Daten zur Analyse, die von mir in den Studien für jede Untersuchungsperson gleich vorgenommen wurde (5 und 15 Minuten vor der Befragung). In unseren Studien könnte es sein, dass die jeweilige Untersuchungsperson in diesem Zeitraum der Tätigkeit gar nicht ihr Erleben konkret bewertet hat, sondern z.~B. den Zeitraum davor. 

Die vorliegende Arbeit ist eine der ersten Arbeiten, die Zusammenhänge von Flow-Erleben und physiologischen Merkmalen unter einer physisch beanspruchenden Tätigkeit untersuchte. Ich versuchte in meiner Forschung an die Arbeiten von \citet{deManzano2010, Keller2011, Gaggioli2013, Peifer2014, Tozman2015, Harmat2015}, die allesamt Zusammenhänge zwischen Flow-Erleben und der Herzfrequenzvariabilität nachweisen konnten, anzuknüpfen. Gerade die Interpretation von \ac{HRV}-Merkmalen, die die sympathische und parasympathische Aktivität des \ac{VNS} nachweisen sollen, sind unter physischer Belastung nahezu unmöglich. Grund dafür ist die dramatische Reduktion der \ac{HRV} unter physischer Belastung \citep{Hoos2010}. Eine Möglichkeit, trotzdem die \ac{HRV} bei physisch beanspruchenden Tätigkeiten einzusetzen, sind Vorher-Nachher-Messungen im Zustand der Ruhe. In Fall einer Vorher-Nachher-Messung bleibt uns aber ein Einblick in die Prozesse des Herzens während der Tätigkeit verwehrt und die Ergebnisse würde nicht direkt zu einer prozessorientierten Lösung führen. 

Ein weiterer Punkt ist die Untersuchung selbst. In unseren Studien haben wir versucht, mit Flow-Erleben einen Bewusstseinszustand zu erfassen. Allgemein wird dieser Bewusstseinszustand oder das Erleben durch das Tragen des Equipments, durch die Untersuchungssituation selbst und die Anweisungen durch den Untersuchungsleiter gegenüber der eigentlichen Tätigkeit verändert, was Einfluss auf die expliziten Bewertungen, aber auch auf die physiologischen Daten haben kann.

Die Auswahl meiner Kandidaten für die implizite Erfassung des Flow-Erlebens ist theoretisch begründet. Einen Kandidaten -- die Atmung -- habe ich dabei unberücksichtigt gelassen. Die Atmung ist ein wesentlicher Faktor bei sportlicher Ertüchtigung; \citet{Harmat2015} fanden bei ihrer Untersuchung einen Zusammenhang zwischen Flow-Erleben und langsamer tiefer Atmung beim Tetris-Spielen. Beim Laufen kann man, wie es der BioHarness 3 macht, die Erhebung des Brustkorbs messen. Diese Messung ist aber bisher sehr unpräzise. 

% section datensammlung (end)
\section{Zusammenhänge} 

% (fold)
\label{sec:zusammenhange}

Im Kontext des \acs{BMBF}-Projekts suchten wir nach Kandidaten für ein implizites Messverfahren des Flow-Erlebens, die auf der Biomechanik des Gehens und Laufens beruhen. Dabei stieß Nassrin Hajinejad bei \citet[][S.~121]{Meinel2007} auf den Bewegungsfluss.  Ich knüpfte zur Quantifizierung an die Arbeit von \citet{Hreljac2000} über den Bewegungsaufwand an (Abschnitt~\ref{ssub:der_bewegungsfluss}). Er nutzte den Bewegungsaufwand, um eine Gruppe ambitionierter Läufer mit einer Gruppe von Freizeitläufern beim Laufen und schnellem Gehen zu vergleichen. 

Ich konnte anhand der vorliegenden Arbeit keinen direkten Zusammenhang zwischen Flow-Erleben und Bewegungsaufwand nachweisen. Ein Grund könnte die Filterung durch die individuellen Grenzfrequenzen darstellen, da eine zu hohe Filterung viele markante Merkmale der Beschleunigung entfernt. In meiner finalen Laufstudie habe ich den Bewegungsaufwand vernachlässigt, da der Nachweis von Flow-Erleben durch den Bewegungsaufwand personenindividuell ist und bei variierenden Laufgeschwindigkeiten eine komplexere Aufgabe darstellt.

Für einen solchen Nachweis wäre eine Laufbandstudie nach dem Beispiel von \citet{Reinhardt2006} die geeignete Methode, um zumindest die Laufgeschwindigkeit konstant zu halten und für alle Untersuchungspersonen vergleichbar zu machen. Weiterhin würde eine Laufbandstudie externe Störfaktoren, die in einer Studie in einem natürlichen Handlungsumfeld immer vorhanden sind und auch Einfluss auf das Erleben besitzen, eliminieren. Es ist aber zu bedenken, dass die Ergebnisse nur bedingt auf die Realität übertragbar sind. \citet{Strohrmann2012} stellten zum Beispiel fest, dass das Laufen auf einem Laufband sich von dem Laufen in Außen-Umgebungen in Bezug auf kinematische Merkmale unterscheidet. 

In der intraindividuellen Studie konnte ich einen Nachweis erbringen, dass ein Zusammenhang zwischen Flow-Erleben und optimaler Doppelschrittfrequenz vorhanden ist. In der finalen Studie konnte ich diesen Zusammenhang nicht nachweisen, da jeder der 31 Untersuchungspersonen seine eigene individuelle Schrittfrequenz besaß. 

Ein implizites Merkmal, welches sich unkompliziert interindividuell vergleichen lässt, stellt die kardio-lokomotorische Phasensynchronisation dar (Abschnitt~\ref{ssub:die_kardio_lokomotorische_phasensynchronisation}). \citet[][S.~18]{Niizeki2014} sehen in hoher kardio-lokomotorischer Phasensynchronisation einen energetisch vorteilhaften Zustand des menschlichen Organismus und Barbara Grüter entwickelte im Kontext des \acs{BMBF}-Projekts die Hypothese, dass die kardio-lokomotorische Phasensynchronisation als implizites Merkmal des (Flow-) Erlebens beim Gehen zu verwenden ist. Wegen der beiden positiv zu bewertenden Eigenschaften der kardio-lokomotorischen Phasensynchronisation: 
\begin{itemize}
	
	\item relative Prozessbezogenheit mit markanten Mustern und
	
	\item unkomplizierte interindividuelle Vergleichbarkeit 
\end{itemize}

übernahm ich die kardio-lokomotorische Phasensynchronisation, konnte aber in den vorgestellten Studien nur einen unmittelbaren Zusammenhang mit einer der Dimensionen der \ac{FKS} (\emph{Absorbiertheit}) nachweisen.

Hierfür mache ich teilweise die Operationalisierung des Flow-Erlebens durch die \ac{FKS} verantwortlich. Die Items des \emph{glatten Verlaufs} wurden eher positiv bewertet. Das erschwert eine Differenzierung über diese Subdimension. Eine Bewertung des \emph{glatten Verlaufs} (sechs Items) hat zusätzlich einen größeren Einfluss auf den Generalfaktor als die Bewertung der \emph{Absorbiertheit} (vier Items). Eine bessere Differenzierbarkeit bietet die \emph{Absorbiertheit}, aber auch in dieser Skala gab es ein Problem mit dem Item »Ich fühle mich optimal beansprucht.«. In der beschriebenen finalen Studie wurde dieses Item wie die Items des \emph{glatten Verlaufs} eher positiver bewertet, da \emph{optimal} gegenüber den anderen Items der Skala unterschiedlich aufgefasst werden kann. Eine Untersuchungsperson ist z.~B. optimal beansprucht, wenn sie am Limit ihrer Leistungsfähigkeit ist und eine andere Untersuchungsperson ist optimal beansprucht, wenn sie beim Laufen sprechen kann. Die anderen Items der Skala \emph{Absorbiertheit} sind sprachlich eindeutiger ausgeführt worden -- zeit vergessen, vertieft in die Tätigkeit und selbstvergessen. Damit ist das Item »Ich fühle mich optimal beansprucht.« zu mindestens beim Laufen nicht trennscharf der \emph{Absorbiertheit} zuzuordnen. Die Verwendung der \ac{FKS} beim Gehen und Laufen müsste man bei weiteren Studien überdenken.

Ein weiterer Grund, zumindest in der finalen Laufstudie, könnte das routinemäßige Training ausmachen. In der individuellen Studie konnte der Läufer zum Ende hin immer häufiger von Flow-Zuständen berichten und auch öfter eine kardio-lokomotorische Phasensynchronisation herstellen. Auch wenn ich keinen Zusammenhang feststellen konnte, weist das auf eine sogenannte Laufroutine hin, die einen Experten und einen Novizen unterscheiden könnte. In der finalen Studie liefen überwiegend Fußballer, die ich nicht als Experten einstufen würde. Eine Studie mit Amateur- bis Profiläufern könnte zumindest, was die kardio-lokomotorische Phasensynchronisation betrifft, eindeutigere Ergebnisse liefern.

Trotz dieser beiden Kritikpunkte habe ich einen Zusammenhang zwischen Flow-Erleben und kardio-lokomotorischer Phasensynchronisation als eine implizit \emph{physiologisch messbare \ac{AFP}} gefunden. Dieser Zusammenhang besagt, dass 
\begin{itemize}
	
	\item die Untersuchungspersonen, die eine kardio-lokomotorische Phasensynchronisation herstellen konnten gegenüber den anderen Untersuchungspersonen vertiefter in die Tätigkeit des Laufens waren (gemessen an der \emph{Absorbiertheit}).
\end{itemize}

Die \emph{Absorbiertheit} stellt zwar nur ein Teilkonstrukt des Flow-Erlebens dar, wird aber z.~B. von \citet{Peifer2015} als das eindeutigere Flow-Merkmal bezeichnet und in ihren Studien als explizite Flow-Referenz genutzt. Zusätzlich spiegeln die Werte des Generalfaktors in meiner finalen Laufstudie die gleichen Unterschiede wider. 

% section zusammenhange (end)
\section{Prozess} 

% (fold)
\label{sec:prozess}

In der vorliegenden Arbeit bin ich davon ausgegangen, dass ich die Zustände und die Übergänge zwischen den Zuständen eines Prozesses, die zum Zustand \emph{Flow} führen, messen kann. Wenn ich mir die impliziten Merkmale der mittleren \ac{HR}, der mittleren Schrittfrequenz und des Bewegungsaufwands anschaue, fällt es mir schwer, in diesen Signalen mit vielen Datenpunkten markante Muster (Übergänge) zu erkennen, die etwas mit dem Erleben zu tun haben könnten und sich auch wiederholen. Ein Grund dafür ist die hohe Auflösung der Daten, doch auch eine Reduzierung durch z.~B. einen gleitenden Mittelwert, führte zu keiner objektiven Verbesserung des Problems. Die kardio-lokomotorische Phasensynchronisation enthält trotz -- in der Regel -- gleicher Anzahl an Datenpunkten markante Muster, die wir mit bloßen Augen erkennen können: 
\begin{itemize}
	
	\item Aufsteigende Punkte beschreiben ein Ungleichgewicht, indem die mittlere Schrittfrequenz höher ist als die mittlere \ac{HR} (Abbildung~\ref{fig:prozessorientierte_ansicht_2}, links) -- \emph{Schritt dominiert}.
	
	\item Waagerechte Linien beschreiben ein Gleichgewicht zwischen mittlerer Schrittfrequenz und mittlerer \ac{HR} (Abbildung~\ref{fig:prozessorientierte_ansicht_2}, Mitte) -- \emph{Gleichgewicht}.
	
	\item Absteigende Punkte beschreiben ein Ungleichgewicht, indem die mittlere Schrittfrequenz niedriger ist als die mittlere \ac{HR} (Abbildung~\ref{fig:prozessorientierte_ansicht_2}, rechts) -- \emph{Herz dominiert}. 
\end{itemize}

Dennoch bleibt mir die genaue Bestimmung von Übergängen beim Flow-Erleben im Prozess verwehrt, da ich keine eindeutigen Hinweise darauf habe, dass der Eintritt in eine hohe kardio-lokomotorische Phasensynchronisation (\emph{Gleichgewicht}) mit dem Eintritt in den Zustand \emph{Flow} (Abbildung~\ref{fig:prozessorientiertes_flow_modell_2}) und die Rückkehr in eine niedrige kardio-lokomotorische Phasensynchronisation (\emph{Schritt dominiert} und \emph{Herz dominiert}) mit dem Eintritt in den Zustand \emph{nicht Flow} (Abbildung~\ref{fig:prozessorientiertes_flow_modell_2}) einhergeht. Eine Präzision der Bestimmung könnte auch hier die Einführung des Video-Recalls bringen.

% section prozess (end)
\section{Integration} 

% (fold)
\label{sec:integration}

Die technische mobile Umsetzung in Echtzeit der kardio-lokomotorischen Phasensynchronisation als eine implizit \emph{physiologisch messbare \ac{AFP}}, ist heutzutage durch die hohen Rechenleistungen von Smartphones bedenkenlos möglich. Das zusätzliche Equipment wäre aber für den Freizeitgebrauch noch nicht preiswert und komfortabel genug. Die Untersuchungsperson musste in meinen Studien mindestens eine \ac{IMU} am unteren Bein und einen hochwertigen Brustgurt zur Herzfrequenzmessung tragen, damit der \ac{PPC} die benötigten Daten erheben konnte. Mit dem BioHarness 3 nutzte ich einen Brustgurt für über 500 EUR, der eine hohe \ac{EKG}-Qualität gewährleistet; das macht den BioHarness 3 zu kostspieligen Equipment für den Freizeitbedarf. 

Eine laufende Person müsste zusätzlich zum BioHarness 3 mindestens eine tragbare \ac{IMU} tragen. Eine Möglichkeit, auf ein zusätzliches technisches Gerät zu verzichten, ist, ein Konzept zu entwickeln, dass das Smartphone selbst als Datenerhebungsgerät verwendet \citep{Strohrmann2013, Strohrmann2014}. Wie ich in Abschnitt~\ref{sec:closed_loop_ansatz} dokumentiert habe, ist ein solches Konzept beim Gehen realisierbar, in dem die Person das Smartphone in der Hosentasche trägt. Damit ist das Konzept aber auf Jeans mit eng anliegenden Taschen beschränkt und darum nicht für jeden Benutzer alltagstauglich. Beim Laufen sehe ich diese Möglichkeit, das Smartphone in der Hosentasche zu tragen, noch weniger, da viele Sporthosen gar keine oder weite Hosentaschen haben. Eine Alternative zur Bestimmung des Bewegungsablaufs am Bein ist die Bestimmung des Bewegungsablaufs am Oberarm, da die Armarbeit entsprechend den Gesetzen der Kreuzkoordination bei geübten Läufern die Beinarbeit bestimmt \citep[][S.~70]{Marquardt2011}.

% section integration (end)

%!TEX root = /Users/sbogutzky/Entwicklung/projects/bogutzky/repositories/2939413/final-draft.tex
\chapter{Flow-Erleben messen}
\label{cha:flow_erleben_messen}
Im gegenwärtigen Kapitel vermittele ich das Wissen über das Flow-Erleben als messbare Größe und veranschauliche die Merkmale, die das Messen eines Bewusstseinszustands wie Flow-Erleben ermöglichen. Gleichzeitig betone ich die Voraussetzungen für das Erleben von Flow. Im Anschluss beleuchte ich die expliziten Messverfahren des Flow-Erlebens. Darauf folgen Lösungsansätze zur impliziten Messung von Flow-Erleben, deren Datenerhebung eine Echtzeitverarbeitung ermöglichen.

In Abschnitt~\ref{sec:was_ist_flow} widme ich mich dem Zustand Flow und dessen Komponenten. Die aus den Komponenten abgeleiteten Merkmale und Voraussetzungen stelle ich in Abschnitt~\ref{sub:merkmale_voraussetzungen_und_folgen} vor. Eine Form der Modellierung von Flow-Erleben erläutere ich in Abschnitt~\ref{sub:modellierung}. In Abschnitt~\ref{sec:explizite_messverfahren_des_flow_erlebens} beschäftige ich mit expliziten Messverfahren des Flow-Erlebens, bevor ich in Abschnitt~\ref{sec:loesungsansaetze_zur_impliziten_messung_von_flow_erleben} auf Lösungsansätze zur impliziten Messung von Flow-Erleben eingehe. In den Abschnitten~\ref{sub:der_bewegungsfluss} und~\ref{sub:die_kardio_lokomotorische_phasensynchronisation} präsentiere ich mit dem Bewegungsfluss und der kardio-lokomotorischen Phasensynchronisation eigene im Projekt "`Flow-Maschinen Körperbewegung und Klang"' entwickelte Lösungsansätze zur impliziten Messung von Flow während des Gehens und Laufens. Abschnitt~\ref{zusammenfassung_3} fasst das Kapitel zusammen.

\section{Was ist Flow?}
\label{sec:was_ist_flow}

Csikszentmihalyi fand 1975 erstmals über qualitative Methoden Zugang zum Flow-Phänomen. Er beschrieb es als das "`holistische Gefühl bei völligem Aufgehen in einer Tätigkeit \citep[S.~58f.]{Csikszentmihalyi2010}"'. Seither sind unterschiedliche Definitionen des Flow-Erlebens entstanden, die sich in ihren Merkmalen und Voraussetzungen nicht wesentlich unterscheiden. Zum Beispiel definieren \citet[vgl.][S.~263]{Rheinberg2003} Flow als den Zustand des reflexionsfreien, gänzlichen Aufgehens in einer glatt laufenden Tätigkeit und Rheinberg differenziert zusammenfassend sechs Komponenten des Flow-Erlebens \citep[S.~153ff.]{Rheinberg2008}:

\begin{enumerate}
\item Passung zwischen Fähigkeit und Anforderung. Man fühlt sich optimal beansprucht und hat trotz hoher Anforderungen das sichere Gefühl, das Geschehen noch unter Kontrolle zu haben.
\item Handlungsanforderungen und Rückmeldungen werden als klar und interpretationsfrei erlebt, so dass man jederzeit und ohne nachzudenken weiß, was jetzt richtig zu tun ist.
\item Der Handlungsablauf wird als glatt erlebt. Ein Schritt geht flüssig in den nächsten über, als liefe das Geschehen gleitend wie aus einer inneren Logik. (Aus dieser Komponente rührt wohl die Bezeichnung "`Flow"'.)
\item Man muss sich nicht willentlich konzentrieren, vielmehr kommt die Konzentration wie von selbst, ganz so wie die Atmung. Es kommt zum Ausblenden aller Kognitionen, die nicht unmittelbar auf die jetzige Ausführungsregulation gerichtet sind.
\item Das Zeiterleben ist stark beeinträchtigt; man vergisst die Zeit und weiß nicht, wie lange man schon dabei ist. Stunden vergehen wie Minuten.
\item Man erlebt sich selbst nicht mehr abgehoben von der Tätigkeit, man geht vielmehr gänzlich in der eigenen Aktivität auf (sog. "`Verschmelzen"' von Selbst und Tätigkeit). Es kommt zum Verlust von Reflexivität und Selbstbewusstheit.
\end{enumerate}

\subsection{Merkmale, Voraussetzungen und Folgen}
\label{sub:merkmale_voraussetzungen_und_folgen}
Die Merkmale und die Voraussetzungen des Flow-Erlebens sind den sechs Komponenten (siehe Tabelle~\ref{tab:merkmale_eines_flow_zustandes} und Tabelle~\ref{tab:voraussetzungen_fuer_einen_flow_zustand}). Demzufolge nutzt die vorliegende Arbeit die Flow-Definition von \citet{Henk2014}, da sie Merkmale und Voraussetzungen nicht miteinander vermischend als ein Verschmelzen von Handlung und Bewusstsein definiert, das sich durch das gleichzeitige Erleben des Handlungsverlaufs als glatt und fließend und des gänzlichen Aufgehens in der Tätigkeit auszeichnet.

Flow-Erleben führt zu einer intrinsischen Motivation, die die Menschen verleitet, eine Tätigkeit erneut auszuführen \citep[vgl.][S.~602]{Csikszentmihalyi2005}. Zu den motivierenden Faktoren gehören u. a. die Gemüts- und Gefühlszustände, Freude und Wohlbefinden. Die wiederholende Ausführung unter Freude führt längerfristig zu einer Verbesserung der Fähigkeiten.

\begin{table}[h]
	\caption[Merkmale eines Flow-Zustands]{Merkmale eines Flow-Zustands nach \citet{Henk2014}}
	\label{tab:merkmale_eines_flow_zustandes}
	\begin{tabularx}{\textwidth}{*{2}{>{\RaggedRight\arraybackslash}X}}
\toprule
Kernmerkmale & weitere Merkmale \\
\midrule
Verschmelzen von Handlung und Bewusstsein: & Zentrierung der Aufmerksamkeit \\
- gänzliches Aufgehen in der Tätigkeit & Selbstvergessenheit \\
- glatter Handlungsverlauf & Verlust des Zeitgefühls \\
& keine Besorgtheit über Misserfolg \\
\bottomrule
\end{tabularx}
\end{table}

\begin{table}[t]
	\caption[Voraussetzungen für einen Flow-Zustand]{Voraussetzungen für einen Flow-Zustand nach \citet{Henk2014}}
	\label{tab:voraussetzungen_fuer_einen_flow_zustand}
	\begin{tabularx}{\textwidth}{*{1}{>{\RaggedRight\arraybackslash}X}}
\toprule
Voraussetzungen \\
\midrule
Gleichgewicht zwischen Anforderungen der Tätigkeit und eigenen Fähigkeiten \\
klare Handlungsschritte und -ziele \\
unmittelbare, eindeutige Rückmeldungen \\
\bottomrule
\end{tabularx}
\end{table}

\subsection{Modellierung}
\label{sub:modellierung}

Die positiven Folgen des Erlebens von Flow veranlassen mich und viele Wissenschaftler, aus unterschiedlichen Forschungsdisziplinen, ihn als Maß nutzen zu wollen. Die bisherige Flow-Forschung stützt sich bei der Quantifizierung des Flow-Erlebens auf zwei dimensionale Modelle. Auf der Grundlage von Befragungen nach Tätigkeiten wie Schach spielen, Klettern im Felsen, Rock-Tanzen und der Chirurgie entwickelte \citet[S.~75]{Csikszentmihalyi2010} das erste und bisher eingängigste Modell der Flow-Theorie. Das Kanalmodell beschreibt das Flow-Erleben anhand des Zusammenhangs zwischen dem Verhältnis von Anforderungen einer Tätigkeit und Fähigkeiten einer Person. Es besagt, dass sobald die Anforderungen und Fähigkeiten auf demselben Niveau liegen, ist eine Person in der Lage ist, Flow zu erleben (siehe Abbildung~\ref{fig:3_1_kanalmodell}). Im Laufe der Flow-Forschung entwickelten unterschiedliche Wissenschaftler verfeinerte Modelle wie das Quadrantenmodell \citep[S.~286]{Csikszentmihalyi1995} oder das Oktantenmodell \citep[S.~296]{Massimini1995}. Die genannten Modelle betrachten das Erleben von Flow als einen konkreten Zustand, der nur Eintritt, wenn ein Gleichgewicht zwischen Anforderungen und Fähigkeiten sowie ein konkretes Niveau an Anforderung vorhanden ist.

\begin{figure}[t]
	\centering
		\includegraphics[width=1.00\textwidth]{3-1-kanalmodell.pdf}
	\caption[Das Kanalmodell des Flow-Erlebens]{Das Kanalmodell des Flow-Erlebens nach \citet[S.~75]{Csikszentmihalyi2010}}
	\label{fig:3_1_kanalmodell}
\end{figure}

Ich verfolge in der vorliegenden Arbeit einen prozessorientierten Ansatz, der das Flow-Erleben als Momente während eines Prozesses (Anforderungssituation) definiert. Ein Prozess besteht aus mehreren einzelnen Momenten (Zuständen und Übergängen zwischen Zuständen). Als Erstes stellen wir die Voraussetzung her, Flow zu erleben. Im Anschluss folgt ggf. das Erleben, das erhalten bleibt, sich verändert oder wieder verschwindet. Flow-Erleben besitzt in der beschriebenen Sicht Intensitätsunterschiede nicht nur zwischen verschiedenen Anforderungssituationen, sondern genauso im Verlauf der ein und derselben Anforderungssituation.

Flow-Erleben ist bildlich vergleichbar mit dem Schlafen. Zunächst schaffen wir die Voraussetzungen. Wir ziehen uns den Schlafanzug an und decken uns zu. Im Anschluss schließen wir die Augen und beenden unsere Gedankengänge. Wir fallen in den Schlaf (Erleben). Wir sind nicht in der Lage selbstständig zu bestimmen, wann wir einschlafen. Der Schlaf besitzt unterschiedliche Phasen, z.~B. die Tiefschlafphase (Intensitätsunterschiede). Am morgen erwachen wir, sind aber ohne externe Helfer wie z.~B. dem Wecker nicht in der Lage, das Erwachen selbstständig zu beeinflussen.

In der vorliegenden Arbeit gehe ich davon aus, dass ich in der Lage bin, die Zustände und die Übergänge zwischen den Zuständen eines Prozesses, der zu Flow-Erleben führt, zu messen. Der erste Schritt zu Bestätigung der Annahme ist, die Identifizierung von impliziten Merkmalen des Flow-Erlebens beim Gehen und Laufen. Implizite Merkmale kategorisiere ich in der vorliegenden Arbeit in Merkmale des Gehirns, physiologische und motorische Merkmale. Physiologische Eigenschaften wie die Herzfrequenz, die Atemfrequenz, das Schwitzen oder das Anspannen konkreter Muskelpartien steuert das \ac{VNS} ohne bewusste menschliche Kontrolle. Im Gegensatz sind wir in der Lage, die Motorik mehr oder weniger willkürlich in einem Bewegungsablauf zu steuern. Durch die Trennung von Physiologie und Motorik erhalte ich für jeden Moment im Prozess drei Ebenen unter den Funktionen des Bewusstseins (siehe Abbildung~\ref{fig:3_2_modellraum-1}).

\begin{sidewaysfigure}
	\includegraphics[width=1.00\textwidth]{3-2-modellraum-1.pdf}
	\caption[Ein Modellraum des Flow-Erlebens (Ausgangspunkt)]{Ein Modellraum des Flow-Erlebens (Ausgangspunkt). Quelle: Eigene Darstellung}
	\label{fig:3_2_modellraum_1}
\end{sidewaysfigure}

\section{Explizite Messverfahren des Flow-Erlebens}
\label{sec:explizite_messverfahren_des_flow_erlebens}

Subjektive, nachträgliche Auskünfte über das Erleben durch strukturierte Interviews oder durch standardisierte psychometrische Skalen gehören zu den erprobtesten und zuverlässigen Flow-Messverfahren. Als Grundlage der expliziten Messverfahren dienen die in Tabelle~\ref{tab:merkmale_eines_flow_zustandes} und Tabelle~\ref{tab:voraussetzungen_fuer_einen_flow_zustand} vorgestellten Merkmale und Voraussetzungen. Im nachfolgenden Abschnitt gebe ich einen kurzen Überblick über ausgewählte Befragungsmethoden zur expliziten Erfassung des Flow-Erlebens auf der Bewusstseinsebene.

\subsection{Strukturierte Interviews und Fragebögen}

Strukturierte Interviews und Fragebögen fragen das Vorhandensein von Flow-Voraussetzungen und -Merkmalen ab. Die Erhebung mit Hilfe von strukturierten Interviews oder Fragebögen findet in der Regel im Anschluss an die Tätigkeit (Anforderungssituation) statt. Die Untersuchenden sind in der Lage mit einer solchen Erhebung festzustellen, ob die befragte Person bei der Durchführung der jeweiligen Tätigkeit Flow erlebte.

Für ein Interview und für die Beantwortung eines Fragebogens gilt: Umso länger die Zeitspanne zwischen der jeweiligen Tätigkeit und der Befragungsmethode ist, desto größer ist die Gefahr, dass die befragte Person Einzelheiten vergisst oder sich Ungenauigkeiten bei der Reflexion einschleichen \citep[vgl.][S.~87]{Henk2014}. Die Aufzeichnung der jeweiligen Tätigkeit durch eine Videokamera und das Durchgehen der Aufzeichnung mit der befragten Person erhöht die Präzision der Reflexion \citep[\emph{Video-Recall}, ][S.~566]{Leuchter2006}.

Fragebögen mit offenen Fragen stellen die eingängigste Form der Fragebögen dar. Sie führen zu einer qualitativen Datenerhebung. Die Auswertung qualitativer Daten ist in der Regel schwierig und zeitaufwendig. Ein Mittel, die Auswertung zu vereinfachen und die Ergebnisse zu vereinheitlichen, ist die Strukturierung der Fragen und die Einführung von psychometrischen Skalen.

\subsection{Psychometrische Skalen}
Eine psychometrische Skala besteht aus mehreren Items, die die Befragten auf einem ordinalskalierten Maß (z.~B. "`trifft nicht zu"' = 1 bis "`trifft zu"' = 5) bewerten. Anhand der bewerteten Items lassen sich die persönlicher Einstellungen oder das Befinden der Befragten messen. In den nachfolgenden Abschnitten stelle ich die \ac{FKS} \citep{Rheinberg2003}, die \emph{\ac{FSS}} \citep{Jackson1996} und den Fragebogen von \citet{Keller2008} vor. Die genannten drei psychometrischen Skalen kamen überwiegend in Forschungsarbeiten zu Lösungsansätzen zur impliziten Messung des Flow-Erlebens zum Einsatz.

\subsubsection{Flow-Kurzskala}

Die \ac{FKS} besteht aus insgesamt 16 Items. Die ersten zehn Items bilden anhand einer 7-Punkte-Likert-Skala ("`trifft nicht zu"' = 1 bis "`trifft zu"' = 7) die Komponenten des Flow-Erlebens ab (vgl.~\ref{sec:was_ist_flow}). Diese zehn Items fassen \citet{Rheinberg2003} zum Generalfaktor zusammen. Zur Differenzierung des Flow-Konstrukts ist der Generalfaktor der \ac{FKS} in zwei Faktoren (Unterdimensionen) unterteilt. Faktor I umfasst sechs Items, die Aussagen zum "`Glatten automatisierten Verlauf"' einer Tätigkeit zusammenfassen. Faktor II beinhaltet vier Items, die mit "`Absorbiertheit"' in Zusammenhang stehen. Der Reliabilitätskoeffizient der zehn Items im Generalfaktor (Cronbachs Alpha) liegt nach Angaben von \citet[S.~9]{Rheinberg2003} im Bereich um $\alpha$ = 0,90. \citet{Rheinberg2003} erweiterten die \ac{FKS} um eine Besorgniskomponente, da sie davon ausgehen, dass bei der Durchführung einer Tätigkeit bzw. in einer Anforderungssituation nicht ausschließlich Flow entsteht. Die Besorgniskomponente besteht aus drei Items (Nr. 11 bis Nr. 13, Cronbachs $\alpha$ = 0,80 bis $\alpha$ = 0,90). Das Ende der \ac{FKS} fragt die persönlich erlebte Anforderung der Tätigkeit und dessen Gleichgewicht mit den eigenen Fähigkeiten ab. Dieser Teil der \ac{FKS} besteht aus drei Items mit jeweils einer 9-Punkte-Likert-Skala. Das Item 14 fokussiert sich auf einen Vergleich der Schwierigkeit der jetzigen Tätigkeit mit allen anderen Tätigkeiten (leicht vs. schwer) und das Item 15 auf die eigene Leistungsfähigkeit (niedrig vs. hoch). Das Item 16 fragt direkt, auf die aktuelle Tätigkeit bezogen, nach der subjektiv wahrgenommenen \ac{AFP} (zu gering vs. zu hoch).

\subsubsection{Flow-State-Scale}

Die \ac{FSS} dient der Ermittlung des Flow-Status in konkreten Situationen im Sport. Sie enthält neun Dimensionen mit jeweils vier Items. Die Dimensionen der Skala repräsentieren die acht von \citet[S. 73-101]{Csikszentmihalyi1992} diskutierten Komponenten der Freude: "`Herausfordernde Aktivität, für die man besondere Geschicklichkeit braucht"', "`Der Zusammenfluss von Handeln und Bewusstsein"', "`Klare Ziele und Rückmeldung"', "`Konzentration auf die anstehende Aufgabe"', "`Das Paradox der Kontrolle"', "`Der Verlust des Selbstgefühls"', "`Die Veränderung der Zeit"' und "`Die autotelische Erfahrung"'. Sie fungieren im weiteren Verlauf als neun Dimensionen des Flow-Erlebens, da klare Ziele und eindeutige Rückmeldungen zwei von einander getrennte Dimension darstellen. Die \ac{FSS} in ihrer originalen Form besitzt einen Cronbachs Alpha von 0,80. Auf ihrer Grundlage entstanden eine revidierte Version (FSS-2, 36 Items, 9 Dimensionen) und eine gekürzte Version (short FSS-2, 9 Items) \citep[vgl.][]{Jackson2002, Jackson2008}. Zusätzlich gibt es zu jeder Version, eine \emph{Trait}- oder \emph{Dispositional}-Version (\acs{DFS} und DFS-2, jeweils 36 Items und short DFS 9 Items). Die \emph{Dispositional}-Versionen sind im Wortlaut und Zeitform geändert und dienen zur Ermittlung einer generellen Tendenz der Befragten Flow in ihrem Sport zu erleben \citep[vgl.][S.~356]{Jackson1998}.

\subsubsection{Keller und Bless Fragebogen}

Der Fragebogen von Keller und Bless fragt mit mehreren Items auf einer 7-Punkte-Likert-Skala ("`trifft nicht zu"' = 1 bis "`trifft zu"' = 7) die nachfolgenden Dimensionen ab: Gefühl der Kontrolle, Involvierung und Vergnügen, wahrgenommene Passung von Fähigkeiten und Aufgabenanforderungen. Zusätzlich misst der Fragebogen das Zeitgefühl auf einer 10~cm langen Linie. Die erste Dimension "`Gefühl der Kontrolle"' umfasst zehn Items, die Aussagen zur Kontrolle über die Ergebnisse einer Tätigkeit zusammenfassen. Die Reliabilität der zehn Items liegt bei einem Cronbachs Alpha von 0,93. Die 14 Items der Dimension "`Involvierung und Vergnügen"' fassen die Involvierung und das Vergnügen in und an einer Tätigkeit zusammen. Deren Cronbachs Alpha liegt bei 0,95. Dimension 3 beinhaltet ein Item, das das Gleichgewicht von Anforderungen der Tätigkeit und eigenen Fähigkeiten abfragt. Keller und Bless setzten ihren Fragebogen in zwei Studien ein \citep{Keller2008}. In der zweiten Studie ersetzten sie die Items der Dimension Gefühl der Kontrolle durch Items, die das Ausmaß einer deutlichen Affektreaktion (z.~B. Unruhe) der befragten Person in Erfahrung bringen. Diese neun Items besitzen einen Reliabilitätskoeffizienten Cronbachs Alpha von 0,88.

\subsubsection{Zusammenfassung}

Psychometrische Skalen vereinfachen auf der einen Seite die Arbeit des Untersuchenden, auf der anderen Seite führen sie in manchen Fällen zu unklaren und widersprechenden Ergebnissen. Ist die befragte Person bei einem Interview und bei der Beantwortung eines offenen Fragebogen noch in der Lage zusätzliche Auskünfte über das Flow-Erleben wie z.~B. Zeitpunkt zu geben, bleibt bei psychometrischen Skalen unklar, wie die Antwort in Form eines Kreuzes an einer Stelle der Skala zustande kommt.

\subsection{Experience Sampling Method}

Beim standardmäßigen Einsatz von strukturierten Interviews, Fragebögen und psychometrischen Skalen beim Gehen und Laufen in einem natürlichen Handlungsumfeld kommt es zu zwei wesentlichen Problemen:
\begin{enumerate}
	\item Das Gehen und das Laufen sind keine natürlich strukturierten Tätigkeiten, die natürliche Pausen besitzen, wie z.~B. das Golfen. Das erschwert einem Außenstehenden, eine Befragung während der Tätigkeit durchzuführen.
	\item Es ist zeitaufwendig, eine gehende oder laufende Person außerhalb von Laboratorien mit Hilfe von Videokameras aufzunehmen.
\end{enumerate}

Um das Verhalten und das Erleben in natürlichen Handlungsumgebungen zu erheben, entwickelten \citep{Larson1983} die \ac{ESM}. Die originale \ac{ESM} von Larson und Csikszentmihalyi wurde von mehreren Wissenschaftlern \citep[z.~B. ][]{Schallberger2001, Rheinberg2003} durch verbesserte Fragebögen bzw. psychometrische Skalen optimiert. Heutzutage fassen wir unter \ac{ESM} eine Reihe von ambulanten Assessment-Strategien zusammen. Ambulantes Assessment ist die Erhebung von Daten im Alltag der zu Untersuchenden.

In einer typischen \ac{ESM}-Studie erhalten die Teilnehmer für den gewählten Testzeitraum einen akustischen Signalgeber sowie Formulare oder Fragebögen in gehefteter Form. Ein akustisches Signal fordert den Teilnehmer auf, eine Selbstauskunft abzugeben. Zu Variablen, die die Wahrnehmung, Motivation und den Gemütszustand des Teilnehmers widerspiegeln, fragen Wissenschaftler in einer üblichen \ac{ESM}-Studie Variablen zur Situation, wie Ort, Aktivität und sozialer Kontext, ab.

Durch die Mehrfachabfrage lässt sich der Verlauf der untersuchten Variablen jedes Teilnehmers über den Testzeitraum abhängig von der zeitlichen Auflösung rekonstruieren. Es gibt drei Vorgehen bei der \ac{ESM}: (a) intervall-kontingente, (b) ereignis-kontingente und (c) signal-kontingente Protokolle \citep[vgl.][S.~198ff.]{Reis2000}. Bei intervall-kontingenten Messungen geben die Teilnehmer zu vorbestimmten Intervallen (z.~B. jede Stunde oder zur gleichen Tageszeit) eine Selbstauskunft ab. Teilnehmer einer ereignis-kontingenten Messstudie geben nur bei konkreten Ereignissen eine Selbstauskunft ab. Beim signal-kontingenten Vorgehen erhalten die Teilnehmer zu beliebigen Zeitpunkten ein Signal, das sie auffordert, eine Selbstauskunft abzugeben. Die Wahl des Vorgehens hängt von der Aufgabe der Studie ab. Der Vorteil der \ac{ESM} ist, dass wir die Messung direkt nach der unterbrochenen Tätigkeit durchführen. Ein Nachteil ist genau die Unterbrechung der Tätigkeit bzw. des Erlebens durch den Signalgeber.

\subsection{Computergestützte Experience Sampling Verfahren}

Computergestützte \emph{Experience Sampling} Verfahren nutzen heutzutage die technischen Möglichkeiten von mobilen Geräten, um eine Datenerhebung automatisch zu initiieren. 2003 entwickelten \citet{Intille2003} eine Version der \ac{ESM} mit Hilfe eines \ac{PDA}s. Sie beabsichtigten ein kontextsensitives \emph{Experience Sampling}, das die beschriebenen Erhebungsprotokolle ergänzt. Ein Beispiel für einen Kontextbezug ist die Ortsinformation. Mit der Hilfe eines \ac{GPS}-Empfängers fordert z.~B. der \ac{PDA} den Teilnehmer automatisch auf, eine Selbstauskunft zu geben, wenn sich der jeweilige Teilnehmer an einem konkreten Ort aufhält.

Die Plattform \emph{MyExperience} läuft auf einem Smartphone mit \emph{Windows} Betriebssystem und ermöglicht kontextbezogene Erhebungsprotokolle \citep[vgl.][]{Froehlich2007}. Sie benutzt die interne Sensorik des Smartphones, um z.~B. eine Lokalisierung über \ac{GPS} oder \acs{GSM} vorzunehmen und um Informationen über die Smartphone-Nutzung zu sammeln. Sensorereignisse ermöglichen dem Untersuchenden, benutzerdefinierte Aktionen wie die Synchronisation von Daten über das Internet, das Senden von \acs{SMS}-Nachrichten an den Untersuchenden oder die Anzeige eines Fragebogens auszuführen. Mit \emph{MyExperience} sind die Untersuchenden in der Lage, Selbstauskünfte nicht nur zeitlich, sondern auch bei vordefinierten Ereignissen abzufragen.

Eine kommerzielle computergestützte \ac{ESM}-Lösung ist \emph{Breakthrough Research} von \emph{MetricWire}. \emph{Breakthrough Research} ist ein geschlossenes \emph{Server-Client} System und ermöglicht \emph{smartphone-gestütztes Experience Sampling}. Die Erstellung einer Befragung erfolgt über eine Webplattform und die Darstellung in einer App für das \emph{Android OS} oder das \emph{iOS}. Die Lösung unterstützt die Lokalisierung durch \ac{GPS}.

\citet{Gaggioli2013} setzen in ihrer Studie zum Flow-Erleben anstelle von Fragebögen in gehefteter Form ein computergestütztes \emph{Experience Sampling} Verfahren auf einem Smartphone ein. Die Forschungsgruppe um Gaggioli entwickelte eine Plattform mit dem Namen \emph{PsychLog} für das mobile \emph{Windows} Betriebssystem zur mobilen Erhebung von subjektiven und kardiovaskulären Daten \citep{Gaggioli2013a}.

Der jetzige Stand der Technik ermöglicht computergestützte \emph{Experience Sampling} Verfahren mit komplexeren Erhebungsprotokollen bei gleichzeitiger Erhebung von physiologischen, kinematischen und kontextbezogenen Daten. Das versetzt die Untersuchenden in die Lage, eine präzisere und weniger aufwendige Kontrolle darüber zu erhalten, wann und wie oft ein Teilnehmer aufgefordert wird, eine digitale Selbstauskunft abzugeben. Zusätzlich reduziert die digitale Form der Erhebung menschliche Fehler bei der Verarbeitung der Daten. Die physiologischen, kinematischen und kontextbezogenen Daten geben ihnen Aufschluss darüber, welche Tätigkeit der Teilnehmer vor einer Selbstauskunft durchführte und wie der Organismus des Teilnehmers auf die jeweilige Tätigkeit reagierte.

Wir erhalten nur explizite Hinweise auf das Erleben, egal ob wir \ac{ESM} oder computergestützte \emph{Experience Sampling} Verfahren anwenden. Die gleichzeitige Aufnahme von physiologischen, kinematischen und kontextbezogenen Daten und deren Analyse in Bezug auf das Erleben im Nachhinein ermöglicht es uns, implizite Hinweise über das Erleben zu erlangen. Aus dem Grund betrachte ich im nachfolgenden Abschnitt Lösungsansätze zu impliziten Messverfahren, die in der Flow-Forschung zum Einsatz gekommen sind.

\section{Lösungsansätze zur impliziten Messung von Flow-Erleben}
\label{sec:loesungsansaetze_zur_impliziten_messung_von_flow_erleben}

Die vorgestellten expliziten Messverfahren haben introspektive Aussagen über das Erlebte als Ergebnis. Introspektive Aussagen sind für Flow als nicht bewusst bzw. nicht reflektiert erlebten Zustand nicht verlässlich, da eventuell keine oder nur unzureichend bewusst zugängliche Gedächtnisinhalte im Flow-Zustand angelegt werden \citep[vgl.][S.~82]{Henk2014}. Damit ist die Entwicklung direkter bzw. impliziter Messverfahren von Flow-Erleben auch für die Flow-Forschung mittelfristig unumgänglich \citep[vgl.][S.~86]{Henk2014}.

Potentielle Kandidaten für implizite Messverfahren für die vorliegende Arbeit sind die Messungen der Gehirnaktivität und die Messungen von physiologischen und motorischen Merkmalen. Die Daten der genannten Messungen erlauben eine computergestützte Echtzeitverarbeitung. Abbildung~\ref{fig:3_2_modellraum_1} veranschaulicht den vorläufigen Modellraum für die vorliegende Arbeit. In der Praxis bewährte sich die Messung des Stresshormons Cortisol \citep{Keller2011, Peifer2014, Peifer2015}. Für die vorliegende Arbeit ist das Verfahren nicht praktikabel, da der Cortisolgehalt erst 30 Minuten nach der Anforderungssituation im Speichel für die konkrete Anforderungssituation messbar ist.

\subsection{Flow im Gehirn}

Eine theoretische Erklärung, Flow-Erleben anhand der Gehirnaktivität zu erklären gibt \citet{Dietrich2004}. \citet[S.~758f.]{Dietrich2004} geht davon aus, dass ein Zustand der \emph{transienten Hypofrontalität} für das Flow-Erleben notwendig ist. Er beschreibt den Zustand der \emph{transienten Hypofrontalität} als eine Aktivitätsabnahme im präfrontalen Kortex, einem Areal des Gehirns, das wir für die explizite Kontrolle einer ausgeführten Tätigkeit benötigen \citep[vgl.][]{Dietrich2003, Dietrich2004}. Die Unterfunktion des präfrontalen Kortexes führt zu einem Zustand zwischen Wachheit und Schlaf \citep[vgl.][S.~241]{Dietrich2003} und zu einer impliziten Ausführung, die das Bewusstsein umgeht \citep[vgl.][S.~753]{Dietrich2004}.

Dietrich \citep[][S.~757]{Dietrich2004} vergleicht in einer Gegenüberstellung die implizite Ausführung im Zustand der \emph{transienten Hypofrontalität} mit den neun Dimensionen des Flow-Erlebens nach \citet{Csikszentmihalyi1992}. Die Gegenüberstellung wurde von \citet{Henk2014} in einer Tabelle~\ref{tab:funktionen_des_praefrontalen_kortexes} zu den Merkmalen des Flow-Erlebens zusammengefasst. Sie verdeutlicht den Zusammenhang der beiden Zustände.

\begin{table}[t]
	\caption[Funktionen des präfrontalen Kortexes und Merkmale eines Flow-Zustands]{Funktionen des präfrontalen Kortexes und Merkmale eines Flow-Zustands (\citet{Henk2014} nach \citet{Dietrich2004})}
	\label{tab:funktionen_des_praefrontalen_kortexes}
	\begin{tabularx}{\textwidth}{*{2}{>{\RaggedRight\arraybackslash}X}}
\toprule
Funktionen des präfrontalen Kortexes & Merkmale eines Flow-Zustands \\
\midrule
Dauerhafte, gerichtete Aufmerksamkeit & Zentrierung der Aufmerksamkeit \\
Selbstreflektierendes Bewusstsein, Vorstellungen über die eigene Person & Verlust der reflexiven Selbstbewusstheit \\
Zeitliche Integration & Verlust des Zeitgefühls \\
Handlungsplanung & keine Besorgtheit über Misserfolg \\
\bottomrule
\end{tabularx}
\end{table}

Zur Messung der Gehirnaktivität setzen Wissenschaftler die \ac{EEG} ein. Hierzu platzieren sie Elektroden auf die Kopfhaut des zu Untersuchenden. Mehrkanalsysteme dienen der klinischen Untersuchung. Je mehr Kanäle ein Messsystem besitzt, umso besser lassen sich Hirnströme identifizieren und aktive Gehirnareale wie der präfrontale Kortex im Nachhinein lokalisieren. Eine große Anzahl von Elektroden, die mit Kabel verbunden sind, erhöhen wiederum die Störungsanfälligkeit eines Mehrkanalsystems. Das macht den Einsatz eines Mehrkanalsystems außerhalb von Laboratorien unmöglich, da ein faradayscher Käfig zur Abschirmung vor externen Störquellen erforderlich ist. Durch die Nutzung eines Zweikanalsystems sind Wissenschaftler in der Lage auch außerhalb von Laboratorien \ac{EEG}-Messungen durchzuführen.

\citet{Hugentobler2011} setzte beide Arten von Systemen ein. Er versuchte Flow-Erleben mit der Hilfe eines Computerspiels zu induzieren und führte gleichzeitig \ac{EEG}-Messungen durch. Als explizites Vergleichsmessverfahren nutzte er die \ac{FKS}. Nachdem er bedeutende Gehirnareale im Flow-Erleben mit Hilfe des Mehrkanalsystems identifiziert hatte, nutzte er ein Zweikanalsystem. Hugentoblers Ergebnisse lassen darauf schließen, dass Flow-Erleben mit höheren Theta-Wellen in Verbindung steht. "`Theta-Wellen beschreiben im Wachzustand Intuition und Erinnerung, allenfalls auch tiefe Entspannung — ein Zustand zwischen Wachheit und Schlaf"' \citep[S.~149]{Hugentobler2011}. Damit stehen die Ergebnisse in einer Linie mit dem theoretischen Erklärungsversuch von Dietrich.

\subsection{Elektromyographie}

Emotionale Gemütszustände lassen sich bei vielen Menschen im Gesicht ablesen. Die \ac{EMG} misst die elektrische Muskelaktivität und lässt sich einsetzen, um emotionale Gemütszustände anhand von Muskelgruppen im Gesicht automatisch zu bestimmen. \citet{Kivikangas2006, Nacke2008, deManzano2010} nutzten die \ac{EMG}-Analyse in ihren Studien, da sie davon ausgehen, dass Flow-Erleben positive Gemütszustände, wie Freude, zur Folge hat. Die drei Untersuchungen kommen zu gegensätzlichen Ergebnissen. \citet[vgl.][S.~153]{Peifer2012} vermutet, dass das unter anderem an den unterschiedlichen Bedingungen der Untersuchungen liegt. \citet{Kivikangas2006} und \citet{Nacke2008} untersuchten Spieler beim Spielen eines Ego-Shooters; \citet{deManzano2010} untersuchten hingegen Pianospieler, die ihr Lieblingsstück spielten. \citet{Kivikangas2006} rekrutierte studentische Freiwillige, wobei \citet{Nacke2008} männliche Hardcore-Computerspieler und \citet{deManzano2010} professionelle Pianospieler rekrutierten. Nach \citet[vgl.][S.~153]{Peifer2012} sind zusätzliche Studien nötig, um die gegensätzlichen Resultate zu klären, und um die \ac{EMG}-Messung zur Identifikation von Flow-Erleben zu nutzen. Daneben identifizieren wir mit der \ac{EMG}-Analyse eine Folge und kein Merkmal des Flow-Erlebens.

\subsection{Vegetatives Nervensystem}

In der heutigen Flow-Forschung gehen die Wissenschaftler davon aus, dass die physiologischen Eigenschaften des Menschen im Flow-Erleben mit anderen Bewusstseinszuständen vergleichbar sind. Wie in Abschnitt~\ref{sub:modellierung} beschrieben, lassen sich der Eintritt in das Flow-Erleben, der Erhalt und die Veränderung des Flow-Erlebens sowie und der Austritt aus dem Flow-Erleben mit dem Einschlafen, den Schlafphasen und den Erwachen vergleichen. Wie für das Schlafen sind wir in der Lage für das Erleben von Flow die in Tabelle~\ref{tab:voraussetzungen_fuer_einen_flow_zustand} dargestellten Voraussetzungen zu schaffen. Wir sind aber nicht in der Lage den Zeitpunkt des Einschlafens bzw. des Eintretens in das Flow-Erleben durch unsere Willenskraft zu bestimmen.

Für jeden Zustand passt das \ac{VNS} die Aktivität und die Stoffwechselprozesse unseres Organismus ohne eigenes Dazutun an. Darum bezeichnen wir es auch als autonomes Nervensystem. Maßgeblich tragen die beiden Kontrahenten des \ac{VNS}, Sympathikus und Parasympathikus, die Verantwortung für die Anpassung. Der Sympathikus bereitet ihn auf körperliche und geistige Leistungen vor. Er sorgt unter anderem dafür, dass das Herz schneller und kräftiger schlägt, sich weniger Speichel bildet und sich die Schweißsekretion erhöht. Der Parasympathikus wiederum kümmert sich um die Körperfunktionen in Ruhe sowie um die Regeneration und um den Aufbau körpereigener Reserven. Er verlangsamt unter anderem den Herzschlag, erhöht die Speichelsekretion und vermindert die Schweißsekretion.

Anhand von physiologischen Messungen und derer Analyse sind wir in der Lage nach zu vollziehen, welcher der beiden Nervenstränge während eines Zustands aktiv ist. Die Intensität der jeweiligen Aktivierung ermöglicht herauszufinden, ob und in welcher Intensität ein Mensch einen Zustand erlebt. In den nachfolgenden Abschnitten gebe ich einen Überblick über physiologische Messungen, die Wissenschaftler in der Flow-Forschung einsetzten.

\subsection{Elektrodermale Aktivität}

Die \ac{EDA} beschreibt den elektrischen Leitungswiderstand der Haut. Eine Änderung der \ac{EDA} kommt durch eine Anpassung im \ac{VNS} und die als Folge dessen angeregte oder abgeschwächte Schweißsekretion zustande. Den Anpassungsprozess des \ac{VNS}s wollten \citet{Kivikangas2006, Nacke2008} als Merkmal für Flow-Erleben nutzen. Sie begründen ihren Ansatz mit der Änderung des Gemütszustands zur Erregung, die durch eine Aktivierung des Sympathikus beim sitzenden Computerspieler gekennzeichnet ist. Sie folgern, dass die Erregung eine Folge der erhöhten mentalen Anstrengung ist, die auf eine Zentrierung der Aufmerksamkeit bzw. eine Fokussierung auf die Tätigkeit schließen lässt.

\citet{Kilpatrick1972} unterscheidet eine ereignisbasierte Hautleitfähigkeitsreaktion (engl. \emph{\ac{SCR}}) und einen Langzeit-Hautleitfähigkeitslevel (engl. \emph{\ac{SCL}}). \citet[][S.~158]{Peifer2012} vermutet, dass das \ac{SCL} gegenüber der \ac{SCR} ein vielversprechender Indikator für Flow-Erleben ist. \citet[][S.~158]{Peifer2012} begründet ihre Annahme damit, dass Flow-Erleben kein diskreter Zustand ist, den eine Person in der Lage ist, durch konkrete Stimuli oder Erfahrungen hervorzurufen, sondern eher durch lang andauernde „glatte“ Aktivitäten begünstigen wird. Zusammenhang von \ac{SCL} und Flow-Erleben. \citet{Nacke2008} benutzten das \ac{SCL}, um unterschiedlich konzipierte Level des Computerspiels \emph{Half-Life 2} zu vergleichen. Sie fanden einen Zusammenhang von ansteigendem \ac{SCL} und dem Leveldesign, das sie für das Flow-Erleben konzipierten.

\subsection{Kardiovaskuläre Messungen}
\label{sub:kardiovaskulaere_messungen}

Kardiovaskuläre Messungen betrachten die Herzaktivität. Das Herz besitzt einen sinusförmigen Impuls von 60 bis 80 Schlägen pro Minute. Abgetragene Schläge pro Minute (engl. \emph{\ac{BPM}}) messen die \ac{HR}. Das \ac{VNS} beeinflusst das Herz, um die Leistung des Herzens ständig wechselnden äußeren und inneren Bedingungen anzupassen. Allgemein erhöht der Sympathikus die \ac{HR} und vergrößert die Gefäße, damit der menschliche Organismus in der Lage ist den äußeren Anforderungen zu begegnen \citep[vgl.][S.~226]{Porges1995}. Der Parasympathikus bewirkt das Gegenteil im Zuge der Erholung \citep[vgl.][S.~226]{Porges1995}. Im Verlauf der Erholung dominiert der Parasympathikus und die \ac{HR} ist gering. Eine hohe \ac{HRV} begleitet gewöhnlich eine geringe \ac{HR}. Die \ac{HRV} repräsentiert die Variabilität des zeitlichen Abstands von zwei aufeinander folgenden Herzschlägen. Wir messen die zeitlichen Abstände mit der Hilfe eines \ac{EKG}s oder eines Plethysmographen.

Die darauffolgende Analyse der Variabilität der Abstände in ruhenden bzw. sitzenden Tätigkeiten erlaubt uns einen Einblick in den Steuermechanismus des \ac{VNS}s \citep[vgl.][]{Jalife1983}. Zur Bestimmung der sympathischen und parasympathischen Aktivität analysierten wir die NN-Intervalle (bei R-Zacke zu R-Zacke, RR-Intervalle) zeit- oder frequenzbezogen. Bei ruhenden bzw. sitzenden Tätigkeiten ist die physiologische Interpretation der zeit- und frequenzbezogenen \ac{HRV}-Parameter weitestgehend manifestiert. Bei Messungen unter 24 Stunden unterscheiden wir üblicherweise bei der Frequenzanalyse die sehr tiefen Frequenzen (engl. \emph{\ac{VLF}}) im Bereich von 0,033~Hz bis 0,04 Hz, von den tiefen Frequenzen (engl. \emph{\ac{LF}}) im Bereich von 0,04~Hz bis 0,15~Hz und den hohen Frequenzen (engl. \emph{\ac{HF}}) im Bereich von 0,15 bis 0,4~Hz \citep[vgl.][S.~360]{TaskForce1996}.

Parasympathische Aktivität wirkt sich in der Regel auf die hohen Frequenzen aus \citep[vgl.][S.~365]{TaskForce1996}. Uneinigkeit besteht in Bezug auf die tiefen Frequenzen. Studien legen nahe, dass die tiefen Frequenzen in normalisierten Einheiten ein Zeichen für sympathische Anpassung ist \citep[vgl.][S.~365]{TaskForce1996}. Andere Studien sehen die tiefen Frequenzen als Hinweis für sowohl sympathische und parasympathische Aktivität an \citep[vgl.][S.~365]{TaskForce1996}. In Folge dessen betrachten Wissenschaftler auf der einen Seite das Verhältnis von sympathischer und parasympathischer Aktivität als Indiz für das Gleichgewicht zwischen den beiden Nervensträngen \citep[vgl.][S.~365]{TaskForce1996}. Die anderen interpretieren das Verhältnis als sympathische Anpassung \citep[vgl.][S.~365]{TaskForce1996}. Die Interpretation von den sehr niedrigen Frequenzen von kurzen \ac{EKG}-Messungen (unter fünf Minuten) ist zweifelhaft und benötigt zusätzliche Untersuchungen \citep[vgl.][S.~365]{TaskForce1996}. Des Weiteren begleitet eine merkbare Reduktion des gesamten Frequenzbandes den resultierenden erhöhten Herzschlag im Verlauf der sympathischen Aktivität \citep[vgl.][S.~366]{TaskForce1996}. Bei parasympathischer Aktivität ist das Gegenteil der Fall \citep[vgl.][S.~366]{TaskForce1996}. Eine merkbare Erhöhung des Gesamtfrequenzbands begleitet einen verringerten Herzschlag \citep[vgl.][S.~366]{TaskForce1996}.

Eine alternative Gruppe der Analysemethoden der \ac{HRV} sind die nichtlinearen Analysemethoden. Die nichtlinearen Analysemethoden kamen in der Flow-Forschung bisher nicht zum Einsatz, da ihre physiologische Interpretation weitestgehend unklar ist. Eine mathematische Beschreibung der zeit-, frequenzbezogenen und nichtlinearen \ac{HRV}-Parameter gebe ich in Tabelle~\ref{tab:mathematische_beschreibung_der_hrv}.

Die Analyse der \ac{HRV} dient wie die Analyse der \ac{EDA} als Hinweis für eine Änderung des Gemütszustands zur Erregung, was auf einen Anstieg der sympathischen Aktivität schließen lässt. Eine zweite Vermutung ist, dass durch den automatischen und effizienteren Ablauf der Handlungsschritte im Flow-Erleben eine Mühelosigkeit entsteht und die mentale und körperliche Anstrengung abnimmt \citep[vgl.][S.~308]{deManzano2010}. Die Abnahme lässt auf eine Aktivierung des parasympathischen Zweigs des \ac{VNS}s und eine steigende \ac{HRV} schließen. Im Nachfolgenden erörtere ich die bisherigen Methoden und Ergebnisse der Studien zu Flow-Erleben und der \ac{HRV}-Analyse.

\begin{longtabu} to \textwidth {X[15,l]<{\strut} X[15,l]<{\strut} X[12,l]<{\strut} X[12,l]<{\strut} X[46,l]<{\strut}}
	\caption[Mathematische Beschreibung der HRV]{Mathematische Beschreibung der HRV nach \citet{Sammito2015}.} \\
	\toprule
	Analyse & Methode & Maß & Einheit & Definition und Erklärung \\
	\bottomrule
	\endhead
	\label{tab:mathematische_beschreibung_der_hrv}
Zeitbe"-zogene & Statistisch & SDNN & \emph{ms} & Standardabweichung der NN-Intervalle im Messzeitbereich; frequenzabhängiger Parameter für die Höhe der Gesamtvariabilität (hohe SDNN $\rightarrow{}$ hohe HRV, niedrige SDNN $\rightarrow{}$ niedrige HRV). \\
\cline{3-5}

& & SDANN & \emph{ms} & Mittelwert der Standardabweichungen aller konsekutiven 5-min-NN-Intervalle; zur Abschätzung der HRV bei Langzeitmessungen. \\
\cline{3-5}

& & RMSSD & \emph{ms} & Quadratwurzel des Mittelwerts der Summe aller quadrierten Differenzen zwischen benachbarten NN-Intervallen; Parameter der Kurzzeitvariabilität; zur Betrachtung des parasympathischen Einflusses. \\
\cline{3-5}

& & SDNN-Index & \emph{ms} & Mittelwert der Standardabweichung von allen normalen NN-Intervallen von 5-min-Segmenten aus der 24-h-EKG-Aufzeichnung; keine klare Zuordnung. \\
\cline{3-5}

& & NN50 & \emph{k. E.} & Anzahl der Paare benachbarter NN, die mehr als 50~ms voneinander abweichen; zeigt den parasympathischen Einfluss. \\
\cline{3-5}

& & pNN50 & \emph{\%} & Prozentsatz aufeinanderfolgender NN-Intervalle, die mehr als 50~ms voneinander abweichen; ein hoher pNN50-Wert bedeutet hohe spontane Änderung der Herzschlagfrequenz; zur Betrachtung des parasympathischen Einflusses. \\
\cline{2-5}

& Geo"-me"-trisch & HRV-Trian"-gular"-index & \emph{k. E.} & Integral der Dichteverteilung [Anzahl aller NN-Intervalle, dividiert durch das Maximum (Höhe) der Dichteverteilung] bzw. Quotient aus der absoluten Anzahl der NN-Intervalle zur Anzahl der modalen NN-Intervalle; keine klare Zuordnung. \\
\cline{3-5}

& & TINN & \emph{ms} & Länge der Basis des minimalen quadratischen Unterschieds der triangulären Interpolation für den höchsten Wert des Histogramms aller NN-Intervalle; keine klare Zuordnung. \\
\hline

Frequenz"-bezogene & FFT und autoregressives (AR-) Modell & \emph{Total power} & $ms^{2}$ & Gesamtleistung oder gesamtes Spektrum; \emph{Total Power} entspricht Energiedichte im Spektrum von 0,00001 bis 0,4 Hz; Parameter der Gesamtvariabilität. \\
\cline{3-5}

& & \acs{UVLF} & $ms^{2}$ & Leistungsdichtespektrum unter 0,003 Hz, Periodendauer größer als 5:30 min. \\
\cline{3-5}

& & VLF & $ms^{2}$ & Leistungsdichtespektrum von 0,003 bis 0,04 Hz, Periodendauer von 25~s bis 5:30 min. \\
\cline{3-5}

& & LF & $ms^{2}$ & Leistungsdichtespektrum im Frequenzbereich von 0,04 bis 0,15 Hz, Periodendauer von 7 bis 25 s; daran ist sowohl der Sympathikus als auch der Parasympathikus beteiligt, wobei der Anteil des Sympathikus überwiegt. \\
\cline{3-5}

& & HF & $ms^{2}$ & Leistungsdichtespektrum im Frequenzbereich von 0,15 bis 0,40 Hz, Periodendauer 2,5 bis 7 s; zeigt ausschließlich den parasympathischen Stimmungsanteil. \\
\cline{3-5}

& & LF nu & \emph{k. E.} & \emph{Low Frequency normalized unit}; entspricht $LF/(Total Power - VLF) \times 100$. \\

& & LF nu & \emph{k. E.} & High Frequency normalized unit; entspricht $HF/(Total Power - VLF) \times 100$. \\
\cline{3-5}

& & LF/HF & \emph{k. E.} & Quotient der sympathovagalen Balance; als Wert des Zusammenspiels von Parasympathikus (HF) und Sympathikus (LF) LF/HF $\uparrow{}$ = Sympathikus $\uparrow{}$ (Sympathikusaktivität steigt an) LF/HF $\downarrow{}$ = Parasympathikus $\uparrow{}$. \\
\cline{3-5}

& & VLF-Peak & \emph{Hz} & Frequenzgipfel im VLF-Band; Thermoregulations-Peak. \\
\cline{3-5}

& & LF-Peak & \emph{Hz} & Frequenzgipfel im LF-Band; Baroreflex-Peak \\
\cline{3-5}

& & HF-Peak & \emph{Hz} & Frequenzgipfel im HF-Band; respiratorischer Peak \\
\hline

Nicht"-lineare & Poincaré-Plot & DL & \emph{ms} & Länge des Längsdurchmessers der 95 \%-Vertrauensellipse; beschreibt die Langzeitabweichung der Herzschlagfrequenz. \\
\cline{3-5}

& & DQ & \emph{ms} & Länge des Querdurchmessers der 95 \%-Vertrauensellipse; beschreibt kurzzeitige Änderungen der Herzschlagfrequenz. \\
\cline{3-5}

& & SD1 & \emph{ms} & Standardabweichung der Punktabstände zum Querdurchmesser; quantifiziert die spontane (kurzzeitige) Variabilität. \\
\cline{3-5}

& & SD2 & \emph{ms} & Standardabweichung der Punktabstände zum Längsdurchmesser; beschreibt langfristige HRV-Änderungen. \\
\cline{2-5}

& Trend"-be"-rein"-ig"-en"-de Fluk"-tu"-a"-ti"-ons"-an"-al"-y"-se (engl. \emph{\acf{DFA}}) & DFA1 & \emph{k. E.} & Grad der Zufälligkeit/Korrelation; reicht von 0,5 (zufällig) bis 1,5 (korreliert) mit Normalwerten um 1,0; wird häufig als nichtlinearer Parameter für kurze NN-Intervall-Daten genutzt. \\
\cline{3-5}

& & DFA2 & \emph{k. E.} & Wird häufig als nichtlinearer Parameter für längere RR-Intervalle genutzt; reduzierte Werte sind assoziiert mit einer schlechten Prognose. \\
\bottomrule

\end{longtabu}

\subsubsection{\citet{deManzano2010}}

\citet{deManzano2010} fanden im Gegensatz zu ihrer aufgestellten Annahme eine Abnahme der parasympathischen Aktivität während des Flow-Erlebens beim Pianospielen. In ihrer Studie mit 21 männlichen Pianospielern identifizierten sie vielmehr eine Aktivierung des sympathischen Zweigs anhand eines ansteigenden LF/HF-Verhältnisses und einer steigenden \ac{HR} während des Flow-Erlebens. Die rekrutierten Pianospieler spielten jeweils fünf Wiederholungen eines von ihnen ausgewählten Musikstückes (Länge: drei bis sieben Minuten). In den Unterbrechungen der Wiederholungen nutzen \citet{deManzano2010} eine Teilmenge der \ac{FSS}, um den erlebten Flow der Pianospieler explizit zu ermitteln. Zur Durchführung der \ac{HRV}-Analyse zeichneten sie kontinuierlich \ac{EKG}-Daten auf.

\subsubsection{\citet{Keller2011}}
\citet{Keller2011} nutzen die Voraussetzung vom "`Gleichgewicht zwischen Anforderungen der Tätigkeit und eigenen Fähigkeiten"' in ihrem Vorgehen. Sie stellten in einem computerbasierten Fragespiel eine Langeweilekondition, eine Vereinbarkeitskondition bzw. Flow-Kondition und eine Überlastungskondition her. Acht Teilnehmer beantworteten jeweils fünf Minuten Wissensfragen in den drei Konditionen während die Wissenschaftler \ac{EKG}-Daten aufzeichneten. Am Ende jeder Kondition füllten die Teilnehmer einen Fragebogen \citep{Keller2008} aus, um den erlebten Flow explizit zu ermitteln. Die Ergebnisse zeigen eine Verringerung des zeitbezogenen \ac{HRV}-Parameters \acs{RMSSD} in der Vereinbarkeitskondition im Gegensatz zu den anderen Konditionen. Das lässt auf eine fallende parasympathische Aktivität im Flow-Erleben schließen.

\subsubsection{\citet{Gaggioli2013}}
\citet{Gaggioli2013} untersuchten den Zusammenhang zwischen Flow-Erleben und \ac{HRV}-Parametern im Laufe des Alltags von 15 Studenten über eine Dauer von sieben Tagen. Die Autoren sind bisher die Einzigen, die Messungen von psychologischen Eigenschaften zu Erforschung von Flow-Erleben in einer natürlichen Handlungsumgebung durchführten. Sie nutzen ein computergestütztes \emph{Experience Sampling} Verfahren. Sie statteten die Studenten mit einem Smartphone und einem mobilen \ac{EKG}-Sensor aus. Im Laufe des Alltags nutzten sie \emph{PsychLog}, um von den Studenten zu zufälligen Zeiten subjektive Daten und Informationen über ihre Tätigkeit zu erhalten. Flow-Erleben identifizierten sie anhand eines überdurchschnittlichen Gleichgewichts zwischen der wahrgenommenen Herausforderung und der Fähigkeitsangabe und den überdurchschnittlichen Werten von positiven Emotionen, wie Freunde, Zufriedenheit, usw. In ihren 561 Selbstberichten und 377 validen \ac{EKG}-Aufnahmen fanden sie damit 32 Flow-Erlebnisse. Ihre Aufnahmen der \ac{EKG}-Daten begannen jeweils 20 Minuten vor jeder Befragung und endeten 20 Minuten nach dem Start der Befragung. Sie fanden damit eine Beziehung zwischen steigendender \ac{HR}, ansteigendem LF/HF-Verhältnis und erlebten Flow. Beides lässt sie auf eine steigende sympathische Aktivierung im Verlauf des Erlebens von Flow schließen.

\subsubsection{\citet{Peifer2014}}
\label{ssub:peifer2014}
Alle drei bisher vorgestellten Ergebnisse sprechen gegen die Annahme einer Mühelosigkeit im Flow-Erleben. \citet{deManzano2010, Peifer2012} folgerten aus den Ergebnissen von \citet{deManzano2010}, dass nicht geringe und nicht hohe sympathische Aktivität, sondern eine optimale Aktivität dazwischen, ideal für das Flow-Erleben ist. \citet{Peifer2014} belegten die Schlussfolgerung in einer Studie, in der sie die Beziehung zwischen Flow und Stress untersuchten. Als Indikatoren für sympathische und parasympathische Aktivität dienten ihnen die frequenzbezogenen \ac{HRV}-Parameter \ac{LF} und \ac{HF}. 22 männliche Teilnehmer führten eine komplexe Aufgabe am Computer für 60 Minuten durch und bewerteten im Anschluss ihren Zustand mit Hilfe der \ac{FKS}. Zur \ac{HRV}-Analyse zeichneten \citet{Peifer2014} kontinuierlich die \ac{EKG}-Daten auf. Die Wissenschaftler werteten die Faktoren "`glatter automatisierten Verlauf"' und "`Absorbiertheit"' der \ac{FKS} aus. Zur Identifikation des explizit erlebten Flows nutzten sie nur die Absorbiertheit. Das begründeten sie wie folgt:

Ein glatter automatisierter Verlauf kann auch entstehen, wenn die Fähigkeiten die Handlungsmöglichkeiten übersteigen. Das führt eher zur Langeweile statt zum Flow-Erleben. Absorbiertheit wiederum entsteht nur, wenn die Anforderungen und die eigenen Fähigkeiten in einem Gleichgewicht stehen.

Ihre Ergebnisse zeigen eine umgekehrte u-förmige Beziehung zwischen erlebten Flow mit dem \ac{HRV}-Parameter \ac{LF}. Der \ac{HRV}-Parameter \ac{HF} verhielt sich linear und positiv zum erlebten Flow. Die Ergebnisse legen nahe, dass moderate sympathische Aktivität und eine Co-Aktivierung beider Zweige des \ac{VNS} das Erleben von Flow charakterisieren.

\subsubsection{\citet{Tozman2015}}

\citet{Tozman2015} entwickelten einen Fahrsimulator mit drei verschiedenen Schwierigkeitsgraden. Sie stellten wie bei \citet{Keller2011} eine Langeweilekondition, eine Vereinbarkeitskondition bzw. Flow-Kondition und eine Überlastungskondition her. 15 Teilnehmer fuhren die drei Simulationen jeweils für sechs Minuten in unterschiedlicher Reihenfolge. Nach jeder Simulation beantworteten die Teilnehmer eine \ac{FKS}. Die Wissenschaftler zeichneten während jeder Simulation \ac{EKG}-Daten des Teilnehmers auf. Anders als \citet{Peifer2014} nutzten \citet{Tozman2015} den Generalfaktor der \ac{FKS} als explizites Flow-Merkmal. Den Wert für \ac{LF} und \ac{HF} subtrahierten sie mit jeweils einem Wert aus einer \emph{Baseline}-Messung. Ihre Ergebnisse zeigen eine negative Beziehung zwischen dem frequenzbezogenen \ac{HRV}-Parameter \ac{LF} und Flow-Erleben in der Flow-Kondition. Zusätzlich fanden sie jeweils eine umgekehrte u-förmige Beziehung zwischen dem Generalfaktor der \ac{FKS} und den frequenzbezogenen \ac{HRV}-Parametern \ac{LF} und \ac{HF} in der Überlastungskondition. Die Ergebnisse legen eine Abnahme der Aktivität oder eine moderate Aktivität des sympathischen Zweigs des \ac{VNS}s bei Überlastung nahe.

\subsubsection{Übersicht}
In Tabelle~\ref{tab:studienuebersicht_zu_flow_erleben} gebe ich eine Übersicht der beschriebenen Studie und fasse ihre Ergebnisse zusammen.

\begin{table}[t]
	\caption[Studienübersicht zu Flow-Erleben und HRV]{Studienübersicht zu Flow-Erleben und HRV}
	\label{tab:studienuebersicht_zu_flow_erleben}
	\begin{tabularx}{\textwidth}{p{.17\textwidth} p{.06\textwidth} p{.07\textwidth} p{.06\textwidth} X}
\toprule
Studie & Teil"-neh"-mer & Sitz"-ung (pp.) & Da"-ten"-paa"-re & Ergebnis \\
\midrule
\citet{deManzano2010} & 21 & 5 & 105 & Flow = LF/HF $\uparrow{}$ + HR $\uparrow{}$ + HF $\downarrow{}$ \newline Sympathikus $\uparrow{}$ + Parasympathikus $\downarrow{}$ \\
\citet{Keller2011} & 8 & 3 & 24 & Flow = RMSSD $\downarrow{}$ \newline Parasympathikus $\downarrow{}$ \\
\citet{Gaggioli2013} & 15 & & 377 & Flow = LF/HF $\uparrow{}$ + HR $\uparrow{}$ \newline Sympathikus $\uparrow{}$ \\
\citet{Peifer2014} & 22 & 1 & 22 & Flow = LF $\cap$ + HF $\uparrow{}$ \newline Sympathikus $\cap$ + Parasympathikus $\uparrow{}$ \\
\citet{Tozman2015} & 15 & 3 & 45 & Flow = LF $\downarrow{}$ || LF $\cap$ + HF $\cap$ \newline Sympathikus $\downarrow{}$ || \newline Sympathikus $\cap$ + Parasympathikus $\cap$ \\
\bottomrule
\end{tabularx}
\end{table}

\subsection{Der Bewegungsfluss}
\label{sub:der_bewegungsfluss}

Auf der Grundlage, dass Menschen explizite durchgeführte Tätigkeiten effizienter ausführen \citep[vgl.][S.~753]{Dietrich2004}, was ich für das Flow-Erleben annehme, betrachte ich im gegenwärtigen Abschnitt einen eigenen Lösungsansatz zur impliziten Flow-Messung durch ein kinematisches Merkmal beim Gehen und Laufen.

Beim Gehen und Laufen erhalten die Füße die größte Aufmerksamkeit des \ac{VNS}s, um eine möglichst "`glatte"' Bewegung durchzuführen \citep[vgl.][S.~193]{Brooks1986}. Die bedeutendeste motorische Funktion im Bewegungsablauf des Gehens und des Laufens ist die Kontrolle über die Fußflugbahn, um ein sicheres Abstoßen des Fußes (\ac{IS}) und eine sanfte Landung des Fußes (\ac{IC}) zu erreichen \citep[vgl.][S.~197]{Winter1989}. Die beiden genannten Schlüsselereignisse der Standphase gelten als die bedeutendesten biomechanischen Merkmale für eine verzögerungsarme Erkennung zusätzlicher Eigenschaften \citep[vgl.][]{Aminian2002, Lee2011}. Anhand eines Schlüsselereignisses wie \ac{IS}, \ac{IC} oder \ac{MS} lässt sich der Bewegungsablauf in einzelne Zyklen trennen. Jeden Zyklus betrachte ich als Handlungsschritt der Tätigkeit Gehen und Laufen, den der Mensch mehr oder weniger glatt bzw. fließend oder eben effizient ausführt. Die Effizienz einer Bewegung drückt sich im Bewegungsfluss aus. Demzufolge gehe ich davon ausgehen, dass Menschen eine explizite aufgeführte Bewegung fließend bzw. effizient ausführen.

Die Quantifizierung des Bewegungsflusses erfolgt über den Bewegungsaufwand (engl. \emph{jerk-cost}) \citep[vgl.][]{Nelson1983, Hogan1984, Flash1985}. Sie definieren \emph{jerk} als zeitliche Änderung der Beschleunigung. Der Bewegungsaufwand ist die Summe der im Verlauf einer Bewegung erfolgten Beschleunigungsänderungen \citep[vgl.][]{Schneider1990}. \citet[][S.~1698]{Flash1985}) führten an, dass Bewegungen um so fließender sind, je kleiner der Beschleunigungsaufwand ist. Sie überprüften ihre Hypothese für eindimensionale Armbewegungen und verglichen die durchgeführten Bewegungen des Experimentes mit den berechneten Ergebnissen des geringst möglichen Bewegungsaufwand (engl.\emph{ minimum-jerk movement}).

Den Bewegungsaufwand nutzte \citet[][]{Hreljac2000}, um eine Gruppe ambitionierter Läufer mit einer Gruppe von Freizeit-Läufern beim Laufen und schnellem Gehen zu vergleichen. Er filmte die Läufer auf einem Laufband mit einer Videokamera und extrahierte 2D-Bewegungsdaten. Er berechnete den Beschleunigungsaufwand an der Ferse, da die Ferse als Teil des Fußes eine bedeutende Aufgabe im Bewegungsfluss des Gehens und Laufens besitzt. \citet[][]{Hreljac2000} zeigte, dass sich der Bewegungsfluss der beiden Tätigkeiten objektiv durch Auswertung des Beschleunigungsaufwandes an der Ferse quantifizieren lässt und dass ambitionierte Läufer eine "`glattere"' Bewegung beim Laufen und schnellem Gehen durchführen als Freizeit-Läufer.

Auf der Grundlage von Dietrich’s These der transienten Hypofrontalität, gehe ich davon aus, dass sich Flow-Erleben anhand des Bewegungsflusses implizit messen lässt. Die nachfolgende Passage aus \citet[][S.~121]{Meinel2007} unterstützt die vorgestellte Annahme:

"`Als eine psychische Erscheinungsform sind Emotionen zu werten, die seit langem als ‚Flow‘-Erlebnis bekannt sind. Gemeint ist damit das positive Befinden, das Erfolgserlebnis, das mit einem reibungslosen, ‚fließenden‘ Bewegungsablauf verbunden ist [\textellipsis]."'

\subsection{Die kardio-lokomotorische Phasensynchronisation}
\label{sub:die_kardio_lokomotorische_phasensynchronisation}

Einen Versuch Flow-Erleben implizit zu messen starteten wir im Projekt "`Flow-Maschinen: Körperbewegung und Klang"' mithilfe der kardio-lokomotorischen Phasensynchronisation. Bei der kardio-lokomotorischen Phasensynchronisation handelt es sich um ein zeitbezogenes Phänomen, das das Zusammenwirken des Herzens und der Bewegung betrachtet. Wir errechnen sie mit Hilfe der Zeitpunkte von Herzschlägen und Schritten. Die Phasensynchronisation bzw. Kopplung stellt sich in unterschiedlichen Verhältnisse dar (z.~B. ein Herzschlag folgt auf zwei Schritte). Zur Quantifizierung der kardio-lokomotorischen Phasensynchronisation dienen verschiedene Indizes, z.~B. ein Index basierend auf der Intensität des Mittelwerts der Fourier Reihe der Verteilung der Herzschläge im Bewegungszyklus -- Phasenkohärenz Index \citep{Rosenblum2003} oder basierend auf der Shannon Entropie -- Shannon Entropie Index \citep{Tass1998, Niizeki2005}. Zur Darstellung der kardio-lokomotorischen Phasensynchronisation dient ein \emph{Phasen Stroboscope} \citep[vgl.][]{Mrowka2000} oder auch die Synchrogramm Technik \citep[vgl.][]{Schafer1999}.

Ein ökonomisches Zusammenwirken der Systeme Herz und Bewegungsapparat betrachten für \citet[S.~18]{Niizeki2014} als einen energetisch vorteilhaften Zustand des menschlichen Organismus. Die Ergebnisse von \citet{Phillips2013} legen z.~B. für das Laufen nahe, dass herbeigeführte kardio-lokomotorische Phasensynchronisation zu einer Verbesserung der Laufleistung führt.

Wir gehen davon aus, dass eine starke kardio-lokomotorische Phasensynchronisation der ideale Zustand zwischen beiden Systemen ist. In dem keine Anpassung des menschlichen Organismus an die Anforderungssituation mehr nötig ist und dem zufolge die mentale und körperliche Beanspruchung abnimmt. Aus den vorgestellten Annahmen schließen wir auf einen Zusammenhang von den beiden Idealzuständen "`Flow-Erleben"' und "`hoher kardio-lokomotorischer Phasensynchronisation"'. In der vorliegenden Arbeit gehe ich davon aus, dass eine hohe kardio-lokomotorischer Phasensynchronisation eine Voraussetzung für Flow-Erleben beim Gehen und Laufen ist.

\section{Zusammenfassung}
\label{zusammenfassung_3}
Auf der Grundlage des gewärtigen Kapitels gehe ich davon aus, dass die einzelnen Ebenen des Modellraums in Verbindung stehen. Abbildung~\ref{fig:3_3_modellraum_2} stellt die Annahmen beruhend auf der bisherigen Forschung auf den einzelnen Ebenen und eventuelle Beziehungen zwischen den Ebenen mit Pfeilen dar.

\begin{sidewaysfigure}
	\includegraphics[width=1.00\textwidth]{3-3-modellraum-2.pdf}
	\caption[Ein Modellraum des Flow-Erlebens (Ergänzung)]{Ein Modellraum des Flow-Erlebens (Ergänzung). Quelle: Eigene Darstellung}
	\label{fig:3_3_modellraum_2}
\end{sidewaysfigure}

Bei allen impliziten Ansätzen mangelt es an der empirischen Fundierung. Der größte Teil der Studien befasst sich mit dem Zusammenhang von \ac{HRV} und Flow-Erleben bei sitzenden Tätigkeiten. Sie weisen unterschiedliche Ergebnissen auf (siehe Tabelle~\ref{tab:studienuebersicht_zu_flow_erleben}). Zudem lassen sich ihre Ergebnisse nicht bedenkenlos auf physische Tätigkeiten im Freien übertragen. Bei den anderen impliziten Ansätzen fehlt es entweder am eindeutigen Bezug zum Bewusstseinszustand Flow (z.~B. \ac{EMG}) oder an den Voraussetzungen für eine effiziente Erfassung bei physischen Tätigkeiten im Freien (z.~B. \ac{EDA} und \ac{EEG}). Die genannten Probleme erörtere ich detailliert in Abschnitt~\ref{sec:probleme_bei_messungen}.

In Abbildung~\ref{fig:3_4_datenraum_1} zeige ich den Datenraum mit den potenziellen Kandidaten für ein implizites Messverfahren des Flow-Erlebens. Bei dem Bewegungsfluss und der kardio-lokomotorischen Phasensynchronisation handelt es sich um zwei eigene Ansätze, um Flow-Erleben implizit zu messen. Der Bewegungsfluss, qualifiziert durch den Bewegungsaufwand, ist ein kinematisches Merkmal und gehört zur motorischen Ebene des Datenraums. Die kardio-lokomotorische Phasensynchronisation stellt einen Sonderfall dar, da ihre Analyse auf zwei Datenreihen aus zwei unterschiedlichen Ebenen basiert. Für beide Messverfahren fehlt bisher eine empirische Fundierung für einen Zusammenhang mit dem Flow-Erleben.

\begin{sidewaysfigure}
	\includegraphics[width=1.00\textwidth]{3-4-datenraum-1.pdf}
	\caption[Ein Datenraum für die Entwicklung eines impliziten Messverfahrens des Flow-Erlebens]{Ein Datenraum für die Entwicklung eines impliziten Messverfahrens des Flow-Erlebens. \\ \hspace{\textwidth}Quelle: Eigene Darstellung}
	\label{fig:3_4_datenraum_1}
\end{sidewaysfigure}
%!TEX root = /Users/sbogutzky/Entwicklung/projects/bogutzky/repositories/2939413/final-draft.tex
\chapter{Einleitung} % (fold)
\label{cha:einleitung}

\section{Motivation} % (fold)
\label{sec:motivation}

Warum schnüren Millionen von Menschen in der Welt tagtäglich ihre Sportschuhe und laufen? Weil sie Freude an der Ausübung der Tätigkeit empfinden.

Den Gemütszustand der Freude verbinden wir bei vielen Tätigkeiten bzw. Anforderungssituation mit dem Erleben von Flow. Flow-Erleben ist ein Bewusstseinszustand, in dem wir auf unsere Tätigkeit fokussiert sind. Sportler verknüpfen Flow-Erleben in der Regel mit hervorragenden Momenten, in denen ihr Körper und ihr Geist mühelos zusammen arbeiten \citep[S.~5]{Jackson1999}. Das Flow-Konzept ist untrennbar mit dem Namen Mihaly Csikszentmihalyi verbunden. Csikszentmihalyi untersuchte in den 1970er-Jahren, warum Menschen intrinsisch (ihrer selbst willen) motiviert sind, zu spielen. Csikszentmihalyis Untersuchungen mündeten in der Entdeckung von Flow. Er fand ihn z.~B. beim Schach spielen, Klettern im Felsen, Rock-Tanzen und Arbeiten.

\citet[][S.~58~f.]{Csikszentmihalyi2010} beschreibt das Erleben von Flow als völliges Aufgehen in einer Tätigkeit. Florian \citet[][S.~13]{Henk2014} definiert ihn als ein Verschmelzen von Handlung und Bewusstsein, das sich durch das gleichzeitige Erleben des Handlungsverlaufs als glatt und fließend und des gänzlichen Aufgehens in der Tätigkeit auszeichnet. Für \citet[][S.~602]{Csikszentmihalyi2005} ist Flow der ausschlaggebende Grund, warum Menschen eine Tätigkeit als intrinsisch belohnend empfinden. Die Ergebnisse der Studie von \citet[][S.~174]{Schuler2009} lassen darauf schließen, dass Flow durch die intrinsische Belohnung zur langfristigen Erhaltung der Ausübung einer sportlichen Aktivität, wie Laufen, beiträgt.

Körperliche Bewegung wie z.~B. Gehen und Laufen ist eine der Voraussetzungen von Gesundheit. Die Weltgesundheitsorganisation \emph{\ac{WHO}} definiert Gesundheit als „vollständiges körperliches, geistiges und soziales Wohlergehen [\textellipsis]“ \citep[S.~100]{WorldHealthOrganization1948}. Mangelnde körperliche Bewegung im Alltag und ungesunde Ernährung gefährden den Zustand und verursachen Erkrankungen wie Fettleibigkeit, Haltungsschäden und Herzerkrankungen. Programme zur Prävention und Rehabilitation der Erkrankungen setzen darauf, ihre Teilnehmer zu regelmäßiger körperlicher Bewegung zu motivieren. Nicht selten scheitern die Teilnehmer daran, regelmäßige körperliche Bewegung in ihren Alltag zu integrieren.

% section motivation (end)

\section{Kontext der Arbeit} % (fold)
\label{sec:kontext_der_arbeit}

Mit dem Ziel spontane Körperbewegung von Kindern und Jugendlichen, Erwachsen und Alten durch Klang zu unterstützen, das Erleben von Flow zu ermöglichen und zur Integration von ambulanter und stationärer Rehabilitation beizutragen, war der Gegenstand des \acs{BMBF}-Projekts „Flow-Maschinen: Körperbewegung und Klang“ (10/2012 bis 10/2015) an der Hochschule Bremen die Gestaltung, Entwicklung und Erprobung von \emph{Flow-Maschinen}. \emph{Flow-Maschinen} sind mobile Applikationen (Apps), die
\begin{itemize}

	\item das Erleben von Flow beim Gehen unterstützen,

	\item die eine unkomplizierte Nutzung im Alltag der Benutzer ermöglichen,

	\item auf alltagsüblichen Smartphones laufen,

	\item zur Integration von ambulanter und stationärer Rehabilitation im Gesundheitswesen beitragen,

	\item sektorübergreifende patientenrelevante Daten erheben.

\end{itemize}

Das Projektvorhaben war in drei in Grenzen eigenständige Teilvorhaben organisiert, die bei der Entwicklung eines vorläufigen computergestützten Modells des Erlebens beim Gehen (Protomodell) und bei der iterativen Entwicklung der \emph{Flow-Maschinen} integriert zusammenwirkten:

\begin{itemize}

	\item Die Theoriebildung und die prozessorientierte Modellierung von Erleben beim Gehen wurde durch Prof.\ Barbara Grüter, Diplompsychologin und Professorin der Mensch-Computer-Interaktion (\emph{human-computer interaction}), vorangetrieben. Ihre Ergebnisse umfassen die Weiterentwicklung von Theorie und Methode der Prozessorientierung des Erlebens beim Gehen (Protomodell), Hypothesen zu Flow und Synchronisation von Herz und Gang und die Idee „Walking Primitives“ als Vermittler zwischen Messung von impliziten Flow-Indikatoren und der Unterstützung des Erlebens durch Klang zu verwenden.

	\item Die Möglichkeiten zur Unterstützung des Erlebens beim Gehen mit dem Smartphone durch Klang wurde von Nassrin Hajinejad entwickelt und untersucht \citep{Hajinejad2013}. Ihre Ergebnisse umfassen die Konzepte zum prozessorientierten Erleben („Walking Primitives“ und „Gangklang“), sowie prototypische Instanzen der Entwicklung und Erprobung dieser Konzepte \citep{Hajinejad2015}.

	\item Die Suche nach potentiellen Kandidaten für ein implizites Flow-Messverfahren, das eine prozessorientierte Objektivierung für den Einsatz in Apps gewährleistet, wurde von mir durchgeführt und stellt den Hintergrund meiner Arbeit dar. Meine Ergebnisse stelle ich in der vorliegenden Dissertation vor.

\end{itemize}

% section kontext_der_arbeit (end)

\section{Gegenstand, Problemstellung und Ziele} % (fold)
\label{sec:gegenstand_problemstellung_und_ziele}

Um Flow-Erleben durch die Anwendungen der Mensch-Computer-Interaktion zu unterstützen, ist ein implizites Messverfahren zwingend notwendig. Ein implizites Messverfahren beschreibt ein Messverfahren, das ohne die Unterbrechung der ausgeübten Tätigkeit auskommt und das Erleben beiläufig ohne mündliche Auskünfte erfasst.

Aus diesem Grund beschäftige ich mich in der vorliegenden Dissertation mit der Suche nach physiologischen und motorischen Eigenschaften, die eine implizierte Messung des Flow-Erlebens während des Gehens und Laufens in Echtzeit ermöglichen. 

Psychologische Merkmale und Voraussetzungen für das Flow-Erleben beim Laufen sind in mehreren Studien dokumentiert und untersucht worden \citep{Stoll2005, Reinhardt2006, Schuler2009, Jimenez-Torres2013}. Ihre Erkenntnisse basieren alle auf explizit erfragte Merkmale des Flow-Erlebens. Es fehlen jedoch Erkenntnisse über psycho-physiologischen Zusammenhänge des Flow-Erlebens (implizite Indikatoren) unter physischer Belastung. Studien zu psycho-physiologischen Zusammenhängen des Flow-Erlebens betrachten überwiegend sitzende Tätigkeiten \citep{deManzano2010, Keller2011, Peifer2014, Tozman2015}. 

Im Flow erlebt der Handelnde „den Prozess als ein einheitliches ‚Fließen‘ von einem Augenblick zum nächsten, wobei er Meister seines Handelns ist“ \citep[][S.~59]{Csikszentmihalyi2010}. Flow ist das „reflexionsfreie, gänzliche Aufgehen in einer glatt laufenden Tätigkeit, die man trotz hoher Beanspruchung noch unter Kontrolle hat“ \citep[][S.~156]{Rheinberg2003}. Im Flow verschmelzen Handlung und Bewusstsein und der Handelnde erlebt den Handlungsverlauf „als glatt und fließend“ \citep[][S.~13]{Henk2014}.

Betrachten wir die vorherigen Beschreibungen des Flow-Erlebens, deuten die Abschnitte, die das Erleben als fließend oder glatt beschreiben, auf \emph{psycho-motorische} Zusammenhänge im Flow-Erleben hin. Ungeachtet dessen gibt es keine empirischen Untersuchungen.


% Das Ziel der vorliegenden Arbeit ist Flow-Erleben als Maß in Apps für das Gehen und das Laufen zu verwenden. Lauf-Apps dienen als Anwendungsbeispiel, in denen Flow-Erleben als Maß des Wohlbefindens, die überwiegend leistungsbezogenen Kennzahlen wie z.~B. zurückgelegte Strecke, Kalorienverbrauch, usw. ergänzt. Damit möchte ich das Fundament schaffen, um Lauf-Apps zu entwickeln, die die Voraussetzungen verbessern, Flow beim Laufen zu erleben.

% section gegenstand_problemstellung_und_ziele (end)



% \section{Aufbau der Arbeit}
%
% In Kapitel~\ref{cha:technologie_beim_laufen} schaffe ich ein Verständnis für die zahlreichen Aspekte, mit denen ich mich in der vorliegenden Arbeit im Bereich der mobilen Mensch-Computer-Interaktion in Bezug auf Sport und Laufen beschäftige. Es zeigt im vorgestellten Bereich der mobilen Mensch-Computer-Interaktion technologische Anknüpfungspunkte für ein impliziertes Messverfahren des Flow-Erlebens auf.
%
% Ich vermittele in Kapitel~\ref{cha:flow_erleben_messen} das Wissen über das Flow-Erleben als messbare Größe und veranschauliche die Merkmale, die das Messen eines Bewusstseinszustands wie Flow-Erleben ermöglichen. Ich beleuchte die expliziten Messverfahren des Flow-Erlebens und stelle Lösungsansätze zur impliziten Messung von Flow-Erleben, deren Datenerhebung eine Echtzeitverarbeitung ermöglichen, vor.
%
% In Kapitel~\ref{cha:flow_erleben_beim_gehen_und_laufen_messen_anforderungen} gebe ich einen Überblick über die Anforderungen an ein implizites Messverfahren, das eine mobile und computergestützte Echtzeitverarbeitung beim Gehen und Laufen zulässt. Ich stelle die technischen Arbeitsschritte vor, die zu der Entwicklung einer App notwendig sind, die die Voraussetzungen verbessert, Flow beim Gehen und Laufen zu erleben.
%
% Ich erläutere und diskutiere in Kapitel~\ref{cha:studien_zur_mobilen_und_prozessorientierten_messung} drei Studien mit ihren Methoden und Ergebnissen. In den Studien untersuchte ich Zusammenhänge zwischen expliziten Flow-Merkmale gemessenen durch die \ac{FKS} und Kandidaten für ein implizites Messverfahren des Flow-Erlebens beim Laufen und Gehen.
%
% In Kapitel~\ref{cha:integration_in_eine_assistierende_echtzeit_benutzerschnittstelle} veranschauliche ich die mögliche Integration einer impliziten Messung des Flow-Erlebens in eine Echtzeit-Benutzerschnittstelle. Dabei stelle ich den Demonstrator der \emph{Flow-Maschine} und Teile der Projektstudie des \acs{BMBF}-Projekts vor.
%
% Eine zusammenfassende Diskussion der vorliegenden Arbeit liefere ich in Kapitel~\ref{cha:diskussion}. Abschließend fasse ich in Kapitel~\ref{cha:wissenschaftlicher_beitrag_und_ausblick} die mit der vorliegenden Dissertation geleisteten wissenschaftlichen Beiträge zusammen und gebe einen Ausblick auf Anknüpfungspunkte für fortführende Forschungsarbeiten.

% chapter einleitung (end)


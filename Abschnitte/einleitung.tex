

%!TEX root = /Users/sbogutzky/Entwicklung/projects/bogutzky/repositories/2939413/final-draft.tex
\chapter{Einleitung} 

% (fold)
\label{cha:einleitung}

\section{Motivation} 

% (fold)
\label{sec:motivation}

Warum schnüren Millionen von Menschen in der Welt tagtäglich ihre Sportschuhe und laufen? Weil sie Freude an der Ausübung der Tätigkeit empfinden.

Den Gemütszustand der Freude verbinden wir bei vielen Tätigkeiten bzw. Anforderungssituation mit dem Erleben von Flow. Flow-Erleben ist ein Bewusstseinszustand, in dem wir auf unsere Tätigkeit fokussiert sind. Sportler verknüpfen Flow-Erleben in der Regel mit hervorragenden Momenten, in denen ihr Körper und ihr Geist mühelos zusammen arbeiten \citep[S.~5]{Jackson1999}. Das Flow-Konzept ist untrennbar mit dem Namen Mihaly Csikszentmihalyi verbunden. Csikszentmihalyi untersuchte in den 1970er-Jahren, warum Menschen intrinsisch (ihrer selbst willen) motiviert sind, zu spielen. Csikszentmihalyis Untersuchungen mündeten in der Entdeckung von Flow. Er fand ihn z.~B. beim Schach spielen, Klettern im Felsen, Rock-Tanzen und Arbeiten.

\citet[][S.~58~f.]{Csikszentmihalyi2010} beschreibt das Erleben von Flow als völliges Aufgehen in einer Tätigkeit. Florian \citet[][S.~13]{Henk2014} definiert ihn als ein Verschmelzen von Handlung und Bewusstsein, das sich durch das gleichzeitige Erleben des Handlungsverlaufs als glatt und fließend und des gänzlichen Aufgehens in der Tätigkeit auszeichnet. Für \citet[][S.~602]{Csikszentmihalyi2005} ist Flow der ausschlaggebende Grund, warum Menschen eine Tätigkeit als intrinsisch belohnend empfinden. Die Ergebnisse der Studie von \citet[][S.~174]{Schuler2009} lassen darauf schließen, dass Flow durch die intrinsische Belohnung zur langfristigen Erhaltung der Ausübung einer sportlichen Aktivität, wie Laufen, beiträgt.

% section motivation (end)
\section{Gegenstand, Problemstellung und Ziele} 

% (fold)
\label{sec:gegenstand_problemstellung_und_ziele}

Um Flow-Erleben durch die Anwendungen der Mensch-Computer-Interaktion (human-computer interaction) zu unterstützen, ist ein implizites Messverfahren zwingend notwendig. Ein implizites Messverfahren beschreibt ein Messverfahren, das ohne die Unterbrechung der ausgeübten Tätigkeit auskommt und das Erleben beiläufig ohne mündliche Auskünfte erfasst.

Beim Laufen sind psychologische Merkmale und Voraussetzungen für das Flow-Erleben in mehreren Studien dokumentiert und untersucht worden \citep{Stoll2005, Reinhardt2006, Schuler2009, Jimenez-Torres2013}. Ihre Erkenntnisse basieren allesamt auf explizit erfragte Merkmale des Flow-Erlebens durch Selbstauskünfte. Es fehlen jedoch Erkenntnisse über psycho-physiologischen Zusammenhänge des Flow-Erlebens (implizite Merkmale) unter \emph{physischer Belastung}. 

Studien zu impliziten Merkmalen des Flow-Erlebens betrachten überwiegend sitzende Tätigkeiten \citep{deManzano2010, Keller2011, Peifer2014, Tozman2015, Harmat2015}. Ihre Ergebnisse basieren überwiegend auf experimentellen Untersuchungsplänen mit mehreren Untersuchungspersonen und interindividuellen Vergleichen. Nach \citet[][S.77]{Henk2014} wären \emph{intraindividuelle Untersuchungen} für aussagekräftige Schlussfolgerungen notwendig: Auf individueller Ebene kann beispielsweise ein Zusammenhang zwischen einem erlebten Flow-Zustand und hoher Effizienz der Bewegung beim Gehen bestehen, auch wenn die Effizienz interindividuell gesehen als gering einzuordnen ist.

Im Flow erlebt der Handelnde „den Prozess als ein einheitliches ‚Fließen‘ von einem Augenblick zum nächsten, wobei er Meister seines Handelns ist“ \citep[][S.~59]{Csikszentmihalyi2010}. Flow ist das „reflexionsfreie, gänzliche Aufgehen in einer glatt laufenden Tätigkeit, die man trotz hoher Beanspruchung noch unter Kontrolle hat“ \citep[][S.~156]{Rheinberg2003}. Im Flow verschmelzen Handlung und Bewusstsein und der Handelnde erlebt den Handlungsverlauf „als glatt und fließend“ \citep[][S.~13]{Henk2014}.

Betrachten wir die vorherigen Beschreibungen des Flow-Erlebens, deuten die Abschnitte, die das Erleben als fließend oder glatt beschreiben, auf \emph{psycho-motorische} Zusammenhänge im Flow-Erleben hin. Ungeachtet dessen gibt es keine empirischen Untersuchungen.

Flow-Erleben auf der Basis von explizit erfragten Merkmalen im Prozess der Tätigkeit Laufen dokumentierten \citet{Reinhardt2006, Schuler2009} und rekonstruierten den Verlauf des Flow-Erlebens über die Zeit. \emph{Prozessorientierte Messungen von physiologischen und motorischen Merkmalen} zum Vergleich mit mehrfachen Selbstauskünften über das Flow-Erleben wurde hingehen noch nicht durchgeführt.

Angesichts der vier Forschungslücken 
\begin{itemize}
	
	\item fehlende implizite Merkmale des Flow-Erlebens unter physischer Belastung,
	
	\item fehlende Berücksichtigung der Individualität bei der Ausführung der Tätigkeit und beim Erleben von Flow,
	
	\item fehlende Empirie zu psycho-motorische Zusammenhänge des Flow-Erlebens und
	
	\item fehlender Vergleich von physiologischen und motorischen Merkmalen mit explizit erfragten Merkmalen des Flow-Erlebens im Prozess 
\end{itemize}
suche und prüfe ich potentiellen Kandidaten für ein implizites Flow-Messverfahren, das eine prozessorientierte Objektivierung für den Einsatz in Apps gewährleistet. 

Lauf-Apps dienen als Anwendungsbeispiel, in denen Flow-Erleben z.~B. als Maß des Wohlbefindens, die überwiegend leistungsbezogenen Kennzahlen wie z.~B. zurückgelegte Strecke, Kalorienverbrauch, usw.\ ergänzt. Damit möchte ich zur Bildung des Fundaments beitragen, Lauf-Apps zu entwickeln, die die Voraussetzungen verbessern, Flow beim Laufen zu erleben. Der Gegenstand der vorliegenden Arbeit mit seinen Problemen und dem von mir verfassten Ziel entstand im Kontext des \acs{BMBF}-Projekts \emph{Flow-Maschinen: Körperbewegung und Klang} (10/2012 bis 10/2015) an der Hochschule Bremen.

% section gegenstand_problemstellung_und_ziele (end)
\section{Kontext der Arbeit} 

% (fold)
\label{sec:kontext_der_arbeit}

Körperliche Bewegung wie z.~B. Gehen und Laufen ist eine der Voraussetzungen von Gesundheit. Die Weltgesundheitsorganisation \ac{WHO} definiert Gesundheit als „vollständiges körperliches, geistiges und soziales Wohlergehen [\textellipsis]“ \citep[S.~100]{WorldHealthOrganization1948}. Mangelnde körperliche Bewegung im Alltag und ungesunde Ernährung gefährden den Zustand und verursachen Erkrankungen wie Fettleibigkeit, Haltungsschäden und Herzerkrankungen. Programme zur Prävention und Rehabilitation der Erkrankungen setzen darauf, ihre Teilnehmer zu regelmäßiger körperlicher Bewegung zu motivieren. Nicht selten scheitern die Teilnehmer daran, regelmäßige körperliche Bewegung in ihren Alltag zu integrieren.

Die Zielsetzung des \acs{BMBF}-Projekts \emph{Flow-Maschinen: Körperbewegung und Klang} war, Menschen im Alltag bei ihren gesundheitsförderlichen Absichten aktiv zu unterstützen. Der Ausgangspunkt des Projektvorhabens war die Annahme, dass das Erleben von Flow während der körperlichen Bewegung uns Menschen intrinsisch motiviert, sich mehr zu bewegen. Zu diesem Zweck ist der Gegenstand des Projekts die Gestaltung, Realisierung und Erprobung von Flow-Maschinen. Flow-Maschinen sind mobile Applikationen (Apps), die das Gehen von Kindern und Jugendlichen, Erwachsenen und älteren Menschen durch Klang unterstützen, sich unkompliziert in den Alltag ihrer Benutzer integrieren lassen und das Erleben von Flow fördern.

Das Projektvorhaben war in drei relativ eigenständige Teilvorhaben organisiert, die bei der Entwicklung eines vorläufigen computergestützten Modells des Erlebens beim Gehen (Protomodell) und bei der iterativen Entwicklung der Flow-Maschinen integriert zusammenwirkten: 
\begin{itemize}
	
	\item Die Theoriebildung und die prozessorientierte Modellierung von Erleben beim Gehen wurde durch Prof.\ Barbara Grüter, Diplompsychologin und Professorin der Mensch-Computer-Interaktion, vorangetrieben.
	
	\item Die Möglichkeiten zur Unterstützung des Erlebens beim Gehen mit dem Smartphone durch Klang wurde von Nassrin Hajinejad entwickelt und untersucht \citep{Hajinejad2013, Hajinejad2015}. Dabei verfolgte sie einen daten- und erfahrungsorientierten Bottom-up-Ansatz.
	
	\item Die Suche nach potentiellen Kandidaten für ein implizites Flow-Messverfahren, das eine prozessorientierte Objektivierung für den Einsatz in Apps gewährleistet, wurde von mir durchgeführt und stellt den Hintergrund der vorliegenden Arbeit dar. Im Gegensatz zu Hajinejad verfolgte ich einen analytischen theoriegeleiteten Top-down-Ansatz, dessen Herangehensweise ich im nachfolgenden Abschnitt vorstelle. 
\end{itemize}

% section kontext_der_arbeit (end)
\section{Herangehensweise} 

% (fold)
\label{sec:herangehensweise}

Zurzeit dominieren explizite zustandsorientierte Messverfahren die Untersuchungen des Flow-Erlebens. Sie sind charakterisiert durch ihre indirekte Beziehung zum Flow-Erleben und ihre Erfassung nach der zu untersuchenden Tätigkeit. Diese Messverfahren erfragen die Komponenten des Flow-Konstrukts, welche in Kapitel~\ref{cha:flow_erleben_messen} vorstelle. 

In drei Schritten näher ich mich einem impliziten Flow-Messverfahren, das eine prozessorientierte Objektivierung für den Einsatz in Apps beim Gehen und Laufen gewährleistet: 
\begin{enumerate}
	\item Um den statischen Charakter der expliziten zustandsorientierten Messverfahren aufzulösen, messe ich wie \citet{Reinhardt2006, Schuler2009} die expliziten Flow-Merkmale durch mehrfache Selbstauskünfte in einer fortlaufenden Tätigkeit. Je nach zeitlichem Abstand kann ich damit den erlebte Flow über die Zeit rekonstruieren. Gleichzeitig messe ich implizite physiologische und kinematische Daten. Zu diesem Zweck entstand eine mobile Messanwendung für Smartphones, auf die ich in Kapitel~\ref{cha:studien_zur_mobilen_und_prozessorientierten_messung} eingehe.
	
	\item Die Daten dienen in akkumulierter Form als implizite Merkmale (Kandidaten), die ich den Komponenten des Flow-Konstrukts zu ordne. Ausgehend von den expliziten Merkmalen untersuche ich die Korrelation zu diesen Kandidaten für ein implizites Messverfahren.
	
	\item Untersuche ich das zeitliche Verhalten (Prozess) der impliziten Daten, um Übergänge zwischen Flow und nicht Flow beschreiben zu können. 
\end{enumerate}

Aufgrund der gegebenen logischen Abhängigkeit von Schritt 2 und 3 liegt das Augenmerk zunächst auf Schritt 2, der Suche nach signifikanten Korrelationen zwischen expliziten Merkmalen und impliziten Kandidaten des Flow-Erlebens. 

% section herangehensweise (end)
\section{Aufbau der Arbeit} 

% (fold)
\label{sec:aufbau_der_arbeit}

In Kapitel~\ref{cha:technologie_beim_laufen} schaffe ich ein Verständnis für die zahlreichen Aspekte, mit denen ich mich in der vorliegenden Arbeit im Bereich der mobilen Mensch-Computer-Interaktion in Bezug auf Sport und Laufen beschäftige. Es zeigt im vorgestellten Bereich der mobilen Mensch-Computer-Interaktion technologische Anknüpfungspunkte für ein implizites Messverfahren des Flow-Erlebens auf.

Ich stelle in Kapitel~\ref{cha:flow_erleben_messen} Flow-Erleben als messbare Größe dar und veranschauliche die Merkmale, die das Messen eines Bewusstseinszustands wie Flow-Erleben ermöglichen. Ich beleuchte die expliziten Messverfahren, bespreche prozessorientierte Ansätze mit expliziten Messverfahren und stelle Lösungsansätze zur impliziten Messung von Flow-Erleben, deren Datenerhebung eine Echtzeitverarbeitung ermöglichen, vor.

In Kapitel~\ref{cha:flow_erleben_beim_gehen_und_laufen_messen_anforderungen} gebe ich einen Überblick über die Anforderungen an ein implizites Messverfahren, das eine mobile und computergestützte Echtzeitverarbeitung beim Gehen und Laufen zulässt. Ich veranschauliche die technischen Arbeitsschritte, die zu der Entwicklung einer App notwendig sind, die die Voraussetzungen verbessert, Flow beim Gehen und Laufen zu erleben.

Ich stelle in Kapitel~\ref{cha:studien_zur_mobilen_und_prozessorientierten_messung} drei Studien vor und diskutiere ihre Methode und Ergebnisse. Bei den ersten beiden Studien (Laufen und Gehen) nutze ich einen intraindividuellen Versuchsaufbau und gehe nach den drei beschriebenen Schritten in Abschnitt~\ref{sec:herangehensweise} vor. In der dritten Studie stelle auf der Grundlage der Ergebnisse der ersten beiden Studien Forschungshypothesen auf und prüfe diese mit einer Gruppe von Läufern (interindividuell).

In Kapitel~\ref{cha:integration_in_eine_assistierende_echtzeit_benutzerschnittstelle} veranschauliche ich die mögliche Integration einer impliziten Messung des Flow-Erlebens in eine Echtzeit-Benutzerschnittstelle. Dabei stelle ich den Demonstrator der Flow-Maschine vor.

Eine zusammenfassende Diskussion mit Bewertung und Einordnung der Ergebnisse der vorliegenden Arbeit liefere ich in Kapitel~\ref{cha:diskussion}. Abschließend fasse ich in Kapitel~\ref{cha:wissenschaftlicher_beitrag_und_ausblick} die mit der vorliegenden Dissertation geleisteten wissenschaftlichen Beiträge zusammen und gebe einen Ausblick auf Anknüpfungspunkte für fortführende Forschungsarbeiten.

% section aufbau_der_arbeit (end)
% chapter einleitung (end)

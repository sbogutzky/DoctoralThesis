%!TEX root = /Users/sbogutzky/Entwicklung/projects/bogutzky/repositories/2939413/final-draft.tex
\chapter{Einleitung}
\section{Motivation}

Warum schnüren Millionen von Menschen in der Welt tagtäglich ihre Sportschuhe und laufen? Weil sie Freude an der Ausübung der Tätigkeit empfinden.

Den Gemütszustand der Freude verbinden wir bei vielen Tätigkeiten bzw. Anforderungssituation mit dem Erleben von Flow. Flow-Erleben ist ein Bewusstseinszustand, in dem wir auf unsere Tätigkeit fokussiert sind. Sportler verknüpfen Flow-Erleben in der Regel mit hervorragenden Momenten, in denen ihr Körper und ihr Geist mühelos zusammen arbeiten \citep[vgl.][S.~5]{Jackson1999}. Das Flow-Konzept ist untrennbar mit dem Namen Mihaly Csikszentmihalyi verbunden. Csikszentmihalyi untersuchte in den 1970er-Jahren, warum Menschen intrinsisch (ihrer selbst willen) motiviert sind, zu spielen. Csikszentmihalyis Untersuchungen mündeten in der Entdeckung von Flow. Er fand ihn z.~B. beim Schach spielen, Klettern im Felsen, Rock-Tanzen und Arbeiten.

\citet[][S.~58~f.]{Csikszentmihalyi2010} beschreibt das Erleben von Flow als völliges Aufgehen in einer Tätigkeit. Florian \citet[][S.~13]{Henk2014} definiert ihn als ein Verschmelzen von Handlung und Bewusstsein, das sich durch das gleichzeitige Erleben des Handlungsverlaufs als glatt und fließend und des gänzlichen Aufgehens in der Tätigkeit auszeichnet. Für \citet[][S.~602]{Csikszentmihalyi2005} ist Flow der ausschlaggebende Grund, warum Menschen eine Tätigkeit als intrinsisch belohnend empfinden. Die Ergebnisse der Studie von \citet[][S.~174]{Schuler2009} lassen darauf schließen, dass Flow durch die intrinsische Belohnung zur langfristigen Erhaltung der Ausübung einer sportlichen Aktivität, wie Laufen, beiträgt.

Körperliche Bewegung wie z.~B. Gehen und Laufen ist eine der Voraussetzungen von Gesundheit. Die Weltgesundheitsorganisation \emph{\ac{WHO}} definiert Gesundheit als "`vollständiges körperliches, geistiges und soziales Wohlergehen [\textellipsis]"' \citep[vgl.][S.~100]{WorldHealthOrganization1948}. Mangelnde körperliche Bewegung im Alltag und ungesunde Ernährung gefährden den Zustand und verursachen Erkrankungen wie Fettleibigkeit, Haltungsschäden und Herzerkrankungen. Programme zur Prävention und Rehabilitation der Erkrankungen setzen darauf, ihre Teilnehmer zu regelmäßiger körperlicher Bewegung zu motivieren. Nicht selten scheitern die Teilnehmer daran, regelmäßige körperliche Bewegung in ihren Alltag zu integrieren.

Mit dem Ziel, Bewegung in Form von Gehen im Alltag der Menschen aktiv zu unterstützen, entwarfen und programmierten wir im \acs{BMBF}-Projekt "`Flow-Maschinen: Körperbewegung und Klang"' (10/2012 bis 10/2015) an der Hochschule Bremen Prototypen von \emph{Flow-Maschinen}. Eine \emph{Flow-Maschine} ist eine mobile Applikation (sog. App), die die Voraussetzungen, um Flow beim Gehen zu erleben, verbessert. Aufgrund des Fehlens eines impliziten Messverfahrens zur Flow-Messung, versuchten wir, mit unseren finalen Demonstrator der \emph{Flow-Maschine} für ein Smartphone, das Gehen durch Klang erfreulicher zu gestalten.

Um Flow-Erleben durch die Anwendungen der Mensch-Computer-Interaktion (engl. \emph{human-computer interaction}) zu ermöglichen, ist ein implizites Messverfahren zwingend notwendig. Ein implizites Messverfahren beschreibt ein Messverfahren, das ohne die Unterbrechung der ausgeübten Tätigkeit auskommt und nicht die Aufmerksamkeit auf eine parallel laufende Tätigkeit lenkt. Ein solches implizites Messverfahren des Flow-Erlebens ermöglicht erst eine automatische Anpassung z.~B. von Klang zur Unterstützung des Flow-Erlebens.

\section{Gegenstand, Problemstellung und Ziele}

Ich beschäftige mich in der vorliegenden Arbeit mit der Suche nach physiologischen und motorischen Eigenschaften, die eine implizierte Messung des Flow-Erlebens während des Gehens und Laufens in Echtzeit ermöglichen.

Psychologische Merkmale und Voraussetzungen für das Flow-Erleben beim Laufen sind in mehreren Studien dokumentiert und untersucht worden \citep{Stoll2005, Reinhardt2006, Schuler2009, Jimenez-Torres2013}. Im Gegensatz fehlen Erkenntnisse über psycho-physiologischen Zusammenhänge des Flow-Erlebens unter physischer Belastung. Studien zu psycho-physiologischen Zusammenhängen des Flow-Erlebens betrachten überwiegend sitzende Tätigkeiten \citep{deManzano2010, Keller2011, Peifer2014, Tozman2015}. Ihre Erkenntnisse lassen sich aufgrund der niedrigen physischen Belastung nur begrenzt auf die Tätigkeiten Gehen und Laufen übertragen.

Die unterschiedlichen Definitionen des Flow-Erlebens deuten auf psycho-motorische Zusammenhänge im Flow-Erleben hin. Ungeachtet dessen gibt es keine empirischen Untersuchungen.

Das Ziel der vorliegenden Arbeit ist Flow-Erleben als Maß in Apps für das Gehen und das Laufen zu verwenden. Lauf-Apps dienen als Anwendungsbeispiel, in denen Flow-Erleben als Maß des Wohlbefindens, die überwiegend leistungsbezogenen Kennzahlen wie z.~B. zurückgelegte Strecke, Kalorienverbrauch, usw. ergänzt. Damit möchte ich das Fundament schaffen, um Lauf-Apps zu entwickeln, die die Voraussetzungen verbessern, Flow beim Laufen zu erleben.

\section{Aufbau der Arbeit}

In Kapitel~\ref{cha:technologie_beim_laufen} schaffe ich ein Verständnis für die zahlreichen Aspekte, mit denen ich mich in der vorliegenden Arbeit im Bereich der mobilen Mensch-Computer-Interaktion in Bezug auf Sport und Laufen beschäftige. Es zeigt im vorgestellten Bereich der mobilen Mensch-Computer-Interaktion technologische Anknüpfungspunkte für ein impliziertes Messverfahren des Flow-Erlebens auf.

Ich vermittele in Kapitel~\ref{cha:flow_erleben_messen} das Wissen über das Flow-Erleben als messbare Größe und veranschauliche die Merkmale, die das Messen eines Bewusstseinszustands wie Flow-Erleben ermöglichen. Ich beleuchte die expliziten Messverfahren des Flow-Erlebens und stelle Lösungsansätze zur impliziten Messung von Flow-Erleben, deren Datenerhebung eine Echtzeitverarbeitung ermöglichen, vor.

In Kapitel~\ref{cha:flow_erleben_beim_gehen_und_laufen_messen_anforderungen} gebe ich einen Überblick über die Anforderungen an ein implizites Messverfahren, das eine mobile und computergestützte Echtzeitverarbeitung beim Gehen und Laufen zu lässt. Ich stelle die technischen Arbeitsschritte vor, die zu der Entwicklung einer App notwendig sind, die die Voraussetzungen verbessert, Flow beim Gehen und Laufen zu erleben.

Ich erläutere und diskutiere in Kapitel~\ref{cha:studien_zur_mobilen_und_prozessorientierten_messung} drei Studien mit ihren Methoden und Ergebnissen. In den Studien untersuchte ich Zusammenhänge zwischen expliziten Flow-Merkmale gemessenen durch die \ac{FKS} und Kandidaten für ein implizites Messverfahren des Flow-Erlebens beim Laufen und Gehen.

In Kapitel~\ref{cha:integration_in_eine_assistierende_echtzeit_benutzerschnittstelle} veranschauliche ich die mögliche Integration einer impliziten Messung des Flow-Erlebens in eine Echtzeit-Benutzerschnittstelle. Dabei stelle ich den Demonstrator der \emph{Flow-Maschine} und Teile der Projektstudie des \acs{BMBF}-Projekts vor.

Eine zusammenfassende Diskussion der vorliegenden Arbeit liefere ich in Kapitel~\ref{cha:diskussion}. Abschließend fasse ich in Kapitel~\ref{cha:wissenschaftlicher_beitrag_und_ausblick} die mit der vorliegenden Dissertation geleisteten wissenschaftlichen Beiträge zusammen und gebe einen Ausblick auf Anknüpfungspunkte für fortführende Forschungsarbeiten.

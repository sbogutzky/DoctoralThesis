%!TEX root = /Users/sbogutzky/Entwicklung/projects/bogutzky/repositories/2939413/final-draft.tex
\section*{Danksagung}

Diese Dissertation ist das Ergebnis von vier Jahren wissenschaftlicher Arbeit zum Thema \emph{Objektivierung und Messung von Flow-Erleben beim Gehen und Laufen für mobile Applikationen} an der Hochschule Bremen. Meine bisher größte Herausforderung war diese Arbeit zu verfassen war  und ich freue mich nun endlich die Gelegenheit nutzen zu können, um mich bei den vielen Menschen zu bedanken, die mir durch ihre Geduld, Anregung und Zeit die nötige Unterstützung, Kraft und Motivation gaben, diese Arbeit abschließen zu können.

Ein besonderer Dank gilt meiner Doktormutter, Frau Prof. Dr. Barbara Grüter, die mit ihrer Projektidee \emph{Flow-Maschinen: Körperbewegung und Klang}, dem Projektantrag und der anschließenden genehmigten Förderungen durch das Bundesministerium für Bildung und Forschung sowie der Hochschule Bremen überhaupt erst das Thema und die wissenschaftliche Stelle für meine Disseration ermöglichte. Für ihr unermüdliches Interesse an dem Thema, ihre Anregungen und die Diskussionen, die Zeit und das Vertrauen, welches sie mir und meiner Arbeit in all den Jahren entgegengebracht hat, möchte ich mich ganz herzlich bedanken.

Dieser Dank gilt auch meinem Zweitgutachter, Herrn Prof. Dr. Rainer Malaka, von der Universität Bremen, der sich die Zeit genommen hat meine Dissertation aufmerksam zu lesen, kritisch zu hinterfragen und mir durch sein Feedback die nötige Motivation gab, die letzten Wochen durchzuhalten und die Arbeit mit gutem Gefühl nun zu beenden.

Das Projekt war von Anfang an auf Teamarbeit angewiesen und so bedanke ich mich bei meinen KollegInnen der Forschungsgruppe \emph{Gangs of Bremen} (Nassrin Hajinejad, Annika Worpenberg, Alina Skamnioti, Iaroslav Sheptykin und Andreas Lochwitz), den studentischen Hilfskräften (Jan Christoph Schrader und Phillip Marsch) und den Projektpartnern (Corinna Peifer und Licinio Roque) für die offenen Ohren, Denkanstöße, Diskussionsrunden und für ihre Arbeit. Bei meinen KollegInnen am ZIMT bedanke ich mich für die gemeinsam verbrachten Pausen, in denen es auch mal nicht um Hypothesen, Analysen und Datenerhebung ging. Ich habe gerne mit euch zusammengearbeitet und bin dankbar, dass wir diesen Weg mit all den Höhen und Tiefen gemeinsam gegangen sind. Besonders erwähnen möchte ich Andreas Lochwitz, der immer die gute Seele in unserem Team war und der uns ganz selbstverständlich in allen Bereichen seine Hilfe und Unterstützung angeboten hat -- vielen Dank dafür! 

Auf diesem Weg möchte ich mich auch bei den Studienteilnehmern aus meinen Fußballmannschaften des TSV Dörverden und aus meinem Freundeskreis bedanken, die bereitwillig bei gutem und schlechten Wetter für das Projekt gelaufen sind, um die Daten für die finale Laufstudie zu liefern. Speziell möchte ich Patrick Buse erwähnen, der sich von Anfang an für das Projekt interessiert hat und ebenfalls mit seinen kilometerlangen Läufen einen großen Beitrag zu den Studien beigetragen hat.

Meiner Familie und meinen Freunden möchte ich ebenfalls einen großen Dank für den Rückhalt und die Geduld aussprechen, die sie mir gegenüber entgegengebracht haben, wenn ich abends, an den Wochenenden oder an Feiertagen keine Zeit hatte oder mit den Gedanken bei der Arbeit war. 

Meinen Eltern danke ich von ganzem Herzen für die seelische und moralische Unterstützung und insbesondere meinem Vater, der so viel Zeit mit Korrekturlesen verbrachte. Vielen Dank dafür, Mama und Papa! 

Abschließend gilt mein Dank meiner lieben Frau und meinen bezaubernden Kindern, die mich mit all meinen Stärken und Schwächen lieben, bedingungslos an mich geglaubt haben und mir unendlich viel Zeit, Kraft und Mut gegeben haben, dieses für mich so wichtige Kapitel in meinem Leben zu beenden. 
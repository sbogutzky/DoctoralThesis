

%!TEX root = /Users/sbogutzky/Entwicklung/projects/bogutzky/repositories/2939413/final-draft.tex
\chapter{Studiendokumente} 

% (fold)
\label{cha:studiendokumente}

\section{Ablauf und Wortlaut der Laufsitzungen der finalen Laufstudie} 

% (fold)
\label{sec:ablauf_und_wortlaut_der_laufsitzungen_der_finalen_laufstudie} 
\begin{enumerate}
	\item Sensoren verbunden 
	\item Sensoren getestet 
	\item Teilnehmer ist in Kabine 2
	
	"`Vielen Dank, dass du heute an meiner Laufuntersuchung teilnimmst. Ich bitte dich, dich umzuziehen und dich hier drinnen auf den Lauf vorzubereiten, z.~B. mit locker machen oder dehnen. Wenn du fragen hast, werde ich diese nach dem Lauf beantworten, um dich nicht zu beeinflussen. Damit keine wichtigen Informationen verloren gehen, nehme ich unsere Gespräche auf. Die Aufnahme wird später selbstverständlich wieder gelöscht. Ist das ok?"'
	
	\item Teilnehmer hat sich umgezogen 
	\item Teilnehmer locker und gedehnt
	
	\item Einstellungen auf \emph{Laufen} gestellt 
	\item Sensoren und Smartphone am Teilnehmer angebracht 
	\item Sensoren testet 
	\item Teilnehmer die Strecke gezeigt. 
	
	"`Nun läufst du die folgende Strecke 3-mal und kehrst hier wieder zurück. Wichtig für den Lauf ist, es geht nicht um die Laufleistung, also nicht darum wie schnell du die Strecke zurücklegst. Ein Trainer bekommt die Ergebnisse nicht mitgeteilt. 
	
	\textbf{Du sollst die Strecke in einem für dich optimalen Tempo laufen, welches sich für dich am besten anfühlt, weil es dich nicht überfordert und auch nicht unterfordert.}
	
	Die Ergebnisse eurer Läufe sollen dabei helfen, eine neue ganzzeitliche Trainingsmethode zu entwickeln."'
	
	\item Teilnehmer nach draußen begleitet 
	\item Aufnahme gestartet 
	\item \acs{GPS} geprüft
	
	--
	
	\item Teilnehmer rein begleitet und hinsetzen lassen 
	\item Aufnahme beendet
	
	"`Jetzt wirst du einen Fragebogen ausfüllen. Versetze die in den eben beendeten Lauf und beziehe die Aussagen den Lauf. Entscheidend ist wieder dabei deine persönliche Einschätzung, es gibt keine richtigen oder falschen Antworten. Je weiter links du anklickst, umso weniger trifft die Aussage für dich zu. Umgekehrt trifft deine Aussage umso mehr zu, je weiter rechts du anklickst.
	
	Selbstverständlich wird alles, was ich aufgezeichnet habe, vertraulich behandelt und anonymisiert ausgewertet."'
	
	\item Teilnehmer hat Konsensformular unterschrieben
	
	"`Danke für deine Teilnahme. Hast du noch Fragen?"' 
\end{enumerate}
\newpage

% section ablauf_und_wortlaut_der_laufsitzungen_der_finalen_laufstudie (end)
\section{Konsensformular} 

% (fold)
\label{sec:konsensformular}

{\Huge Einverständniserklärung}\\
\begin{singlespace}
	Ich \hrulefill 
	\begin{center}
		(Name, Vorname) 
	\end{center}
	Geburtsdatum \hrulefill 
	\begin{center}
		Erkläre, dass ich die Probandeninformation zur Studie\\
		"`Flow beim Laufen"'\\
		und diese Einverständniserklärung zur Studienteilnahme erhalten habe. 
	\end{center}
	\begin{itemize}
		\item Ich wurde für mich ausreichend mündlich und/oder schriftlich über die wissenschaftliche Untersuchung informiert. 
		\item Ich erkläre mich bereit, dass im Rahmen der Studie Daten über mich gesammelt und anonymisiert aufgezeichnet werden. Es wird gewährleistet, dass meine personenbezogenen Daten nicht an Dritte weitergegeben werden. Bei der Veröffentlichung in einer wissenschaftlichen Zeitung wird aus den Daten nicht hervorgehen, wer an dieser Untersuchung teilgenommen hat. Meine persönlichen Daten unterliegen dem Datenschutzgesetz. 
		\item Ich weiß, dass ich jederzeit meine Einverständniserklärung, ohne Angabe von Gründen, widerrufen kann, ohne dass dies für mich nachteilige Folgen hat. 
		\item Mit der vorstehend geschilderten Vorgehensweise bin ich einverstanden und bestätige dies mit meiner Unterschrift. 
	\end{itemize}
	\vspace{15mm} \hrulefill \\
	(Ort, Datum) \hspace{20mm} (Unterschrift) 
\end{singlespace}
\newpage

% section konsensformular (end)
\section{Anweisungstext der Flow-Kurzskala} 

% (fold)
\label{sec:anweisungstext_der_flow_kurzskala}

Im Folgenden findest du einige Aussagen, die sich auf den eben beendeten Lauf beziehen. Bitte \textbf{versetze dich} nun nochmal \textbf{in den eben beendeten Lauf hinein}. Inwieweit treffen die folgenden Aussagen für dich zu?

Entscheidend ist dabei deine persönliche Einschätzung, es gibt keine richtigen oder falschen Antworten. Dazu steht dir die folgende Antwortskala zur Verfügung: Durch Anklicken einer der Antwortalternativen gebe an, wie sehr die jeweilige Aussage auf \textbf{deine persönliche Einschätzung deines Erlebens während des Laufs} zutrifft oder nicht. Je weiter links du anklickst, umso weniger trifft die Aussage für den eben durchgeführte Lauf zu. Umgekehrt trifft eine Aussage umso mehr zu, je weiter rechts du anklickst. 
\newpage

% section anweisungstext_der_flow_kurzskala (end)
\section{Flow-Kurzskala} 

% (fold)
\label{sec:flow_kurzskala} 

von Falko Rheinberg und Regina Vollmeyer

\includepdf{Fragebogen/flow_kurzskala.pdf} 

% section flow_kurzskala (end)
\section{Flow-State-Scale} 

% (fold) 
\label{sec:flow_state_scale} 

von Susan A. Jackson und Robert C. Eklund 

\includepdf[pages=-]{Fragebogen/flow_state_scale.pdf}

% section flow_state_scale (end)

\section{Game Experience Questionnaire} 

% (fold)
\label{sec:game_experience_questionnaire} 

von Wijnand Ijsselsteijn, Yvonne de Kort und Karolien Poels 

\includepdf{Fragebogen/game_experience_questionnaire.pdf}

% section game_experience_questionnaire (end)

\section{Self-Assessment Manikin} 

% (fold)
\label{sec:self_assessment_manikin} 
\begin{figure}
	[!htb] \centering 
	\includegraphics[width=.7
	\textwidth]{Fragebogen/self_assessment_manikin} \caption{Self-Assessment Manikin (SAM). Self-evaluation scales for the dimensions of valence, arousal, and dominance \citep{Bradley1994}} \label{fig:self_assessment_manikin} 
\end{figure}

% section self_assessment_manikin (end)
% chapter studiendokumente (end)

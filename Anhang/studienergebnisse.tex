

%!TEX root = /Users/sbogutzky/Entwicklung/projects/bogutzky/repositories/2939413/final-draft.tex
\chapter{Weitere Studienergebnisse} 

% (fold)
\label{cha:weitere_studienergebnisse}

\section{Item-Faktor-Korrelation der FKS} 

% (fold)
\label{sec:item_faktor_korrelation_der_fks} 
\begin{table}
	[!htb] \centering \caption[Item-Faktor-Korrelation der Items des Generalfaktors (Laufstudie -- intraindividuell).]{Arithmetisches Mittel, Standardabweichung und Item-Faktor-Korrelation der Items des Generalfaktors der ersten Studie zum Flow-Erleben beim Laufen [$N = 24$].} \label{tab:generalfaktor_1} 
	\begin{tabularx}
		{ 
		\textwidth}{p{.65 
		\textwidth} p{.05 
		\textwidth} p{.05 
		\textwidth} p{.25 
		\textwidth}} \toprule & $M$ & $SD$ & Trennschärfe \\
		\midrule Ich fühle mich optimal beansprucht. & 4,75 & 0,53 & 0,53 \\
		Meine Gedanken bzw. Aktivitäten laufen flüssig und glatt. & 4,71 & 0,69 & 0,75 \\
		Ich merke gar nicht, wie die Zeit vergeht. & 4,33 & 0,70 & 0,67 \\
		Ich habe keine Mühe mich zu konzentrieren. & 5,08 & 0,50 & 0,55 \\
		Mein Kopf ist völlig klar. & 4,71 & 0,75 & 0,32 \\
		Ich bin ganz vertieft in das, was ich gerade mache. & 4,42 & 0,58 & 0,78 \\
		Die richtigen Gedanken/ Bewegungen kommen wie von selbst. & 5,00 & 0,78 & 0,81 \\
		Ich weiß bei jedem Schritt, was ich zu tun habe. & 5,25 & 0,68 & 0,39 \\
		Ich habe das Gefühl, den Ablauf unter Kontrolle zu haben. & 5,29 & 0,55 & 0,65 \\
		Ich bin völlig selbstvergessen. & 3,88 & 0,61 & 0,69 \\
		Gesamtmittelwerte & 4,74 & 0,64 & \\
		\bottomrule 
	\end{tabularx}
\end{table}
\begin{table}
	[!htb] \centering \caption[Item-Faktor-Korrelation der Items des glatten Verlaufs (Laufstudie -- intraindividuell).]{Arithmetisches Mittel, Standardabweichung und Item-Faktor-Korrelation der Items des glatten Verlaufs der ersten Studie zum Flow-Erleben beim Laufen [$N = 24$].} \label{tab:glatter_verlauf_1} 
	\begin{tabularx}
		{ 
		\textwidth}{p{.65 
		\textwidth} p{.05 
		\textwidth} p{.05 
		\textwidth} p{.25 
		\textwidth}} \toprule & $M$ & $SD$ & Trennschärfe \\
		\midrule Ich weiß bei jedem Schritt, was ich zu tun habe. & 5,25 & 0,68 & 0,46 \\
		Die richtigen Gedanken/ Bewegungen kommen wie von selbst. & 5,00 & 0,78 & 0,68 \\
		Ich habe das Gefühl, den Ablauf unter Kontrolle zu haben. & 5,29 & 0,55 & 0,63 \\
		Ich habe keine Mühe mich zu konzentrieren. & 5,08 & 0,50 & 0,66 \\
		Mein Kopf ist völlig klar. & 4,71 & 0,75 & 0,52 \\
		Meine Gedanken bzw. Aktivitäten laufen flüssig und glatt. & 4,71 & 0,69 & 0,82 \\
		Gesamtmittelwerte & 5,01 & 0,66 & \\
		\bottomrule 
	\end{tabularx}
\end{table}
\begin{table}
	[!htb] \centering \caption[Item-Faktor-Korrelation der Items der Absorbiertheit (Laufstudie -- intraindividuell).]{Arithmetisches Mittel, Standardabweichung und Item-Faktor-Korrelation der Items der Absorbiertheit der ersten Studie zum Flow-Erleben beim Laufen [$N = 24$].} \label{tab:absorbiertheit_1} 
	\begin{tabularx}
		{ 
		\textwidth}{p{.65 
		\textwidth} p{.05 
		\textwidth} p{.05 
		\textwidth} p{.25 
		\textwidth}} \toprule & $M$ & $SD$ & Trennschärfe \\
		\midrule Ich bin ganz vertieft in das, was ich gerade mache. & 4,42 & 0,58 & 0,72 \\
		Ich fühle mich optimal beansprucht. & 4,75 & 0,53 & 0,71 \\
		Ich bin völlig selbstvergessen. & 3,88 & 0,61 & 0,87 \\
		Ich merke gar nicht, wie die Zeit vergeht. & 4,33 & 0,70 & 0,79 \\
		Gesamtmittelwerte & 4,34 & 0,61 & \\
		\bottomrule 
	\end{tabularx}
\end{table}

\newpage

\begin{table}
	[!htb] \centering \caption[Item-Faktor-Korrelation der Items des Generalfaktors (Gehstudie -- intraindividuell).]{Arithmetisches Mittel, Standardabweichung und Item-Faktor-Korrelation der Items des Generalfaktors der Machbarkeitsstudie zum Flow-Erleben beim Gehen [$N = 23$].} \label{tab:generalfaktor_2} 
	\begin{tabularx}
		{ 
		\textwidth}{p{.65 
		\textwidth} p{.05 
		\textwidth} p{.05 
		\textwidth} p{.25 
		\textwidth}} \toprule & $M$ & $SD$ & Trennschärfe \\
		\midrule Ich fühle mich optimal beansprucht. & 5,22 & 1,59 & 0,97 \\
		Meine Gedanken bzw. Aktivitäten laufen flüssig und glatt. & 5,09 & 1,68 & 0,96 \\
		Ich merke gar nicht, wie die Zeit vergeht. & 5,17 & 1,56 & 0,92 \\
		Ich habe keine Mühe mich zu konzentrieren. & 4,57 & 1,67 & 0,95 \\
		Mein Kopf ist völlig klar. & 4,91 & 1,65 & 0,95 \\
		Ich bin ganz vertieft in das, was ich gerade mache. & 4,83 & 1,59 & 0,93 \\
		Die richtigen Gedanken/ Bewegungen kommen wie von selbst. & 6,09 & 1,12 & 0,93 \\
		Ich weiß bei jedem Schritt, was ich zu tun habe. & 6,09 & 1,12 & 0,93 \\
		Ich habe das Gefühl, den Ablauf unter Kontrolle zu haben. & 5,39 & 1,03 & 0,91 \\
		Ich bin völlig selbstvergessen. & 5,09 & 1,53 & 0,95 \\
		Gesamtmittelwerte & 5,24 & 1,46 & \\
		\bottomrule 
	\end{tabularx}
\end{table}
\begin{table}
	[!htb] \centering \caption[Item-Faktor-Korrelation der Items des glatten Verlaufs (Gehstudie -- intraindividuell).]{Arithmetisches Mittel, Standardabweichung und Item-Faktor-Korrelation der Items des glatten Verlaufs der Machbarkeitsstudie zum Flow-Erleben beim Gehen [$N = 23$].} \label{tab:glatter_verlauf_2} 
	\begin{tabularx}
		{ 
		\textwidth}{p{.65 
		\textwidth} p{.05 
		\textwidth} p{.05 
		\textwidth} p{.25 
		\textwidth}} \toprule & $M$ & $SD$ & Trennschärfe \\
		\midrule Ich weiß bei jedem Schritt, was ich zu tun habe. & 6,09 & 1,12 & 0,94 \\
		Die richtigen Gedanken/ Bewegungen kommen wie von selbst. & 6,09 & 1,12 & 0,94 \\
		Ich habe das Gefühl, den Ablauf unter Kontrolle zu haben. & 5,39 & 1,03 & 0,92 \\
		Ich habe keine Mühe mich zu konzentrieren. & 4,57 & 1,67 & 0,96 \\
		Mein Kopf ist völlig klar. & 4,91 & 1,65 & 0,95 \\
		Meine Gedanken bzw. Aktivitäten laufen flüssig und glatt. & 5,09 & 1,68 & 0,96 \\
		Gesamtmittelwerte & 5,36 & 1,38 & \\
		\bottomrule 
	\end{tabularx}
\end{table}
\begin{table}
	[!htb] \centering \caption[Item-Faktor-Korrelation der Items der Absorbiertheit (Gehstudie -- intraindividuell).]{Arithmetisches Mittel, Standardabweichung und Item-Faktor-Korrelation der Items der Absorbiertheit der Machbarkeitsstudie zum Flow-Erleben beim Gehen [$N = 23$].} \label{tab:absorbiertheit_2} 
	\begin{tabularx}
		{ 
		\textwidth}{p{.65 
		\textwidth} p{.05 
		\textwidth} p{.05 
		\textwidth} p{.25 
		\textwidth}} \toprule & $M$ & $SD$ & Trennschärfe \\
		\midrule Ich bin ganz vertieft in das, was ich gerade mache. & 4,83 & 1,59 & 0,94 \\
		Ich fühle mich optimal beansprucht. & 5,22 & 1,59 & 0,96 \\
		Ich bin völlig selbstvergessen. & 5,09 & 1,53 & 0,96 \\
		Ich merke gar nicht, wie die Zeit vergeht. & 5,17 & 1,56 & 0,94 \\
		Gesamtmittelwerte & 5,08 & 1,57 & \\
		\bottomrule 
	\end{tabularx}
\end{table}

\newpage

\begin{table}
	[!htb] \centering \caption[Item-Faktor-Korrelation der Items des Generalfaktors (Laufstudie -- interindividuell).]{Arithmetisches Mittel, Standardabweichung und Item-Faktor-Korrelation der Items des Generalfaktors der finalen Studie zum Flow-Erleben beim Laufen [$N = 31$].} \label{tab:generalfaktor_3} 
	\begin{tabularx}
		{ 
		\textwidth}{p{.65 
		\textwidth} p{.05 
		\textwidth} p{.05 
		\textwidth} p{.25 
		\textwidth}} \toprule & $M$ & $SD$ & Trennschärfe \\
		\midrule Ich fühle mich optimal beansprucht. & 5,26 & 1,37 & 0,28 \\
		Meine Gedanken bzw. Aktivitäten laufen flüssig und glatt. & 4,58 & 1,26 & 0,76 \\
		Ich merke gar nicht, wie die Zeit vergeht. & 5,03 & 1,45 & 0,40 \\
		Ich habe keine Mühe mich zu konzentrieren. & 4,74 & 1,21 & 0,74 \\
		Mein Kopf ist völlig klar. & 4,61 & 1,36 & 0,62 \\
		Ich bin ganz vertieft in das, was ich gerade mache. & 5,10 & 1,54 & 0,66 \\
		Die richtigen Gedanken/ Bewegungen kommen wie von selbst. & 4,87 & 1,38 & 0,60 \\
		Ich weiß bei jedem Schritt, was ich zu tun habe. & 5,29 & 1,19 & 0,56 \\
		Ich habe das Gefühl, den Ablauf unter Kontrolle zu haben. & 5,06 & 1,26 & 0,81 \\
		Ich bin völlig selbstvergessen. & 4,00 & 1,46 & 0,51 \\
		Gesamtmittelwerte & 4,85 & 1,35 & \\
		\bottomrule 
	\end{tabularx}
\end{table}
\begin{table}
	[!htb] \centering \caption[Item-Faktor-Korrelation der Items des glatten Verlaufs (Laufstudie -- interindividuell).]{Arithmetisches Mittel, Standardabweichung und Item-Faktor-Korrelation der Items des glatten Verlaufs der finalen Studie zum Flow-Erleben beim Laufen [$N = 31$].} \label{tab:glatter_verlauf_3} 
	\begin{tabularx}
		{ 
		\textwidth}{p{.65 
		\textwidth} p{.05 
		\textwidth} p{.05 
		\textwidth} p{.25 
		\textwidth}} \toprule & $M$ & $SD$ & Trennschärfe \\
		\midrule Ich weiß bei jedem Schritt, was ich zu tun habe. & 5,29 & 1,19 & 0,65 \\
		Die richtigen Gedanken/ Bewegungen kommen wie von selbst. & 4,87 & 1,38 & 0,66 \\
		Ich habe das Gefühl, den Ablauf unter Kontrolle zu haben. & 5,06 & 1,26 & 0,79 \\
		Ich habe keine Mühe mich zu konzentrieren. & 4,74 & 1,21 & 0,82 \\
		Mein Kopf ist völlig klar. & 4,61 & 1,36 & 0,71 \\
		Meine Gedanken bzw. Aktivitäten laufen flüssig und glatt. & 4,58 & 1,26 & 0,82 \\
		Gesamtmittelwerte & 4,86 & 1,28 & \\
		\bottomrule 
	\end{tabularx}
\end{table}
\begin{table}
	[!htb] \centering \caption[Item-Faktor-Korrelation der Items der Absorbiertheit (Laufstudie -- interindividuell).]{Arithmetisches Mittel, Standardabweichung und Item-Faktor-Korrelation der Items der Absorbiertheit der finalen Studie zum Flow-Erleben beim Laufen [$N = 31$].} \label{tab:absorbiertheit_3} 
	\begin{tabularx}
		{ 
		\textwidth}{p{.65 
		\textwidth} p{.05 
		\textwidth} p{.05 
		\textwidth} p{.25 
		\textwidth}} \toprule & $M$ & $SD$ & Trennschärfe \\
		\midrule Ich bin ganz vertieft in das, was ich gerade mache. & 5,10 & 1,54 & 0,64 \\
		Ich fühle mich optimal beansprucht. & 5,26 & 1,37 & 0,47 \\
		Ich bin völlig selbstvergessen. & 4,00 & 1,46 & 0,66 \\
		Ich merke gar nicht, wie die Zeit vergeht. & 5,03 & 1,45 & 0,64 \\
		Gesamtmittelwerte & 4,85 & 1,45 & \\
		\bottomrule 
	\end{tabularx}
\end{table}


\cleardoublepage

% section item_faktor_korrelation_der_fks (end)
\section{Merkmal-Übersichten} 

% (fold)
\label{sec:merkmal_ubersichten} 
\begin{sidewaystable}
	\centering \caption[Übersicht der expliziten und impliziten Merkmale nach Läufen der ersten Studie.]{Übersicht der expliziten und impliziten Merkmale nach Läufen der ersten Studie: Arithmetisches Mittel $\pm$ Standardabweichung zu den sechs Läufen [$N = 4$]. \\
	\hspace{ 
	\textwidth}\emph{Anmerkung}: Bew. = Bewegungsaufwand.} \label{tab:ubersicht_nach_laufen_1} 
	\begin{tabular}
		{lyyyyyyy} \toprule & 03.10 & 10.10 & 17.10 & 24.10 & 31.10 & 07.11 & Gesamt \\
		\midrule Generalfaktor $[1{,} 7]$ & 4{,}57 ; 0{,}05 & 4{,}60 ; 0{,}28 & 4{,}20 ; 0{,}28 & 5{,}00 ; 0{,}16 & 5{,}10 ; 0{,}16 & 4{,}97 ; 0{,}42 & 4{,}74 ; 0{,}39 \\
		Glatter Verlauf $[1{,} 7]$ & 4{,}79 ; 0{,}16 & 4{,}92 ; 0{,}10 & 4{,}46 ; 0{,}34 & 5{,}25 ; 0{,}21 & 5{,}38 ; 0{,}21 & 5{,}25 ; 0{,}50 & 5{,}01 ; 0{,}41 \\
		Absorbiertheit $[1{,} 7]$ & 4{,}25 ; 0{,}20 & 4{,}12 ; 0{,}60 & 3{,}81 ; 0{,}31 & 4{,}62 ; 0{,}14 & 4{,}69 ; 0{,}12 & 4{,}56 ; 0{,}62 & 4{,}34 ; 0{,}47 \\
		AFP $[1{,} 9]$ & 2{,}75 ; 0{,}50 & 2{,}75 ; 0{,}50 & 3{,}75 ; 0{,}50 & 4{,}00 ; 0{,}00 & 3{,}50 ; 0{,}58 & 3{,}50 ; 0{,}58 & 3{,}38 ; 0{,}65 \\
		Herzfrequenz ($1/min$) & 168{,}58 ; 0{,}99 & 175{,}41 ; 1{,}37 & 174{,}19 ; 1{,}32 & 177{,}39 ; 4{,}37 & 175{,}79 ; 4{,}94 & 175{,}02 ; 5{,}94 & 174{,}40 ; 4{,}35 \\
		RMSSD ($ms$) & 10{,}29 ; 1{,}43 & 13{,}11 ; 0{,}86 & 11{,}39 ; 5{,}76 & 10{,}69 ; 1{,}69 & 7{,}59 ; 1{,}76 & 7{,}77 ; 1{,}78 & 10{,}14 ; 3{,}14 \\
		Norm. Shan. Entr. Index & 0{,}03 ; 0{,}03 & 0{,}30 ; 0{,}12 & 0{,}30 ; 0{,}09 & 0{,}14 ; 0{,}12 & 0{,}13 ; 0{,}15 & 0{,}19 ; 0{,}17 & 0{,}18 ; 0{,}15 \\
		Doppelschrittfr. ($1/min$) & 87,83 ; 0,69 & 87,78 ; 0,43 & 87,13 ; 0,72 & 87,98 ; 0,30 & 87,49 ; 0,57 & 87,92 ; 0,63 & 87,69 ; 0,59 \\
		Bew. ($\times 10^3 \: m^2 \cdot s^{-5}$) & 18{,}41 ; 0{,}69 & 16{,}51 ; 0{,}59 & 15{,}52 ; 0{,}74 & 15{,}77 ; 0{,}57 & 15{,}85 ; 0{,}36 & 20{,}46 ; 1{,}26 & 17{,}09 ; 1{,}95 \\
		\bottomrule 
	\end{tabular}
\end{sidewaystable}
\begin{table}
	[!htb] \centering \caption[Übersicht der expliziten und impliziten Merkmale nach Messzeitpunkten der ersten Studie zum Laufen.]{Übersicht der expliziten und impliziten Merkmale nach Messzeitpunkten der ersten Studie zum Laufen: Arithmetisches Mittel $\pm$ Standardabweichung zu den drei Messzeitpunkten [$N = 6$]. \\
	\hspace{ 
	\textwidth}\emph{Anmerkung}: Bew. = Bewegungsaufwand} \label{tab:ubersicht_nach_messzeitpunkten_1} 
	\begin{tabular}
		{lyyyyy} \toprule & 15' & 30' & 45' & 60' & Gesamt \\
		\midrule Generalfaktor $[1{,} 7]$ & 4{,}78 ; 0{,}46 & 4{,}73 ; 0{,}31 & 4{,}75 ; 0{,}51 & 4{,}70 ; 0{,}37 & 4{,}74 ; 0{,}39 \\
		Glatter Verlauf $[1{,} 7]$ & 5{,}19 ; 0{,}39 & 5{,}00 ; 0{,}33 & 4{,}89 ; 0{,}53 & 4{,}95 ; 0{,}42 & 5{,}01 ; 0{,}41 \\
		Absorbiertheit $[1{,} 7]$ & 4{,}17 ; 0{,}61 & 4{,}33 ; 0{,}30 & 4{,}54 ; 0{,}58 & 4{,}33 ; 0{,}38 & 4{,}34 ; 0{,}47 \\
		AFP $[1{,} 9]$ & 2{,}83 ; 0{,}75 & 3{,}33 ; 0{,}52 & 3{,}67 ; 0{,}52 & 3{,}67 ; 0{,}52 & 3{,}38 ; 0{,}65 \\
		Herzfrequenz ($1/min$) & 171{,}22 ; 2{,}99 & 173{,}74 ; 2{,}60 & 175{,}23 ; 3{,}66 & 177{,}39 ; 5{,}80 & 174{,}40 ; 4{,}35 \\
		RMSSD ($ms$) & 9{,}97 ; 3{,}92 & 10{,}55 ; 1{,}58 & 9{,}05 ; 2{,}76 & 11{,}00 ; 4{,}16 & 10{,}14 ; 3{,}14 \\
		Norm. Shan. Entr. Index & 0{,}11 ; 0{,}12 & 0{,}31 ; 0{,}15 & 0{,}19 ; 0{,}13 & 0{,}11 ; 0{,}12 & 0{,}18 ; 0{,}15 \\
		Doppelschrittfr. ($1/min$) & 88,14 ; 0,23 & 87,71 ; 0,44 & 87,19 ; 0,46 & 87,71 ; 0,77 & 87,69 ; 0,59 \\
		Bew. ($\times 10^3 \: m^2 \cdot s^{-5}$) & 17{,}80 ; 2{,}13 & 16{,}86 ; 1{,}79 & 16{,}35 ; 1{,}63 & 17{,}33 ; 2{,}38 & 17{,}09 ; 1{,}95 \\
		\bottomrule 
	\end{tabular}
\end{table}
\begin{sidewaystable}
	\centering \caption[Übersicht der expliziten und impliziten Merkmale nach Gängen der Machbarkeitsstudie.]{Übersicht der expliziten und impliziten Merkmale nach Gängen der Machbarkeitsstudie: Arithmetisches Mittel $\pm$ Standardabweichung zu den sechs Gängen [$N \approx 4$].} \label{tab:ubersicht_nach_gangen_2} 
	\begin{tabular}
		{lyyyyyyy} \toprule & 27.05 & 28.05 & 03.06 & 04.06 & 05.06 & 06.06 & Gesamt \\
		\midrule Generalfaktor $[1{,} 7]$ & 4{,}20 ; 1{,}73 & 5{,}08 ; 1{,}54 & 5{,}33 ; 1{,}30 & 5{,}70 ; 1{,}16 & 5{,}60 ; 1{,}38 & 5{,}58 ; 1{,}40 & 5{,}24 ; 1{,}37 \\
		Glatter Verlauf $[1{,} 7]$ & 4{,}42 ; 1{,}71 & 5{,}21 ; 1{,}44 & 5{,}50 ; 1{,}26 & 5{,}71 ; 1{,}16 & 5{,}67 ; 1{,}30 & 5{,}67 ; 1{,}35 & 5{,}36 ; 1{,}31 \\
		Absorbiertheit $[1{,} 7]$ & 3{,}88 ; 1{,}80 & 4{,}88 ; 1{,}70 & 5{,}08 ; 1{,}38 & 5{,}69 ; 1{,}16 & 5{,}50 ; 1{,}51 & 5{,}44 ; 1{,}53 & 5{,}08 ; 1{,}49 \\
		Herzfrequenz ($1/min$) & 110{,}16 ; 2{,}53 & 110{,}49 ; 1{,}01 & 113{,}72 ; 1{,}91 & 103{,}59 ; 0{,}74 & 112{,}82 ; 2{,}36 & 109{,}37 ; 0{,}81 & 110{,}02 ; 3{,}66 \\
		Norm. Shan. Entr. Index & 0{,}21 ; 0{,}05 & 0{,}29 ; 0{,}07 & 0{,}27 ; 0{,}07 & 0{,}16 ; 0{,}12 & 0{,}13 ; 0{,}04 & 0{,}23 ; 0{,}04 & 0{,}22 ; 0{,}09 \\
		Doppelschrittfr. ($1/min$) & 54{,}89 ; 0{,}76 & 55{,}31 ; 0{,}73 & 57{,}08 ; 1{,}19 & 53{,}52 ; 0{,}77 & 54{,}47 ; 0{,}69 & 55{,}23 ; 0{,}92 & 55{,}08 ; 1{,}33 \\
		\bottomrule 
	\end{tabular}
\end{sidewaystable}
\begin{table}
	[!htb] \centering \caption[Übersicht der expliziten und impliziten Merkmale nach Messzeitpunkten der Machbarkeitsstudie.]{Übersicht der expliziten und impliziten Merkmale nach Messzeitpunkten der Machbarkeitsstudie: Arithmetisches Mittel $\pm$ Standardabweichung zu den sechs Gängen [$N \approx 6$].} \label{tab:ubersicht_nach_messzeitpunkten_2} 
	\begin{tabular}
		{lyyyyy} \toprule & 15' & 30' & 45' & 60' & Gesamt \\
		\midrule Generalfaktor $[1, 7]$ & 3,62 ; 0,68 & 4,72 ; 0,84 & 6,30 ; 0,70 & 6,56 ; 0,24 & 5,24 ; 1,37 \\
		Glatter Verlauf $[1, 7]$ & 3,78 ; 0,65 & 4,86 ; 0,80 & 6,39 ; 0,55 & 6,60 ; 0,28 & 5,36 ; 1,31 \\
		Absorbiertheit $[1, 7]$ & 3,38 ; 0,74 & 4,50 ; 0,92 & 6,17 ; 0,97 & 6,50 ; 0,35 & 5,08 ; 1,49 \\
		Herzfrequenz ($1/min$) & 110,18 ; 4,36 & 110,40 ; 3,83 & 109,27 ; 2,61 & 110,25 ; 4,49 & 110,02 ; 3,66 \\
		Norm. Shan. Entr. Index & 0,18 ; 0,07 & 0,20 ; 0,08 & 0,25 ; 0,11 & 0,24 ; 0,09 & 0,22 ; 0,09 \\
		Doppelschrittfr. ($1/min$) & 55,29 ; 1,76 & 55,04 ; 1,07 & 54,99 ; 1,04 & 55,02 ; 1,67 & 55,08 ; 1,33 \\
		\bottomrule 
	\end{tabular}
\end{table}
